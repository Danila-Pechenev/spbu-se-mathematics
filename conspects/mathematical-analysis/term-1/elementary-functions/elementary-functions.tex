\documentclass[12pt]{article}

% Автор стиля: Сергей Копелиович
% Автор конспекта: Илья Дудников

\usepackage{cmap}
\usepackage[T2A]{fontenc}
\usepackage[utf8]{inputenc}
\usepackage[russian]{babel}
\usepackage{graphicx}
\usepackage{amsthm,amsmath,amssymb}
\usepackage{listings}
\usepackage{color}
\usepackage{xcolor}
\usepackage{array}
\usepackage{mathrsfs}
\usepackage{epigraph}

\usepackage[russian,colorlinks=true,urlcolor=red,linkcolor=blue]{hyperref}
\usepackage{enumerate}
\usepackage{datetime}
\usepackage{fancyhdr}
\usepackage{lastpage}
\usepackage{verbatim}
\usepackage{tikz}
\usetikzlibrary{arrows,decorations.markings,decorations.pathmorphing}
\usepackage{pgfplots}

\usepackage{ifthen}
\usepackage{mathtools}

%\usepackage{tabls}
%\usepackage{tabularx}
%\usepackage{xifthen}
%\listfiles

\usepackage{import}
\usepackage{xifthen}
\usepackage{pdfpages}
\usepackage{transparent}

\def\NAME{Лекция, 10/20/21}

\sloppy
\voffset=-20mm
\textheight=235mm
\hoffset=-22mm
\textwidth=180mm
\headsep=12pt
\footskip=20pt

\parskip=0em
\parindent=0em

\setlength\epigraphwidth{.8\textwidth}

\newlength{\tmplen}
\newlength{\tmpwidth}
\newcounter{listcounter}

% Список с маленькими отступами
\newenvironment{MyList}[1][4pt]{
  \begin{enumerate}[1.]
  \setlength{\parskip}{0pt}
  \setlength{\itemsep}{#1}
}{       
  \end{enumerate}
}
% Вложенный список с маленькими отступами
\newenvironment{InnerMyList}[1][0pt]{
  \vspace*{-0.5em}
  \begin{enumerate}[(a)]
  \setlength{\parskip}{-0pt}
  \setlength{\itemsep}{#1}
}{       
  \end{enumerate}
  \vspace*{-0.5em}
}
% Список с маленькими отступами
\newenvironment{MyItemize}[1][4pt]{
  \begin{itemize}
  \setlength{\parskip}{0pt}
  \setlength{\itemsep}{#1}
}{       
  \end{itemize}
}

% Основные математические символы
\def\TODO{{\color{red}\bf TODO}}
\def\N{\mathbb{N}}       %
\def\R{\mathbb{R}}       %
\def\F2{\mathbb{F}_2}    %
\def\Z{\mathbb{Z}}       %
\def\INF{\t{+}\infty}    % +inf
\def\EPS{\varepsilon}    %
\def\EMPTY{\varnothing}  %
\def\PHI{\varphi}        %
\def\SO{\Rightarrow}     % =>
\def\EQ{\Leftrightarrow} % <=>
\def\t{\texttt}          % mono font
\def\c#1{{\rm\sc{#1}}}   % font for classes NP, SAT, etc
\def\O{\mathcal{O}}      %
\def\NO{\t{\#}}          % #
\def\XOR{\text{ {\raisebox{-2pt}{\ensuremath{\Hat{}}}} }}
\renewcommand{\le}{\leqslant}
\renewcommand{\ge}{\geqslant}
\newcommand{\q}[1]{\langle #1 \rangle}               % <x>
\newcommand\URL[1]{{\footnotesize{\url{#1}}}}        %
% \newcommand{\sfrac}[2]{{\scriptscriptstyle\frac{#1}{#2}}}  % Очень маленькая дробь
% \newcommand{\mfrac}[2]{{\scriptstyle\frac{#1}{#2}}}    % Небольшая дробь
\newcommand{\sfrac}[2]{{\scriptstyle\frac{#1}{#2}}}  % Очень маленькая дробь
\newcommand{\mfrac}[2]{{\textstyle\frac{#1}{#2}}}    % Небольшая дробь

\newcommand{\fix}[1]{{\color{fixcolor}{#1}}} % \underline
\def\bonus{\t{\red{(*)}}}
\def\ifbonus#1{\ifthenelse{\equal{#1}{}}{}{\bonus}}
\def\smallsquare{$\scalebox{0.5}{$\square$}$}

\newlength{\myItemLength}
\setlength{\myItemLength}{0.3em}
\def\ItemSymbol{\smallsquare}
\def\Item{\vspace*{\myItemLength}\ItemSymbol \ \ }

\newcommand{\LET}{%
  % [line width=0.6pt]
  \begin{tikzpicture}%
  \draw(0.8ex,0) -- (0.8ex,1.6ex);%
  \draw(0,1.6ex) -- (0.8ex,1.6ex);%
  \end{tikzpicture}%
  \hspace*{0.1em}%
}

% Отступы
\def\makeparindent{\hspace*{\parindent}\unskip}
\def\up{\vspace*{-0.5em}}%{\vspace*{-\baselineskip}}
\def\down{\vspace*{0.5em}}
\def\LINE{\vspace*{-1em}\noindent \underline{\hbox to 1\textwidth{{ } \hfil{ } \hfil{ } }}}
\def\BOX#1{\mbox{\fbox{\bf{#1}}}}
\def\Pagebreak{\pagebreak\vspace*{-1.5em}}

% Мелкий заголовок
\newcommand{\THEE}[1]{
  \vspace*{0.5em}
  \noindent{\bf \underline{#1}}%\hspace{0.5em}
  \vspace*{0.2em}
}
% Другой тип мелкого заголовка
\newcommand{\THE}[1]{
  \vspace*{0.5em} $\bullet$
  \noindent{\bf #1}%\hspace{0.5em}
  \vspace*{0.2em}
}

\newenvironment{MyTabbing}{
  \t\bgroup
  \vspace*{-\baselineskip}
  \begin{tabbing}
    aaaa\=aaaa\=aaaa\=aaaa\=aaaa\=aaaa\kill
}{
  \end{tabbing}
  \t\egroup
}

% Код с правильными отступами
\lstnewenvironment{code}{
  \lstset{}
%  \vspace*{-0.2em}
}%
{
%  \vspace*{-0.2em}
}
\lstnewenvironment{codep}{
  \lstset{language=python}
}%
{
}

% Формулы с правильными отступами
\newenvironment{smallformula}{
 
  \vspace*{-0.8em}
}{
  \vspace*{-1.2em}
  
}
\newenvironment{formula}{
 
  \vspace*{-0.4em}
}{
  \vspace*{-0.6em}
  
}

% Большая квадратная скобка
\makeatletter
\newenvironment{sqcases}{%
  \matrix@check\sqcases\env@sqcases
}{%
  \endarray\right.%
}
\def\env@sqcases{%
  \let\@ifnextchar\new@ifnextchar
  \left\lbrack
  \def\arraystretch{1.2}%
  \array{@{}l@{\quad}l@{}}%
}
\makeatother

\DeclareMathOperator{\sort}{sort}

% Определяем основные секции: \begin{Lm}, \begin{Thm}, \begin{Def}, \begin{Rem}
\renewcommand{\qedsymbol}{$\blacksquare$}
\theoremstyle{definition} % жирный заголовок, плоский текст
\newtheorem{Thm}{\underline{Теорема}}[subsection] % нумерация будет "<номер subsection>.<номер теоремы>"
\newtheorem{Lm}[Thm]{\underline{Lm}} % Нумерация такая же, как и у теорем
\newtheorem{Ex}[Thm]{Упражнение} % Нумерация такая же, как и у теорем
\newtheorem{Example}[Thm]{Пример} % Нумерация такая же, как и у теорем
\newtheorem{Solution}[Thm]{Решение}
\newtheorem{Code}[Thm]{Код} % Нумерация такая же, как и у теорем
\theoremstyle{plain} % жирный заголовок, курсивный текст
\newtheorem{Def}[Thm]{Def.} % Нумерация такая же, как и у теорем
\theoremstyle{remark} % курсивный заголовок, плоский текст
\newtheorem{Cons}[Thm]{Следствие} % Нумерация такая же, как и у теорем
\newtheorem{Conj}[Thm]{Гипотеза} % Нумерация такая же, как и у теорем
\newtheorem{Prop}[Thm]{Утверждение} % Нумерация такая же, как и у теорем
\newtheorem{Rem}[Thm]{Замечание} % Нумерация такая же, как и у теорем
\newtheorem{Remark}[Thm]{Замечание} % Нумерация такая же, как и у теорем
\newtheorem{Algo}[Thm]{Алгоритм} % Нумерация такая же, как и у теорем

% Определяем ЗАГОЛОВКИ
\def\SectionName{Матанализ}
\def\AuthorName{Илья Дудников}

\newlength{\sectionvskip}
\setlength{\sectionvskip}{0.5em}
\newcommand{\Section}[4][]{
  % Заголовок
  \pagebreak
%  \ifthenelse{\isempty{#1}}{
    \refstepcounter{section}
%  }{}
  \vspace{0.5em}
%  \ifthenelse{\isempty{#1}}{
%    \addtocontents{toc}{\protect\addvspace{-5pt}}%
    \addcontentsline{toc}{section}{\arabic{section}. #2}
%  }{}
  


  % Запомнили название и автора главы
  \gdef\SectionName{#2}
  \gdef\AuthorName{#4}

  % Заголовок страницы
  \lhead{Конспект, \NAME}
  \chead{}
  \rhead{\SectionName}
  \renewcommand{\headrulewidth}{0.4pt}
    
  \lfoot{}
  \cfoot{\thepage\t{/}\pageref*{LastPage}}
  \rfoot{Автор: \AuthorName}
  \renewcommand{\footrulewidth}{0.4pt}
}

\newcommand{\Subsection}[2][]{
  \refstepcounter{subsection}
  \vspace*{1em}
  \ifthenelse{\equal{#1}{}}
    {\addcontentsline{toc}{subsection}{\arabic{section}.\arabic{subsection}. #2}}
    {\addcontentsline{toc}{subsection}{\arabic{section}.\arabic{subsection}. \bonus\,#2}}
  {\color{blue}\bf\large \arabic{section}.\arabic{subsection}. \ifbonus{#1}\,{#2}} 
  \vspace*{0.5em}
  \makeparindent
}
\newcommand{\Subsubsection}[2][]{
  \refstepcounter{subsubsection}
  \vspace*{1em}
  \ifthenelse{\equal{#1}{}}
    {\addcontentsline{toc}{subsubsection}{\arabic{section}.\arabic{subsection}.\arabic{subsubsection}. #2}}
    {\addcontentsline{toc}{subsubsection}{\arabic{section}.\arabic{subsection}.\arabic{subsubsection}. \bonus\,#2}}
  {\color{blue}\bf\large \arabic{section}.\arabic{subsection}.\arabic{subsubsection}. \ifbonus{#1}\,#2}
  \vspace*{0.5em}
  \makeparindent
}

\newcommand{\Header}{
  \pagestyle{empty}
  \renewcommand{\dateseparator}{--}
  \begin{center}
    {\Large\bf 
    Матанализ 1 семестр ПИ,\\
    \vspace{0.3em}
     \NAME}\\
    \vspace{0.7em}
    {Собрано {\today} в {\currenttime}}
  \end{center}

  \LINE
  \vspace{0em}

  \renewcommand{\baselinestretch}{0.98}\normalsize
  \tableofcontents
  \renewcommand{\baselinestretch}{1.0}\normalsize
  \pagebreak
}

\newcommand{\BeginConspect}{
  \pagestyle{fancy}
  \setcounter{page}{1}
}

\definecolor{mygray}{rgb}{0.7,0.7,0.7}
\definecolor{ltgray}{rgb}{0.9,0.9,0.9}
\definecolor{fixcolor}{rgb}{0.7,0,0}
\definecolor{red2}{rgb}{0.7,0,0}
\definecolor{dkred}{rgb}{0.4,0,0}
\definecolor{dkblue}{rgb}{0,0,0.6}
\definecolor{dkgreen}{rgb}{0,0.6,0}
\definecolor{brown}{rgb}{0.5,0.5,0}

\newcommand{\green}[1]{{\color{green}{#1}}}
\newcommand{\black}[1]{{\color{black}{#1}}}
\newcommand{\red}[1]{{\color{red}{#1}}}
\newcommand{\dkred}[1]{{\color{dkred}{#1}}}
\newcommand{\blue}[1]{{\color{blue}{#1}}}
\newcommand{\dkgreen}[1]{{\color{dkgreen}{#1}}}

\newcommand{\Mod}[1]{\ (\mathrm{mod}\ #1)}

\begin{document}

\Header

\BeginConspect

\Section{Элементарные функции}{}{Илья Дудников}

\Subsection{Постоянная}

$f(x) = c, x \mapsto c$, непрерывна на $\R$

\Subsection{Степенная функция}

$e_\alpha (x) = x^\alpha$

При $\alpha = 1 \ e_1 (x) = x$ -- непрерывна на $\R$

При $\alpha = n \in \N$ 
\[e_\alpha (x) = x^n\]
Следовательно $e_n (x)$ непрерывна на $\R$ как произведение непрерывных.

При $\alpha = -n, n \in \N$
\[x^{-n} = \frac{1}{x^n}, \t x \in \R \setminus \{0\}\] 
Непрерывна на $\R \setminus \{0\}$ как частное непрерывных.

При $\alpha = 0$ полагаем $x^0 = 1$ при всех $x \neq 0$. Можно доопределить до непрерывности ( $0^0 = 1$ )

Если $n$ нечётно, то $e_n$ строго возрастает на $\R, \sup_{x \in \R} e_n(x) = +\infty, \inf_{x \in \R} e_n(x) = -\infty$. По теореме о сохранении промежутка $e_n(\R) = \R$.

Если $n$ четно, то функция $e_n$ строго возрастает на $\R_+, \sup_{x \in \R_+} e_n(x) = +\infty, \min_{x \in \R_+} e_n(x) = 0, e_n(\R_+) = \R_+$.
По теореме о существовании и непрерывности обратной функции существует и непрерывна функция

\[e_{\frac{1}{n}} = \begin{cases}
    e_n^{-1}, n \ \not\vdots \ 2 \\
    \left(e_n|_{R_+}\right)^{-1}, n \ \vdots \ 2 
\end{cases}\]
Это $\sqrt[n]{x}$, строго возрастает и непрерывна на $\R_+$

Теперь определим $x^\alpha$ при рациональном $\alpha = r = \frac{p}{q}, p \in \Z, q \in \N, \frac{p}{q}$ несократима.
\[x^r = (x^p)^{\frac{1}{q}} (e_r = e_{\frac{1}{q}} \circ e_p)\] 
Таким образом, $x^r$ определено следующим образом.
\begin{align*}
    &x > 0, r \text{ любое}, \\
    &x = 0, r \geqslant 0, \\ 
    &x < 0, q \ \not\vdots 2
\end{align*}
$e_r$ непрерывна на своей области определения, строго возрастает на $[0, +\infty)$ при $r > 0$, строго убывает на $(0, +\infty)$ при $r < 0$

\Pagebreak
\Subsection{Показательная функция}

$0^x = 0 \ \forall x > 0$

Пусть $a > 0$. Пока что $a^x$ определена только для $x \in \mathbb{Q}$. Обозначим эту функцию $a^x |_\mathbb{Q}$. Её свойства:
\begin{MyList}
    \item $r < s \SO a^r < a^s, a > 1$ и $a^r > a^s, 0 < a < 1$
    \item $a^{r + s} = a^r a^s$
    \item $(a^r)^s = a^{rs}$
    \item $(ab)^r = a^r b^r$   
\end{MyList}  

\begin{Def}
    Пусть $a > 0, x \in \R$ Положим
    \[a^x = \lim_{r \to x} a^r |_\mathbb{Q}\] 
\end{Def}

\begin{Lm}
    Пусть $a > 0, \{r_n\}$ -- последовательность рациональных чисел, $r_n \to 0$. Тогда $a^{r_n} \to 1$.   
\end{Lm}

\begin{proof}
    При $a = 1$ лемма очевидно, т.к. $a^{r_n} = 1 \ \forall n$.
    
    Пусть $a > 1$. Докажем лемму в частном случае $r_n = \frac{1}{n}$. Поскольку $a^{\frac{1}{n}} > 1$, имеем $a^{\frac{1}{n}} = 1 + \alpha_n, \alpha_n > 0$. Тогда по неравенству Бернулии
    \[a = (1 + \alpha_n)^n \geqslant 1 + n \alpha_n\]
    Откуда $0 < \alpha_n < \frac{a - 1}{n} \SO \alpha_n \to 0 \SO a^{\frac{1}{n}} \to 1$.

    Далее, по доказанному 
    \[a^{-\frac{1}{n}} = \frac{1}{a^{\frac{1}{n}}} \to \frac{1}{1} = 1\]
    Пусть теперь $\{r_n\}$ -- произвольная последовательность из условия леммы. Возьмем $\varepsilon > 0$. $\exists N_0:$
    \[1 - \varepsilon < a^{-\frac{1}{N_0}} < a^{\frac{1}{N_0}} < 1 + \varepsilon\]
    Поскольку $r_n \to 0$, найдется такой номер $N$, что $\forall n > N \to -\frac{1}{N_0} < r_n < \frac{1}{N_0}$. В силу строгой монотонности показательной функции рационального аргумента
    \[1 - \varepsilon < a^{-\frac{1}{N_0}} < a^{r_n} < a^{\frac{1}{N_0}} < 1 + \varepsilon\]
    Значит $a^{r_n} \to 1$
    
    Если $0 < a < 1$, то $\frac{1}{a} > 1$, и по доказанному
    \[a^{r_n} = \frac{1}{\left(\frac{1}{a}\right)^{r_n}} \to 1\]  
\end{proof}

\begin{Lm}
    Пусть $a > 0, x \in \R, \{r_n\}$ -- последовательность рациональных чисел, $r_n \to x$. Тогда существует конечный предел последовательности $\{a^{r_n}\}$
\end{Lm}

\begin{proof}
    При $a = 1$ лемма очевидна. 
    
    Пусть $a > 1$. Возьмем какую-либо возрастающую последовательность $\{s_n\}$ рациональных чисел, стремящуюся к $x$. Например
    \[s_n = \frac{[10^n x]}{10^n}\]
    Тогда $x - \frac{1}{10^n} < s_n \leqslant x \SO s_n \to x$. Докажем, что последовательность $\{s_n\}$ возрастает. Пусть $A = 10^n x$.
    Тогда $s_n \leqslant s_{n + 1} \EQ 10[A] \leqslant [10A]$, но $10[A]$ -- целое число, не превосходящее $10A$.
    
    $\{a^{s_n}\}$ возрастает и ограничена сверху числом $a^{[x] + 1}$. Значит $\{a^{s_n}\}$ сходится к некоторому пределу $L$. Но тогда    
    \[a^{r_n} = a^{r_n - s_n}a^{s_n} \to L\]
    Потому что $a^{r_n - s_n} \to 1$ по предыдущей лемме.
    
    Если $0 < a < 1$, то $\frac{1}{a} > 1$ и по доказанному $\left(\frac{1}{a}\right)^{r_n} \to L, L > 0$. Тогда
    \[a^{r_n} = \frac{1}{\left(\frac{1}{a}\right)^{r_n}} \to \frac{1}{L}\]   
\end{proof}

\end{document}