\documentclass[12pt]{article}

% Автор стиля: Сергей Копелиович
% Автор конспекта: Илья Дудников

\usepackage{cmap}
\usepackage[T2A]{fontenc}
\usepackage[utf8]{inputenc}
\usepackage[russian]{babel}
\usepackage{graphicx}
\usepackage{amsthm,amsmath,amssymb}
\usepackage{listings}
\usepackage{color}
\usepackage{xcolor}
\usepackage{array}
\usepackage{mathrsfs}
\usepackage{epigraph}

\usepackage[russian,colorlinks=true,urlcolor=red,linkcolor=blue]{hyperref}
\usepackage{enumerate}
\usepackage{datetime}
\usepackage{fancyhdr}
\usepackage{lastpage}
\usepackage{verbatim}
\usepackage{tikz}
\usetikzlibrary{arrows,decorations.markings,decorations.pathmorphing}
\usepackage{pgfplots}

\usepackage{ifthen}
\usepackage{mathtools}

%\usepackage{tabls}
%\usepackage{tabularx}
%\usepackage{xifthen}
%\listfiles

\usepackage{import}
\usepackage{MnSymbol}
\usepackage{xifthen}
\usepackage{pdfpages}
\usepackage{transparent}

\def\NAME{Лекция, 11/10/21}

\sloppy
\voffset=-20mm
\textheight=235mm
\hoffset=-22mm
\textwidth=180mm
\headsep=12pt
\footskip=20pt

\parskip=0em
\parindent=0em

\setlength\epigraphwidth{.8\textwidth}

\newlength{\tmplen}
\newlength{\tmpwidth}
\newcounter{listcounter}

% Список с маленькими отступами
\newenvironment{MyList}[1][4pt]{
  \begin{enumerate}[1.]
  \setlength{\parskip}{0pt}
  \setlength{\itemsep}{#1}
}{       
  \end{enumerate}
}
% Вложенный список с маленькими отступами
\newenvironment{InnerMyList}[1][0pt]{
  \vspace*{-0.5em}
  \begin{enumerate}[(a)]
  \setlength{\parskip}{-0pt}
  \setlength{\itemsep}{#1}
}{       
  \end{enumerate}
  \vspace*{-0.5em}
}
% Список с маленькими отступами
\newenvironment{MyItemize}[1][4pt]{
  \begin{itemize}
  \setlength{\parskip}{0pt}
  \setlength{\itemsep}{#1}
}{       
  \end{itemize}
}

% Основные математические символы
\def\TODO{{\color{red}\bf TODO}}
\def\Q{\mathbb{Q}}
\def\N{\mathbb{N}}       %
\def\R{\mathbb{R}}       %
\def\F2{\mathbb{F}_2}    %
\def\Z{\mathbb{Z}}       %
\def\INF{\t{+}\infty}    % +inf
\def\EPS{\varepsilon}    %
\def\EMPTY{\varnothing}  %
\def\PHI{\varphi}        %
\def\SO{\Rightarrow}     % =>
\def\EQ{\Leftrightarrow} % <=>
\def\t{\texttt}          % mono font
\def\c#1{{\rm\sc{#1}}}   % font for classes NP, SAT, etc
\def\O{\mathcal{O}}      %
\def\NO{\t{\#}}          % #
\def\XOR{\text{ {\raisebox{-2pt}{\ensuremath{\Hat{}}}} }}
\renewcommand{\le}{\leqslant}
\renewcommand{\ge}{\geqslant}
\newcommand{\q}[1]{\langle #1 \rangle}               % <x>
\newcommand\URL[1]{{\footnotesize{\url{#1}}}}        %
% \newcommand{\sfrac}[2]{{\scriptscriptstyle\frac{#1}{#2}}}  % Очень маленькая дробь
% \newcommand{\mfrac}[2]{{\scriptstyle\frac{#1}{#2}}}    % Небольшая дробь
\newcommand{\sfrac}[2]{{\scriptstyle\frac{#1}{#2}}}  % Очень маленькая дробь
\newcommand{\mfrac}[2]{{\textstyle\frac{#1}{#2}}}    % Небольшая дробь

\newcommand{\fix}[1]{{\color{fixcolor}{#1}}} % \underline
\def\bonus{\t{\red{(*)}}}
\def\ifbonus#1{\ifthenelse{\equal{#1}{}}{}{\bonus}}
\def\smallsquare{$\scalebox{0.5}{$\square$}$}

\newlength{\myItemLength}
\setlength{\myItemLength}{0.3em}
\def\ItemSymbol{\smallsquare}
\def\Item{\vspace*{\myItemLength}\ItemSymbol \ \ }

\newcommand{\LET}{%
  % [line width=0.6pt]
  \begin{tikzpicture}%
  \draw(0.8ex,0) -- (0.8ex,1.6ex);%
  \draw(0,1.6ex) -- (0.8ex,1.6ex);%
  \end{tikzpicture}%
  \hspace*{0.1em}%
}

% Отступы
\def\makeparindent{\hspace*{\parindent}\unskip}
\def\up{\vspace*{-0.5em}}%{\vspace*{-\baselineskip}}
\def\down{\vspace*{0.5em}}
\def\LINE{\vspace*{-1em}\noindent \underline{\hbox to 1\textwidth{{ } \hfil{ } \hfil{ } }}}
\def\BOX#1{\mbox{\fbox{\bf{#1}}}}
\def\Pagebreak{\pagebreak\vspace*{-1.5em}}

% Мелкий заголовок
\newcommand{\THEE}[1]{
  \vspace*{0.5em}
  \noindent{\bf \underline{#1}}%\hspace{0.5em}
  \vspace*{0.2em}
}
% Другой тип мелкого заголовка
\newcommand{\THE}[1]{
  \vspace*{0.5em} $\bullet$
  \noindent{\bf #1}%\hspace{0.5em}
  \vspace*{0.2em}
}

\newenvironment{MyTabbing}{
  \t\bgroup
  \vspace*{-\baselineskip}
  \begin{tabbing}
    aaaa\=aaaa\=aaaa\=aaaa\=aaaa\=aaaa\kill
}{
  \end{tabbing}
  \t\egroup
}

% Код с правильными отступами
\lstnewenvironment{code}{
  \lstset{}
%  \vspace*{-0.2em}
}%
{
%  \vspace*{-0.2em}
}
\lstnewenvironment{codep}{
  \lstset{language=python}
}%
{
}

% Формулы с правильными отступами
\newenvironment{smallformula}{
 
  \vspace*{-0.8em}
}{
  \vspace*{-1.2em}
  
}
\newenvironment{formula}{
 
  \vspace*{-0.4em}
}{
  \vspace*{-0.6em}
  
}

% Большая квадратная скобка
\makeatletter
\newenvironment{sqcases}{%
  \matrix@check\sqcases\env@sqcases
}{%
  \endarray\right.%
}
\def\env@sqcases{%
  \let\@ifnextchar\new@ifnextchar
  \left\lbrack
  \def\arraystretch{1.2}%
  \array{@{}l@{\quad}l@{}}%
}
\makeatother

\DeclareMathOperator{\sort}{sort}

% Определяем основные секции: \begin{Lm}, \begin{Thm}, \begin{Def}, \begin{Rem}
\renewcommand{\qedsymbol}{$\blacksquare$}
\theoremstyle{definition} % жирный заголовок, плоский текст
\newtheorem{Thm}{\underline{Теорема}}[subsection] % нумерация будет "<номер subsection>.<номер теоремы>"
\newtheorem{Lm}[Thm]{\underline{Lm}} % Нумерация такая же, как и у теорем
\newtheorem{Ex}[Thm]{Упражнение} % Нумерация такая же, как и у теорем
\newtheorem{Example}[Thm]{Пример} % Нумерация такая же, как и у теорем
\newtheorem{Solution}[Thm]{Решение}
\newtheorem{Code}[Thm]{Код} % Нумерация такая же, как и у теорем
\theoremstyle{plain} % жирный заголовок, курсивный текст
\newtheorem{Def}[Thm]{Def.} % Нумерация такая же, как и у теорем
\theoremstyle{remark} % курсивный заголовок, плоский текст
\newtheorem{Cons}[Thm]{Следствие} % Нумерация такая же, как и у теорем
\newtheorem{Conj}[Thm]{Гипотеза} % Нумерация такая же, как и у теорем
\newtheorem{Prop}[Thm]{Утверждение} % Нумерация такая же, как и у теорем
\newtheorem{Rem}[Thm]{Замечание} % Нумерация такая же, как и у теорем
\newtheorem{Remark}[Thm]{Замечание} % Нумерация такая же, как и у теорем
\newtheorem{Algo}[Thm]{Алгоритм} % Нумерация такая же, как и у теорем

% Определяем ЗАГОЛОВКИ
\def\SectionName{Матанализ}
\def\AuthorName{Илья Дудников}

\def\SEASON{}

\newlength{\sectionvskip}
\setlength{\sectionvskip}{0.5em}
\newcommand{\Section}[4][]{
  % Заголовок
  \pagebreak
%  \ifthenelse{\isempty{#1}}{
    \refstepcounter{section}
%  }{}
  \vspace{0.5em}
%  \ifthenelse{\isempty{#1}}{
%    \addtocontents{toc}{\protect\addvspace{-5pt}}%
    \addcontentsline{toc}{section}{\arabic{section}. #2}
%  }{}
  \begin{center}
    {\Large \bf Раздел \NO{\arabic{section}}: #2} \\ 
    \vspace{\sectionvskip}
    \ifthenelse{\equal{#3}{}}{}{{\large #3}\\}
  \end{center}

  \LINE

  % Запомнили название и автора главы
  \gdef\SectionName{#2}
  \gdef\AuthorName{#4}

  % Заголовок страницы
  \lhead{\SEASON}
  \chead{}
  \rhead{\SectionName}
  \renewcommand{\headrulewidth}{0.4pt}

  \lfoot{Глава \NO{\arabic{section}}. #3.}
  \cfoot{\thepage\t{/}\pageref*{LastPage}}
  \rfoot{Автор: \AuthorName}
  \renewcommand{\footrulewidth}{0.4pt}
}

\newcommand{\Subsection}[2][]{
  \refstepcounter{subsection}
  \vspace*{1em}
  \ifthenelse{\equal{#1}{}}
    {\addcontentsline{toc}{subsection}{\arabic{section}.\arabic{subsection}. #2}}
    {\addcontentsline{toc}{subsection}{\arabic{section}.\arabic{subsection}. \bonus\,#2}}
  {\color{blue}\bf\large \arabic{section}.\arabic{subsection}. \ifbonus{#1}\,{#2}} 
  \vspace*{0.5em}
  \makeparindent
}
\newcommand{\Subsubsection}[2][]{
  \refstepcounter{subsubsection}
  \vspace*{1em}
  \ifthenelse{\equal{#1}{}}
    {\addcontentsline{toc}{subsubsection}{\arabic{section}.\arabic{subsection}.\arabic{subsubsection}. #2}}
    {\addcontentsline{toc}{subsubsection}{\arabic{section}.\arabic{subsection}.\arabic{subsubsection}. \bonus\,#2}}
  {\color{blue}\bf\large \arabic{section}.\arabic{subsection}.\arabic{subsubsection}. \ifbonus{#1}\,#2}
  \vspace*{0.5em}
  \makeparindent
}

\newcommand{\Header}{
  \pagestyle{empty}
  \renewcommand{\dateseparator}{--}
  \begin{center}
    {\Large\bf 
    Матанализ 1 семестр ПИ,\\
    \vspace{0.3em}
     \NAME}\\
    \vspace{0.7em}
    {Собрано {\today} в {\currenttime}}
  \end{center}

  \LINE
  \vspace{0em}

  \renewcommand{\baselinestretch}{0.98}\normalsize
  \tableofcontents
  \renewcommand{\baselinestretch}{1.0}\normalsize
  \pagebreak
}

\newcommand{\BeginConspect}{
  \pagestyle{fancy}
  \setcounter{page}{1}
}

\definecolor{mygray}{rgb}{0.7,0.7,0.7}
\definecolor{ltgray}{rgb}{0.9,0.9,0.9}
\definecolor{fixcolor}{rgb}{0.7,0,0}
\definecolor{red2}{rgb}{0.7,0,0}
\definecolor{dkred}{rgb}{0.4,0,0}
\definecolor{dkblue}{rgb}{0,0,0.6}
\definecolor{dkgreen}{rgb}{0,0.6,0}
\definecolor{brown}{rgb}{0.5,0.5,0}

\newcommand{\green}[1]{{\color{green}{#1}}}
\newcommand{\black}[1]{{\color{black}{#1}}}
\newcommand{\red}[1]{{\color{red}{#1}}}
\newcommand{\dkred}[1]{{\color{dkred}{#1}}}
\newcommand{\blue}[1]{{\color{blue}{#1}}}
\newcommand{\dkgreen}[1]{{\color{dkgreen}{#1}}}

\DeclareMathOperator{\lcm}{lcm}

\newcommand{\Mod}[1]{\ (\mathrm{mod}\ #1)}

\begin{document}

\Header

\BeginConspect

\begin{Rem}
    Если $f$ дифференцируема в точке $a$, то $f$ непрерывна в точке $a$.
    \begin{proof}
        \[f(x) - f(a) = f'(a)(x - a) + o(x - a), x \to a \SO f(x) \xrightarrow[x \to a]{} f(a)\]
    \end{proof}
    Обратное не выполняется. Например, $f(x) = |x|$ 
\end{Rem}

\Subsection{Правила дифференцирования}

\begin{Thm}[Производная композиции]
    $f: \langle A, B\rangle \to \langle C, D\rangle, g: \langle C, D\rangle \to \R, a \in \langle A, B\rangle$.
    Если $f$ дифференцируема в точке $a$, $g$ дифференцируема в точке $f(a)$, то $g \circ f$ дифференцируема в точке $a$ и 
    \[(g \circ f)'(a) = g'(f(a)) \cdot f'(a)\] 
\end{Thm}

\begin{proof}
    $\exists F : \langle A, B\rangle \to \R, F(a) = f'(a)$ и $f(x) - f(a) = F(x)(x - a), x \in \langle A, B\rangle$, $F$ непрерывна в точке $a$ \\
    $\exists G : \langle C, D\rangle \to \R, G(f(a)) = g'(f(a))$ и $g(y) - g(f(a)) = G(y)(y - f(a)), y \in C, D\rangle$, $G$ непрерывна в точке $f(a)$ 
    Подставим $y = f(x)$
    \[g(f(x)) - g(f(a)) = G(f(x))(f(x) - f(a)) = G(f(x))F(x)(x - a) = H(x)(x - a)\]
    $H(x)$ -- непрерывна в точке $x = a$, $H : \langle A, B\rangle \to \R$.
    Тогда $(g \circ f)'(a) = H(a) = G(f(a)) \cdot F(a) = g'(f(a)) \cdot f'(a)$.
\end{proof}

\begin{Rem}
    Это "правило цепочки".
    \[(g \circ h \circ f)'(a) = g'(h \circ f(a)) \cdot h'(f(a)) \circ f'(a)\]
\end{Rem}

\begin{Thm}[Арифметические операции]
    $f, g: \langle A, B\rangle \to \R, a \in \langle A, B\rangle, f, g$ -- дифференцируемы в точке $a$. Тогда

    \begin{MyList}
        \item $\forall \alpha, \beta \in \R$, то $\alpha f + \beta g$ -- дифференцируемая в точке $a$ функция и 
        \[(\alpha f + \beta g)'(a) = \alpha f'(a) + \beta g'(a)\]
        \item $f \cdot g$ -- дифференцируема в точке $a$ и
        \[(f \cdot g)'(a) = f'(a) \cdot g(a) + f(a) \cdot g'(a)\]
        \item если $g(a) \neq 0$, то $\frac{f}{g}$ -- дифференцируема в точке $a$ и
        \[\left(\frac{f}{g}\right)'(a) = \frac{f'(a) \cdot g(a) - f(a) \cdot g'(a)}{g^2(a)}\]
    \end{MyList}
\end{Thm}

\begin{proof}
    \begin{MyList}
        \item $(\alpha f + \beta g)'(a) = \lim_{x \to a} \frac{(\alpha f(x) + \beta g(x)) - (\alpha f (a) + \beta g(a))}{x - a} =$ 
        \[= \lim_{x \to a} \left(\alpha \cdot \frac{f(x) - f(a)}{x - a} + \beta \cdot \frac{g(x) - g(a)}{x - a}\right) = \alpha \lim_{x \to a} \frac{f(x) - f(a)}{x - a} + \beta \lim_{x \to a} \frac{g(x) - g(a)}{x - a} = \alpha f'(a) + \beta g'(a)\]
        
        \item Докажем частный случай $g = f$, т.е. докажем $(f^2)'(a) = 2f'(a)f(a)$. 
        Возьмем $h(t) = t^2$, тогда $f^2(x) = (h \circ f)(x)$. Тогда по предыдущей теореме
        \[(f^2)'(a) = h'(f(a)) \cdot f'(a) = 2 \cdot f(a) \cdot f'(a)\]
        Вернемся к общей формуле:
        \[f \cdot g = \frac{1}{4}((f + g)^2 - (f - g)^2) \SO (f \cdot g)'(a) = \frac{1}{4}((f + g)^2 - (f - g)^2)'(a) = \frac{1}{4} (2 \cdot (f(a) + g(a)) \cdot (f'(a) + g'(a)) - 2(f(a) - g(a)) \cdot (f'(a) - g'(a))) =\] 
        \[= \frac{1}{2}(2 f(a) \cdot g'(a) + 2f'(a) \cdot g(a)) = f(a)g'(a) + f'(a)g(a)\]

        \begin{Ex}
            Получить эту формулу непосредственно из определения производной
        \end{Ex}

        \item $\left(\frac{1}{g}\right)'(a) = - \frac{g'(a)}{g^2(a)}$. Возьмем $h(t) = \frac{1}{t} \SO \frac{1}{g(x)} = (h \circ g)(x)$ 
        \[\left(\frac{1}{g}\right)'(a) = h'(g(a)) \cdot g'(a) = -\frac{1}{g^2(a)} \cdot g'(a)\]
        Теперь $f \cdot \frac{1}{g}$.
        \[\left(\frac{f}{g}\right)'(a) = \left(f \cdot \frac{1}{g}\right)'(a) = f'(a) \cdot \frac{1}{g(a)} + f(a) \cdot - \frac{g'(a)}{g^2(a)} = \frac{f'(a)g(a) - f(a)g'(a)}{g^2(a)}\]
    \end{MyList}
\end{proof}

\begin{Cons}
    \[(f \cdot (h \cdot g))'(a) = f'(a) \cdot (h \cdot g)(a) + f(a) \cdot (h \cdot g)'(a) = f'(a) \cdot h(a) \cdot g(a) + f(a)(h'(a)g(a) + h(a)g'(a)) = \] 
    \[= f'(a) h(a)g(a) + f(a)h'(a) + g(a) + f(a)h(a)g'(a)\]
\end{Cons}

\begin{Thm}[Дифференцирование обратной функции]
    $f$ -- строго монотонная непрерывная функция на $\langle A, B\rangle$, $a \in \langle A, B\rangle, f$ -- дифференцируема в точке $a$ и $f'(a) \neq 0$. 
    Тогда $f^{-1} $ -- дифференцируема в точке $f(a)$ и $(f^{-1})'(f(a)) = \frac{1}{f'(a)}$.
    
    \begin{Rem}
        Геометрический смысл. Рисунок: \TODO 
    \end{Rem}
\end{Thm}

\begin{proof}
    $g(x) = f^{-1}(x), f(a) = b$. $f: \langle A, B\rangle \xrightarrow{\text{на}} \langle C, D\rangle, g : \langle C, D\rangle \xrightarrow{\text{на}} \langle A, B\rangle$ -- непрерывны.
    
    $f$ -- дифференцируема, тогда $\exists F(x) : \langle A, B\rangle$ непрерывная в точке $a$ 
    \[F(a) = f'(a), f(x) - f(a) + F(x)(x - a)\]
    $f$ строго монотонна $\SO \forall x \neq a \ f(x) \neq f(a) \SO F(x) \neq 0$ если $x \neq a$ и по условию $f'(a) = F(a) \neq 0$,
    т.е. $F(x) \neq 0 \ \forall x \in \langle A, B\rangle$ 

    \[x = g(y) \ (y = f(x))\]
    Тогда $y - b = f(x) - f(a) = F(x)(x - a) = F(g(y))(g(y) - g(b)) \SO g(y) - g(b) = \frac{1}{F(g(y))} (y - b) = H(y)(y - b)$ \\
    $H$ определена на $\langle C, D\rangle$, непрерывна в точке $b = f(a) \SO g'(b) = H(b) = \frac{1}{F(g(b))} = \frac{1}{F(a)} = \frac{1}{f'(a)}$ 
\end{proof}

\Subsection{Формулы для вычисления производных}

$f'(a), a \in E$ $a \mapsto f'(a)$

\begin{MyList}
    \item $f(x) \equiv 1, a \in \R$
    \[f'(a) = \lim_{x \to a} \frac{f(x) - f(a)}{x - a} = \lim_{x \to a} \frac{1 - 1}{x - a} = 0\]
    
    \item $f(x) = b^x, b > 0, a \in \R$
    \[f'(x) = \lim_{x \to a} \frac{b^x - b^a}{x - a} = \lim_{x \to a} b^a \cdot \frac{b^{x - a} - 1}{x - a} = b^a \cdot \ln a\]
    В частности, $(e^x)' = e^x$

    \item $f(x) = \log_b x, b > 0, b \neq 1, a \in (0, +\infty)$
    \[\lim_{x \to a} \frac{\log_b x - \log_b a}{x - a} = \lim_{x \to a} \frac{\log_b \frac{x}{a}}{x - a}\]
    \[\frac{x}{a} \xrightarrow[x \to a]{} 1 \SO \log_b \frac{x}{a} = \frac{\ln \frac{x}{a}}{\ln b} = \ln \frac{\left(1 + \left(\frac{x}{a} - 1\right)\right)}{\ln b} \thicksim \frac{\frac{x}{a} - 1}{\ln b}\]
    \[\SO \lim_{x \to a} \frac{\log_b \frac{x}{a}}{x - a} = \lim_{x \to a} \frac{\frac{x}{a} - 1}{(x - a)\ln b} = \lim_{x \to a} \frac{x - a}{a(x - a)\ln b} = \frac{1}{a\ln b}\]
    Значит
    \[(\log_b x)' = \frac{1}{x\ln b}\]
    В частности, $(\ln x)' = \frac{1}{x}$ 

    \item $f(x) = x^\alpha, \alpha \neq 0$
    \subitem $\alpha \in \N, x \in \R$
    \subitem $\alpha \in \Z \setminus \N, x \in \R \setminus \{0\}$ 
    \subitem $\alpha \in \Q, \alpha = \frac{m}{2n + 1}, n \in \N, x \in \R (\alpha > 0), x \in \R \setminus \{0\}(\alpha < 0)$ 
    \subitem $\alpha = \frac{m}{2n}, \alpha \in \R \setminus \Q, x \in [0, +\infty) (\alpha > 0), x \in (0, +\infty) (\alpha < 0)$
    
    \[\lim_{x \to a} \frac{x^\alpha - a^\alpha}{x - a} = \lim_{x \to a} a^\alpha \cdot \frac{\left(\frac{x}{a}\right)^\alpha - 1}{x - a} = \lim_{x \to a} a^\alpha \frac{\left(1 + \left(\frac{x}{a} - 1\right)\right)^\alpha - 1}{x - a} = \lim_{x \to a} a^\alpha \cdot \frac{\alpha \cdot \left(\frac{x}{a} - 1\right)}{x - a} = \]
    \[= \lim_{x \to a} a^\alpha \cdot \frac{\alpha(x - a)}{a(x - a)} = \alpha \cdot a^{\alpha - 1}\]

    \[f'(0) = \lim_{x \to 0} \frac{x^\alpha}{x} = \lim_{x \to 0} x^{\alpha - 1} = \begin{cases}
        0, \alpha > 1 \\
        1, \alpha = 1 \\
        \infty, \alpha < 1
    \end{cases}\]
    Выводы: $(x^\alpha)' = \alpha \cdot x^{\alpha - 1}$ (с точностью до области определения функции).

    \item $f(x) = \sin x, a \in \R$ 
    \[f'(a) = \lim_{x \to a} \frac{\sin x - \sin a}{x - a} = \lim_{x \to a} \frac{2 \sin \frac{x - a}{2} \cos \frac{x + a}{2}}{x - a} = \lim_{x \to a}2 \frac{ \frac{x - a}{2} \cdot \cos a}{x - a} = \cos a\]
    
    \item $f(x) = \cos x$
    \[(\cos x)' = \left(\sin \left(\frac{\pi}{2} - x\right)\right)' = \cos \left(\frac{\pi}{2} - x\right) \cdot \left(\frac{\pi}{2} - x\right) = -\sin x\]

    \item $f(x) = \tg x, a \neq \frac{\pi}{2} + \pi k, k \in \Z$ 
    \[(\tg x)' = \left( \frac{\sin x}{\cos x}\right)' = \frac{(\sin x)' \cos x - \sin x (\cos x)'}{\cos^2 x} = \frac{\cos^2 x + \sin^2x}{\cos^2 x} = \frac{1}{\cos^2 x}\]

    \item $f(x) = \ctg x, a \neq \pi k, k \in \Z$ 
    \[(\ctg x)' = \left(\tg \left(\frac{\pi}{2} - x\right)\right)' = \frac{1}{\cos \left(\frac{\pi}{2} - x\right)} \cdot \left(\frac{\pi}{2} - x\right)' = -\frac{1}{\sin^2 x}\]

    \item $f(x) = \arcsin x, x \in [-1, 1]$. Пусть $g(y) = \sin y \SO b = \arcsin a, g'(b) = \cos b > 0$, т.к. $b \in \left[-\frac{\pi}{2}, \frac{\pi}{2}\right]$ 
    \[f'(a) = \frac{1}{g'(b)} = \frac{1}{\cos b} = \frac{1}{\sqrt{1 - \sin^2 b}} = \frac{1}{\sqrt{1 - a^2}}\]
    \[(\arcsin x)' = \frac{1}{\sqrt{1 - x^2}}, x \in (-1, 1)\]

    \item $(\arccos x)' = \left(\frac{\pi}{2} - \arcsin x\right) = -\frac{1}{\sqrt{1 - x^2}}, x \in (-1, 1)$
    \item $f(x) = \arctg x, g(y) = \tg y, b \arctg a, b \in \left(-\frac{\pi}{2}, \frac{\pi}{2}\right)$
    \[f'(a) = \frac{1}{g'(b)} = \cos^2 b = \frac{1}{\tg^2 b + 1} = \frac{1}{a^2 + 1}\]
    \[(\arctg x)' = \frac{1}{1 + x^2}\]

    \item $(\arcctg x)' = \left(\frac{\pi}{2} - \arctg x\right)' = -\frac{1}{x^2 + 1}$ 
\end{MyList} 

\Subsection{Теоремы о средних}

\begin{Thm}[Теорема Ферма]
    $a \in (A, B), f : \langle A, B\rangle \to \R$ -- дифференцируема в точке $a$. 
    Если $f(a) = \max_{a \in \langle A, B\rangle} f$ или $f(a) = \min_{a \in \langle A, B\rangle} f$, то $f'(a) = 0$.
    
    Геометрический смысл: Картинка \TODO. 
\end{Thm}

\begin{proof}
    $f(a) = \max_{\langle A, B\rangle} \SO f(x) - f(a) \leqslant 0 \ \forall x \in \langle A, B\rangle$. 
    Если $x > a$, то $ \frac{f(x) - f(a)}{x - a} \leqslant 0$ 
    \[f_+'(a) = \lim_{x \to a+} \frac{f(x) - f(a)}{x - a} \leqslant 0\] 
    Если $x < a$, то $ \frac{f(x) - f(a)}{x - a} \geqslant 0$ 
    \[f_-'(a) = \lim_{x \to a-} \frac{f(x) - f(a)}{x - a} \geqslant 0\]
    $f$ дифференцируема в точке $a \SO f_-'(a) = f_+'(a) = f'(a) \SO f'(a) = 0$ 
\end{proof}
\end{document}