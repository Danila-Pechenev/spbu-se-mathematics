\documentclass[12pt]{article}

% Автор стиля: Сергей Копелиович
% Автор конспекта: Илья Дудников    

\usepackage{cmap}
\usepackage[T2A]{fontenc}
\usepackage[utf8]{inputenc}
\usepackage[russian]{babel}
\usepackage{graphicx}
\usepackage{amsthm,amsmath,amssymb}
\usepackage{listings}
\usepackage{color}
\usepackage{xcolor}
\usepackage{array}
\usepackage{epigraph}

\usepackage[russian,colorlinks=true,urlcolor=red,linkcolor=blue]{hyperref}
\usepackage{enumerate}
\usepackage{datetime}
\usepackage{fancyhdr}
\usepackage{lastpage}
\usepackage{verbatim}
\usepackage{tikz}
\usetikzlibrary{arrows,decorations.markings,decorations.pathmorphing}

\usepackage{ifthen}
\usepackage{mathtools}

%\usepackage{tabls}
%\usepackage{tabularx}
%\usepackage{xifthen}
%\listfiles

\def\NAME{Лекция}


\sloppy
\voffset=-20mm
\textheight=235mm
\hoffset=-22mm
\textwidth=180mm
\headsep=12pt
\footskip=20pt

\parskip=0em
\parindent=0em

\setlength\epigraphwidth{.8\textwidth}

\newlength{\tmplen}
\newlength{\tmpwidth}
\newcounter{listcounter}

% Список с маленькими отступами
\newenvironment{MyList}[1][4pt]{
  \begin{enumerate}[1.]
  \setlength{\parskip}{0pt}
  \setlength{\itemsep}{#1}
}{       
  \end{enumerate}
}
% Вложенный список с маленькими отступами
\newenvironment{InnerMyList}[1][0pt]{
  \vspace*{-0.5em}
  \begin{enumerate}[(a)]
  \setlength{\parskip}{-0pt}
  \setlength{\itemsep}{#1}
}{       
  \end{enumerate}
  \vspace*{-0.5em}
}
% Список с маленькими отступами
\newenvironment{MyItemize}[1][4pt]{
  \begin{itemize}
  \setlength{\parskip}{0pt}
  \setlength{\itemsep}{#1}
}{       
  \end{itemize}
}

% Основные математические символы
\def\TODO{{\color{red}\bf TODO}}
\def\N{\mathbb{N}}       %
\def\R{\mathbb{R}}       %
\def\F2{\mathbb{F}_2}    %
\def\Z{\mathbb{Z}}       %
\def\INF{\t{+}\infty}    % +inf
\def\EPS{\varepsilon}    %
\def\EMPTY{\varnothing}  %
\def\PHI{\varphi}        %
\def\SO{\Rightarrow}     % =>
\def\EQ{\Leftrightarrow} % <=>
\def\t{\texttt}          % mono font
\def\c#1{{\rm\sc{#1}}}   % font for classes NP, SAT, etc
\def\O{\mathcal{O}}      %
\def\NO{\t{\#}}          % #
\def\XOR{\text{ {\raisebox{-2pt}{\ensuremath{\Hat{}}}} }}
\renewcommand{\le}{\leqslant}
\renewcommand{\ge}{\geqslant}
\newcommand{\q}[1]{\langle #1 \rangle}               % <x>
\newcommand\URL[1]{{\footnotesize{\url{#1}}}}        %
% \newcommand{\sfrac}[2]{{\scriptscriptstyle\frac{#1}{#2}}}  % Очень маленькая дробь
% \newcommand{\mfrac}[2]{{\scriptstyle\frac{#1}{#2}}}    % Небольшая дробь
\newcommand{\sfrac}[2]{{\scriptstyle\frac{#1}{#2}}}  % Очень маленькая дробь
\newcommand{\mfrac}[2]{{\textstyle\frac{#1}{#2}}}    % Небольшая дробь

\newcommand{\fix}[1]{{\color{fixcolor}{#1}}} % \underline
\def\bonus{\t{\red{(*)}}}
\def\ifbonus#1{\ifthenelse{\equal{#1}{}}{}{\bonus}}
\def\smallsquare{$\scalebox{0.5}{$\square$}$}

\newlength{\myItemLength}
\setlength{\myItemLength}{0.3em}
\def\ItemSymbol{\smallsquare}
\def\Item{\vspace*{\myItemLength}\ItemSymbol \ \ }

\newcommand{\LET}{%
  % [line width=0.6pt]
  \begin{tikzpicture}%
  \draw(0.8ex,0) -- (0.8ex,1.6ex);%
  \draw(0,1.6ex) -- (0.8ex,1.6ex);%
  \end{tikzpicture}%
  \hspace*{0.1em}%
}

% Отступы
\def\makeparindent{\hspace*{\parindent}\unskip}
\def\up{\vspace*{-0.5em}}%{\vspace*{-\baselineskip}}
\def\down{\vspace*{0.5em}}
\def\LINE{\vspace*{-1em}\noindent \underline{\hbox to 1\textwidth{{ } \hfil{ } \hfil{ } }}}
\def\BOX#1{\mbox{\fbox{\bf{#1}}}}
\def\Pagebreak{\pagebreak\vspace*{-1.5em}}

% Мелкий заголовок
\newcommand{\THEE}[1]{
  \vspace*{0.5em}
  \noindent{\bf \underline{#1}}%\hspace{0.5em}
  \vspace*{0.2em}
}
% Другой тип мелкого заголовка
\newcommand{\THE}[1]{
  \vspace*{0.5em} $\bullet$
  \noindent{\bf #1}%\hspace{0.5em}
  \vspace*{0.2em}
}

\newenvironment{MyTabbing}{
  \t\bgroup
  \vspace*{-\baselineskip}
  \begin{tabbing}
    aaaa\=aaaa\=aaaa\=aaaa\=aaaa\=aaaa\kill
}{
  \end{tabbing}
  \t\egroup
}

% Код с правильными отступами
\lstnewenvironment{code}{
  \lstset{}
%  \vspace*{-0.2em}
}%
{
%  \vspace*{-0.2em}
}
\lstnewenvironment{codep}{
  \lstset{language=python}
}%
{
}

% Формулы с правильными отступами
\newenvironment{smallformula}{
 
  \vspace*{-0.8em}
}{
  \vspace*{-1.2em}
  
}
\newenvironment{formula}{
 
  \vspace*{-0.4em}
}{
  \vspace*{-0.6em}
  
}

% Большая квадратная скобка
\makeatletter
\newenvironment{sqcases}{%
  \matrix@check\sqcases\env@sqcases
}{%
  \endarray\right.%
}
\def\env@sqcases{%
  \let\@ifnextchar\new@ifnextchar
  \left\lbrack
  \def\arraystretch{1.2}%
  \array{@{}l@{\quad}l@{}}%
}
\makeatother

% Определяем основные секции: \begin{Lm}, \begin{Thm}, \begin{Def}, \begin{Rem}
\renewcommand{\qedsymbol}{$\blacksquare$}
\theoremstyle{definition} % жирный заголовок, плоский текст
\newtheorem{Thm}{\underline{Теорема}}[subsection] % нумерация будет "<номер subsection>.<номер теоремы>"
\newtheorem{Lm}[Thm]{\underline{Lm}} % Нумерация такая же, как и у теорем
\newtheorem{Ex}[Thm]{Упражнение} % Нумерация такая же, как и у теорем
\newtheorem{Example}[Thm]{Пример} % Нумерация такая же, как и у теорем
\newtheorem{Code}[Thm]{Код} % Нумерация такая же, как и у теорем
%\theoremstyle{plain} % жирный заголовок, курсивный текст
\newtheorem{Def}[Thm]{Def.} % Нумерация такая же, как и у теорем
\theoremstyle{remark} % курсивный заголовок, плоский текст
\newtheorem{Cons}[Thm]{Следствие} % Нумерация такая же, как и у теорем
\newtheorem{Conj}[Thm]{Гипотеза} % Нумерация такая же, как и у теорем
\newtheorem{Prop}[Thm]{Утверждение} % Нумерация такая же, как и у теорем
\newtheorem{Rem}[Thm]{Замечание} % Нумерация такая же, как и у теорем
\newtheorem{Remark}[Thm]{Замечание} % Нумерация такая же, как и у теорем
\newtheorem{Algo}[Thm]{Алгоритм} % Нумерация такая же, как и у теорем

% Определяем ЗАГОЛОВКИ
\def\SectionName{Матанализ}
\def\AuthorName{Илья Дудников}

\newlength{\sectionvskip}
\setlength{\sectionvskip}{0.5em}
\newcommand{\Section}[4][]{
  % Заголовок
  \pagebreak
%  \ifthenelse{\isempty{#1}}{
    \refstepcounter{section}
%  }{}
  \vspace{0.5em}
%  \ifthenelse{\isempty{#1}}{
%    \addtocontents{toc}{\protect\addvspace{-5pt}}%
    \addcontentsline{toc}{section}{\arabic{section}. #2}
%  }{}
  


  % Запомнили название и автора главы
  \gdef\SectionName{#2}
  \gdef\AuthorName{#4}

  % Заголовок страницы
  \lhead{Конспект, \NAME}
  \chead{}
  \rhead{\SectionName}
  \renewcommand{\headrulewidth}{0.4pt}
    
  \lfoot{}
  \cfoot{\thepage\t{/}\pageref*{LastPage}}
  \rfoot{Автор: \AuthorName}
  \renewcommand{\footrulewidth}{0.4pt}
}

\newcommand{\Subsection}[2][]{
  \refstepcounter{subsection}
  \vspace*{1em}
  \ifthenelse{\equal{#1}{}}
    {\addcontentsline{toc}{subsection}{\arabic{section}.\arabic{subsection}. #2}}
    {\addcontentsline{toc}{subsection}{\arabic{section}.\arabic{subsection}. \bonus\,#2}}
  {\color{blue}\bf\large \arabic{section}.\arabic{subsection}. \ifbonus{#1}\,{#2}} 
  \vspace*{0.5em}
  \makeparindent
}
\newcommand{\Subsubsection}[2][]{
  \refstepcounter{subsubsection}
  \vspace*{1em}
  \ifthenelse{\equal{#1}{}}
    {\addcontentsline{toc}{subsubsection}{\arabic{section}.\arabic{subsection}.\arabic{subsubsection}. #2}}
    {\addcontentsline{toc}{subsubsection}{\arabic{section}.\arabic{subsection}.\arabic{subsubsection}. \bonus\,#2}}
  {\color{blue}\bf\large \arabic{section}.\arabic{subsection}.\arabic{subsubsection}. \ifbonus{#1}\,#2}
  \vspace*{0.5em}
  \makeparindent
}

\newcommand{\Header}{
  \pagestyle{empty}
  \renewcommand{\dateseparator}{--}
  \begin{center}
    {\Large\bf 
     Матанализ,
    \vspace{0.3em}
     \NAME}\\
    \vspace{0.7em}
    {Собрано {\today} в {\currenttime}}
  \end{center}

  \LINE
  \vspace{0em}

  \renewcommand{\baselinestretch}{0.98}\normalsize
  \tableofcontents
  \renewcommand{\baselinestretch}{1.0}\normalsize
  \pagebreak
}

\newcommand{\BeginConspect}{
  \pagestyle{fancy}
  \setcounter{page}{1}
}

\definecolor{mygray}{rgb}{0.7,0.7,0.7}
\definecolor{ltgray}{rgb}{0.9,0.9,0.9}
\definecolor{fixcolor}{rgb}{0.7,0,0}
\definecolor{red2}{rgb}{0.7,0,0}
\definecolor{dkred}{rgb}{0.4,0,0}
\definecolor{dkblue}{rgb}{0,0,0.6}
\definecolor{dkgreen}{rgb}{0,0.6,0}
\definecolor{brown}{rgb}{0.5,0.5,0}

\newcommand{\green}[1]{{\color{green}{#1}}}
\newcommand{\black}[1]{{\color{black}{#1}}}
\newcommand{\red}[1]{{\color{red}{#1}}}
\newcommand{\dkred}[1]{{\color{dkred}{#1}}}
\newcommand{\blue}[1]{{\color{blue}{#1}}}
\newcommand{\dkgreen}[1]{{\color{dkgreen}{#1}}}

\begin{document}

\Header

\BeginConspect

\Section{Последовательности}{}{Илья Дудников}
\begin{Def}
    Последовательность --- это отображение $f: \mathbb{N} \to \mathbb{R}$ 
\end{Def}

\begin{Example}
    $x_n = n^2 : x_n = \{1, 4, 9, ...\}$ 
\end{Example}

\Subsection{Предел последовательности и его свойства}

\begin{Def}
    Предел последовательности - это такое число $l = lim_{n \to \infty}x_n$, что 
    \[\forall \varepsilon > 0 \to \exists N \in \mathbb{N} : \forall n \geqslant N \to |x_n - l| < \varepsilon\] 

    Также говорят, что вне любого интервала, содержащего $l$, лежит лишь конечно число элементов $\{X_n\}_{n = 1}^{\infty}$ 
\end{Def}

\begin{Example}
    $x_n = \frac{1}{n}, \frac{1}{n} < \varepsilon \Leftrightarrow n > \frac{1}{\varepsilon}$. Тогда 
    $N_\varepsilon = \left[\frac{1}{\varepsilon} + 1\right]$  
\end{Example}

\begin{Rem}
    $N$ необязательно наименьшее.
\end{Rem}

\begin{Def}
    Последовательность называется \textbf{сходящейся}, если она имеет конечный предел.
\end{Def}

\begin{Def}
    $\lim_{n \to \infty}x_n = +\infty \Leftrightarrow \forall E \in \mathbb{R} \to \exists N \in \mathbb{N} : \forall n \geqslant N \to x_n > E$ 
\end{Def}

\begin{Def}
    $\lim_{n \to \infty}x_n = -\infty \Leftrightarrow \forall E \in \mathbb{R} \to \exists N \in \mathbb{N} : \forall n \geqslant N \to x_n < E$ 
\end{Def}

\begin{Def}[Беззнаковая бесконечность]
    $$\lim_{n \to \infty}x_n = \infty \Leftrightarrow \forall E \in \mathbb{R} \to \exists N \in \mathbb{N} : \forall n \geqslant N \to |x_n| > E$$
\end{Def}

\begin{Def}
    Последовательность называется бесконечно большой, если она стремится к бесконечности 
\end{Def}

\begin{Def}
    Последовательность называется бесконечно малой, если она стремится к нулю
\end{Def}

Свойства пределов последовательности:
\begin{MyList}
    \item Последовательность не может иметь двух различных пределов.
    
    \begin{proof}
        Пусть $a \neq b$ - пределы, $a < b$. Возьмем $\varepsilon = \left(\frac{b - a}{3}\right)$. Тогда по определнию предела вне $\varepsilon$-окрестности $a$ лежит конечно число членов последовательности, и
        вне $\varepsilon$-окрестности $b$ лежит конечно число членов последовательности $\Rightarrow$ сама последовательности конечна !?    
    \end{proof}
    \item Сходящаяся последовательность ограничена.
    \begin{proof}
        Пусть $\lim \{x_n\} = a$. По определению предела для $\varepsilon = 1$ найдем номер $N$ такой, что при всех $n \geqslant N$ имеет место неравенство $|x_n - a| < 1$.
        Так как модуль суммы не превосходит суммы модулей, то 
        \[|x_n| = |x_n - a + a| \leqslant |x_n - a| + |a|\]

        Поэтому при всех $n \geqslant N$ выполняется неравенство
        \[|x_n| < 1 + |a|\]
        Положим $M = \max(1 + |a|, |x_1|, ..., |x_{N - 1}|)$. Тогда $|x_n| \leqslant M \ \forall n \in \N$   
    \end{proof}

    \begin{Prop}
        Пусть $\lim x_n = a, \lim y_n = b$. Тогда $$\forall \varepsilon > 0 \to \exists N \in \N : \forall n \geqslant N \to \begin{cases}
            |x_n - a| < \varepsilon \\
            |y_n - b| < \varepsilon
        \end{cases}$$

        \begin{proof}
            $N_1$ - номер из определения $\lim x_n = a$ \\
            $N_2$ - номер из определения $\lim y_n = b$ \\
            $N = \max \{N_1; N_2\}$ 
        \end{proof}
    \end{Prop}
    \item Пусть $\lim x_n = a, \lim y_n = b, \forall n \in \N \to x_n \leqslant y_n$. Тогда $a \leqslant b$ (предельный переход в неравенстве).
    \begin{proof}
        Пусть $a > b$. Тогда $\exists N$, начиная с которого в $\varepsilon$-окрестности $b$ лежит бесконечное число членов $y_n$, а $\varepsilon$-окрестности $a$ лежит бесконечное число 
        членов $x_n$. Но тогда, если бы возьмем $\varepsilon = \frac{a - b}{3}$, то $\exists y_n \in (b - \varepsilon; b + \varepsilon) : y_n < x_n, x_n \in (a - \varepsilon; a + \varepsilon)$    
    \end{proof}
    \begin{Cons}
        \begin{MyList}
            \item $\lim x_n = a, \forall n \to x_n \leqslant b \Rightarrow a \leqslant b$ 
            \item $\lim y_n = b, \forall n \to y_n \geqslant a \Rightarrow a \leqslant b$  
        \end{MyList} 
    \end{Cons}
    \item \begin{Thm}[Теорема о сжатой последовательности, теорема о двух милиционерах]
        Пусть $\forall n \in \N \to x_n \leqslant z_n \leqslant y_n \wedge \lim x_n = \lim y_n = a$. Тогда $\lim z_n = a$ 
    \end{Thm}
    \begin{proof}
        $\forall \varepsilon > 0 \to \exists N \in \N : \forall n \geqslant N \to x_n, y_n \in (a - \varepsilon; a + \varepsilon)$. Т.к. $x_n \leqslant z_n \leqslant y_n \Rightarrow z_n \in (a - \varepsilon; a + \varepsilon)$  
    \end{proof}
\end{MyList}

\Subsection{Монотонные последовательности}

\begin{Def}
    Последовательность называется возрастающей, если $x_1 \leqslant x_2 \leqslant x_3 \leqslant ...$ 
\end{Def}
\begin{Def}
    Последовательность называется убывающей, если $x_1 \geqslant x_2 \geqslant x_3 \geqslant ...$ 
\end{Def}

\begin{Def}
    Последовательность называется монотонной, если она возрастающая или убывающая.
\end{Def}

\begin{Thm}[О монотонной ограниченной последовательности]
    \begin{MyList}
        \item Возрастающая и ограниченная сверху последовательность сходится
        \item Убывающая и ограниченная снизу последовательность сходится
    \end{MyList}
\end{Thm}

\begin{proof}
    Пусть множество $E = \{x_1, x_2, ...\}, c = \sup E$ 

    $\forall n \in \N \to x_n \leqslant c$. Тогда
    \[\forall \varepsilon > 0 \to \exists N \in \N : x_N > c - \varepsilon\]
    Т.к. $x_n$ возрастает, то \[\forall n > N \to x_n \geqslant x_N > c - \varepsilon \wedge x_n < c + \varepsilon \Leftrightarrow |x_n - c| < \varepsilon\] 
\end{proof}
\begin{Rem}
    \begin{MyList}
        \item Возрастающая и неограниченная сверху последовательность стремится к $+\infty$.
        \item Убывающая и неограниченная снизу последовательность стремится к $-\infty$
    \end{MyList}
    \begin{proof}
        $\forall E \to \exists N : x_N > E$ и $x_n \geqslant x_N$ $\forall n \geqslant N$  
    \end{proof}
\end{Rem}

\Pagebreak
\Subsection{Теорема об арифметических действиях с пределами}
\begin{Thm}[Теорема об арифметических действиях с пределами]
    Пусть $\lim x_n = a, \lim y_n = b, a, b \in \R$. Тогда
    \begin{MyList}
        \item $\lim |x_n| = |a|$ 
        \item $\lim (x_n + y_n) = a + b$
        \item $\lim (x_n - y_n) = a - b$ 
        \item $\lim (x_n \cdot y_n) = a \cdot b$ 
        \item $\forall n \in \N \to b \neq 0 \wedge y_n \neq 0$, то $\lim \frac{x_n}{y_n} = \frac{a}{b}$
    \end{MyList} 

    \begin{proof}
        \begin{MyList}
            \item $\forall \varepsilon > 0 \to \exists N : \forall n \geqslant N \to |x_n - a| < \varepsilon$. Заметим, что \[||x_n| - |a|| < |x_n - a| < \varepsilon\]
            \item $|(x_n + y_n) - (a + b)| \leqslant |x_n - a| + |y_n - b| < \varepsilon$ 
            \item Вместо $y_n$ рассмотрим $-y_n$
            \item $\forall \varepsilon > 0 \to \exists N : \forall n \geqslant N \to \begin{cases}
                |x_n - a| < \frac{\varepsilon}{M + |a|} \\
                |y_n - b| < \frac{\varepsilon}{M + |a|}
            \end{cases}, M : \forall n \to |y_n| < M$  \\
            \begin{equation*}
                |x_n \cdot y_n - ab| = |x_n y_n - ab - a \cdot y_n + a \cdot y_n| = |y_n (x_n - a) + a(y_n - b)| \leqslant |y_n||x_n - a| + |a||y_n - b| < \varepsilon
            \end{equation*}
            \item Достаточно доказать, что $\lim \frac{1}{y_n} = \frac{1}{b}$. 
            \[\left|\frac{1}{y_n} - \frac{1}{b}\right| = \frac{|y_n - b}{|y_n||b|} \leqslant \frac{|y_n - b|}{|\frac{b}{2}||b|} < \frac{\frac{b^2 \varepsilon}{2}}{|\frac{b}{2}||b|} = \varepsilon\]
        \end{MyList}
    \end{proof}
\end{Thm}

\begin{Prop}
    $x_n$ - бесконечно малая, $y_n$ - ограниченная. Тогда $\lim x_n \cdot y_n = 0$
    \begin{proof}
        $|x_n y_n| < |x_n| \cdot M < \varepsilon \cdot M, M : \forall n \in \N \to |y_n| < M$ 
    \end{proof} 
\end{Prop}

\begin{Prop}
    $\forall n \to x_n \neq 0$. Тогда $x_n$ - бесконечно большая $\Leftrightarrow \frac{1}{x_n}$ - бесконечно малая
    
    \begin{proof}
        $|x_n| > E \Leftrightarrow \frac{1}{|x_n|} < \frac{1}{E}$ 
    \end{proof}
\end{Prop}

\Pagebreak
\Subsection{Арифметические действия с бесконечностями}
\begin{MyList}
    \item $\lim x_n = +\infty, y_n$ ограничено снизу. Тогда $\lim (x_n + y_n) = +\infty$
    \item $\lim x_n = -\infty, y_n$ ограничено сверху. Тогда $\lim (x_n + y_n) = -\infty$
    \item $\lim x_n = +\infty, y_n \geqslant c > 0$. Тогда $\lim (x_n \cdot y_n) = +\infty$
    \item $\lim x_n = +\infty, y_n \leqslant c < 0$. Тогда $\lim (x_n \cdot y_n) = -\infty$
    \item $\lim x_n = a \neq 0, \lim y_n = 0$. Тогда $\lim \left(\frac{x_n}{y_n}\right) = \infty$
    \item $\lim x_n = a \in \R, \lim y_n = \infty$. Тогда $\lim \left( \frac{x_n}{y_n} \right) = 0$
    \item $\lim x_n = \infty, \lim y_n = b \in \R \wedge y_n \neq 0$. Тогда $\lim \left( \frac{x_n}{y_n} \right) = \infty$
\end{MyList}

\begin{Rem}
    $\lim x_n = a \in \overline{\R}, \lim y_n = b \in \overline{\R}$, то $\lim (x_n * y_n) = a * b$  
\end{Rem}

Запрещенные операции (неопределённости):
\begin{MyList}
    \item $\pm \infty + (\mp \infty)$ 
    \item $\pm \infty - (\pm \infty)$
    \item $0 \cdot \infty$ 
    \item $\frac{0}{0}$ 
    \item $\frac{\infty}{\infty}$  
\end{MyList}
\Pagebreak
\Subsection{Неравенство Бернулли}
\begin{Thm}[Неравенство Бернулли]
    Пусть $x > -1, n \in \N$. Тогда $(1 + x)^n \geqslant 1 + nx$
    
    \begin{proof}
        База $n = 1 : (1 + x)^1 \geqslant 1 + 1 \cdot x$ \\
        Индукционный переход $n \to n + 1$
        \[(1 + x)(1 + x)^n \geqslant (1 + x)(1 + nx) = 1 + n(x + 1) + nx^2 \geqslant 1 + (n + 1)x\] 
    \end{proof}
\end{Thm}
\end{document}