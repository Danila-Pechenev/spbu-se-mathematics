\documentclass[12pt]{article}
\usepackage[russian,colorlinks=true,urlcolor=red,linkcolor=blue]{hyperref}
\usepackage{cmap}
\usepackage[T2A]{fontenc}
\usepackage[utf8]{inputenc}
\usepackage[english, russian]{babel}
\usepackage{graphicx}
\usepackage{amsthm,amsmath,amssymb}
\usepackage{listings}
\usepackage{color}
\usepackage{xcolor}
\usepackage[normalem]{ulem}
\usepackage{array}
\usepackage{epigraph}

\usepackage{datetime}
\usepackage{fancyhdr}
\usepackage{lastpage}
\usepackage{verbatim}
\usepackage{tikz}
\usetikzlibrary{arrows,decorations.markings,decorations.pathmorphing}
\usepackage{pgfplots}
\usepackage{ifthen}
\usepackage{mathtools}
%\usepackage{tabls}
%\usepackage{tabularx}
%\usepackage{xifthen}
%\listfiles

\def\SEASON{a}

%\input code-format.tex

\sloppy
\voffset=-20mm
\textheight=235mm
\hoffset=-22mm
\textwidth=180mm
\headsep=12pt
\footskip=20pt

\parskip=0em
\parindent=0em

\title{Лекция по математическому анализу №2}
\author{Дудников Илья}
\date{\today}

% Отступы
\def\makeparindent{\hspace*{\parindent}\unskip}
\def\up{\vspace*{-0.5em}}%{\vspace*{-\baselineskip}}
\def\down{\vspace*{0.5em}}
\def\LINE{\vspace*{-1em}\noindent \underline{\hbox to 1\textwidth{{ } \hfil{ } \hfil{ } }}}
\def\BOX#1{\mbox{\fbox{\bf{#1}}}}
\def\Pagebreak{\pagebreak\vspace*{-1.5em}}

\newtheorem{theorem}{Теорема}
\newtheorem{formula}{Формула}
\newtheorem{statement}{Утверждение}

\theoremstyle{definition}
\newtheorem{definition}{Определение}
\newtheorem{consequence}{Следствие}

\renewcommand{\qedsymbol}{$\blacksquare$}
%\renewcommand{\thetheorem}{}
%\renewcommand{\thedefinition}{}

\begin{document}
    \maketitle
    \tableofcontents
    \section{Продолжение про вещественные числа}
    \subsection{Теорема о вложенных отрезках}
    
    \begin{theorem}[Теорема о вложенных отрезка]
        Пусть $[a_1, b_1] \supset [a_2, b_2] \supset [a_3, b_3] \supset ... \supset [a_n, b_n]$. Тогда $$\exists a \in \bigcap_{n = 1}^{\infty}[a_n, b_n]$$
    \end{theorem}

    \begin{proof}
        $a_1 \leqslant a_2 \leqslant a_3 \leqslant ... \leqslant a_n \leqslant b_n \leqslant b_{n - 1} \leqslant ... \leqslant b_1$
        
        Значит, $\forall k, m \to a_k \leqslant b_m$ \\
        Пусть $A = \{a_n\}, B = \{b_n\}$. По аксиоме полноты $\exists c \in \mathbb{R} : \forall k, m \in \mathbb{N} \to a_k \leqslant c \leqslant b_m \Rightarrow c \in \bigcap_{n = 1}^{\infty} [a_n, b_n]$  
    \end{proof}
    Замечания: 1) $\bigcap_{n = 1}^{\infty} \left(0; \frac{1}{n}\right] = \varnothing$. Важно, что именно отрезки, а не интервалы или полуинтервалы.  \\
    2) $\bigcap_{n = 1}^{\infty} [n; +\infty) = ?$ \\
    3) Без аксиомы полноты не работает. Например 
    \[[1.4; 1.5] \supset [1.41; 1.42] \supset ...\]
    \[\bigcap_{n = 1}^{\infty}[a_n, b_n] = \{\sqrt{2}\}, \text{но не в } \mathbb{Q}\]

    \subsection{Принцип математической индукции}
    $\{P_n\}_{n = 1}^{\infty}$ - утверждения. Если
    \begin{enumerate}
        \item $P_1$ верно - база
        \item $\forall n \in \mathbb{N} \to P_n \Rightarrow P_{n + 1}$ - индукционный переход  
    \end{enumerate}
    Тогда $\forall n \in \mathbb{N} \to P_n$.
    \begin{definition}
        $M \subset \mathbb{R}$ - индуктивное, если $1 \in M \wedge (x \in M \Rightarrow x + 1 \in M)$.  
    \end{definition} 

    \begin{definition}
        $\mathbb{N}$ - минимальное индуктивное подмножество $\mathbb{R}$  
    \end{definition}

    \begin{definition}[Сдвиг индекса суммирования]
        $$\sum_{n = m}^k a_n = \sum_{j = m + p}^{k + p}{a_{j - p}}, p \in \mathbb{Z}$$
    \end{definition}

    \begin{definition}
        $k!!$ - произведение целых чисел до $k$ включительно одной четности с $k$.
    \end{definition}

    \begin{definition}[Биномиальные коэффициенты]
        $$C_n^k = \frac{n!}{k!(n - k)!} = \binom{n}{k}$$
    \end{definition}

    \begin{theorem}[Формула бинома Ньютона]
        Пусть $n \in \mathbb{Z}, x, y, \in \mathbb{R}$. Тогда 
        \[(x + y)^n = \sum_{k = 0}^{n}{C_n^k x^k y^{n - k}}\]
    \end{theorem}

    \begin{proof}
        $n = 0 \to 1 = 1$, верно
        
        Индукционный переход: \\
        \begin{align*}
            &(x + y)^{n + 1} = (x + y)(x + y)^n = (x + y)(\sum_{k = 0}^{n}{C_n^k x^k y^{n - k}}) = \\
            &=\sum_{k = 0}^{n}{C_n^k x^{k + 1}y^{n - k}} + \sum_{k = 0}^{n}C_n^k x^k y^{n - k + 1} = \sum_{j = 1}^{n + 1}C_n^{j - 1}x^j y^{n + 1 - j} + \sum_{k = 0}^{n}C_n^k x^k y^{n - k + 1} = \\
            &=\sum_{k = 1}^{n + 1}C_n^{k - 1}x^k y^{n + 1 - k} + \sum_{k = 0}^{n}C_n^k x^k y^{n - k + 1} = C_n^n x^{n + 1} y^0 + \sum_{k = 1}^{n} \left(C_n^{k - 1} x^k y^{n + 1 - k} + C_n^k x^k y^{n + 1 - k}\right) + C_n^0 x^0 y^{n + 1} = \\
            &=C_{n + 1}^{n + 1} x^{n + 1} y^0 + \sum_{k = 1}^{n} \left(C_n^{k - 1} + C_n^k\right)x^k y^{n + 1 - k} + C_{n + 1}^0 x^0 y^{n + 1} = \sum_{k = 0}^{n + 1}C_{n + 1}^k x^k y^{n + 1 - k} 
        \end{align*}
    \end{proof}

    \subsection{Супремум и инфимум}


    \begin{definition}
        $E \subset \mathbb{R}$ - ограниченное сверху, если $\exists A : \forall x \in E \to x \leqslant A$ 
    \end{definition}
    
    \begin{definition}
        $E \subset \mathbb{R}$ - ограниченное снизу, если $\exists B : \forall x \in E \to x \geqslant B$  
    \end{definition}

    \begin{definition}
        $E \subset \mathbb{R}$ - ограниченное, если оно ограничено и снизу, и сверху.
    \end{definition}

    \begin{definition}
        $M \in \mathbb{R}$ называется максимумом мн-ва $E$, если $\forall x \in E \to x \leqslant M \wedge M \in E$ 
    \end{definition}    

    \begin{definition}
        $K \in \mathbb{R}$ называется минимумом мн-ва $E$, если $\forall x \in E \to x \geqslant K \wedge K \in E$ 
    \end{definition}

    \begin{theorem}[Существование минимума и максимума у конечного множества из $\mathbb{R}$ ]
        Во всяком конечном непустом подмножестве $\mathbb{R}$ есть наибольший и наименьший элементю
    \end{theorem}

    \begin{proof}
        $n = 1$ - количество элементов (База) \\
        Индукционный переход: $\exists \max\{x_1, x_2, ..., x_n\} = C$ \\
        Добавим $x_{n + 1}$: если $x_{n + 1} > C \Rightarrow \max\{x_1, ..., x_{n + 1}\} = x_{n + 1}$ \\
        если $x_{n + 1} \leqslant C \Rightarrow \max\{x_1, ..., x_{n + 1}\} = C$     
    \end{proof}

    \begin{consequence}
        $\forall E \neq \varnothing \wedge E \subset \mathbb{Z}\wedge E \text{ - огр.} \to \exists \max E \wedge \min E $ 
    \end{consequence}

    \begin{consequence}
        $\forall E \subset \mathbb{N}, E \neq \varnothing \to \exists \min E$ 
    \end{consequence}

    Далее везде $E \subset \mathbb{R}, E \neq \varnothing$ \\

    \begin{definition}
        Пусть $E$ ограничено сверху, тогда $\sup{E}$ - наименьшаяя из верхних границ. (точная верхняя граница)
    \end{definition}

    \begin{definition}
        Пусть $E$ ограничено снизу, тогда $\inf{E}$ - наибольшая из нижних границ. (точная нижняя граница)
    \end{definition}

    \begin{theorem}
        $E \neq \varnothing$. Если $E$ ограничено снизу, то $\exists ! \inf{E}$ 
    \end{theorem}

    \begin{proof}
        Пусть $A$ - множество всех нижних границ $E (A \neq \varnothing)$ \\
        $\forall a \in A, b \in E \to a \leqslant b$ \\
        Тогда по аксиоме полноты $\Rightarrow \exists c \in \mathbb{R}: a \leqslant c \leqslant b \text{ } \forall a \in A, b \in E \Rightarrow \\ \Rightarrow \begin{cases}
            c \leqslant b \text{ } \forall b \in E \text{ - } c \text{ - нижняя граница}, \\
            c \geqslant a \text{ } \forall a \in A \text{ - } c \text{ - наибольшее}
        \end{cases}$ 
    \end{proof}

    \begin{definition}
        \begin{align*}
            &l = \sup{E} \Leftrightarrow \begin{cases}
                \forall x \in E \to x \leqslant l \\
                \forall \varepsilon > 0 \to \exists y \in E : y > l - \varepsilon
            \end{cases}\\
            &m = \inf{E} \Leftrightarrow \begin{cases}
                \forall x \in E \to x \geqslant m \\
                \forall \varepsilon > 0 \to \exists y \in E: y < l + \varepsilon
            \end{cases}
        \end{align*}
         
    \end{definition}

    Если $E$ не ограничено сверху, то $\sup{E} = +\infty$ \\
    Если $E = \varnothing$, то чаще всего $\sup{E} \text{ и } \inf{E}$ не определены, но иногда $\sup{\varnothing} = -\infty, \inf{\varnothing} = +\infty$  

    \begin{statement}
        $\varnothing \neq B \subset A \subset \mathbb{R}$. Тогда если $A$ ограничено снизу, то $\inf{A} \leqslant \inf{B}$ 
    \end{statement}

    \begin{proof}
        Если $C$ - нижняя граница $A$, то $\forall x \in A \to C \leqslant x \Rightarrow \forall y \in B \to C \leqslant y \Rightarrow C$ - нижняя граница $B \Rightarrow \inf{A}$ - тоже нижняя граница $B \Rightarrow \inf{A} \leqslant \inf{B}$  
    \end{proof}

    \begin{statement}
        $\varnothing \neq B \subset A \subset \mathbb{R}$. Тогда если $A$ ограничено сверху, то $\sup{A} \geqslant \sup{B}$  
    \end{statement}
    \newpage

    \section{Отображения}
    $f: A \to B$ 
    $f(x) = y \\
    y - \text{образ элемента } X \\ 
    x - \text{прообраз } y \\
    f(A) - \text{образ множества } A \\
    f^{-1}(B) - \text{прообраз множества } B
    $ \\
    $G_f = \{(x, y) : x \in A, y = f(x)\}$ 

    \begin{definition}
        $f: A \to B$. Если $f(A) = B$, то $f$ сюръективно.
    \end{definition}
    \begin{definition}
        $f: A \to B$. Если $(x_1 \neq x_2 \in A) \Leftrightarrow (f(x_1) \neq f(x_2))$, то $f$ инъективно.
    \end{definition}
    \begin{definition}
        Биекция - $f$ инъективно и сюръективно. 
    \end{definition}
    \begin{definition}[Композиция]
        $g(x), f(x). \\h(x) = g \circ f(x) = g(f(x))$ \\
        $f: X \to Y$
        $g: Y_0 \to Z, f(x) \subset Y_0$   
    \end{definition}
    \begin{definition}
        $id_x $ - тождественное отображение: $f(x) = x$  
    \end{definition}
    \begin{definition}
        $f: X \to Y, X_0 \subset X \\ f |_{X_0}$ - сужение отображения $f$ на $X_0$ 
    \end{definition}
\end{document}