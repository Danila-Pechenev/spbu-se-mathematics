\documentclass[12pt]{article}

% Автор стиля: Сергей Копелиович
% Автор конспекта: Илья Дудников

\usepackage{cmap}
\usepackage[T2A]{fontenc}
\usepackage[utf8]{inputenc}
\usepackage[russian]{babel}
\usepackage{graphicx}
\usepackage{amsthm,amsmath,amssymb}
\usepackage{listings}
\usepackage{color}
\usepackage{xcolor}
\usepackage{array}
\usepackage{epigraph}

\usepackage[russian,colorlinks=true,urlcolor=red,linkcolor=blue]{hyperref}
\usepackage{enumerate}
\usepackage{datetime}
\usepackage{fancyhdr}
\usepackage{lastpage}
\usepackage{verbatim}
\usepackage{tikz}
\usetikzlibrary{arrows,decorations.markings,decorations.pathmorphing}
\usepackage{pgfplots}

\usepackage{ifthen}
\usepackage{mathtools}

%\usepackage{tabls}
%\usepackage{tabularx}
%\usepackage{xifthen}
%\listfiles

\def\NAME{Лекция, 10/06/21}

\sloppy
\voffset=-20mm
\textheight=235mm
\hoffset=-22mm
\textwidth=180mm
\headsep=12pt
\footskip=20pt

\parskip=0em
\parindent=0em

\setlength\epigraphwidth{.8\textwidth}

\newlength{\tmplen}
\newlength{\tmpwidth}
\newcounter{listcounter}

% Список с маленькими отступами
\newenvironment{MyList}[1][4pt]{
  \begin{enumerate}[1.]
  \setlength{\parskip}{0pt}
  \setlength{\itemsep}{#1}
}{       
  \end{enumerate}
}
% Вложенный список с маленькими отступами
\newenvironment{InnerMyList}[1][0pt]{
  \vspace*{-0.5em}
  \begin{enumerate}[(a)]
  \setlength{\parskip}{-0pt}
  \setlength{\itemsep}{#1}
}{       
  \end{enumerate}
  \vspace*{-0.5em}
}
% Список с маленькими отступами
\newenvironment{MyItemize}[1][4pt]{
  \begin{itemize}
  \setlength{\parskip}{0pt}
  \setlength{\itemsep}{#1}
}{       
  \end{itemize}
}

% Основные математические символы
\def\TODO{{\color{red}\bf TODO}}
\def\N{\mathbb{N}}       %
\def\R{\mathbb{R}}       %
\def\F2{\mathbb{F}_2}    %
\def\Z{\mathbb{Z}}       %
\def\INF{\t{+}\infty}    % +inf
\def\EPS{\varepsilon}    %
\def\EMPTY{\varnothing}  %
\def\PHI{\varphi}        %
\def\SO{\Rightarrow}     % =>
\def\EQ{\Leftrightarrow} % <=>
\def\t{\texttt}          % mono font
\def\c#1{{\rm\sc{#1}}}   % font for classes NP, SAT, etc
\def\O{\mathcal{O}}      %
\def\NO{\t{\#}}          % #
\def\XOR{\text{ {\raisebox{-2pt}{\ensuremath{\Hat{}}}} }}
\renewcommand{\le}{\leqslant}
\renewcommand{\ge}{\geqslant}
\newcommand{\q}[1]{\langle #1 \rangle}               % <x>
\newcommand\URL[1]{{\footnotesize{\url{#1}}}}        %
% \newcommand{\sfrac}[2]{{\scriptscriptstyle\frac{#1}{#2}}}  % Очень маленькая дробь
% \newcommand{\mfrac}[2]{{\scriptstyle\frac{#1}{#2}}}    % Небольшая дробь
\newcommand{\sfrac}[2]{{\scriptstyle\frac{#1}{#2}}}  % Очень маленькая дробь
\newcommand{\mfrac}[2]{{\textstyle\frac{#1}{#2}}}    % Небольшая дробь

\newcommand{\fix}[1]{{\color{fixcolor}{#1}}} % \underline
\def\bonus{\t{\red{(*)}}}
\def\ifbonus#1{\ifthenelse{\equal{#1}{}}{}{\bonus}}
\def\smallsquare{$\scalebox{0.5}{$\square$}$}

\newlength{\myItemLength}
\setlength{\myItemLength}{0.3em}
\def\ItemSymbol{\smallsquare}
\def\Item{\vspace*{\myItemLength}\ItemSymbol \ \ }

\newcommand{\LET}{%
  % [line width=0.6pt]
  \begin{tikzpicture}%
  \draw(0.8ex,0) -- (0.8ex,1.6ex);%
  \draw(0,1.6ex) -- (0.8ex,1.6ex);%
  \end{tikzpicture}%
  \hspace*{0.1em}%
}

% Отступы
\def\makeparindent{\hspace*{\parindent}\unskip}
\def\up{\vspace*{-0.5em}}%{\vspace*{-\baselineskip}}
\def\down{\vspace*{0.5em}}
\def\LINE{\vspace*{-1em}\noindent \underline{\hbox to 1\textwidth{{ } \hfil{ } \hfil{ } }}}
\def\BOX#1{\mbox{\fbox{\bf{#1}}}}
\def\Pagebreak{\pagebreak\vspace*{-1.5em}}

% Мелкий заголовок
\newcommand{\THEE}[1]{
  \vspace*{0.5em}
  \noindent{\bf \underline{#1}}%\hspace{0.5em}
  \vspace*{0.2em}
}
% Другой тип мелкого заголовка
\newcommand{\THE}[1]{
  \vspace*{0.5em} $\bullet$
  \noindent{\bf #1}%\hspace{0.5em}
  \vspace*{0.2em}
}

\newenvironment{MyTabbing}{
  \t\bgroup
  \vspace*{-\baselineskip}
  \begin{tabbing}
    aaaa\=aaaa\=aaaa\=aaaa\=aaaa\=aaaa\kill
}{
  \end{tabbing}
  \t\egroup
}

% Код с правильными отступами
\lstnewenvironment{code}{
  \lstset{}
%  \vspace*{-0.2em}
}%
{
%  \vspace*{-0.2em}
}
\lstnewenvironment{codep}{
  \lstset{language=python}
}%
{
}

% Формулы с правильными отступами
\newenvironment{smallformula}{
 
  \vspace*{-0.8em}
}{
  \vspace*{-1.2em}
  
}
\newenvironment{formula}{
 
  \vspace*{-0.4em}
}{
  \vspace*{-0.6em}
  
}

% Большая квадратная скобка
\makeatletter
\newenvironment{sqcases}{%
  \matrix@check\sqcases\env@sqcases
}{%
  \endarray\right.%
}
\def\env@sqcases{%
  \let\@ifnextchar\new@ifnextchar
  \left\lbrack
  \def\arraystretch{1.2}%
  \array{@{}l@{\quad}l@{}}%
}
\makeatother

\DeclareMathOperator{\sort}{sort}

% Определяем основные секции: \begin{Lm}, \begin{Thm}, \begin{Def}, \begin{Rem}
\renewcommand{\qedsymbol}{$\blacksquare$}
\theoremstyle{definition} % жирный заголовок, плоский текст
\newtheorem{Thm}{\underline{Теорема}}[subsection] % нумерация будет "<номер subsection>.<номер теоремы>"
\newtheorem{Lm}[Thm]{\underline{Lm}} % Нумерация такая же, как и у теорем
\newtheorem{Ex}[Thm]{Упражнение} % Нумерация такая же, как и у теорем
\newtheorem{Example}[Thm]{Пример} % Нумерация такая же, как и у теорем
\newtheorem{Code}[Thm]{Код} % Нумерация такая же, как и у теорем
\theoremstyle{plain} % жирный заголовок, курсивный текст
\newtheorem{Def}[Thm]{Def.} % Нумерация такая же, как и у теорем
\theoremstyle{remark} % курсивный заголовок, плоский текст
\newtheorem{Cons}[Thm]{Следствие} % Нумерация такая же, как и у теорем
\newtheorem{Conj}[Thm]{Гипотеза} % Нумерация такая же, как и у теорем
\newtheorem{Prop}[Thm]{Утверждение} % Нумерация такая же, как и у теорем
\newtheorem{Rem}[Thm]{Замечание} % Нумерация такая же, как и у теорем
\newtheorem{Remark}[Thm]{Замечание} % Нумерация такая же, как и у теорем
\newtheorem{Algo}[Thm]{Алгоритм} % Нумерация такая же, как и у теорем

% Определяем ЗАГОЛОВКИ
\def\SectionName{Дискретная математика}
\def\AuthorName{Илья Дудников}

\newlength{\sectionvskip}
\setlength{\sectionvskip}{0.5em}
\newcommand{\Section}[4][]{
  % Заголовок
  \pagebreak
%  \ifthenelse{\isempty{#1}}{
    \refstepcounter{section}
%  }{}
  \vspace{0.5em}
%  \ifthenelse{\isempty{#1}}{
%    \addtocontents{toc}{\protect\addvspace{-5pt}}%
    \addcontentsline{toc}{section}{\arabic{section}. #2}
%  }{}
  


  % Запомнили название и автора главы
  \gdef\SectionName{#2}
  \gdef\AuthorName{#4}

  % Заголовок страницы
  \lhead{Конспект, \NAME}
  \chead{}
  \rhead{\SectionName}
  \renewcommand{\headrulewidth}{0.4pt}
    
  \lfoot{}
  \cfoot{\thepage\t{/}\pageref*{LastPage}}
  \rfoot{Автор: \AuthorName}
  \renewcommand{\footrulewidth}{0.4pt}
}

\newcommand{\Subsection}[2][]{
  \refstepcounter{subsection}
  \vspace*{1em}
  \ifthenelse{\equal{#1}{}}
    {\addcontentsline{toc}{subsection}{\arabic{section}.\arabic{subsection}. #2}}
    {\addcontentsline{toc}{subsection}{\arabic{section}.\arabic{subsection}. \bonus\,#2}}
  {\color{blue}\bf\large \arabic{section}.\arabic{subsection}. \ifbonus{#1}\,{#2}} 
  \vspace*{0.5em}
  \makeparindent
}
\newcommand{\Subsubsection}[2][]{
  \refstepcounter{subsubsection}
  \vspace*{1em}
  \ifthenelse{\equal{#1}{}}
    {\addcontentsline{toc}{subsubsection}{\arabic{section}.\arabic{subsection}.\arabic{subsubsection}. #2}}
    {\addcontentsline{toc}{subsubsection}{\arabic{section}.\arabic{subsection}.\arabic{subsubsection}. \bonus\,#2}}
  {\color{blue}\bf\large \arabic{section}.\arabic{subsection}.\arabic{subsubsection}. \ifbonus{#1}\,#2}
  \vspace*{0.5em}
  \makeparindent
}

\newcommand{\Header}{
  \pagestyle{empty}
  \renewcommand{\dateseparator}{--}
  \begin{center}
    {\Large\bf 
        Матанализ 1 семестр ПИ,\\
    \vspace{0.3em}
     \NAME}\\
    \vspace{0.7em}
    {Собрано {\today} в {\currenttime}}
  \end{center}

  \LINE
  \vspace{0em}

  \renewcommand{\baselinestretch}{0.98}\normalsize
  \tableofcontents
  \renewcommand{\baselinestretch}{1.0}\normalsize
  \pagebreak
}

\newcommand{\BeginConspect}{
  \pagestyle{fancy}
  \setcounter{page}{1}
}

\definecolor{mygray}{rgb}{0.7,0.7,0.7}
\definecolor{ltgray}{rgb}{0.9,0.9,0.9}
\definecolor{fixcolor}{rgb}{0.7,0,0}
\definecolor{red2}{rgb}{0.7,0,0}
\definecolor{dkred}{rgb}{0.4,0,0}
\definecolor{dkblue}{rgb}{0,0,0.6}
\definecolor{dkgreen}{rgb}{0,0.6,0}
\definecolor{brown}{rgb}{0.5,0.5,0}

\newcommand{\green}[1]{{\color{green}{#1}}}
\newcommand{\black}[1]{{\color{black}{#1}}}
\newcommand{\red}[1]{{\color{red}{#1}}}
\newcommand{\dkred}[1]{{\color{dkred}{#1}}}
\newcommand{\blue}[1]{{\color{blue}{#1}}}
\newcommand{\dkgreen}[1]{{\color{dkgreen}{#1}}}

\begin{document}

\Header

\BeginConspect

\Section{Пределы функций}{}{Илья Дудников}

\Subsection{$\varepsilon$-окрестности}

\begin{Def}
    $\varepsilon$-окрестность точки $a - V_\varepsilon (a) : (a - \varepsilon, a + \varepsilon)$ \\
    проколотая $\varepsilon$-окрестность $a - \dot{V}_\varepsilon : (a - \varepsilon, a) \cup (a + \varepsilon)$   
\end{Def}

\begin{Def}
    $D \subset \R, a \in \R$. Точка $a$ называется точкой сгущения $D$, если в любой окрестности $a$ найдется точка из $D$, отличная от $a$
    \[\forall \dot{V}(a) \ \exists x \in D : x \in \dot{V}(a) \wedge x \neq a\]
\end{Def}

\begin{Example}
    $D = [1, 2)$. Точки сгущения: $[1, 2]$ 
\end{Example}

\begin{Rem}
    Точка сгущения может принадлежать множеству, а может и не принадлежать.
\end{Rem}

\begin{Rem}
    Если $a$ -- точка сгущения, тогда в $\forall \dot{V}(a)$ бесконечно много точек из $D$.
\end{Rem}

\begin{Rem}
    Точки сгущения называют предельными точками множества. \\
    $a$ -- точка сгущения $\EQ \exists \{x_n\} : x_n \in D, x_n \neq a, x_n \to a$ 
\end{Rem}

\begin{proof}
    "$\SO$". $\varepsilon = \frac{1}{k} \SO |x_k - a| < \frac{1}{k} \SO 0 \leqslant \lim |x_k - a| < 0 \SO \exists \lim x_k = 0$ \\
    "$\Leftarrow$". В $\forall V(a)$ лежит бесконечно много точек $\{x_n\}, x_n \neq a \SO a$ -- точка сгущения. 
\end{proof}

\begin{Def}
    $a \in D$, но $a$ -- не предельная точка. Тогда $a$ называется изолированной точкой множества $D$
\end{Def}

\begin{Rem}
    $+\infty$ может быть предельной точкой множества
    \[\dot{V}(+\infty) = (E, +\infty)\] 
\end{Rem}

\Subsection{Предел функции}

\begin{Def}
    $f: D \to \R, D \subset \R, a \in \R$ -- предельная точка $D$.

    Число $A \in \R$ называется пределом $f$ в точке $a$
    \[\lim_{x \to a} f(x) = A \text{ или } f(x) \xrightarrow[x \to a]{} A\]
    если выполняется одно из следующих условий:

    \begin{MyList}
        \item $\forall \varepsilon > 0 \ \exists \delta > 0 : \forall x \in D \setminus \{a\} : |x - a| < \delta \to |f(x) - A| < \varepsilon$ (Определение по Коши, определение на языке $\delta, \varepsilon$ )
        \item $\forall V(A) \ \exists V(a) : f(\dot{V}(a) \cap D) \subset V(A)$ (Определение на языке окрестностей)
        \item $\forall \{x_n\} : x_n \in D, x_n \neq a, x_n \to a \SO f(x_n) \to A$ (Определение по Гейне, на языке последовательностей)
    \end{MyList}
\end{Def}

\Pagebreak
\begin{Thm}[Эквивалентность определения по Коши и по Гейне]
    Определения 1) и 3) эквивалентны.
\end{Thm}

\begin{proof}
    1) $\SO$ 3). Рассмотрим какую-то $\{x_n\} : x_n \neq a, x_n \in D, x_n \to a$ (она существует по доказанному).
    Нужно доказать, что $f(x_n) \to A$.  \\
    Пусть $x_n \to a$, то 
    \[\forall \delta > 0 \ \exists N : \forall n \geqslant N \to |x_n - a| < \delta \SO |f(x_n) - A| < \varepsilon\]

    3) $\SO$ 1). Пусть это не так, т.е. 1) не выполнено
    \[\exists \varepsilon > 0 : \forall \delta > 0 \ \exists x \in D, x \neq a, |x - a| < \delta: |f(x) - A| \geqslant \varepsilon\]
    Возьмем последовательность $\delta_n = \frac{1}{n}$.
    \[|x_n - a| < \frac{1}{n} \SO x_n \to a \SO f(x_n) \to A, \text{ но } |f(x_n) - A| \geqslant \varepsilon\] 
\end{proof}

\begin{Rem}
    в $\overline{\R}$ 
    \begin{MyList}
        \item \[\lim_{x \to 5} f(x) = \infty \EQ \forall E \ \exists \delta > 0 : \forall x \in D \setminus \{5\}, |x - 5| < \delta \to f(x) > E\]
        \item \[\lim_{x \to -\infty} f(x) = 2 \EQ \forall E > 0 \ \exists \Delta : \forall x \in D, x < \Delta \to |f(x) - 2| < E\]
    \end{MyList}
\end{Rem}

\begin{Rem}
    В определении по Гейне есть "$\forall \{x_n\}$". Если $x_n$ и $y_n$ подходят под условия, то $\lim f(x_n) = \lim f(y_n)$ 
\end{Rem}

\begin{proof}
    Возьмем $z_n: z_1 = x_1, z_2 = y_1, z_3 = x_2, z_4 = y_2$ и т.д. $\{z_n\}$ подходит под определение $\SO \exists \lim f(z_n)$ 
\end{proof}

\begin{Rem}
    В определении предела функции не участвует значения функции в точке $a$.
\end{Rem}

\begin{Rem}
    Последовательность -- частный случай функции.
\end{Rem}

\end{document}