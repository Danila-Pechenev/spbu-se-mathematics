\documentclass[12pt]{article}

% Автор стиля: Сергей Копелиович
% Автор конспекта: Илья Дудников

\usepackage{cmap}
\usepackage[T2A]{fontenc}
\usepackage[utf8]{inputenc}
\usepackage[russian]{babel}
\usepackage{graphicx}
\usepackage{amsthm,amsmath,amssymb}
\usepackage{listings}
\usepackage{color}
\usepackage{xcolor}
\usepackage{array}
\usepackage{epigraph}

\usepackage[russian,colorlinks=true,urlcolor=red,linkcolor=blue]{hyperref}
\usepackage{enumerate}
\usepackage{datetime}
\usepackage{fancyhdr}
\usepackage{lastpage}
\usepackage{verbatim}
\usepackage{tikz}
\usetikzlibrary{arrows,decorations.markings,decorations.pathmorphing}
\usepackage{pgfplots}

\usepackage{ifthen}
\usepackage{mathtools}

%\usepackage{tabls}
%\usepackage{tabularx}
%\usepackage{xifthen}
%\listfiles

\def\NAME{Лекция, 09/29/21}

\sloppy
\voffset=-20mm
\textheight=235mm
\hoffset=-22mm
\textwidth=180mm
\headsep=12pt
\footskip=20pt

\parskip=0em
\parindent=0em

\setlength\epigraphwidth{.8\textwidth}

\newlength{\tmplen}
\newlength{\tmpwidth}
\newcounter{listcounter}

% Список с маленькими отступами
\newenvironment{MyList}[1][4pt]{
  \begin{enumerate}[1.]
  \setlength{\parskip}{0pt}
  \setlength{\itemsep}{#1}
}{       
  \end{enumerate}
}
% Вложенный список с маленькими отступами
\newenvironment{InnerMyList}[1][0pt]{
  \vspace*{-0.5em}
  \begin{enumerate}[(a)]
  \setlength{\parskip}{-0pt}
  \setlength{\itemsep}{#1}
}{       
  \end{enumerate}
  \vspace*{-0.5em}
}
% Список с маленькими отступами
\newenvironment{MyItemize}[1][4pt]{
  \begin{itemize}
  \setlength{\parskip}{0pt}
  \setlength{\itemsep}{#1}
}{       
  \end{itemize}
}

% Основные математические символы
\def\TODO{{\color{red}\bf TODO}}
\def\N{\mathbb{N}}       %
\def\R{\mathbb{R}}       %
\def\F2{\mathbb{F}_2}    %
\def\Z{\mathbb{Z}}       %
\def\INF{\t{+}\infty}    % +inf
\def\EPS{\varepsilon}    %
\def\EMPTY{\varnothing}  %
\def\PHI{\varphi}        %
\def\SO{\Rightarrow}     % =>
\def\EQ{\Leftrightarrow} % <=>
\def\t{\texttt}          % mono font
\def\c#1{{\rm\sc{#1}}}   % font for classes NP, SAT, etc
\def\O{\mathcal{O}}      %
\def\NO{\t{\#}}          % #
\def\XOR{\text{ {\raisebox{-2pt}{\ensuremath{\Hat{}}}} }}
\renewcommand{\le}{\leqslant}
\renewcommand{\ge}{\geqslant}
\newcommand{\q}[1]{\langle #1 \rangle}               % <x>
\newcommand\URL[1]{{\footnotesize{\url{#1}}}}        %
% \newcommand{\sfrac}[2]{{\scriptscriptstyle\frac{#1}{#2}}}  % Очень маленькая дробь
% \newcommand{\mfrac}[2]{{\scriptstyle\frac{#1}{#2}}}    % Небольшая дробь
\newcommand{\sfrac}[2]{{\scriptstyle\frac{#1}{#2}}}  % Очень маленькая дробь
\newcommand{\mfrac}[2]{{\textstyle\frac{#1}{#2}}}    % Небольшая дробь

\newcommand{\fix}[1]{{\color{fixcolor}{#1}}} % \underline
\def\bonus{\t{\red{(*)}}}
\def\ifbonus#1{\ifthenelse{\equal{#1}{}}{}{\bonus}}
\def\smallsquare{$\scalebox{0.5}{$\square$}$}

\newlength{\myItemLength}
\setlength{\myItemLength}{0.3em}
\def\ItemSymbol{\smallsquare}
\def\Item{\vspace*{\myItemLength}\ItemSymbol \ \ }

\newcommand{\LET}{%
  % [line width=0.6pt]
  \begin{tikzpicture}%
  \draw(0.8ex,0) -- (0.8ex,1.6ex);%
  \draw(0,1.6ex) -- (0.8ex,1.6ex);%
  \end{tikzpicture}%
  \hspace*{0.1em}%
}

% Отступы
\def\makeparindent{\hspace*{\parindent}\unskip}
\def\up{\vspace*{-0.5em}}%{\vspace*{-\baselineskip}}
\def\down{\vspace*{0.5em}}
\def\LINE{\vspace*{-1em}\noindent \underline{\hbox to 1\textwidth{{ } \hfil{ } \hfil{ } }}}
\def\BOX#1{\mbox{\fbox{\bf{#1}}}}
\def\Pagebreak{\pagebreak\vspace*{-1.5em}}

% Мелкий заголовок
\newcommand{\THEE}[1]{
  \vspace*{0.5em}
  \noindent{\bf \underline{#1}}%\hspace{0.5em}
  \vspace*{0.2em}
}
% Другой тип мелкого заголовка
\newcommand{\THE}[1]{
  \vspace*{0.5em} $\bullet$
  \noindent{\bf #1}%\hspace{0.5em}
  \vspace*{0.2em}
}

\newenvironment{MyTabbing}{
  \t\bgroup
  \vspace*{-\baselineskip}
  \begin{tabbing}
    aaaa\=aaaa\=aaaa\=aaaa\=aaaa\=aaaa\kill
}{
  \end{tabbing}
  \t\egroup
}

% Код с правильными отступами
\lstnewenvironment{code}{
  \lstset{}
%  \vspace*{-0.2em}
}%
{
%  \vspace*{-0.2em}
}
\lstnewenvironment{codep}{
  \lstset{language=python}
}%
{
}

% Формулы с правильными отступами
\newenvironment{smallformula}{
 
  \vspace*{-0.8em}
}{
  \vspace*{-1.2em}
  
}
\newenvironment{formula}{
 
  \vspace*{-0.4em}
}{
  \vspace*{-0.6em}
  
}

% Большая квадратная скобка
\makeatletter
\newenvironment{sqcases}{%
  \matrix@check\sqcases\env@sqcases
}{%
  \endarray\right.%
}
\def\env@sqcases{%
  \let\@ifnextchar\new@ifnextchar
  \left\lbrack
  \def\arraystretch{1.2}%
  \array{@{}l@{\quad}l@{}}%
}
\makeatother

% Определяем основные секции: \begin{Lm}, \begin{Thm}, \begin{Def}, \begin{Rem}
\renewcommand{\qedsymbol}{$\blacksquare$}
\theoremstyle{definition} % жирный заголовок, плоский текст
\newtheorem{Thm}{\underline{Теорема}}[subsection] % нумерация будет "<номер subsection>.<номер теоремы>"
\newtheorem{Lm}[Thm]{\underline{Lm}} % Нумерация такая же, как и у теорем
\newtheorem{Ex}[Thm]{Упражнение} % Нумерация такая же, как и у теорем
\newtheorem{Example}[Thm]{Пример} % Нумерация такая же, как и у теорем
\newtheorem{Code}[Thm]{Код} % Нумерация такая же, как и у теорем
\theoremstyle{plain} % жирный заголовок, курсивный текст
\newtheorem{Def}[Thm]{Def.} % Нумерация такая же, как и у теорем
\theoremstyle{remark} % курсивный заголовок, плоский текст
\newtheorem{Cons}[Thm]{Следствие} % Нумерация такая же, как и у теорем
\newtheorem{Conj}[Thm]{Гипотеза} % Нумерация такая же, как и у теорем
\newtheorem{Prop}[Thm]{Утверждение} % Нумерация такая же, как и у теорем
\newtheorem{Rem}[Thm]{Замечание} % Нумерация такая же, как и у теорем
\newtheorem{Remark}[Thm]{Замечание} % Нумерация такая же, как и у теорем
\newtheorem{Algo}[Thm]{Алгоритм} % Нумерация такая же, как и у теорем

% Определяем ЗАГОЛОВКИ
\def\SectionName{Матанализ}
\def\AuthorName{Илья Дудников}

\newlength{\sectionvskip}
\setlength{\sectionvskip}{0.5em}
\newcommand{\Section}[4][]{
  % Заголовок
  \pagebreak
%  \ifthenelse{\isempty{#1}}{
    \refstepcounter{section}
%  }{}
  \vspace{0.5em}
%  \ifthenelse{\isempty{#1}}{
%    \addtocontents{toc}{\protect\addvspace{-5pt}}%
    \addcontentsline{toc}{section}{\arabic{section}. #2}
%  }{}
  


  % Запомнили название и автора главы
  \gdef\SectionName{#2}
  \gdef\AuthorName{#4}

  % Заголовок страницы
  \lhead{Конспект, \NAME}
  \chead{}
  \rhead{\SectionName}
  \renewcommand{\headrulewidth}{0.4pt}
    
  \lfoot{}
  \cfoot{\thepage\t{/}\pageref*{LastPage}}
  \rfoot{Автор: \AuthorName}
  \renewcommand{\footrulewidth}{0.4pt}
}

\newcommand{\Subsection}[2][]{
  \refstepcounter{subsection}
  \vspace*{1em}
  \ifthenelse{\equal{#1}{}}
    {\addcontentsline{toc}{subsection}{\arabic{section}.\arabic{subsection}. #2}}
    {\addcontentsline{toc}{subsection}{\arabic{section}.\arabic{subsection}. \bonus\,#2}}
  {\color{blue}\bf\large \arabic{section}.\arabic{subsection}. \ifbonus{#1}\,{#2}} 
  \vspace*{0.5em}
  \makeparindent
}
\newcommand{\Subsubsection}[2][]{
  \refstepcounter{subsubsection}
  \vspace*{1em}
  \ifthenelse{\equal{#1}{}}
    {\addcontentsline{toc}{subsubsection}{\arabic{section}.\arabic{subsection}.\arabic{subsubsection}. #2}}
    {\addcontentsline{toc}{subsubsection}{\arabic{section}.\arabic{subsection}.\arabic{subsubsection}. \bonus\,#2}}
  {\color{blue}\bf\large \arabic{section}.\arabic{subsection}.\arabic{subsubsection}. \ifbonus{#1}\,#2}
  \vspace*{0.5em}
  \makeparindent
}

\newcommand{\Header}{
  \pagestyle{empty}
  \renewcommand{\dateseparator}{--}
  \begin{center}
    {\Large\bf 
     Матанализ 1 семестр ПИ,\\
    \vspace{0.3em}
     \NAME}\\
    \vspace{0.7em}
    {Собрано {\today} в {\currenttime}}
  \end{center}

  \LINE
  \vspace{0em}

  \renewcommand{\baselinestretch}{0.98}\normalsize
  \tableofcontents
  \renewcommand{\baselinestretch}{1.0}\normalsize
  \pagebreak
}

\newcommand{\BeginConspect}{
  \pagestyle{fancy}
  \setcounter{page}{1}
}

\definecolor{mygray}{rgb}{0.7,0.7,0.7}
\definecolor{ltgray}{rgb}{0.9,0.9,0.9}
\definecolor{fixcolor}{rgb}{0.7,0,0}
\definecolor{red2}{rgb}{0.7,0,0}
\definecolor{dkred}{rgb}{0.4,0,0}
\definecolor{dkblue}{rgb}{0,0,0.6}
\definecolor{dkgreen}{rgb}{0,0.6,0}
\definecolor{brown}{rgb}{0.5,0.5,0}

\newcommand{\green}[1]{{\color{green}{#1}}}
\newcommand{\black}[1]{{\color{black}{#1}}}
\newcommand{\red}[1]{{\color{red}{#1}}}
\newcommand{\dkred}[1]{{\color{dkred}{#1}}}
\newcommand{\blue}[1]{{\color{blue}{#1}}}
\newcommand{\dkgreen}[1]{{\color{dkgreen}{#1}}}

\begin{document}

\Header

\BeginConspect

\Section{Пределы последовательностей}{}{Илья Дудников}

\Subsection{Число $e$}

Пусть $x_n = \left(1 + \frac{1}{n}\right)^n, y_n = \left(1 + \frac{1}{n}\right)^{n + 1}$.
Тогда 

\begin{align*}
    \frac{y_{n - 1}}{y_n} &= \frac{\left(1 + \frac{1}{n - 1}\right)^n}{\left(1 + \frac{1}{n}\right)^{n + 1}} = \frac{\left(\frac{n}{n - 1}\right)^n}{\left(\frac{n + 1}{n}\right)^{n + 1}} = \frac{\frac{n^n}{(n - 1)^n}}{\frac{(n + 1)^{n + 1}}{n^{n + 1}}} = \frac{n^n}{(n - 1)^n} \cdot \frac{n^{n + 1}}{(n + 1)^{n + 1}} = \frac{n^{2n}}{(n^2 - 1)^n} \cdot \frac{n}{n + 1} \\
    &= \frac{n^{2n + 2}}{(n^2 - 1)^{n + 1}} \cdot \frac{n - 1}{n} = \left(\frac{n^2}{n^2 - 1}\right)^{n + 1} \cdot \frac{n - 1}{n} = \left(1 + \frac{1}{n^2 - 1}\right)^{n + 1} \cdot \frac{n - 1}{n} \geqslant \\
    &\geqslant \left(1 + \frac{n + 1}{n^2 - 1}\right) \cdot \frac{n - 1}{n} = \left(1 + \frac{1}{n - 1}\right) \cdot \frac{n - 1}{n} = \frac{n}{n - 1} \cdot \frac{n - 1}{n} = 1 \Rightarrow \frac{y_{n - 1}}{y_n} \geqslant 1 \Rightarrow \\
    &\Rightarrow y_{n - 1} \geqslant y_n \Rightarrow y_n \text{ убывающая } \Rightarrow \exists \lim y_n \Rightarrow \exists \lim x_n \text{, т.к. } x_n = \frac{y_n}{1 + \frac{1}{n}} \Rightarrow \lim x_n = \lim y_n \\
\end{align*}

\begin{Def}
    $e = \lim_{n \to \infty}{\left(1 + \frac{1}{n}\right)^n}$ 
\end{Def}

\begin{Thm}
    $x_n > 0 \wedge \lim \frac{x_{n + 1}}{x_n} < 1$. Тогда $\exists \lim x_n = 0$  
\end{Thm}

\begin{proof}
    Пусть $q = \lim \frac{x_{n + 1}}{x_n}, q < 1$. $\exists N : \forall n \geqslant N \to \frac{x_{n + 1}}{x_n} < \frac{1 + q}{2}$. Тогда
    \[0 < x_n = \frac{x_n}{x_{n - 1}} \cdot \frac{x_{n - 1}}{x_{n - 2}} \cdot \frac{x_{n - 2}}{x_{n - 3}} \cdot ... \cdot \frac{x_{N + 1}}{x_N} \cdot x_N \leqslant x_N \cdot \left(\frac{1 + q}{2}\right)^{n - N} \to 0\]
\end{proof}

\begin{Cons}
    $a > 1, k \in \N, \lim \frac{n^k}{a^n} = 0$
    \[\frac{x_{n + 1}}{x_n} = \frac{(n + 1)^k}{a^{n + 1}} \cdot \frac{a^n}{n^k} = \left(\frac{n + 1}{n}\right)^k \cdot \frac{1}{a} \to \frac{1}{a} < 1\] 
\end{Cons}

\begin{Cons}
    $\lim \frac{a^n}{n!} = 0$ 
    \[\frac{x_{n + 1}}{x_n} = \frac{a^{n + 1}}{(n + 1)!} \cdot \frac{n!}{a^n} = \frac{a}{n + 1} \to 0 < 1\]
\end{Cons}

\begin{Cons}
    $\lim \frac{n!}{n^n} = 0$ 
    \[ \frac{x_{n + 1}}{x_n} = \frac{(n + 1)!}{(n + 1)^{n + 1}} \cdot \frac{n^n}{n!} = \left(\frac{n}{n + 1}\right)^n = \frac{1}{\left(\frac{n + 1}{n}\right)^n} = \frac{1}{\left(1 + \frac{1}{n}\right)^n} \to \frac{1}{e} < 1 \]
\end{Cons}

\Pagebreak
\Subsection{Теорема Штольца}

\begin{Thm}[Теорема Штольца]
        $y_1 < y_2 < y_3 < ...$. $\lim y_n = + \infty \wedge \lim \frac{x_n - x_{n - 1}}{y_n - y_{n - 1}} = l \in \overline{\R}$  
        Тогда $\exists \lim \frac{x_n}{y_n} = l$
    \begin{proof}
        \begin{MyList}
            \item $l = 0$. $\varepsilon_k = \frac{x_k - x_{k - 1}}{y_k - y_{k - 1}}$. $\forall \varepsilon > 0 \ \exists m : \forall k \geqslant m \to |\varepsilon_k| < \varepsilon$ \\
            $x_k - x_{k - 1} = \varepsilon_k (y_k - y_{k - 1})$
            \[x_n - x_m = (x_n - x_{n - 1}) + (x_{n - 1} - x_{n - 2}) + ... + (x_{m + 1} - x_m) = \sum_{k = m + 1}^{n}(x_k - x_{k - 1}) = \sum_{k=m + 1}^{n} \varepsilon_k (y_k - y_{k - 1})\]  
            \[|x_n - x_m| = \sum_{k=m + 1}^{n} |\varepsilon_k|(y_k - y_{k - 1}) \leqslant \varepsilon \sum_{k=m + 1}^{n} (y_k - y_{k - 1}) = \varepsilon (y_n - y_m) \leqslant \varepsilon \cdot y_n\]  

            Тогда $|x_n| \leqslant |x_m| + \varepsilon y_n$ 
            \[0 \leqslant \left|\frac{x_n}{y_n}\right| \leqslant \left|\frac{x_m}{y_n}\right| + \varepsilon \Rightarrow \lim \frac{x_n}{y_n} = 0\]   
        
            \item $l \neq 0, l \in \R$. Рассмотрим $\widetilde{x}_n = x_n - l \cdot y_n$. Тогда
            \[ \frac{\widetilde{x}_n - \widetilde{x}_{n - 1}}{y_n - y_{n - 1}} = \frac{x_n - l \cdot y_n - x_{n - 1} + l \cdot y_{n - 1}}{y_n - y_{n - 1}} = \frac{x_n - x_{n - 1}}{y_n - y_{n - 1}} - l \to 0 \]
            Тогда по п. 1 $\frac{\widetilde{x}_n}{y_n} \to 0$. $ \frac{x_n}{y_n} = \frac{\widetilde{x}_n + l \cdot y_n}{y_n} = \frac{x_n}{y_n} + l \to l $  

            \item $l = + \infty$. $ \frac{x_n - x_{n - 1}}{y_n - y_{n - 1}} \to +\infty$. Начиная с некоторого номера $> 1$ \\
            $$x_n - x_{n - 1} > y_n - y_{n - 1} \Leftrightarrow x_n - x_m > y_n - y_m \to +\infty$$
            Тогда $x_n$ возрастает и стремится к $+\infty$
            \[ \frac{y_n - y_{n - 1}}{x_n - x_{n - 1}} \to 0 \Rightarrow \frac{y_n}{x_n} \to 0 \Rightarrow \frac{x_n}{y_n} \to +\infty\]
            \item $l \to -\infty$. Следует рассмотреть $\{-x_n\}$ 
        \end{MyList}
    \end{proof}
\end{Thm}

\begin{Thm}
    $y_1 > y_2 > ... > 0 \wedge \lim y_n = \lim x_n = 0$. Если $\lim \frac{x_n - x_{n - 1}}{y_n - y_{n - 1}} = l \in \overline{\R}$, тогда $\exists \lim \frac{x_n}{y_n} = l$   
\end{Thm}

\begin{proof}
    Докажем для $l = 0$.
    \[\varepsilon_k = \frac{x_k - x_{k - 1}}{y_k - y_{k - 1}}, \forall \varepsilon > 0 \ \exists N : \forall k \geqslant N \to |\varepsilon_k| < \varepsilon\]

    Пусть $n > m \geqslant N$
    \[|x_n - x_m| = \sum_{k=n + 1}^{m} |\varepsilon_k| \cdot |y_{k - 1} - y_k| \leqslant \varepsilon \sum_{k=n + 1}^{m} (y_k - y_{k - 1}) = \varepsilon (y_m - y_n)\] 
    \[|x_n - x_m| \leqslant \varepsilon(y_m - y_n) \EQ |x_m| \leqslant \varepsilon \cdot y_m \SO \frac{|x_m|}{y_m} < \varepsilon\]
    Доказали, что $\forall \varepsilon > 0 \ \exists N : \forall m \geqslant N  \to \left|\frac{x_m}{y_m}\right| < \varepsilon$
    
    Для $l \neq 0$ доказывается аналогично предыдущей теореме.
\end{proof}

\Pagebreak
\Subsection{Подпоследовательности}
\begin{Def}
    Пусть дана последовательность $\left\{x_n\right\}_{n = 1}^{\infty}$. Подпоследовательностью этой последовательности называется $\left\{x_{n_k}\right\}_{k = 1}^\infty : n_1 < n_2 < n_3 < ...$  
\end{Def}

\begin{Thm}[О стягивающихся отрезках]
    $[a_1, b_1] \supset [a_2, b_2] \supset [a_3, b_3] \supset ..., \lim (b_n - a_n) = 0$. Тогда $\bigcap_{n = 1}^\infty [a_n, b_n]$ состоит из одной точки.
    Если эта точка $c$, то $\lim a_n = \lim b_n = c$ 
\end{Thm}

\begin{proof}
    $\bigcap_{n = 1}^\infty [a_n, b_n] \neq \varnothing$ (по лемме о вложенных отрезках). Пусть $c < d$ принадлежит этому пересечению.
    \[0 < d - c \leqslant b_n - a_n \to 0 \SO 0 \leqslant c - a_n \leqslant 0 \SO \text{ точка единственна}\] 
    \[0 \leqslant c - a_n \leqslant b_n - a_n \to \SO 0 \leqslant c - a_n \leqslant 0 \SO a_n \to c\]
    \[0 \leqslant b_n - c \leqslant b_n - a_n \SO b_n \to c\]
\end{proof}

\begin{Thm}[Теорема Больцано-Вейерштрасса]
    Из всякой ограниченной последовательности можно извлечь сходящуюся подпоследовательность.
\end{Thm}

\begin{proof}
    Возьмем $a_1 \leqslant b_1$ так, чтобы вся последовательность лежала между ними. 
    $x_{n_1} \in [a_1, b_1]$. Поделим отрезок пополам и возьмем ту половину, в которой лежит бесконечное число членов последовательности. Обозначим её $[a_2, b_2]$. 
    Теперь возьмем $x_{n_2} \in [a_2, b_2]$ и $n_2 > n_1$. $[a_3, b_3]$ - ту половину $[a_2, b_2]$, в которой бесконечное число членов последовательности и т.д.
    
    \[[a_1, b_1] \supset [a_2, b_2] \supset [a_3, b_3] \supset ... \text{ и длина } [a_k, b_k] = \frac{b_1 - a_1}{2^k} \to 0\]

    \[\bigcap_{n = 1}^\infty [a_n, b_n] = \left\{c\right\}, \lim a_n = \lim b_n = c\]

    $n_1 < n_2 < n_3 < ...$. $\left\{x_{n_k}\right\}_{k = 1}^{\infty}$ -- подпоследовательность $\left\{x_n\right\}$ и $$a_k \leqslant x_{n_k} \leqslant b_k \SO \lim_{k \to \infty} x_{n_k} = c$$    
\end{proof}

\begin{Thm}
    \begin{MyList}
        \item Если последовательность неограничена сверху, то из неё можно выделить подпоследовательность, стремящуюся к $+\infty$ 
        \item Если неограничена снизу, то можно выделить подпоследовательность, стремящуюся к $-\infty$ 
    \end{MyList}
\end{Thm}

\begin{proof}
    \begin{align*}
        &\exists n_1 : x_{n_1} > 1 \\
        &\exists n_2 : x_{n_2} > 2 \wedge n_2 > n_1 \\
        &\exists n_k : x_{n_k} > k \wedge n_k > n_{k - 1}
    \end{align*} 
\end{proof}

\Pagebreak
\begin{Cons}
    Из любой последовательсти можно выбрать подпоследовательность имеющую предел (конечный или бесконечный).
\end{Cons}

\begin{Def}
    Частичные пределы последовательности $\left\{x_{n_k}\right\}_{n = 1}^\infty$ -- пределы её подпоследовательностей.
    
    \begin{Rem}
        $\lim x_n = a, \left\{x_{n_k}\right\}$ - подпоследовательность $\SO x_{n_k} \to a$ 
    \end{Rem}
\end{Def}

\begin{Def}
    Последовательность $\left\{x_n\right\}$ -- фундаментальная, если 
    \[\forall \varepsilon > 0 \ \exists N : \forall m, n \geqslant N \to |x_m - x_n| < \varepsilon\]
\end{Def}

Свойства:
\begin{MyList}
    \item Фундаментальная последовательность ограничена
    \item Сходящаяся последовательность фундмаентальна
    \begin{proof}
        $a = \lim x_n$
        \[\forall \varepsilon > 0 \ \exists N : \forall n \geqslant N \to |x_n - a| < \frac{\varepsilon}{2}\]
        \[|x_n - x_m| = |x_n - a + a - x_m| \leqslant |x_n - a| + |x_m - a| < \varepsilon\]  
    \end{proof}
    \item Если у фундаментальной последовательности есть сходящаяся подпоследовательность, то эта последовательность сходится.
    \begin{proof}
        $\lim_{k \to \infty} x_{n_k} = a, \forall \varepsilon > 0 \ \exists K : \forall k \geqslant K \to |x_{n_k} - a| < \frac{\varepsilon}{2}$ 
        \[\forall \varepsilon > 0 \ \exists N : \forall m, n \geqslant N \to |x_n - x_m| < \frac{\varepsilon}{2}\]
        $M = \max \left\{n_K, N\right\}$. Тогда $\forall n \geqslant M \to |x_n - a| \leqslant |x_n - x_{n_k}| + |x_{n_k} - a| < \varepsilon$ 
    \end{proof}
\end{MyList}

\begin{Thm}[Критерий Коши сходимости последовательности]
    Последовательность сходится $\EQ$ она фундаментальна
\end{Thm}

\begin{proof}
    "$\SO$". Свойство 2. \\
    "$\Leftarrow$". Фундаментальа $\SO$ ограничена (свойство 1) $\SO \exists$ сходящаяся подпоследовательность (теорема Больцано-Вейерштрасса) $\SO$ сходится  
\end{proof}

\begin{Def}
    $\left\{x_n\right\}$ -- ограничена сверху. \\
    $\overline{\lim_{n \to \infty}} x_n = \limsup_{n \to \infty} x_k = \lim_{n \to \infty} \sup_{k \geqslant n} x_k$  -- верхний предел. \\
    $\underline{\lim_{n \to \infty}} x_n = \liminf_{n \to \infty} x_k = \lim_{n \to \infty} \inf_{k \geqslant n} x_k$  -- нижний предел. 
\end{Def} 

\Pagebreak
\begin{Thm}
    Пусть $y_n = \inf_{k \geqslant n}x_k, z_n = \sup_{k \geqslant n} x_k$. Тогда
    \[\exists \underline{\lim} x_n, \overline{\lim} x_n \wedge \underline{\lim} x_n \leqslant \overline{\lim} x_n\] 
\end{Thm}

\begin{proof}
    \begin{MyList}
        \item Если неограничена сверху, то $\overline{\lim x_n} = +\infty$
        \item Пусть $\left\{x_n\right\}$ -- ограчичена. $\forall n \to x_n \leqslant M$
        \[z_1 \geqslant z_2 \geqslant z_3 \geqslant ... \wedge z_n \leqslant M \SO \exists \lim z_n\]
        \item Аналогично для $y_n$
        \item $\forall n \to y_n \leqslant z_n \SO \underline{\lim}x_n \leqslant \overline{\lim}x_n$  
    \end{MyList}
\end{proof}

\begin{Thm}
    \begin{MyList}
        \item $\overline{\lim} x_n$ -- наибольший частичный предел $\{x_n\}$  
        \item $\underline{\lim} x_n$ -- наименьший частичный предел $\{x_n\}$ 
        \item $\exists \lim x_n$ в $\overline{\R} \EQ \overline{\lim}x_n = \underline{\lim}x_n$  
    \end{MyList}
\end{Thm}

\begin{proof}
    \begin{MyList}
        \item $\{x_n\}$ -- ограниченная последовательность, $b = \overline{\lim} x_n$. Построим $\{x_{n_k}\} : x_{n_k} \xrightarrow[k \to \infty]{} b$
        $z_1 = \sup \{x_1, x_2, ...\} > b - \frac{1}{2} \SO \exists x_{n_1} > b - \frac{1}{2}$ \\
        $z_{n_1} = \sup \{x_{n_1 + 1}, x_{n_1 + 2}, ...\} > b - \frac{1}{3} \SO \exists x_{n_2} > b - \frac{1}{3}, n_2 > n_1$  
        
        \[b - \frac{1}{k} < x_{n_k} \leqslant z_{n_k} \SO x_{n_k} \to b\]
        Рассмотрим $x_{m_k} \to c$. Тогда $x_{m_k} \leqslant z_{m_k} \SO c \leqslant b$
        
        Если $\{x_n\}$ неограничена 
        \begin{MyItemize}
            \item сверху: $\exists \overline{\lim}x_n = +\infty \SO \exists \{x_{n_k}\} : x_{n_k} \to +\infty$
            \item снизу: а) $\exists \overline{\lim} x_n = b \in \R$ -- аналогично п.1. б) $\exists \overline{\lim}x_n = -\infty \SO x_n \to -\infty$   
        \end{MyItemize} 

        \item Аналогично
        \item "$\SO$". $\exists \lim x_n = l \SO \forall \{x_{n_k}\} \to \lim x_{n_k} = l \SO \overline{\lim} x_n = \underline{\lim} x_n = l$
        
        "$\Leftarrow$". $y_n \leqslant x_n \leqslant z_n \SO \underline{\lim}x_n \leqslant \lim x_n \leqslant \overline{\lim} x_n \SO \exists \lim x_n = \overline{\lim} x_n = \underline{\lim} x_n$ 
    \end{MyList}
\end{proof}

\begin{Thm}[характеристические свойства $\overline{\lim}x_n$ и $\underline{\lim} x_n$]
    \begin{MyList}
        \item \[a = \underline{\lim} x_n \EQ \begin{cases}
            \forall \varepsilon > 0 \ \exists N : \forall n \geqslant N \to x_n > a - \varepsilon \\
            \forall \varepsilon > 0, \forall N \ \exists n \geqslant N : x_n < a + \varepsilon
        \end{cases}\] 

        \item \[b = \overline{\lim} x_n \EQ \begin{cases}
            \forall \varepsilon > 0 \ \exists N : \forall n \geqslant N \to x_n < a + \varepsilon \\
            \forall \varepsilon > 0, \forall N \ \exists n \geqslant N : x_n > a - \varepsilon
        \end{cases}\]
    \end{MyList}
\end{Thm}

\begin{proof}
    \TODO
\end{proof}
\end{document}