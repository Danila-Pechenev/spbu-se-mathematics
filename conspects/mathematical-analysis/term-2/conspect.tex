\documentclass[12pt]{article}

% Автор: Илья Дудников
% Автор стиля: Сергей Копелиович

\usepackage{cmap}
\usepackage[T2A]{fontenc}
\usepackage[utf8]{inputenc}
\usepackage[russian]{babel}
\usepackage{graphicx}
\usepackage{amsthm,amsmath,amssymb}
\usepackage{listings}
\usepackage{color}
\usepackage{xcolor}
\usepackage{array}
\usepackage{epigraph}
\usepackage{multicol}

\usepackage[russian,colorlinks=true,urlcolor=red,linkcolor=blue]{hyperref}
\usepackage{enumerate}
\usepackage{datetime}
\usepackage{fancyhdr}
\usepackage{lastpage}
\usepackage{verbatim}
\usepackage{tikz}
\usepackage{MnSymbol}
\usetikzlibrary{arrows,decorations.markings,decorations.pathmorphing}
\usepackage{pgfplots}

\usepackage{ifthen}
\usepackage{mathtools}

%\usepackage{tabls}
%\usepackage{tabularx}
%\usepackage{xifthen}
%\listfiles

\def\NAME{Лекции}
\def\SEASON{Конспект лекций по матанализу, ПИ, 1 семестр}

\sloppy
\voffset=-20mm
\textheight=235mm
\hoffset=-22mm
\textwidth=180mm
\headsep=12pt
\footskip=20pt

\parskip=0em
\parindent=0em

\setlength\epigraphwidth{.8\textwidth}

\newlength{\tmplen}
\newlength{\tmpwidth}
\newcounter{listcounter}

% Список с маленькими отступами
\newenvironment{MyList}[1][4pt]{
  \begin{enumerate}[1.]
  \setlength{\parskip}{0pt}
  \setlength{\itemsep}{#1}
}{       
  \end{enumerate}
}
% Вложенный список с маленькими отступами
\newenvironment{InnerMyList}[1][0pt]{
  \vspace*{-0.5em}
  \begin{enumerate}[(a)]
  \setlength{\parskip}{-0pt}
  \setlength{\itemsep}{#1}
}{       
  \end{enumerate}
  \vspace*{-0.5em}
}
% Список с маленькими отступами
\newenvironment{MyItemize}[1][4pt]{
  \begin{itemize}
  \setlength{\parskip}{0pt}
  \setlength{\itemsep}{#1}
}{       
  \end{itemize}
}

% Основные математические символы
\def\TODO{{\color{red}\bf TODO}}
\def\C{\mathbb{C}}       %
\def\Q{\mathbb{Q}}       %
\def\N{\mathbb{N}}       %
\def\R{\mathbb{R}}       %
\def\F2{\mathbb{F}_2}    %
\def\Z{\mathbb{Z}}       %
\def\INF{\t{+}\infty}    % +inf
\def\EPS{\varepsilon}    %
\def\EMPTY{\varnothing}  %
\def\PHI{\varphi}        %
\def\SO{\Rightarrow}     % =>
\def\EQ{\Leftrightarrow} % <=>
\def\t{\texttt}          % mono font
\def\c#1{{\rm\sc{#1}}}   % font for classes NP, SAT, etc
\def\O{\mathcal{O}}      %
\def\NO{\t{\#}}          % #
\def\XOR{\text{ {\raisebox{-2pt}{\ensuremath{\Hat{}}}} }}
\renewcommand{\le}{\leqslant}
\renewcommand{\ge}{\geqslant}
\newcommand{\q}[1]{\langle #1 \rangle}               % <x>
\newcommand\URL[1]{{\footnotesize{\url{#1}}}}        %
% \newcommand{\sfrac}[2]{{\scriptscriptstyle\frac{#1}{#2}}}  % Очень маленькая дробь
% \newcommand{\mfrac}[2]{{\scriptstyle\frac{#1}{#2}}}    % Небольшая дробь
\newcommand{\sfrac}[2]{{\scriptstyle\frac{#1}{#2}}}  % Очень маленькая дробь
\newcommand{\mfrac}[2]{{\textstyle\frac{#1}{#2}}}    % Небольшая дробь

\newcommand{\fix}[1]{{\color{fixcolor}{#1}}} % \underline
\def\bonus{\t{\red{(*)}}}
\def\ifbonus#1{\ifthenelse{\equal{#1}{}}{}{\bonus}}
\def\smallsquare{$\scalebox{0.5}{$\square$}$}

\newlength{\myItemLength}
\setlength{\myItemLength}{0.3em}
\def\ItemSymbol{\smallsquare}
\def\Item{\vspace*{\myItemLength}\ItemSymbol \ \ }

\newcommand{\LET}{%
  % [line width=0.6pt]
  \begin{tikzpicture}%
  \draw(0.8ex,0) -- (0.8ex,1.6ex);%
  \draw(0,1.6ex) -- (0.8ex,1.6ex);%
  \end{tikzpicture}%
  \hspace*{0.1em}%
}

% Отступы
\def\makeparindent{\hspace*{\parindent}\unskip}
\def\up{\vspace*{-0.5em}}%{\vspace*{-\baselineskip}}
\def\down{\vspace*{0.5em}}
\def\LINE{\vspace*{-1em}\noindent \underline{\hbox to 1\textwidth{{ } \hfil{ } \hfil{ } }}}
\def\BOX#1{\mbox{\fbox{\bf{#1}}}}
\def\Pagebreak{\pagebreak\vspace*{-1.5em}}

% Мелкий заголовок
\newcommand{\THEE}[1]{
  \vspace*{0.5em}
  \noindent{\bf \underline{#1}}%\hspace{0.5em}
  \vspace*{0.2em}
}
% Другой тип мелкого заголовка
\newcommand{\THE}[1]{
  \vspace*{0.5em} $\bullet$
  \noindent{\bf #1}%\hspace{0.5em}
  \vspace*{0.2em}
}

\newenvironment{MyTabbing}{
  \t\bgroup
  \vspace*{-\baselineskip}
  \begin{tabbing}
    aaaa\=aaaa\=aaaa\=aaaa\=aaaa\=aaaa\kill
}{
  \end{tabbing}
  \t\egroup
}

% Код с правильными отступами
\lstnewenvironment{code}{
  \lstset{}
%  \vspace*{-0.2em}
}%
{
%  \vspace*{-0.2em}
}
\lstnewenvironment{codep}{
  \lstset{language=python}
}%
{
}

% Формулы с правильными отступами
\newenvironment{smallformula}{
 
  \vspace*{-0.8em}
}{
  \vspace*{-1.2em}
  
}
\newenvironment{formula}{
 
  \vspace*{-0.4em}
}{
  \vspace*{-0.6em}
  
}

% Большая квадратная скобка
\makeatletter
\newenvironment{sqcases}{%
  \matrix@check\sqcases\env@sqcases
}{%
  \endarray\right.%
}
\def\env@sqcases{%
  \let\@ifnextchar\new@ifnextchar
  \left\lbrack
  \def\arraystretch{1.2}%
  \array{@{}l@{\quad}l@{}}%
}
\makeatother

% Определяем основные секции: \begin{Lm}, \begin{Thm}, \begin{Def}, \begin{Rem}
\renewcommand{\qedsymbol}{$\blacksquare$}
\theoremstyle{definition} % жирный заголовок, плоский текст
\newtheorem{Thm}{\underline{Теорема}}[subsection] % нумерация будет "<номер subsection>.<номер теоремы>"
\newtheorem{Lm}[Thm]{\underline{Lm}} % Нумерация такая же, как и у теорем
\newtheorem{Ex}[Thm]{Упражнение} % Нумерация такая же, как и у теорем
\newtheorem{Example}[Thm]{Пример} % Нумерация такая же, как и у теорем
\newtheorem{Code}[Thm]{Код} % Нумерация такая же, как и у теорем
\theoremstyle{plain} % жирный заголовок, курсивный текст
\newtheorem{Def}[Thm]{Def} % Нумерация такая же, как и у теорем
\theoremstyle{remark} % курсивный заголовок, плоский текст
\newtheorem{Cons}[Thm]{Следствие} % Нумерация такая же, как и у теорем
\newtheorem{Conj}[Thm]{Гипотеза} % Нумерация такая же, как и у теорем
\newtheorem{Prop}[Thm]{Утверждение} % Нумерация такая же, как и у теорем
\newtheorem{Rem}[Thm]{Замечание} % Нумерация такая же, как и у теорем
\newtheorem{Remark}[Thm]{Замечание} % Нумерация такая же, как и у теорем
\newtheorem{Algo}[Thm]{Алгоритм} % Нумерация такая же, как и у теорем

% Определяем ЗАГОЛОВКИ
\def\SectionName{unknown}
\def\AuthorName{unknown}

\newlength{\sectionvskip}
\setlength{\sectionvskip}{0.5em}
\newcommand{\Section}[4][]{
  % Заголовок
  \pagebreak
%  \ifthenelse{\isempty{#1}}{
    \refstepcounter{section}
%  }{}
  \vspace{0.5em}
%  \ifthenelse{\isempty{#1}}{
%    \addtocontents{toc}{\protect\addvspace{-5pt}}%
    \addcontentsline{toc}{section}{\arabic{section}. #2}
%  }{}
  \begin{center}
    {\Large \bf Раздел \NO{\arabic{section}}: #2} \\ 
    \vspace{\sectionvskip}
    \ifthenelse{\equal{#3}{}}{}{{\large #3}\\}
  \end{center}

  \LINE

  % Запомнили название и автора главы
  \gdef\SectionName{#2}
  \gdef\AuthorName{#4}

  % Заголовок страницы
  \lhead{\SEASON}
  \chead{}
  \rhead{\SectionName}
  \renewcommand{\headrulewidth}{0.4pt}

  \lfoot{Глава \NO{\arabic{section}}.}
  \cfoot{\thepage\t{/}\pageref*{LastPage}}
  \rfoot{Автор: \AuthorName}
  \renewcommand{\footrulewidth}{0.4pt}
}

\newcommand{\Subsection}[2][]{
  \refstepcounter{subsection}
  \vspace*{1em}
  \ifthenelse{\equal{#1}{}}
    {\addcontentsline{toc}{subsection}{\arabic{section}.\arabic{subsection}. #2}}
    {\addcontentsline{toc}{subsection}{\arabic{section}.\arabic{subsection}. \bonus\,#2}}
  {\color{blue}\bf\large \arabic{section}.\arabic{subsection}. \ifbonus{#1}\,{#2}} 
  \vspace*{0.5em}
  \makeparindent
}
\newcommand{\Subsubsection}[2][]{
  \refstepcounter{subsubsection}
  \vspace*{1em}
  \ifthenelse{\equal{#1}{}}
    {\addcontentsline{toc}{subsubsection}{\arabic{section}.\arabic{subsection}.\arabic{subsubsection}. #2}}
    {\addcontentsline{toc}{subsubsection}{\arabic{section}.\arabic{subsection}.\arabic{subsubsection}. \bonus\,#2}}
  {\color{blue}\bf\large \arabic{section}.\arabic{subsection}.\arabic{subsubsection}. \ifbonus{#1}\,#2}
  \vspace*{0.5em}
  \makeparindent
}

\newcommand{\Header}{
  \pagestyle{empty}
  \renewcommand{\dateseparator}{--}
  \begin{center}
    {\Large\bf 
     Матанализ 2 семестр ПИ,\\
    \vspace{0.3em}
    \NAME}\\
    \vspace{0.7em}
    {Собрано {\today} в {\currenttime}}
  \end{center}

  \LINE
  \vspace{0em}

  \renewcommand{\baselinestretch}{0.98}\normalsize
  \tableofcontents
  \renewcommand{\baselinestretch}{1.0}\normalsize
  \pagebreak
}

\newcommand{\BeginConspect}{
  \pagestyle{fancy}
  \setcounter{page}{1}
}

\definecolor{mygray}{rgb}{0.7,0.7,0.7}
\definecolor{ltgray}{rgb}{0.9,0.9,0.9}
\definecolor{fixcolor}{rgb}{0.7,0,0}
\definecolor{red2}{rgb}{0.7,0,0}
\definecolor{dkred}{rgb}{0.4,0,0}
\definecolor{dkblue}{rgb}{0,0,0.6}
\definecolor{dkgreen}{rgb}{0,0.6,0}
\definecolor{brown}{rgb}{0.5,0.5,0}

\newcommand{\green}[1]{{\color{green}{#1}}}
\newcommand{\black}[1]{{\color{black}{#1}}}
\newcommand{\red}[1]{{\color{red}{#1}}}
\newcommand{\dkred}[1]{{\color{dkred}{#1}}}
\newcommand{\blue}[1]{{\color{blue}{#1}}}
\newcommand{\dkgreen}[1]{{\color{dkgreen}{#1}}}

\newcommand{\Mod}[1]{\ (\mathrm{mod}\ #1)}

\DeclareMathOperator{\Real}{Re}
\DeclareMathOperator{\Imag}{Im}
\DeclareMathOperator{\lcm}{lcm}
\DeclareMathOperator{\sign}{sign}
\DeclareMathOperator{\Si}{Si}
\DeclareMathOperator{\const}{const}

\begin{document}

\Header

\BeginConspect

\Section{Интегральное исчисление}{}{Илья Дудников}

\Subsection{Неопределенный интеграл}
\begin{Def}
    $f : \langle A, B\rangle \to \R, F : \langle A, B\rangle \to \R$ называется первообразной функцией $f$, если $F$ дифференцируема на $\langle A, B\rangle, F'(x) = f(x) \ \forall x \in \langle A, B\rangle$.
\end{Def}

\begin{Thm}
    Пусть $f, F, G : \langle A, B\rangle \to \R, F$ -- первообразная $f$. Тогда
    $G$ -- первообразная $f \EQ \exists c \in \R : F(x) + c = G(x)$.
\end{Thm}

\begin{proof}
	$\SO$.  
    Пусть $H(x) = F(x) - G(x)$. Тогда
	\[H'(x) = F'(x) - G'(x) = f(x) - f(x) = 0 \EQ H'(x) = 0 \SO H(x) \equiv \const\]
    $\Leftarrow$. $(F(x) + c)' = (G(x))' \EQ f(x) = F'(x) = G'(x) \SO G$ -- первообразная.  
\end{proof}

\begin{Def}
    $f : \langle A, B\rangle \to \R, F$ -- первообразная $f$. Множество функций 
    $\{F(x) + c, c \in \R\}$ называется неопределенным интегралом $f$.
    \[\int f(x) \,  = F(x) + c, c \in \R \] 
\end{Def}

Далее, $f : \langle A, B\rangle \to \R$. 

\begin{MyList}
	\item Дифференцирование
	\[\left(\int f(x) \,dx\right)' = f(x), x \in \langle A, B\rangle\]

	\item Арифметические действия: 
	\[\int f(x) \,dx + \int g(x) \,dx = \left\{F(x) + G(x) + c, c \in \R\right\}\]
	\[\int f(x) \,dx + H(x) = \left\{F(x) + H(x) + c, c \in \R\right\}\]
	\[\lambda \int f(x) \,dx = \left\{\lambda F(x) + c, c \in \R\right\}, \lambda \neq 0, \lambda \in \R\]
\end{MyList}

\begin{Prop}
	Если функция $f$ непрерывна на $\langle A, B\rangle$, то у неё есть первообразная на~$\langle A, B\rangle$.
\end{Prop}

\begin{Ex}
	$f(x) = \begin{cases}
		1, x \geqslant 0 \\
		-1, x < 0
	\end{cases}$. Есть ли первообразная у этой функции?
\end{Ex}

\begin{Def}
	$E \subset \R, f : E \to \R$. Если $F$ дифференцируема на $E$ и $F'(x) = f(x)$ на $E$, то $F$ -- первообразная $f$ на множестве $E$.
\end{Def}

\Pagebreak
Таблица неопределенных интегралов

\begin{multicols}{2}
	\begin{MyList}
		\item $\int a \,dx = ax + c, a \in \R$ 
		\item $\int x^a \,dx = \frac{x^{a + 1}}{a + 1} + c, a \neq -1$ 
		\item $\int \frac{1}{x}\,dx = \ln |x| + c$ 
		\item $\int e^x \,dx = e^x + c$ 
		\item $\int a^x \,dx = \frac{a^x}{\ln a} + c, a > 0, a \neq 1$ 
		\item $\int \sin x \,dx = -\cos x + c$
		\item $\int \cos x \,dx = \sin x + c$ 
		\item $\int \frac{1}{\cos^2 x}\,dx = \tg x + c$ 
		\item $\int \frac{1}{\sin^2 x}\,dx = -\ctg x + c$ 
		\item $\int \frac{\,dx}{x^2 + a^2} = \frac{1}{a} \arctg \frac{x}{a} + c, a \neq 0$
		\item $\int \frac{\,dx}{\sqrt{a^2 - x^2}} = \arcsin \frac{x}{a} + c, a > 0$
		\item $\int \frac{\,dx}{x^2 - a^2} = \frac{1}{2a}\ln \left| \frac{x - a}{x + a}\right| + c, a \neq 0$ 
		\item $\int \frac{\,dx}{\sqrt{x^2 + a}} = \ln \left| x + \sqrt{x^2 + a}\right| + c, a \in \R$   
	\end{MyList}	
\end{multicols}

\begin{proof}
	Дифференцирование
\end{proof}

\begin{Example}
	$\int \frac{\sin x}{x} \,dx$ -- неберущийся интеграл.
	$\Si(x)$ -- интегральный синус (одна из первообразных, закрепленная при $x \to 0+$ ).
	\[(\Si(x))' = \frac{\sin x}{x}\]
\end{Example}

\begin{Thm}[Линейность неопределенного интеграла]
	$f, g : \langle A, B\rangle \to \R$, имеют первообразные на $\langle A, B\rangle$.
	Тогда $\forall \alpha, \beta \in \R : \alpha, \beta \neq 0$
	\[\int (\alpha f(x) + \beta g(x))\,dx = \alpha \int f(x) \,dx + \beta \int g(x) \,dx\] 
\end{Thm}

\begin{proof}
	Пусть $F$ и $G$ -- первообразные $f$ и $g$ на $\langle A, B\rangle$.
	Правая часть равенства: $\left\{\alpha F(x) + \beta G(x) + c, c \in \R\right\}$.
	\[\left(\alpha F(x) + \beta G(x) + c\right)' = \alpha F'(x) + \beta G'(x) = \alpha f(x) + \beta g(x)\]
\end{proof}

\begin{Thm}[Замена переменной]
	$f : \langle A, B\rangle \to \R, F$ -- первообразная $f$ на $\langle A, B\rangle$, $\PHI : \langle C, D\rangle \to \langle A, B\rangle$ -- дифференцируемая функция.
	Тогда 
	\[\int f(\PHI(x))\PHI'(x) \,dx = F(\PHI(x)) + c\]
\end{Thm}

\begin{proof}
	\[\left(F(\PHI(x)) + c\right)' = F'(\PHI(x)) \cdot \PHI'(x) = f(\PHI(x)) \cdot \PHI'(x)\]
\end{proof}

\begin{Rem}
	$\PHI'(x) \,dx = d \PHI(x)$. Пусть $y = \PHI (x)$ 
	\[\int f(y) dy = F(y) + c = F(\PHI(x)) + c\] 
\end{Rem}

\begin{Example}
	$\int \frac{\ln x}{x}\,dx = \int \ln x \cdot \frac{1}{x}\,dx$. Пусть $y = \ln x \SO dy = \frac{1}{x} \,dx$ 
	\[\SO \int \frac{\ln x}{x}\,dx = \int y \,dy = \frac{y^2}{2} + c = \frac{\ln^2 x}{2} + c\]
\end{Example}

\begin{Cons}
	Пусть в условиях теоремы $\PHI$ имеет обратную функцию $\psi : \langle A, B\rangle \to \langle C, D\rangle$. Если $G(x)$ -- первообразная функции $(f \circ \PHI(x)) \cdot \PHI'(x)$, то \[\int f(x) \,dx = G(\psi (x)) + c\]
\end{Cons}

\begin{proof}
	Пусть $F$ -- первообразная $f$ на $\langle A, B\rangle$.
	$F(\PHI(x))$ -- первообразная $f(\PHI(y)) \PHI'(y)$~(по теореме).
	Рассмотрим $G(x) - F(\PHI(x))$ -- постоянная (т.к. производная равна нулю).
	$y = \PHI(x) \EQ x = \psi(y)$. Тогда
	\[G(\psi(y)) - F(y) = \const \SO \int f(y) \,dy = G(\psi(y)) + c \] 
\end{proof}

\begin{Example}
	$\int \frac{dx}{1 + \sqrt{x}}$. Пусть $t = \sqrt{x}, t > 0 \EQ t^2 = x \SO dx = dt^2 = 2t \,dt$. Тогда
	\[\int \frac{dx}{1 + \sqrt{x}} = \int \frac{2t \,dt}{1 + t} = \int \left( \frac{2t + 2}{t + 1} - \frac{2}{t + 1}\right) \,dt = \int \left(2 - \frac{2}{t + 1}\right) \,dt = 2\int \,dt - 2\int \frac{dt}{t + 1} =\] 
	\[= 2t - \int \frac{d(t + 1)}{t + 1} = 2t - 2\ln |t + 1| + c = 2\sqrt{x} - 2\ln (\sqrt{x} + 1) + c\]
\end{Example}

\begin{Example}
	$\int \sin x \cos x \,dx = \int \sin x \,d\sin x = \frac{\sin^2 x }{2} + c$. \\
	Иначе: $\int \sin x \cos x \,dx = -\int \cos x \,d\cos x = - \frac{\cos^2 x}{2} + c$. \\
	Иначе: $\int \sin x \cos x \,dx = \frac{1}{2} \int \sin 2x \,dx = \frac{1}{2} \cdot \frac{1}{2} \int \sin 2x \,d(2x) = \frac{-\cos 2x}{4} + c$. \\
	Мораль сей басни такова: константы разные, а не $ \frac{\sin^2 x}{2} = - \frac{\cos^2 x}{2} = - \frac{\cos 2x}{4}$. 
\end{Example}

\begin{Thm}[Формула интегрирования по частям]
	$f, g \in C^1 \langle A, B\rangle$. Тогда 
	\[\int f(x)g'(x) \,dx = f(x) g(x) - \int f'(x) g(x) \,dx\] 
\end{Thm}

\begin{proof}
	$H$ -- первообразная $g \cdot f'$. Тогда 
	\[(f(x) g(x) - H(x))' = f'(x) g(x) + f(x)g'(x) - H'(x) = f(x)g'(x)\]
\end{proof}

\begin{Rem}
	$\int u \,dv = uv - \int v \,du$
\end{Rem}

\begin{Example}
	$\int x e^x \,dx$. Пусть $u = x, u' = 1, v' = e^x, v = e^x$
	\[\int x e^x \,dx = xe^x - \int 1 \cdot e^x \,dx = x e^x - e^x + c\] 
\end{Example}

\begin{Example}
	$\int \ln x \,dx$. Пусть $u = \ln x, u' = \frac{1}{x}, v' = 1, v = x$.
	\[\int \ln x \,dx = x \ln x - \int \frac{1}{x} \cdot x \,dx = x \ln x - x + c\]  
\end{Example}

\begin{Ex}
	$\int e^x \cdot \sin x \,dx$
	Пусть $f = \sin x, g = e^x$. Тогда
	\[\int f \, dg = fg - \int g \, df \EQ \int e^x \sin x = e^x\sin x - \int e^x \cos x\]
	Пусть теперь $f = \cos x, g = e^x$. Тогда
	\[\int f \, dg = fg - \int g \, df \EQ \int e^x \cos x = e^x \cos x + \int e^x \sin x\]
	Отсюда 
	\[\int e^x \sin x = e^x\sin x - e^x\cos x - \int e^x \sin x \EQ \int e^x\sin x = \frac{e^x}{2}(\sin x - \cos x)\]
\end{Ex}

\begin{Example}
	Пусть $a \in \R, a \neq 0, I_n = \int \frac{dx}{(x^2 + a)^n}, n \in \N$. Выразим интеграл $I_{n + 1}$ через $I_n$ для произвольного натурального $n$.
	
	Обозначим $f(x) = \frac{1}{(x^2 + a)^n}$ и $g(x) = x$. Тогда
	\[df(x) = \left(\frac{1}{(x^2 + a)^n}\right)' \, dx = -\frac{2nx}{(x^2 + a)^{n + 1}} \, dx, dg(x) = dx\]
	По формуле интегрирования по частям:
	\begin{align*}
		I_n &= \frac{x}{(x^2 + a)^n} + 2n \int \frac{x^2}{(x^2 + a)^{n + 1}} \, dx = \frac{x}{(x^2 + a)^n} + 2n \int \frac{x^2 + a - a}{(x^2 + a)^{n + 1}} \, dx \\
		&= \frac{x}{(x^2 + a)^n} + 2n \int \frac{dx}{(x^2 + a)^n} - 2na \int \frac{dx}{(x^2 + a)^{n + 1}} = \frac{x}{(x^2 + a)^n} + 2n I_n - 2na I_{n + 1} \\
	\end{align*}
	Откуда
	\[2na I_{n + 1} = (2n - 1)I_n + \frac{x}{(x^2 + a)^n}\]
\end{Example} 

\begin{Prop}
	Любая рациональная функция имеет элементарную первообразную.
\end{Prop}

Рассмотрим простешие дроби:

\begin{MyList}
	\item $\frac{a}{(x + p)^n}, n \in \N, a, p \in \R$ 
	\item $ \frac{ax + b}{(x^2 + px + q)^n}$ 
\end{MyList}

Интегралы от простейших дробей первого рода вычисляются по таблице. Для простейших дробей второго рода используется следующий алгоритм:

\begin{MyList}
	\item Если $p \neq 0$, то выделим полный квадрат и выполним замену $y = x + \frac{p}{2}$. Если $p = 0$, тогда
	\[\int \frac{ax + b}{(x^2 + px + q)^n} = a\int \frac{x \,dx}{(x^2 + q)^n} + b \int \frac{dx}{(x^2 + q)^n}\]
	\item Интеграл $\int \frac{x \, dx}{(x^2 + q)^n}$ можно вычислить с помощью замены $y = x^2 + q$, т.к. $dy = 2x \, dx$.
	\item Применяя к интегралу $I_n = \int \frac{dx}{(x^2 + q)^n}$ формулу понижения $n - 1$ раз сведем его к интегралу $I_1$, который является табличным.
\end{MyList}

\begin{Example}[12 и 13 из таблицы]
	\[\int \frac{dx}{x^2 - 4} = \int \left( \frac{\frac{1}{4}}{x - 2} + \frac{-\frac{1}{4}}{x + 2}\right) \,dx = \frac{1}{4} \left(\ln |x - 2| - \ln |x + 2|\right) + c\]	
\end{Example}

\begin{Example}
	$\int \frac{dx}{\sqrt{x^2 + 1}}$. Пусть $x = \sh t, dx = \ch t \,dt $. Тогда
	\[\int \frac{\ch t \,dt}{\sqrt{1 + \sh^2 t}} = \int \frac{\ch t}{\ch t} \,dt = \int \,dt = t + c\]   
\end{Example}

\begin{Ex}
	Найди формулу для $(\sh t)^{-1}$ 
\end{Ex}

Неберущиеся интегралы:

\begin{multicols}{2}
	\begin{MyItemize}
		\item $\int \frac{\sin x}{x} \,dx$ 
		\item $\int \frac{\cos x}{x} \,dx$ 
		\item $\int \frac{\,dx}{\ln x}$ 
		\item $\int \frac{e^x}{x} \,dx$ 
		\item $\int \sin x^2 \,dx$ 
		\item $\int \cos x^2 \,dx$ 
		\item $\int e^{-x^2}\,dx$ 
	\end{MyItemize}
\end{multicols}

\Subsection{Определенный интеграл Римана}

\begin{Def}
	$[a, b], a < b$. Набор точек $\tau = \{x_k\}_{k = 0}^n : x_0 = a < x_1 < x_2 < ... < x_n = b$ -- разбиение (дробление) отрезка $[a, b]$, $\Delta x_k = x_{k + 1} - x_k$ -- длина отрезка $[x_k, x_{k + 1}]$ . 
	$\lambda = \lambda_\tau  = \max_{k \in [0, n - 1]} \Delta x_k$ -- ранг дробления (мелкость), $\xi = \{\xi_k\}_{k = 0}^{n - 1} : \xi_k \in [x_k, x_{k + 1}]$ -- оснащение дробления $\tau$.
	Пара $(\tau, \xi)$ называется оснащенным дроблением.  
\end{Def}

\begin{Def}
	$f : [a, b] \to \R, \sigma_\tau = \sum_{k = 0}^{n - 1} f(\xi_k) \Delta x_k$ -- суммы Римана (интегральные суммы). 
\end{Def}

\begin{figure*}[h]
	\centering
	\def\svgwidth{0.5\columnwidth}
	\input{img/riemann_sum.pdf_tex}
\end{figure*}

\begin{Def}
	$f : [a, b] \to \R$. Число $I \in \R$ называют пределом интегральных сумм при ранге $\to 0 :$
	\[I = \lim_{\lambda_\tau \to 0} \sigma_\tau (f, \xi) \quad (I = \lim_{\lambda \to 0} \sigma)\]
	если $\forall \varepsilon > 0 \ \exists \delta > 0 : \forall \tau : \lambda_\tau < \delta$
	\[|\sigma_\tau (f, \xi) - I| < \varepsilon\] 
\end{Def}

\begin{Rem}
	Последовательность оснащенных дроблений $\{(\tau^{(i)}, \xi^{(i)})\}_{i = 1}^\infty : \lambda^{(i)} \to 0$.
	$\forall \{\tau^{(i)}, \xi^{(i)}\} : \lambda^{(i)} \to 0 \ \sigma_{\tau^{(i)}}(f, \xi^{(i)}) \to I$.  
\end{Rem}

\begin{Def}[Интеграл Римана]
	$f : [a, b] \to \R$. Если $\exists \lim_{\lambda \to 0} \sigma = I $, то $f$ называется интегрируемой по Риману на $[a, b]$, а число $I$ называется интегралом $f$ по $[a, b]$. \\
	$R[a, b]$ -- класс функций, интегрируемых по Риману на $[a, b]$.
	\[\int_a^b f(x)\,dx\]
\end{Def}

\Subsection{Суммы Дарбу}

\begin{Def}
	$f : [a, b] \to \R, \tau = \{x_k\}_{k = 0}^n$ -- дробление $[a, b]$.
	\[M_k = \displaystyle{\sup_{x \in [x_k, x_{k + 1}]}} f(x), m_k = \displaystyle{\inf_{x \in [x_k, x_{k + 1}]}}f(x)\] 
	Суммы
	\[S = S_\tau (f) = \displaystyle{\sum_{k = 0}^{n - 1}} M_k \Delta x_k, s = s_\tau (f) = \displaystyle{\sum_{k = 0}^{n - 1}} m_k \Delta x_k\]
	называются верхними и нижними интегральными суммами.
\end{Def}

\begin{Rem}
	Если $f$ -- непрерывна на $[a, b]$, то это две частные суммы из сумм Римана.
\end{Rem}

\begin{Rem}
	$f$ ограничена сверху $\EQ$ $S$ ограничена.
\end{Rem}

Свойства сумм Дарбу:

\begin{MyList}
	\item $S_\tau (f) = \displaystyle{\sup_\xi \sigma_\tau (f, \xi)}, s_\tau = \displaystyle{\inf_\xi \sigma_\tau (f, \xi)}$ 
	\begin{proof}
		$M_k \geqslant f(\xi_k), k = 0, ..., n - 1$. Тогда $M_k \Delta x_k \geqslant f(\xi_k) \Delta x_k \EQ \sum_{k = 0}^{n - 1} M_k \Delta x_k \geqslant \sum_{k = 0}^{n - 1} f(\xi_k) \Delta x_k \SO S_\tau (f) \geqslant \sigma_\tau $, т.е. $S_\tau$ -- верхняя граница. Докажем, что она является точной верхней границей. \\
		Если $f$ ограничена на $[a, b]$. Фиксируем $\varepsilon > 0$. На каждом кусочке разбиения $\exists \xi_k^* \in [x_k, x_{k + 1}] : f(\xi_k^*) > M_k - \frac{\varepsilon}{b - a}$.
		Тогда $\sigma^* = \sum_{k = 0}^{n - 1} f(\xi_k^*) \Delta x_k > S - \frac{\varepsilon}{b - a}\sum_{k = 0}^{n - 1} \Delta x_k = S - \varepsilon$. \\
		Если $f$ не ограничена на $[a, b] \SO$ не ограничена на каком-то кусочке $[x_l, x_{l + 1}]$. 
		Фиксируем $A > 0$ и выберем $\xi_k^*$ при $k \neq l$ произвольно, а для $\xi_l^*$
		\[f(\xi_l^*) > \frac{1}{\Delta x_l}\left(A - \displaystyle{\sum_{k \neq l} f(\xi_k^*) \Delta x_k}\right)\]

		Тогда 
		\[\sigma^* = \displaystyle{\sum_{k = 0}^{n - 1} f(\xi_k^*) \Delta x_k} > A \SO \sup_\xi \sigma = +\infty = S\] 
	\end{proof}

	\item При добавлении новых точек дробления верхняя сумма не увеличится, а нижняя не уменьшится.
	\begin{proof}
		Докажем для верхних сумм при добавлении одной точки.
		$\tau : \{x_k\}_{k = 0}^{n - 1}$. Добавим точку $c$ в $[x_l, x_{l + 1}] - T$ -- новое дробление. \\
		\[S_\tau = \sum_{k = 0}^{l - 1} M_k \Delta x_k + M_l \Delta x_l + \sum_{k = l + 1}^{n - 1} M_k \Delta x_k\]
		\[S_T = \sum_{k = 0}^{l - 1} M_k \Delta x_k + (c - x_l) \cdot M' + (x_{l + 1} - c) M'' + \sum_{k = l + 1}^{n - 1} M_k \Delta x_k\]
		где $M' = \sup_{x \in [x_l, c]} f, M'' = \sup_{x \in [c, x_{l + 1}]} f$. $M_l \geqslant M', M_l \geqslant M''$, т.к. $[x_l, c] \subset [x_l, x_{l + 1}], [c, x_{l + 1}] \subset [x_l, x_{l + 1}]$.

		Рассмотрим $S_\tau - S_T = M_l \Delta x_l - (c - x_l) M' - (x_{l + 1} - c) M'' \geqslant M_l (x_{l + 1} - x_l - c + x_l - x_{l + 1} + c) = 0$.
		
		Добавить больше точек можно по индукции.
	\end{proof}

	\item Каждая нижняя сумма Дарбу не превосходит каждой верхней.
	\begin{proof}
		$\tau_1, \tau_2$ -- разные дробления $[a, b]$. Докажем, что $s_{\tau_1} \leqslant S_{\tau_2}$. Возьмем $\tau = \tau_1 \cup \tau_2$. Тогда $s_{\tau_1} \leqslant s_\tau \leqslant S_\tau \leqslant S_{\tau_2}$ (по свойству 2).
	\end{proof}
\end{MyList}

\begin{Prop}
	$f \in R[a, b] \SO f$ ограничена на $[a, b]$.
\end{Prop}

\begin{proof}
	Пусть $f$ не ограничена на $[a, b]$ сверху. Тогда $\forall \tau \SO \sup_\xi \sigma_\tau (f, \xi) = +\infty$. Тогда 
	$\forall \tau$ и числа $I \ \exists$ оснащение $\xi' : \sigma_\tau (\xi') > I + 1 \SO$ никакое число $I$ не является пределом интегральных сумм.   
\end{proof}

\begin{Def}
	$f : [a, b] \to \R$. Возьмем
	\[I^* = \inf_\tau S_\tau \qquad I_* = \sup_\tau s_\tau\]
	где $I^*$ -- верхний интеграл Дарбу, $I_*$ -- нижний интеграл Дарбу.
\end{Def}

\begin{Rem}
	$I^* \geqslant I_*$.
\end{Rem}

\begin{Rem}
	$f$ ограничена сверху $\EQ I^*$ ограничена. 
\end{Rem}

\Subsection{Критерии интегрируемости функции}

\begin{Thm}[Критерий интегрируемости функции]
	Пусть $f : [a, b] \to \R$. Тогда $f \in R[a, b] \EQ S_\tau (f) - s_\tau (f) \xrightarrow[\lambda \to 0]{} 0$, т.е.
	\[\forall \varepsilon > 0 \ \exists \delta > 0 : \forall \tau : \lambda_\tau < \delta \ S_\tau(f) - s_\tau(f) < \varepsilon\] 
\end{Thm}

\begin{proof}
	$\SO$. Пусть $f \in R[a, b]$. Обозначим $I = \int_a^b f$. Возьмем $\varepsilon > 0$, подберем $\delta > 0 :$
	\[I - \frac{\varepsilon}{3} < \sigma_\tau (f, \xi) < I + \frac{\varepsilon}{3}\]
	Переходя к супремуму и инфимуму, получим
	\[I - \frac{\varepsilon}{3} \leqslant s_\tau \leqslant S_\tau \leqslant I + \frac{\varepsilon}{3}\]
	откуда $S_\tau - s_\tau \leqslant I + \frac{\varepsilon}{3} - I + \frac{\varepsilon}{3} = \frac{2\varepsilon}{3} < \varepsilon$.

	$\Leftarrow$. Пусть $S_\tau - s_\tau \xrightarrow[\lambda \to 0]{} 0 \SO$ все суммы Дарбу конечны.
	\[s_\tau \leqslant I_* \leqslant I^* \leqslant S_\tau \SO 0 \leqslant I^* - I_* \leqslant S_\tau - s_\tau\]
	$\SO I^* = I_*$ (т.к. это числа). Обозначим $I = I^* = I_*$.
	\[s_\tau \leqslant I \leqslant S_\tau, s_\tau \leqslant \sigma_\tau \leqslant S_\tau \SO |I - \sigma_\tau| \leqslant S_\tau - s_\tau\]
	$\SO \forall \varepsilon > 0 \ \exists \delta > 0 : \forall \tau : \lambda_\tau < \delta \ |I - \sigma_\tau| < \varepsilon$.   
\end{proof}

\begin{Rem}
	Если $f \in R[a, b] \SO s_\tau \leqslant \int_a^b f \leqslant S_\tau$. 
\end{Rem}

\begin{Cons}
	$f \in R[a, b] \SO \lim_{\lambda \to 0} S_\tau = \lim_{\lambda \to 0} s_\tau = \int_a^b f$ 
\end{Cons}

\begin{proof}
	$0 \leqslant S_\tau - \int_a^b f \leqslant S_\tau - s_\tau$, $0 \leqslant \int_a^b f - s_\tau \leqslant S_\tau - s_\tau$.
\end{proof}

\begin{Rem}
	$\lim_{\lambda \to 0} S_\tau = I^*, \lim_{\lambda \to 0} s_\tau = I_*$. 
\end{Rem}

\begin{Prop}[Критерий Дарбу интегрируемости функции по Риману]
	$f \in R[a, b] \EQ f$ ограничена на $[a, b]$ и $I_* = I^*$. 
\end{Prop}

\begin{Prop}[Критерий Римана интегрируемости]
	$f \in R[a, b] \EQ \forall \varepsilon > 0 \ \exists \tau \ S_\tau(f) - s_\tau (f) < \varepsilon$. 
\end{Prop}

\begin{Def}
	$f : D \to \R$. Величина
	\[\omega (f)_D = \sup_{x, y \in D} (f(x) - f(y))\]
	называется колебанием $f$ на $D$. Из определений граней функции ясно, что
	\[\omega (f)_D = \sup_{x \in D} f(x) - \inf_{y \in D} f(y)\]

	Если задано $\tau$ отрезка $[a, b]$, то 
	\[\omega_k (f) = M_k - m_k\]
\end{Def}

Тогда теорему можно записать:
\[f \in R[a, b] \EQ \lim_{\lambda \to 0} \sum_{k = 0}^{n - 1} \omega_k (f) \Delta x_k = 0\]

\begin{Thm}[Интегрируемость непрерывной функции]
	$f : [a, b] \to \R, f \in C[a, b] \SO f \in R[a, b]$.
\end{Thm}

\begin{proof}
	По теореме Кантора $f \in C[a, b] \SO f$ равномерна непрерывна на $[a, b]$.
	\[\forall \varepsilon > 0 \ \exists \delta > 0 : \forall t', t'' \in [a, b] : |t' - t''| < \delta \ |f(t') - f(t'')| < \frac{\varepsilon}{b - a}\]
	По теореме Вейерштрасса $f$ достигает наибольшего и наименьшего значения на любом отрезке, содержащемся в $[a, b]$.
	Поэтому колебание $f$ на всяком отрезке, длина которого меньше $\delta$, будет меньше $\frac{\varepsilon}{b - a}$. Значит, $\forall \tau : \lambda_\tau < \delta$ 
	\[\sum_{k = 0}^{n - 1} \omega_k(f) \Delta x_k < \sum_{k = 0}^{n - 1} \frac{\varepsilon}{b - a} \Delta x_k\]
\end{proof}

\begin{Thm}[Интегрируемость монотонной функции]
	$f$ монотонна на $[a, b] \SO f \in R[a, b]$. 
\end{Thm}

\begin{proof}
	Пусть $f$ монотонно возрастает на $[a, b]$. Если $f(a) = f(b) \SO f$ постоянна $\SO f \in C[a, b] \SO f \in R[a, b]$. \\
	Если $f(a) < f(b)$. $\forall \varepsilon > 0$ возьмем $\delta = \frac{\varepsilon}{f(b) - f(a)}$. Возьмем произвольное $\tau : \lambda_\tau < \delta$ на $[x_k, x_{k + 1}]$. В силу монотонности $f$ верно $\omega_k(f) = f(x_{k + 1}) - f(x_k)$.
	\[\sum_{k = 0}^{n - 1} \omega_k(f) \Delta_k = \sum_{k = 0}^{n - 1} (f(x_{k + 1}) - f(x_k)) \Delta x_k < \sum_{k = 0}^n (f(x_{k + 1}) - f(x_k)) \cdot \frac{\varepsilon}{f(b) - f(a)} = \varepsilon\]   
\end{proof}

\begin{Rem}
	$f \in R[a, b]$. Если изменить значение $f$ в конечном числе точек, то интегрируемость не нарушится и интеграл не изменится.
\end{Rem}

\begin{proof}
	$\widetilde{f}$ -- отличается от $f$ в точках $t_1, t_2, ..., t_m$.
	$|f|$ ограничена на $[a, b] \SO |\widetilde{f}|$ ограничена.
	$|f| \leqslant A$, возьмем $\widetilde{A} = \max \{A, |\widetilde{f}(t_1)|, |\widetilde{f}(t_2)|, ..., |\widetilde{f}(t_m)|\}$.
	В интегральных суммах для $f$ и $\widetilde{f}$ отличаются не более $2m$ слагаемых, поэтому
	\[|\sigma_\tau(f, \xi) - \sigma_\tau(\widetilde{f}, \xi)| \leqslant 2m(A + \widetilde{A}) \lambda_\tau \xrightarrow[\lambda_\tau]{}0\]  
	Поэтому предел $\sigma_\tau (\widetilde{f}, \xi)$ существует и равен пределу $\sigma_\tau (f, \xi)$.  
\end{proof}

\begin{Thm}[Интегрируемость функции и её сужения]
	\begin{MyList}
		\item $f \in R[a, b], [\alpha, \beta] \subset [a, b] \SO f \in R[\alpha, \beta]$
		\item Если $a < c < b, f : [a, b] \to \R$ и $f \in R[a, c], f \in R[c, b]$, то $f \in R[a, b]$.   
	\end{MyList}
\end{Thm}

\begin{proof}
	\begin{MyList}
		\item Возьмем $\varepsilon > 0$, подберем $\delta > 0$ из критерия интегрируемости на $[a, b]$. \\
		$\tau_0$ -- дробление $[\alpha, \beta], \lambda_{\tau_0} < \delta$. Добавим точек до дробления $[a, b]$. Получим $\tau (\lambda_\tau < \delta)$.
		\[S_{\tau_0} - s_{\tau_0} = \sum_{k = l}^{m - 1} \omega_k (f) \Delta x_k \leqslant \sum_{k = 0}^{n - 1} \omega_k (f) \Delta x_k < \varepsilon\]

		\item Пусть $f$ не постоянна, т.е. $\omega(f)_{[a, b]} > 0$.
		Возьмем $\varepsilon > 0$, подберем $\delta_1, \delta_2 : \forall \tau_1 : \lambda_{\tau_1} < \delta_1, \forall \tau_2 : \lambda_{\tau_2} < \delta_2$
		\[S_{\tau_1} - s_{\tau_1} < \frac{\varepsilon}{3}, S_{\tau_2} - s_{\tau_2} < \frac{\varepsilon}{3}\]
		$\delta = \min \{\delta_1, \delta_2, \frac{\varepsilon}{3 \omega}\}$. Пусть $\tau$ -- дробление $[a, b], \lambda_\tau < \delta$.
		Точка $c \in [x_l, x_{l + 1})$. Обозначим $\tau' = \tau \cup \{c\}, \tau_1 = \tau' \cap [a, c], \tau_2 = \tau' \cap [c, b]$
		\[S_\tau - s_\tau \leqslant S_{\tau_1} - s_{\tau_1} + S_{\tau_2} - s_{\tau_1} + \omega_l (f) \delta < \varepsilon\]   
	\end{MyList}
\end{proof}

\gdef\AuthorName{Ксения Кузьмина}

\begin{Def}
	Функция $f:[a,b] \to R$ называется кусочно-непрерывной на $[a,b],$ если множество её точек разрыв пусто или конечно (и все разрывы первого рода)
\end{Def}

\begin{Cons}
	f -- кусочно-непрерывная на $[a, b] \Rightarrow f \in R[a,b]$
\end{Cons}

\begin{proof}
	Возьмём точки $a_1, a_2, ..., a_m$ (может $a_1 = a$ и/или $a_m$ = b). Рассмотрим отрезки $[a_k, a_{k+1}]$. f непрерывна на $(a_k, a_{k+1})$  и $\exists$ 
	конечные $\displaystyle \lim_{x \to a_k+} f(x)$ и $\displaystyle \lim_{x \to a_{k+1}-} f(x) \Rightarrow f \in R[a_k, a_{k+1}] \Rightarrow$  по теореме о сужении $f \in R[a,b]$
\end{proof}

\begin{Def} 
	Множество X называется не более, чем счетным, если оно конечно или счетно. 
\end{Def} 

\begin{Def} 
	$E \subset \R$ -- имеет нулевую меру, если для $\forall \varepsilon > 0$ множество E можно заключить в не более, чем счётное объединение интервалов, суммарная длина
	которых < $\displaystyle  \varepsilon.\\$ \[ \left(\lim_{m \to \infty} \sum_{i=1}^m (b_i-a_i)\right)\]
\end{Def} 

\begin{Example}
	Множество из одной точки.
\end{Example}

\begin{Ex}
	Чему равна мера $\N$?
\end{Ex}

\begin{Thm}[Критерий Лебега интегрируемости по Риману] 
	Пусть $f:[a, b] \to R. \\ f \in R[a,b] \Leftrightarrow f$ ограничена и множество точек разрыва имеет нулевую меру.
\end{Thm}

\begin{Thm}[Арифметические действия над интегрируемыми функциями] 
	$f, g \in \R [a,b]$. Тогда
	\begin{enumerate}
		\item $f+g \in R[a,b]$
		\item $f \cdot g \in R[a,b]$
		\item $\alpha f \in R[a,b], \alpha \in \R$
		\item $|f| \in R[a,b]$
		\item Если $\underset{[a,b]}{\inf} |g| > 0,$ то $\displaystyle \frac{f}{g} \in R[a,b]$
	\end{enumerate}
\end{Thm} 

\begin{proof}
	\begin{enumerate}
		\item $D \subset [a,b].$ $x, y \in D \\ |(f+g)(x) - (f+g)(y)| = |f(x)+g(y) - f(y) - g(y)| \leqslant |f(x)-f(y)|+|g(x)-g(y)| \leqslant \omega_D(f)+\omega_D(g)\\
		\omega_D(f+g) \leqslant \omega_D(f)+\omega_D(g)\\ \underset {[x_k, x_{k+1}]}{\omega} (f+g) \leqslant \underset{[x_k, x_{k+1}]}{\omega} (f) + \underset{[x_k, x_{k+1}]}{\omega} (g)\\
		\omega_k(f+g) \leqslant \omega_k f + \omega_k g$ 
		\[0 \leqslant \sum_{k=0}^{n-1} \omega_k(f+g) \delta x_k \leqslant \sum_{k=0}^{n-1} \omega_k f \Delta x_k+ \sum_{k=0}^{n-1} \omega_k g \Delta x_k (\to 0, \lambda \to 0)\]\\
		$\Rightarrow f+g \in R[a,b]$
		\item $|fg(x) - fg(y)| \leqslant |f(x)g(x)-f(y)g(x)+f(y)g(x)-f(y)g(y)| \leqslant |g(x)||f(x)-f(y)|+ \\ +|f(y)||g(x)-g(y)| \leqslant A|f(x)-f(y)|+ B|g(x)-g(y)| \ \ (\text{т.к. } R[a,b] \Rightarrow \text{ограничена на } [a,b])$
		\item $g(x) = \alpha$
		\item $||f(x)|-|f(y)|| \leqslant |f(x) - f(y)| \\ |\omega_k|f|| \leqslant |\omega_k f|$
		\item $\displaystyle \frac{f}{g} = f \cdot \frac{1}{g}.$ Докажем, что $\displaystyle \frac{1}{g} \in R[a,b]$.\\ $0<m=\underset{[a,b]}{\inf}|g|$ 
		\[\left|\frac{1}{g(x)} - \frac{1}{g(y)}\right| = \left|\frac{g(x)-g(y)}{g(x)g(y)}\right| \leqslant \frac{g(x)-g(y)}{m^2} \EQ \omega_k \left(\frac{1}{g}\right) \leqslant \frac{\omega_k(g)}{m^2}\]
	\end{enumerate}
\end{proof}

\begin{Example}
	1. $\displaystyle \int_{0}^{1} x^2 \,dx\\ x^2 \in C[a,b] \Rightarrow x^2 \in R[a,b].$\\
	Рассмотрим какую-нибудь интегральную сумму: $x_k = \frac{k}{n} = \xi_k$\\
	\[\lim_{n \to \infty} \sum_{k=0}^{n-1}f(\xi_k) \Delta x_k = \lim_{n \to \infty} \sum_{k=0}^{n-1} \left(\frac{k}{n}\right)^2 \cdot \frac{1}{n} = \lim \frac{1}{n^3}  \sum_{k=0}^{n-1} k^2 = \lim \frac{1}{n^3} \cdot \frac{(n-1)n(2n-1)}{6} = \frac{1}{3}\]

	2. $\int_{0}^{1} e^xdx -$ упражнение\\

	3. $f(x) = 
	\begin{cases}
		1, x \in \Q \\
		0, x \notin \Q
	\end{cases}$, $D \notin R[a,b], a<b$

	\begin{proof}
		\[\sum_{k=0}^{n-1} \omega_k(D)\Delta x_k = \sum_{k=0}^{n-1} \Delta x_k = b-a \underset{\lambda \to 0}{\nrightarrow} 0\]
	\end{proof}

	4. $ r(x)
	\begin{cases}
		\frac{1}{q} , x = \frac{p}{q} \in \Q \text{, дробь несократима} \\
		0, x \notin \Q
	\end{cases}$\\
	$r(x)$ непрерывна в каждой точке, разрывна в каждой рациональной.\\
	$r(x) \in R[0,1]$

	\begin{proof}
		Зафиксируем $\displaystyle \varepsilon > 0, N \in \N: \frac{1}{N}<\frac{\varepsilon}{2}$ Рациональные числа из [0,1] со знаменателем $\leqslant N$, конечное число $= C_N$, множество X.\\
		Возьмём $\displaystyle \delta = \frac{\varepsilon}{4C_N}$ и дробление $\tau: \lambda_{\tau} < \delta$\\
		Точки X попадут в не более, чем $2C_N$ отрезков дробления. В отрезках, где нет точек из X наибольшее значение $\displaystyle <\frac{1}{N}$\\
		$s_{\tau}(r) = 0$
		\[S_{\tau}(r)=\sum_{k:M_k \geqslant \frac{1}{N}} M_k \Delta x_k \sum_{k:M_k < \frac{1}{N}}M_l \Delta x_k \leqslant \underbrace{1 \cdot 2C_n}_{\frac{\varepsilon}{2}} 
		\cdot \ \delta + \underbrace{\frac{1}{N}}_{<\frac{\varepsilon}{2}} < \varepsilon\] \\
		$S_{\tau}(r) - s_{\tau}(r) = S_{\tau}(r) \underset{\lambda_r \to 0}{\to} 0 \Rightarrow r \in R[0,1]$ и $\displaystyle \int_{0}^{1} r(x)\,dx = 0$
	\end{proof}

	Если $f \in R_D \ g \in R[a, b]$, то $f(g) \in R[a,b]\text{?} \ \ (D -$ множество значений g$)$\\
	Ответ: нет. Пример: 
	$f(y) = 
	\begin{cases}
		1, y \in [0,1] \\
		0, y = 0
	\end{cases}$  и $g(x) = r(x)$ на $[0,1]$\\
	$f(r(x)) = 
	\begin{cases}
		1, x \in \Q \\
		0, x \notin \Q 
	\end{cases} = D(x) \notin R[0,1]$
\end{Example}

\begin{Thm}[Интегрируемость композиции]
	$\varphi: [\alpha, \beta] \to [a, b], f: [a,b] \to \R,\\ f(\varphi): [\alpha, \beta] \to \R\\
	\varphi \in R[\alpha, \beta], f \in C[a,b]$. Тогда $f \circ \varphi \in R[\alpha, \beta]$ 
\end{Thm} 

\begin{proof}
	Например, из критерия Лебега.
\end{proof}

\Subsection{Свойства интеграла Римана}
\begin{enumerate}
	\item $\displaystyle \int_{b}^{a} f = - \int_{a}^{b} f$ 
	\item $\displaystyle \int_{a}^{b} f = 0 \ (\forall f \text{ на вырожденном отрезке } f \in R [a,a])$
\end{enumerate}

Свойства:

\begin{itemize}
	\item Аддитивность интеграла по отрезку:\\ $a, b, c \in \R, \ f \in R [\min\{a, b, c\}, \max \{a, b, c\}]$
	\[\int_{a}^{b} f = \int_{a}^{c} + \int_{c}^{b} f\]

	\begin{proof}
		$f\in R[a,b] \Rightarrow f in \R[a,c], f \in R[c,b], \{\overline{\tau}^{(n)}, \overline{\xi}^{(n)}\}^{\infty}_{n=1}$ и $\{\overline{\overline{\tau}}^{(n)}, \overline{\overline{\xi}}^{(n)}\}^{\infty}_{n=1}$ --
		последовательности оснащенных дроблений $[a,c]$ и $[c,b]$  (равномерных, т.е. $\overline{\lambda} = \frac{c-a}{n}, \overline{\overline{\lambda}}$)\\
		$\tau^{(n)} = \overline{\tau}^{(n)} \cup \overline{\overline{\tau}}^{(n)} -$ дробление $[a,b]$\\
		$\xi^{(n)} = \overline{\xi}^{(n)} \cup \overline{\overline{\xi}}^{(n)} -$ оснащение $\tau^{(n)}\\
		\sigma = \overline{\sigma} + \overline{\overline{\sigma}}$ при $n \to \infty$\\
		\[\underbrace{\int_{a}^{b} f = \int_{a}^{c} f - \int_{b}^{c} f}_{\text{по доказанному}}  = \int_{a}^{c} f + \int_{c}^{b} f\]
		\[ \int_{a}^{b} f = \int_{a}^{c} f + \int_{c}^{b} f = \int_{a}^{c} f - \int_{b}^{c} f\]
		Все остальные случаи -- аналогично.
	\end{proof}

	\item $f \equiv \alpha$ при $\displaystyle x \in [a,b] \Rightarrow \int_{a}^{b} f = \alpha(b-a)$
	
	\begin{proof}
		\[\sum_{k=0}^{n-1} f(\xi_k) \Delta x_k = \alpha \cdot \sum_{k=0}^{n-1} \Delta x_k = \alpha (b-a)\]
	\end{proof}

	\item Линейность интеграла: $\alpha, \beta \in \R, f, g \in R[a,b]\\
	\displaystyle \int_{a}^{b} (\alpha f + \beta g) = \alpha \int_{a}^{b} + \beta \int_{a}^{b} g$

	\begin{proof}
		$\alpha f + \beta g \in R [a,b]\\
		\sigma_{\tau}(\alpha f + \beta g) = \sigma_{\tau}(\alpha f) + \sigma_{\tau} (\beta g)$ и переход к пределу.
	\end{proof}

	\item Монотонность интеграла: $a < b, \ \ f, g \in R[a,b]$ и $f \leqslant g$ на $\displaystyle [a,b] \Rightarrow \int_{a}^{b} f \leqslant \int_{a}^{b} g$

	\begin{proof}
		$\sigma_{\tau}(f) \leqslant \sigma_{\tau} (g)$
	\end{proof}

	\begin{Cons}
		$a<b, f \in R[a,b]$, если $ f \leqslant M \in \R$ на $\displaystyle [a,b], \text{ то } \int_{a}^{b} f \leqslant M(b-a), \\
		\text{ если } f \geqslant m \text{ на } [a,b] \text{то} \int_{a}^{b} f \geqslant m(b-a)$
	\end{Cons}

	\begin{Cons}
		$\displaystyle f \geqslant 0 \Rightarrow \int_{a}^{b} f \geqslant 0$
	\end{Cons}

	\item $a < b, \underset{f \geqslant 0 \text{ на } [a,b]}{f \in R[a,b]}  \text{ и } \exists c \in [a,b]: f(c)>0$ и $f$ непрерывна в точке C.\\ 
	Тогда $\displaystyle \int_{a}^{b} f > 0$

	\begin{proof}
		Пусть $\varepsilon = \frac{f(c)}{2} > 0 \Rightarrow \exists \delta: \forall x \in \underbrace{[c - \delta; c+\delta] \cap [a,b]}_{[\alpha, \beta]}:
		|f(x) - f(c)| < \varepsilon\\
		\displaystyle f(x) > f(c) - \varepsilon = \frac{f(c)}{2} \Rightarrow \int_{\alpha}^{\beta} f \geqslant \frac{f(c)}{2}(\beta - \alpha)$
		\[ \int_{a}^{b} f = \int_{a}^{\alpha} f + \int_{\alpha}^{\beta} f + \int_{\beta}^{b} \geqslant \int_{\alpha}^{\beta} f \geqslant \frac{f(c)}{2} (\beta - \alpha) > 0\]
	\end{proof}

	\begin{Rem}
		Таким же образом строгий знак в монотонности интеграла.
	\end{Rem}

	\begin{Rem}
		$\displaystyle f \in R[a,b], f > 0 \Rightarrow \int_{a}^{b} f > 0$
	\end{Rem}

	\item $a < b, f \in R[a,b]$
	\[\Big|\int_{a}^{b} f \Big| \leqslant \int_{a}^{b} |f| \]

	\begin{proof}
		$-|f| \leqslant f \leqslant |f|$ 
	\end{proof}

	Если не знаем, что $a \geqslant b \text{ или } b \geqslant a$
	\[\displaystyle \Big| \int_{a}^{b} f \Big| \leqslant \Big|\int_{a}^{b} |f|\Big| \]
\end{itemize}

\Subsection{Интегральные теоремы о средних}

\begin{Thm} 
	$f, g \in R[a,b], g \geqslant 0 \text{ на } [a,b], m \leqslant f \leqslant M. \text{ Тогда } \exists \mu \in [m, M]: \displaystyle \int_{a}^{b} fg = \mu \int_{a}^{b} g$
\end{Thm} 

\begin{proof}
	$mg \leqslant fg \leqslant Mg$ на $[a,b]$
	\[m \int_{a}^{b} g \leqslant \int_{a}^{b} fg \leqslant M \int_{a}^{b} g \]
	Если $\displaystyle \int_{a}^{b} g = 0,$ то $\exists \mu \in [m, M]: 0 = \mu \cdot 0$\\
	Если $\displaystyle \int_{a}^{b} g > 0$, то $m \leqslant \frac{\int_{a}^{b}fg}{\int_{a}^{b} g} \leqslant M$\\
	Возьмём $\displaystyle \mu = \frac{\int_{a}^{b}fg}{\int_{a}^{b} g}$
\end{proof}

\begin{Rem}
	Для $g \leqslant 0$ тоже верно.
\end{Rem}

\begin{Cons}
	\begin{enumerate}
		\item $f \in C[a,b], g \in R[a,b], g \geqslant 0 ($ или $g \leqslant 0)$.\\ 
		Тогда $\displaystyle \exists c \in [a, b]: \int_{a}^{b} f \cdot g = f(c) \cdot \int_{a}^{b} g$ 
		
		\begin{proof}
			По теореме Вейерштрасса: $\exists m = \underset{[a,b]}{\min} f$ и $M = \underset{[a,b]}{\max} f$\\
			Подберём $\mu \in [m, M]$ по предыдущей теореме. Тогда по теореме Больцано-Вейерштрасса $\exists c \in [a,b]: f(c) = M$
		\end{proof}

		\item $f \in R[a,b], m, M \in \R: m \leqslant f \leqslant M \text{ на } [a,b]$. Тогда $\displaystyle \exists \mu \in [m, M]: \int_{a}^{b} f = \mu(b-a)$
		
		\begin{proof}
			$g \equiv 1$ в теореме.
		\end{proof}

		\item $f \in C[a,b]$. Тогда $\displaystyle \exists c \in [a,b]: \int_{a}^{b} f = f(c)(b-a)$
		
		\begin{proof}
			$g \equiv 1$ в следствии 1.
		\end{proof}
	\end{enumerate}
\end{Cons}

\begin{Rem}
	Теорему и следствия называют ещё теоремами о средних. Почему?
\end{Rem}

\begin{Def}  
	$f \in R[a, b], a<b$\\
	$\displaystyle \frac{1}{b-a} \int_{a}^{b} f$ -- интегральное среднее f на $[a, b]$\\ 
	Если возьмём равномерное разбиение $[a, b],$ то $\displaystyle \sigma_n =\sum_{k=0}^{n-1} f(\xi_k) \cdot \frac{b-a}{n}$\\
	То есть $\displaystyle \frac{\sigma_n}{b-a} \to \frac{1}{b-a} \int_{a}^{b} f$, где $\displaystyle \frac{\sigma_n}{b-a}$ -- среднее арифметическое значений функции в точках $\xi_k$ 
\end{Def}

\begin{Def} 
	$E \subset \R$ -- невырожденный промежуток (может быть и лучом), $f: E \to \R$, $f$ -- интегрируема на 
	каждом отрезке, содержащемся в $E$. $a \in E$.\\
	$\displaystyle \Phi(x) = \int_{a}^{x} f(t)\,dt, x \in E$ -- интеграл с переменным верхним пределом.
\end{Def} 

\begin{Thm}[Барроу, об интеграле с переменным верхним пределом] 
	$E \subset \R$ -- невырожденный промежуток, $f: E \to \R$, интегрируема на каждом отрезке из $E$, $a \in E$,
	$\displaystyle \Phi(x) = \int_{a}^{x} f, x\in E$. Тогда
	\begin{enumerate}
		\item $\Phi(x) \in C(E)$
		\item Если f непрерывна в точке $x_0 \in E$, то $\Phi$ -- дифференцируема в точке $x_0, \ \ \Phi'(x_0) = 
		f(x_0)$
	\end{enumerate}
\end{Thm} 

\begin{proof}
	\begin{enumerate}
		\item Пусть $x_0 \in E, \text{ подберем }\delta > 0 [x_0 - \delta; x_0 + \delta] \cap E = [A,B]\\
		|f| \text{ на } [A, B]$ ограничена числом M. $\displaystyle \Delta x: x_0 + \Delta x \in [A, B]\\
		\left|\Phi(x_0 + \Delta x) - \Phi(x_0)\right| = \left|\int_{a}^{x_0 + \Delta x} f - \int_{a}^{x_0} f\right| = 
		\left|\int_{x_0}^{x_0 + \Delta x} f\right| \leqslant \left|\int_{x_0}^{x_0+\Delta x} |f|\right| \leqslant 
		|\Delta x| \cdot M \underset{\Delta x \to 0}{\to} 0$
		\item Проверим, что $ \displaystyle \frac{\Phi(x_0 + \Delta x) - \Phi(x_0)}{\Delta x} \xrightarrow[\Delta x \to 0]{} f(x_0)$\\ %!
		Возьмем $\varepsilon > 0$ и $\delta > 0: \forall t: |t-x_0| < \delta \ |f(t) - f(x_0)| < \varepsilon$ (по
		непрерывности.)\\ $\displaystyle \left|\frac{\Phi(x_0 + \Delta x) - \Phi(x_0)}{\Delta x} - f(x_0) \right| = \left|\frac{1}{\Delta x} \int_{x_0}^{x_0 + \Delta x} 
		f(t)\,dt = f(x_0)\right| = \left| \frac{1}{\Delta x} \int_{x_0}^{x_0+\Delta x} (f(t) - f(x_0))\,dt \right| < \\ < \frac{1}{|\Delta x|}
		\cdot \varepsilon \cdot |\Delta x| = \varepsilon$, $\displaystyle k = \int_{a}^{b} k \cdot \frac{1}{b-a}$
	\end{enumerate}
\end{proof}

\begin{Example}
	$\displaystyle \Phi(x) = \int_{1}^{x} \frac{\sin t}{t} \,dt, x>1$\\
	$\displaystyle \Phi'(x) = \frac{\sin x}{x} \Rightarrow \Si'(x) = \frac{\sin x}{x}$
\end{Example}

\begin{Ex}
	$\int \Si(x)\,dx = ?$
\end{Ex}

\begin{Cons}
	Функция, непрерывная на промежутке имеет на нём первообразную. Ей является интеграл с переменным верхним пределом.
\end{Cons}

\begin{Def} 
	$\displaystyle \Psi(x) = \int_{x}^{a} f$ (Условия на f  и a прежние) -- интеграл с переменным нижним пределом.\\
	$\Rightarrow \Psi'(x) = - f(x)$ (Если f непрерывна).
\end{Def} 

\begin{Thm}[Формула Ньютона-Лейбница] 
	$f \in R[a,b], F -$ первообразная $f$ на $[a,b]$. Тогда: $\displaystyle \int_{a}^{b} f = F(b) - F(a)$
\end{Thm} 

\begin{proof}
	Для каждого $n \in \N$:\\ %!
	$\displaystyle F(x_1) - F(x_0) + F(x_2) - F(x_1) + F(x_3) - F(x_2) + ... + F(x_n) - F(x_{n-1}) = 
	\sum_{k=0}^{n-1} (F(x_{k+1}) - F(x_k)) = F(b) - F(a)$\\
	По теореме Лагранжа $\displaystyle \exists \xi_{k,n} \in (x_k, x_{k+1})\\ F(x_{k+1}) - F(x_k) = F'(\xi_{k, n})(x_{k+1} - x_k) =
	f(\xi_{k,n}) \Delta x_k\\
	\int_{a}^{b} f = \lim_{n \to \infty} \sum_{k=0}^{n-1} f(\xi_{k, n}) \Delta x_k = \lim (F(b) - F(a)) = F(b) - F(a)$
\end{proof}

\begin{Rem}
	$\displaystyle \int_{a}^{b} f = F \Big|^b_a\\
	\int_{a}^{b} f(x)\,dx = F(x) \Big|^b_{x = a}$ -- двойная подстановка.
\end{Rem}

\begin{Rem}
	$G(x) = F(x) + C$ -- тоже первообразная.\\
	$G(b) - G(a) = F(b) - F(a)$
\end{Rem}

\begin{Example}
	$\displaystyle \int_{0}^{1} x^2 \,dx = \left. \frac{x^3}{3} \right|^1_0 = \frac{1}{3}$
\end{Example}

\begin{Example}
	$\displaystyle \int_{-1}^{1} \frac{1}{x^2} \,dx = - \left. \frac{1}{x} \right|^1_{-1} = -2$ - чушь!
	\begin{enumerate}
		\item $\left(-\frac{1}{x}\right)' = \frac{1}{x^2}$ -- не везде на $[-1,1]$
		\item $\frac{1}{x^2}$ не интегрируема на $[-1; 1], $ т.к. не ограничена.
	\end{enumerate}
\end{Example}

\begin{Rem}
	Обобщение теоремы.\\
	$f \in R[a; b], F \in C[a,b], \ F$ -- первообразная f на $[a,b]$ за исключением некоторого конечного числа точек.\\
	Тогда $\displaystyle \int_{a}^{b} f	= F(b) - F(a)$
\end{Rem}

\begin{proof} 
	Пусть $\alpha_0 = a, \alpha_m = b$, $\alpha_1, \alpha_2, ..., \alpha_{m-1}$ -- все точки на $(a,b)$, в которых $F' \neq f$
	$\displaystyle \int_{a}^{b} f = \sum_{k=0}^{m-1} \int_{\alpha_k}^{\alpha_{k+1}} f = \sum_{k=0}^{m-1} (F(\alpha_{k+1}) - F(\alpha_k)) = F(b) - F(a)$.\\
	(Рассмотрим $\displaystyle \int_{\alpha_k}^{\alpha_{k+1}} f = \lim_{\varepsilon \to	0+} \int_{\alpha_k+\varepsilon}^{\alpha_{k+1}
	 - \varepsilon} f =\lim_{\varepsilon \to 0+} (F(\alpha_{k+1} - \varepsilon) - F(\alpha_k + a)) = F(\alpha_{k+1}) - F(\alpha_k)$)
\end{proof}

\begin{Rem}
	Без непрерывности $F$ не получится: на $[-1,1]$\\
	$ F(x) = \sign x = 
	\begin{cases}
		1, x>0\\
		0, x = 0\\
		-1, x<0	
	\end{cases}, f(x) = 0$\\
	$\displaystyle 0 = \int_{-1}^{1} f \neq F \Big|^1_{-1} = 2$
\end{Rem}

\begin{Rem}
	$\displaystyle \int_{a}^{b} F'(x)\,dx = F(b) - F(a)$.\\ F дифференцируема, $F'$ интегрируема. 
\end{Rem}

\begin{Rem}
	$F' \in R[a, b]$ -- существенно.\\
	$F(x) = \begin{cases}
		x^2 \cdot \sin \frac{1}{x^2}, x \neq 0\\
		0, x = 0
	\end{cases}$\\

	$\displaystyle F'(x) = 
	\begin{cases}
		2x \sin \frac{1}{x^2} - \frac{2}{x} \cdot \cos \frac{1}{x^2}, x \neq 0 \\
		0, x = 0
	\end{cases}$\\ $F'$ не ограничена, а значит не интегрируема.
\end{Rem}

\begin{Rem}
	Интегрируемость $\overset{?}{\Leftrightarrow} \exists$ первообразной.\\
	$\cancel{\Rightarrow} \sign x$ интегрируема на [-1. 1], но первообразной нет.\\
	$\cancel{\Leftarrow}$ Предыдущее замечание.   
\end{Rem}

\begin{Thm}[Интегрирование по частям в определенном интеграле.] 
	$f, g$ -- дифференцируемы на $[a,b]$, $f', g' \in R[a,b]$. Тогда\\
	$\displaystyle \int_{a}^{b} fg' = fg \Big|_a^b - \int_{a}^{b} f'g$
\end{Thm} 

\begin{proof}
	$f, g -$ дифференцируемы $\Rightarrow$ непрерывны $\Rightarrow$ интегрируемы.\\
	$(f \cdot g)' = f' \cdot g + g' \cdot f \in R[a,b]$\\
	$\displaystyle \int_{a}^{b} (fg)' = fg \Big|_a^b$\\
	$\displaystyle \int_a^b (fg)' = \int_{a}^{b} (f'g + g'f)$
\end{proof}

\begin{Rem}
	$\displaystyle \int_{a}^{b} f \,dg = fg \Big|_a^b - \int_{a}^{b} g \,df$\\
	$dg(x) = g'(x)\,dx$
\end{Rem}

\begin{Thm}[Замена переменной в определенном интеграле]
	$\varphi: [\alpha, \beta] \to [A,B],$ дифференцируема на $[\alpha, \beta], \varphi' \in R[\alpha, \beta]$\\
	$f \in C[A; B].$ Тогда
	\[\int_{\alpha}^{\beta} f(\varphi) \cdot \varphi' = \int_{\varphi (\alpha)}^{\varphi (\beta)} f\]
\end{Thm} 

\begin{proof}
	$f(\varphi) \in C[\alpha, \beta] \Rightarrow f(\varphi) \in R[a,b] \Rightarrow f(\varphi) \cdot \varphi' \in R[a,b]$\\
	Пусть F - первообразная $f$ на $[A,B] \Rightarrow F(\varphi)$ -- первообразная $f(\varphi) \cdot \varphi'$ на $[\alpha, \beta]$\\
	$\displaystyle \int_{\alpha}^{\beta} f(\varphi) \cdot \varphi' = F(\varphi) \Big|_{\alpha}^{\beta} = 
	F(\varphi(\beta)) - F(\varphi(\alpha))\\
	\int_{\varphi(\alpha)}^{\varphi(\beta)} f = F \Big|^{\varphi(\beta)}_{\varphi(\alpha)} = F(\varphi(\beta))
	- F(\varphi(\alpha))$
\end{proof}

\begin{Ex}
	Пусть $f$ четная функция. Доказать, что $\displaystyle \int_{-a}^{a} = 2 \int_{0}^{a} f$\\
	Пусть $f$ нечетная функция. Доказать, что $\displaystyle \int_{-a}^{a} f = 0$
\end{Ex}

\begin{Thm}[Формула Тейлора с остатком в интегральной форме] 
	$n \in \N_0,\\ f \in C^{n+1} \langle A; B \rangle, a, x \in \langle A; B \rangle .$ Тогда $\displaystyle f(x) = \sum_{k=0}^{n} \frac{f^{(k)}(a)}{k!}
	(x-a)^k + \frac{1}{n!} \int_{a}^{x} f^{(n+1)}(t)(x-t)^n \,dt$
\end{Thm} 

\begin{proof}
	По индукции:\\
	База: $\displaystyle n=0: f(x) = f(a) + \int_{a}^{x} f'(t) \,dt$ (Формула Ньютона-Лейбница)\\
	Пусть верно для $n-1$. Докажем для $n$.\\
	$\displaystyle f(x) = \sum_{k=0}^{n-1} \frac{f^{(k)}(a)}{k!} (x-a)^k + \frac{1}{(n-1)!} \int_{a}^{x} 
	f^{(n)}(t) (x-t)^{n-1}\,dt$. Проинтегрируем остаток по частям: $\displaystyle u = f^{(n)}(t), u' = f^{(n+1)}(t), 
	v' = (x-t)^{n-1}, v = \frac{(x-t)^n}{n}$\\

	\begin{align*}
		&\displaystyle \sum_{k=0}^{n-1} \frac{f^{(k)}(a)}{k!} (x-a)^k + \frac{1}{(n-1)!} \int_{a}^{x} f^{(n)}(t) (x-t)^{n-1}\,dt = \\
		= &\sum_{k=0}^{n-1} \frac{f^{(k)}(a)}{k!}(x-a)^k 
		+ \frac{1}{(n-1)!} \left(-f^{(n)}(t) \cdot \left. \frac{(x-t)^n}{n} \right|^x_{t=a} + 
		\int_{a}^{x} \frac{f^{n+1}(t)(x-t)^n}{n}\,dt\right) = \\
		= &\sum_{k=0}^{n} \frac{f^{(k)}(a)}{k!} (x-a)^k + \frac{1}{n!} \int_{a}^{x} f^{(n+1)}(t) (x-t)^n \,dt
	\end{align*}
\end{proof}

\begin{Rem}
	$\displaystyle \exists c: \in (a,x) \int_{a}^{x} f^{(n+1)} (t) (x-t)^n \,dt = f^{(n+1)}(c) \int_{a}^{x} (x-t)^m \,dt = 
	f^{(n+1)}(c) \frac{(x-t)^{n+1}}{n+1}$ (Т.е. остаток в форме Лагранжа 
	следует отсюда)
\end{Rem}

Последовательность $\{x_n\}: x_i \in \Q, x_n \to \pi$

\begin{Lm}
	$\displaystyle m \in \N_0\\ \int_{0}^{\frac{\pi}{2}} \sin^m xdx = \int_{0}^{\frac{\pi}{2}} \sin^{m-1}
	x \cdot \sin x \,dx = - \sin^{m-1} \cdot \cos x \Big|_0^{\frac{\pi}{2}} + (m-1)
	\int_{0}^{\frac{\pi}{2}} \sin^{m-2} x \cdot \cos^2x \,dx = \\ = (m-1) \int_{0}^{\frac{\pi}{2}} 
	\sin^{n-2} x (1 - \sin^2 x)\,dx\\
	I_m = (m-1) \cdot (I_{m-2} - I_m) \Rightarrow I_m = \frac{m-1}{m} I_{m-2}\\
	I_0 = \int_{0}^{\frac{\pi}{2}} \sin^0 x \,dx = \frac{\pi}{2}\\
	I_1 = \int_{0}^{\frac{\pi}{2}} \sin x \,dx = - \cos x \Big|^{\frac{\pi}{2}}_0 = 1\\$
	$I_m =
	\begin{cases}
		\frac{(m-1)!!}{m!!} \cdot \frac{\pi}{2}, m - \text{ четно}\\
		I_m = \frac{(m-1)!!}{m!!} \cdot 1, m - \text { нечётно}
	\end{cases}$ 
\end{Lm}

\begin{Ex}
	$f: [-1; 1] \to \R$ - непрерывна. \\Доказать, что $\displaystyle \int_{0}^{\frac{\pi}{2}} f
	(\sin x) \,dx = \int_{0}^{\frac{\pi}{2}} f(\cos x) \,dx$
\end{Ex}

\begin{Thm}[Формула Валлиса]
	$\displaystyle \pi = \lim_{n \to \infty} \frac{1}{n} \Bigg(\frac{(2n)!!}{(2n-1)!!} \Bigg)^2$
\end{Thm} 

\begin{proof}
	$\displaystyle \forall x \in (0; \frac{\pi}{2}) \ \ \ \sin x(0;1)\\
	\forall n \in \N \ \ \ \sin^{2n+1} < \sin^{2n} x < \sin^{2n-1} x \Rightarrow 
	\int_{0}^{\frac{\pi}{2}} \sin^{2n+1} x \,dx < \int_{0}^{\frac{\pi}{2}} 
	\sin^{2n} x \,dx < \int_{0}^{\frac{\pi}{2}} \sin ^{2n-1} x \,dx$\\
	$\displaystyle \frac{(2n)!!}{(2n+1)!!}  < \frac{\pi}{2} \cdot \frac{(2n-1)!!}{(2n)!!} < \frac{(2n-2)!!}{(2n-1)!!}\\
	< \frac{\pi}{2} < \frac{(2n-2)!! \cdot (2n)!!)}{((2n-1)!!)^2}\\
	\frac{1}{2n+1}\cdot \Bigg(\frac{(2n)!!}{(2n+1)!!} \Bigg)^2 < \frac{\pi}{2} < \frac{1}{2n} \Bigg(\frac{(2n)!!}{(2n-1)!!} \Bigg)^2$

	$\displaystyle x_n = \frac{1}{n} \Bigg(\frac{(2n)!!}{(2n-1)!!}\Bigg)^2 \Rightarrow \pi < x_n < \frac{2n+1}{2n} \pi, \Rightarrow \ x_n \to \pi$
\end{proof}

\begin{Thm}[Вторая теорема о среднем для интегралов, Бонне] 
	$f \in C[a,b], \\ g \in C^1[a,b], g$ монотонна на $[a,b].$ Тогда $\exists c \in [a,b]:$
	\[\int_{a}^{b} fg = g(a) \int_{a}^{c} f + g(b) \int_{c}^{b} f\]
\end{Thm} 

\begin{proof}
	$\displaystyle F(x) = \int_{a}^{x} f, \ \ F' = f, F(a) = 0$
	\begin{gather*}
		\int_{a}^{b} fg = Fg \Big|^b_a - \int_{a}^{b} - \int_{a}^{b} Fg' = g(b) \int_{a}^{b} f - \int_{a}^{b} Fg' = \\
		= g(b) \int_{a}^{b} f - \int_{a}^{c} f \cdot (g(b) - g(a)) = g(a) \int_{a}^{c} f + g(b) \int_{c}^{b} f
	\end{gather*}
\end{proof}

\begin{Ex}
	Оценить $\int_{100\pi}^{200\pi} \frac{\sin x}{x} \,dx$\\
	\begin{enumerate}
		\item По первой теореме о среднем.
		\item По второй теореме о среднем.
	\end{enumerate}
	
\end{Ex}

\Subsection{Интегральные неравенства}

\def\AuthorName{Илья Дудников}

\begin{Thm}[Неравенство Йенсена]
	$f$ -- выпукла и непрерывна на $\langle A, B\rangle$, $\PHI : [a, b] \to \langle A, B\rangle$ -- непрерывна,
	$\lambda : [a, b] \to [0, +\infty)$ -- непрерывна, $\int_a^b \lambda = 1$. Тогда
	\[f\left( \int_a^b \lambda \PHI\right) \leqslant \int_a^b \lambda \cdot f(\PHI)\]
\end{Thm}

\begin{Ex}
	Доказать.
\end{Ex}

\begin{Rem}
	Строкое неравенство, если $f$ строго выпукла и $\PHI \not\equiv \const$. 
\end{Rem}

\begin{Thm}[Неравенство Гельдера]
	$p, q > 1, \frac{1}{p} + \frac{1}{q} = 1, f, g \in C[a, b]$. Тогда
	\[\left|\int_a^b fg\right| \leqslant \left(\int_a^b |f|^p\right)^{\frac{1}{p}} \cdot \left(\int_a^b |g|^q\right)^{\frac{1}{q}}\]
\end{Thm}

\begin{proof}
	Пусть $x_k = \frac{k(b - a)}{n} + a, \xi_k = x_k$.
	Обозначим $a_k = f(x_k) (\Delta x_k)^{\frac{1}{p}}, b_k = g(x_k) (\Delta x_k)^{\frac{1}{q}} \SO a_k b_k = f(x_k) g(x_k) \Delta x_k$. Тогда
	\[\left| \sum_{k=0}^{n - 1} a_k b_k \right| \leqslant \left(\sum_{k=0}^{n - 1} |a_k|^p\right)^{\frac{1}{p}} \cdot \left(\sum_{k = 0}^{n - 1}|b_k|^q\right)^{\frac{1}{q}}\]
	\[\left| \sum_{k=0}^{n - 1} f(x_k)g(x_k) \Delta x_k \right| \leqslant \left(\sum_{k = 0}^{n - 1} |f(x_k)|^p \Delta x_k\right)^{\frac{1}{p}} \cdot \left(\sum_{k = 0}^{n - 1}|g(x_k)|^q\right)^{\frac{1}{q}}\]
	Выполним предельный переход:
	\[\left|\int_a^b fg\right| \leqslant \left(\int_a^b |f|^p\right)\frac{1}{p} \cdot \left(\int_a^b |g|^q\right)\frac{1}{q}\] 	
\end{proof}

\begin{Cons}[Неравенство Коши-Буняковского]
	$f, g \in C[a, b] \SO \left|\int_a^b fg\right| \leqslant \sqrt{\int_a^b f^2} \cdot \sqrt{\int_a^b g^2}$ 
\end{Cons}

\begin{Thm}[Неравенство Минковского]
	$f, g \in C[a, b], p \geqslant 1$.
	\[\left(\int_a^b |f + g|^p\right)^{\frac{1}{p}} \leqslant \left(\int_a^b |f|^p\right)^{\frac{1}{p}} + \left(\int_a^b |g|^q\right)^{\frac{1}{q}}\] 
\end{Thm}

\begin{Def}
	Пусть $f \in C[a, b]$.
	\begin{MyList}
		\item Величина
		\[\frac{1}{b - a}\int_a^b f\]
		называется интегральным средним арифметическим функции $f$ на $[a, b]$.

		\item Если $f > 0$, то величина
		\[\exp \left(\frac{1}{b - a}\int_a^b f\right)\]
		называется интегральным средним геометрическим функции $f$ на $[a, b]$.
	\end{MyList}
\end{Def}

\begin{Rem}
	Интегральное среднее геометрическое есть пределы при $n \to \infty$ последовательности
	\[\sqrt[n]{\prod_{k = 0}^{n - 1} f(x_k)} = \exp \left(\frac{1}{n} \sum_{k = 0}^{n - 1} \ln f(x_k)\right) = \exp \left(\frac{1}{b - a} \sum_{k = 0}^{n - 1} \ln f(x_k) \Delta x_k\right)\]
	при $x_k = a + \frac{k(b - a)}{n}$.  
\end{Rem}

\begin{Thm}[Об интегральных средних]
	$f \in C[a, b], f > 0$. Тогда 
	\[\exp\left(\frac{1}{b - a}\int_a^b \ln f\right) \leqslant \frac{1}{b - a}\int_a^b f\]
\end{Thm}

\begin{proof}
	Предельный переход в неравенстве для сумм, либо применить неравенство Йенсена для $\ln x$.
\end{proof}

\Subsection{Несобственные интегралы}

\begin{Def}
	$f$ локально интегрируема (по Риману) на промежутке $E$, если она интегрируема на каждом отрезке из $E$.
\end{Def}

\begin{Rem}
	Непрерывность влечет локальную интегрируемость.
\end{Rem}

\begin{Def}
	Пусть $-\infty < a < b \leqslant +\infty, f \in R_{loc}[a, b]$. Тогда
	$\int_a^{\to b}f$ -- несобственный интеграл. 
	\[\lim_{t \to b-} \int_a^t f = \int_a^{\to b}\]
	если предел существует в $\overline{\R}$. 
\end{Def}

\begin{Def}
	Несобственный интеграл называется сходящимся, если из $\R$. 
\end{Def}

\begin{Def}
	Аналогично, для $-\infty \leqslant a < b < +\infty, f \in R_{loc}(a, b]$ 
	\[\int_{\to a}^b f = \lim_{t \to a+} \int_t^b f\]
	если предел существует в $\overline{\R}$.  
\end{Def}

\begin{Thm}[Критерий Больцано-Коши сходимости интегралов]
	Пусть $-\infty < a < b \leqslant +\infty, f \in R_{loc}[a, b)$. Тогда сходимость интеграла $\int_a^b f$ равносильна условию
	\[\forall \varepsilon > 0 \ \exists \Delta \in (a, b) : \forall t_1, t_2 \in (\Delta, b) \ \left|\int_{t_1}^{t_2} f\right| < \varepsilon\]	
\end{Thm}

\begin{proof}
	$\Phi (t) = \int_a^t f$. $\int_a^b$ сходится $\EQ \exists$ конечный $\lim_{t \to b-} \Phi (t)$.
	Согласно критерию Больцано-Коши существования предела функции
	\[\exists \varepsilon > 0 \ \exists \Delta \in (a, b) : \forall t_1, t_2 \in (\Delta, b) \ |\Phi(t_2) - \Phi(t_1)| < \varepsilon\]
	и по аддитивности интеграла $\Phi(t_2) - \Phi(t_1) = \int_{t_1}^{t_2} f$.
\end{proof}

\begin{Rem}
	Расходимость $\int_a^b f \EQ \exists \varepsilon > 0 \ \forall \Delta \in (a, b) \ \exists t_1, t_2 \in (\Delta, b) \ \left| \int_{t_1}^{t_2} f\right| \geqslant \varepsilon$
\end{Rem}

\begin{Rem}
	Запись:
	\[\int_a^b f = \lim_{t \to b-}\int_a^t f = \lim_{t \to b-} (F(t) - F(a)) = F(b-) - F(a)\]
\end{Rem}

\begin{Example}
	$\int_1^{+\infty} \frac{1}{x^{\alpha}} \, dx$
	\[\int_1^{+\infty} \frac{1}{x^\alpha}\,dx = \begin{cases}
		\left.\frac{x^{1 - \alpha}}{1 - \alpha}\right|_{1}^{+\infty}, \alpha \neq 1 \\
		\left.\ln x\right|_1^{+\infty}, \alpha = 1
	\end{cases} = \begin{cases}
		\frac{1}{\alpha - 1}, \alpha > 1 \\
		+\infty, \alpha \leqslant 1
	\end{cases}\] 
\end{Example}

\begin{Example}
	$\int_0^1 \frac{1}{x^\alpha} \,dx = \begin{cases}
		+\infty, \alpha \geqslant 1 \\
		\frac{1}{1 - \alpha}, \alpha < 1.
	\end{cases}$ 
\end{Example}

\Subsubsection{Свойства несобственного интеграла}

Будем считать, что $f$ локально интегрируема на рассматриваемых промежутках.

\begin{MyList}
	\item \textbf{Аддитивность по промежутку.}  Если $\int_a^b f$ сходится, то $\forall c \in (a, b)$ интеграл $\int_c^b$ тоже сходится и
	\[\int_a^b = \int_a^c f + \int_c^b f\]	
	В обратную сторону, если при $c \in (a, b)$ интеграл $\int_c^b f$ сходится, то сходится и интеграл $\int_a^b f$.
	\begin{proof}
		$\forall t \in (a, b) \ \int_a^t f = \int_a^c f + \int_c^t f$ -- по аддитивности определенного интеграла.
		Переидем к пределу при $t \to b-$ предел левой части и правой части существует или не существует одновременно.
	\end{proof}

	\item Если $\int_a^b f$ сходится, то $\underbrace{\int_t^b f \xrightarrow[t \to b-]{} 0}_{\text{остаток интеграла}}$.
	\begin{proof}
		\[\int_t^b f = \int_a^b f - \int_a^t \xrightarrow[t \to b-]{}\int_a^b f - \int_a^b f = 0\]
	\end{proof}

	\item \textbf{Линейность несобственного интеграла.}  Если интегралы $\int_a^b f, \int_a^b g$ сходятся, $\alpha, \beta \in \R$, то интеграл $\int_a^b(\alpha f + \beta g)$ сходится и
	\[\int_a^b (\alpha f + \beta g) = \alpha\int_a^b f + \beta\int_a^b g\]
	\begin{proof}
		Для доказательства надо перейти к пределу в равенстве для частичных интегралов
		\[\int_a^t (\alpha f + \beta g) = \alpha \int_a^b f + \beta \int_a^t g\]
	\end{proof}

	\begin{Rem}
		Если интеграл $\int_a^b f$ расходится, а интеграл $\int_a^b g$ сходится, то интеграл $\int_a^b(f + g)$ расходится.
		Действительно, если $f + g$ сходится, то сходится и интеграл от $f = (f + g) - f$ (?!).
	\end{Rem}

	\item \textbf{Монотонность несобственного интеграла.} Если интегралы $\int_a^b f, \int_a^b g$ существуют в $\overline{R}, f \leqslant g$ на $[a, b)$, то
	\[\int_a^b f \leqslant \int_a^b g\]

	\begin{proof}
		Переидем к пределу в неравенстве для частичных пределов
		\[\int_a^t f \leqslant \int_a^t g\]	
	\end{proof}

	\begin{Rem}
		Аналогично, с помощью предельного перехода, на несобственные интегралы переносятся неравенства Йенсена, Гельдера, Минковского.
	\end{Rem}

	\item \textbf{Интегрирование по частям в несобственном интеграле.} Пусть $f, g$ дифференцируемы на $[a, b), f', g' \in R_{loc}[a, b)$.
	Тогда
	\[\int_a^b fg' = \left.fg\right|_a^b - \int_a^b f'g\]
	Если два из этих трез пределов конечны, то третий предел также существует и конечен.

	\begin{proof}
		Устремим $t$ к $b$ слева в равенстве
		\[\int_a^t fg' = \left.fg\right|_a^t - \int_a^t f'g\]
	\end{proof}

	\item \textbf{Замена переменной в несобственном интеграле.} 
	Пусть $\PHI : [\alpha, \beta) \to [A, B)$ -- дифференцируема на $[\alpha, \beta)$, $\PHI' \in R_{loc}[\alpha, \beta)$, существует 
	$\PHI(\beta -) \in \overline{R}, f \in C[A, B)$. Тогда
	\[\int_\alpha^\beta (f \circ \PHI)\PHI' = \int_{\PHI(\alpha)}^{\PHI(\beta-)} f\]
	Опять же, если существует один из интегралов, то существует и другой.    

	\begin{proof}
		Обозначим
		\[\Phi(t) = \int_\alpha^t (f \circ \PHI) \PHI', \quad \Psi(y) = \int_{\PHI(\alpha)}^y f\]
		По формуле замены переменной в собственном интеграле 
		\[\Phi(t) = \Psi(\PHI(t))\]

		\begin{MyList}
			\item Пусть $\exists \int_{\PHI(\alpha)}^{\PHI(\beta)} f = I \in \overline{R}$. Докажем, что $\exists \int_\alpha^\beta f(\PHI)\PHI' = I$, т.е.
			$\Phi(t) \xrightarrow[t \to \beta-]{} I$. Возьмем $\{t_n\} : t_n \to \beta, t_n < \beta$. 
			Тогда $\PHI(t_n) \to \PHI(b-), \PHI(t_n) \in [A, B)$. Поэтому $\Phi(t_n) = \Psi(\PHI(t_n)) \to I$. В силу произвольности выбора $\{t_n\}$, $\Phi(t) \to I$ при $t \to \beta-$.
			
			\item Пусть существует интеграл $\int_\alpha^\beta (f \circ \PHI) \PHI' = J \in \overline{R}$. 
			Докажем, что интеграл $\int_{\PHI(\alpha)}^{\PHI(\beta-)} f$ существует, и тогда по пункту 1 будет следовать, что он равен $J$. 
			Если $\PHI(\beta -) \in [A, B)$, то интеграл собственный. 
			Пусть $\PHI(\beta-) = B$. Возьмем $\{y_n\}, y_n \in [A, B), y_n \to B$. Не уменьшая общности, можно считать, что $y_n \in [\PHI(\alpha), B)$.
			Тогда $\exists \gamma_n \in [\alpha, \beta) : \PHI(\gamma_n) = y_n$ (по теореме Больцано-Коши).

			Докажем, что $\gamma_n \to \beta$. Пусть $\beta' \in [\alpha, \beta)$. Т.к. $\max_{[\alpha, \beta']} \PHI < \beta$, а
			$\PHI(\gamma_n) \to B$, то, начиная с некоторого номера, $\gamma_n \in (\beta', \beta)$. Поэтому $\gamma_n \to \beta$, откуда $\Psi(y_n) = \Phi(\gamma_n) \to J$.     
		\end{MyList}
	\end{proof}

	\begin{Example}
		$\int_0^\pi \frac{dx}{2 + \cos x}$. Пусть $t = \tg \frac{x}{2}$. Тогда $x = 2 \arctg t, \cos x \frac{1 - t^2}{1 + t^2}, dx = \frac{2}{1 + t^2}\,dt$. 
		Если $x = 0$, то $t = 0$. Если $x = \pi$, то $t = +\infty$. Тогда
		\begin{gather*}
		\int_0^\pi \frac{dx}{2 + \cos x} = \int_0^{+\infty} \frac{1}{2 + \frac{1 - t^2}{1 + t^2}} \cdot \frac{2}{1 + t^2}\,dt = \int_0^{+\infty} \frac{2dt}{(1 + t^2) \cdot 2 + 1 - t^2} = 2 \int_0^{+\infty} \frac{dt}{t^2 + 3} = \\
		=  \left.2 \cdot \frac{1}{\sqrt{3}} \arctg{\frac{t}{\sqrt{3}}}\right|_0^{+\infty} = \frac{2}{\sqrt{3}}\left(\frac{\pi}{2} - 0\right) = \frac{\pi}{\sqrt{3}}
		\end{gather*}
	\end{Example}

	\begin{Rem}
		$a < b \in \R$. Пусть $x = b - \frac{1}{t}$.
		\[\int_a^b f(x) \,dx = \int_{\frac{1}{b - a}}^{+\infty} f\left(b - \frac{1}{t}\right)\cdot \frac{1}{t^2}\,dt\]
	\end{Rem}

	\begin{Example}
		\[\int_1^{+\infty} \cos x \,dx = \left.\sin x\right|_1^{+\infty} = \lim_{x \to +\infty} \sin x - \sin 1 \textrm{ -- не существует}\]
	\end{Example}
\end{MyList}


\Subsubsection{Признаки сходимости несобственных интегралов}

\begin{Lm}
	$f \in R_{loc} [a, b), f \geqslant 0$. Тогда $\int_a^b f$ сходится $\EQ$ $F(t) = \int_a^t f$ на $[a, b)$ ограничена сверху. 
\end{Lm}

\begin{proof}
	$F(t)$ возрастает на $[a, b)$ $\left(t_1, t_2 \ F(t_2) - F(t_1) = \int_{t_1}^{t_2} f \geqslant 0\right)$.
	$\displaystyle{\exists \lim_{t \to b-} F(t) \in \R \EQ F}$ возрастает и $F$ ограничена сверху. 
\end{proof}

\begin{Rem}
	Если $f \geqslant 0$, то $\int_a^b f \in \overline{R}$.
\end{Rem}

\begin{Thm}[Признак сравнения]
	$f, g \in R_{loc} [a, b), f, g \geqslant 0$
	\[f(x) = O(g(x)) \qquad \textrm{при } x \to b-\]
	Тогда 
	\begin{MyList}
		\item Если $\int_a^b g$ сходится, то $\int_a^b f$ сходится.
		\item Если $\int_a^b f$ расходится, то $\int_a^b g$ расходится.
	\end{MyList}
\end{Thm}

\begin{proof}
	\begin{MyList}
		\item По определению $O$-большого найдутся такие $\Delta \in (a, b)$ и $K > 0$, что $f(x) \leqslant Kg(x)$ при всех $x \in [\Delta, b)$. Следовательно,
		\[\int_\Delta^b f \leqslant K \int_\Delta^b g < +\infty\]
		то есть остаток интеграла $\int_a^b f$ сходится, а тогда и сам интеграл $\int_a^b f$ сходится.

		\item Если бы интеграл $\int_a^b g$ сходился, то по пункту 1 сходился бы и интеграл $\int_a^b f$.
	\end{MyList} 
\end{proof}

\def\AuthorName{Ксения Кузьмина}

\begin{Cons}[Признак сравнения в предельной форме]
	$f, g \in R_{loc}[a;b), f \geqslant 0, g > 0$ и $\displaystyle \exists \lim_{x \to b-} \frac{f(x)}{g(x)} = l \in [0;+\infty]$. Тогда
	\begin{enumerate}
		\item Если $l \in [0, + \infty)$ и $\displaystyle \int_{a}^{b} g$ сходится, то $\displaystyle \int_{a}^{b} f$ сходится
		\item Если $l \in (0, + \infty]$ и $\displaystyle \int_{a}^{b} f$ сходится, то $\displaystyle \int_{a}^{b} g$ сходится
		\item Если $l \in (0, + \infty)$, то $\displaystyle \int_{a}^{b} f$ и $\displaystyle \int_{a}^{b} g$ сходятся или расходятся одновременно
	\end{enumerate}
\end{Cons}

\begin{proof}
	\begin{enumerate}
		\item $\displaystyle \frac{f}{g}$ ограничено в $(b - \varepsilon; b) \Rightarrow f(x) = O_b(g(x))$ при $x \to b- \Rightarrow$ по теореме $\displaystyle \int_{a}^{b} f$ сходится
		\item Т.к. $l>0$, то $f > 0$ в $(b - \varepsilon; b).$ Тогда поменяем $f$ и $g$ местами в п.1
		\item Следует из пунктов 1 и 2.
	\end{enumerate}
\end{proof}

\begin{Cons}
	Интегралы от неотрицательных эквивалентных функций сходятся или расходятся одновременно.
\end{Cons}

\begin{Ex}
	$\displaystyle \int_{5}^{+\infty} \frac{dx}{x^{\alpha}ln^7x} $
\end{Ex}

\begin{Example}
	Докажем, что $\displaystyle f \geqslant 0, \int_{a}^{+\infty} f$ сходится $\cancel{\Rightarrow} f(x) \underset{x \to +\infty}{\to} 0$
\end{Example}

\begin{proof}
	$\displaystyle E = \bigcup_{k=1}^{+\infty} \left(k - \frac{1}{k^2(k+1)}; k + \frac{1}{k^2(k+1)} \right)$\\

	$f(x) = \begin{cases}
		0, x \in \R \backslash E\\
		k, x = k\\
		\text{линейно и непрерывном соединим точки}, x \in E
	\end{cases}$.

	$\displaystyle \int_{0}^{+ \infty} f = \lim_{b \to \infty} \int_{a}^{b} f\\
	\int_{a}^{b} f(x)dx = \sum_{k=1}^{N} \int_{k - \frac{1}{k^2(k+1)}}^{k+ \frac{1}{k^2(k+1)}} f(x) dx =
	\sum_{k=1}^{N} \frac{1}{2}k \cdot \frac{2}{k^2(k+1)} = \sum_{k=1}^{N} \frac{1}{k(k+1)} = 
	\sum_{k=1}^{N} \left(\frac{1}{k} - \frac{1}{k+1}\right) =\\= 1 - \frac{1}{N+1} \underset{N \to \infty}{\to}  1$
\end{proof}

\begin{Rem}
	Можно построить пример с $\displaystyle g>0$. $\displaystyle g(x) = f(x) + \frac{1}{x^2}$ 
\end{Rem}

\Subsection{Интегралы от знакопеременных функций}
\begin{Def}
	$\displaystyle -\infty < a < b \leqslant + \infty, f \in R_{loc}[a;b)\\
	\int_{a}^{b} f$ сходится абсолютно, если cходится $\displaystyle \int_{a}^{b} |f|$
\end{Def}

\begin{Rem}
	Если $\displaystyle \int_{a}^{b} f$ и $\displaystyle \int_{a}^{b} g$ сходится абсолютно, то $\displaystyle \int_{a}^{b} (\alpha f + \beta g)$
	сходится абсолютно $\forall \alpha, \beta \in \R$
\end{Rem}

\begin{proof}
	$|\alpha f + \beta g| \leqslant |\alpha| \cdot |f| + |\beta| \cdot |g|$ + признак сравнения для неотрицательных функций.
\end{proof}

\begin{Rem}
	Если $\displaystyle \int_{a}^{b} f \in \overline{\R},$ то $\displaystyle \left| \int_{a}^{b} f \right| \leqslant \int_{a}^{b}|f|$
\end{Rem}

\begin{Lm}
	Если интеграл сходится абсолютно, то он сходится.
\end{Lm}

\begin{proof}
	$\displaystyle \int_{a}^{b} |f|$ сходится $\displaystyle \Rightarrow \forall \varepsilon > 0 \exists \Delta \in (a;b) \int_{\Delta}^{b} |f| < \varepsilon$\\
	Тогда $\displaystyle \left| \int_{\Delta}^{b} f \right| < \int_{\Delta}^{b} |f| < \varepsilon \Rightarrow 
	\int_{a}^{b} f = \int_{a}^{\Delta} f + \int_{\Delta}^{b} f$ сходится по критерию Больцано-Коши.
\end{proof}

\begin{Def} 
	$$x_+= \max\{x,0\} = \begin{cases} x, x \geqslant 0\\ 0, x<0 \end{cases} - \textrm{ положительная часть }x$$
	$$x_-= \max\{-x,0\} = \begin{cases} 0, x > 0\\ -x, x \leqslant 0 \end{cases} - \textrm{ отрицательная часть } x$$
	$$x_+ - x_- = x \displaystyle \Rightarrow x_+ = \frac{|x|+x}{2}$$
	$$x_+ + x_- = |x| \Rightarrow x_- = \frac{|x|-x}{2}$$
	$0 \leqslant x_{\pm} \leqslant |x|, f_+ = \max \{f;0\}, f_- = \max \{-f;0\}$
\end{Def} 

\begin{proof}[Второе доказательство леммы]
	$\displaystyle \int_{a}^{b} |f|$ сходится $\displaystyle \underset{0 \leqslant f_{\pm} \leqslant|f|}{\Rightarrow} \int_{a}^{b} f_+$ и $\displaystyle \int_{a}^{b} f_-$ -- сходятся $\Rightarrow$ \\ 
	$\displaystyle \underset{f = f_+ - f_-}{\Rightarrow} \int_{a}^{b} f$ сходится 
\end{proof}

\begin{Rem}
	Обратное утверждение к лемме неверно:
	$\displaystyle \int_{a}^{b} f$ сходится $\displaystyle \cancel{\Rightarrow} \int_{a}^{b} |f|$ сходится.
\end{Rem}

\begin{Def} 
	Если $\displaystyle \int_{a}^{b} f$ сходится, а $\displaystyle \int_{a}^{b} |f|$ расходится, то 
	$\displaystyle \int_{a}^{b} f$ называют условно сходящимся.
\end{Def} 

\begin{Rem}
	$\displaystyle \int_{a}^{b} f$ сходится абсолютно, $\displaystyle \int_{a}^{b} g$ сходится условно $\displaystyle \Rightarrow \int_{a}^{b} 
	(f+g)$ сходится условно, т.к. $g = (f+g)-f$.
\end{Rem}

\begin{Thm}[Признаки Абеля и Дирихле сходимости несобственных интегралов]
	$f \in C[a;b), g \in C^1[a;b], g$ монотонна.

	\begin{MyList}
		\item \textbf{Признак Дирихле.}
		Если функция $F(t) = \int_a^t f$ ограничена, а $g(x) \xrightarrow[x \to b-]{} 0$, то интеграл $\int_a^b fg$ сходится.

		\item \textbf{Признак Абеля.}
		Если интеграл $\int_a^b f$ сходится, а $g$ ограничена, то интеграл $\int_a^b fg$ сходится.
	\end{MyList}
\end{Thm} 

\begin{proof}
	\begin{MyList}
	\item Проинтегрируем по частям:
	$$\displaystyle \int_{a}^{b} fg = \int_{a}^{b} F'g = Fg \Big|_a^b - \int_{a}^{b} Fg'$$
	Двойная подстановка обнуляется, поэтому сходимость исходного интеграла равносильна сходимости интеграла $\int_a^b Fg'$. Докажем, что $\int_{a}^{b} Fg'$ сходится абсолютно. 
	\[ \int_{a}^{b} |Fg'| \leqslant \underset{|F| \leqslant K}{K}  \int_{a}^{b} |g'| = K \left|\int_{a}^{b} g'\right| = 
	K \cdot |g \Big|^b_a| = K|g(a)|\]

	\item $g$ ограничена и монотонна $\Rightarrow \alpha = \lim_{x \to b-} g(x)$\\
	Функция $g - \alpha$ монотонна, $\xrightarrow[x \to b-]{} 0 \Rightarrow 
	\int_{a}^{b} f(g - \alpha)$ сходится по признаку Дирихле. Поэтому интегра $\int_a^b f(g - \alpha)$ сходится, а интеграл $\int_a^b fg$ сходится как сумма двух сходящихся: 
	$$\displaystyle \int_{a}^{b} fg = \int_{a}^{b} f(g-\alpha) + \int_{a}^{b} f \cdot \alpha$$
	\end{MyList}
\end{proof}

\begin{Rem}
	Можно ослабить условия: $f \in R_{loc}[a;b), g$ монотонна на $[a;b)$
\end{Rem}

\begin{Def} 
	v.p. $\displaystyle \int_{a}^{b} f = \lim_{\varepsilon \to 0} \left(\int_{a}^{c-\varepsilon} f + \int_{c+\varepsilon}^{b} f  \right)$ -- главное значение.\\
\end{Def} 

\begin{Example}
	\begin{align*}
		\displaystyle &\int_{-1}^{1} \frac{dx}{x} = 0 \\
		\displaystyle &\int_{-\infty}^{\infty} x dx = 0 \\
		\displaystyle &\int_{-\infty}^{\infty} x^2 dx = +\infty
	\end{align*}
\end{Example}

\begin{Example}
	\begin{MyList}
		\item $\displaystyle \int_{1}^{+\infty} f(x) \cdot \sin x \, dx, f(x) \geqslant 0$.\\
		\begin{MyItemize}
			\item Если $\displaystyle \int_{1}^{+\infty} f$ сходится, то $\displaystyle \int_{1}^{+\infty} f(x) \sin x dx$ сходится абсолютно.\\
			$0 \leqslant |f(x) \cdot \sin x| \leqslant |f(x)| = f(x)$\\
			\item Если $\displaystyle \int_{1}^{+\infty} f$ расходится\\
			$\displaystyle l = \lim_{x \to + \infty} f(x)$

			\begin{MyList}
				\item $l = 0$ и $f$ монотонна, то признак Дирихле и $\displaystyle \int_{1}^{+\infty} f(x) \sin x dx$ -- сходится.\\
				Но: $\displaystyle \int_{1}^{+\infty} |f(x) \sin x| dx$ не сходится.\\
				$\displaystyle |\sin x| \geqslant \sin^2 x = \frac{1 - \cos 2x}{2}$\\
				$\displaystyle \int_{1}^{\infty} f(x) |\sin x| dx \geqslant \underbrace{\int_{1}^{+\infty} \frac{1}{2}f(x)dx}_{\text{расходится}} - 
				\underbrace{\int_{1}^{+\infty} \frac{1}{2} f(x) \cos 2x dx}_{\text{сходится}}$
				\item $\displaystyle l > 0 \Rightarrow \int_{1}^{+ \infty} f \sin x dx$ расходится.\\
				$\displaystyle \int_{a_k}^{b_k} f(x) \cdot \sin x dx \geqslant \frac{1}{2} \int_{a_k}^{b_k}
				f(x)dx \geqslant \frac{1}{2} \cdot \frac{2 \pi}{3} \cdot \min\{f(a_k), f(b_k)\} 
				\underset{k \to \infty}{\to} \frac{\pi}{3} \cdot l = \varepsilon > 0$
			\end{MyList}
		\end{MyItemize}
		
		\item $\displaystyle \int_{1}^{+\infty} \frac{\sin x}{x} dx$ сходится условно.\\
		$\displaystyle \int_{1}^{+\infty} \frac{\sin x}{x^2} dx$ сходится абсолютно по признаку сравнения.\\ 
		$\displaystyle \int_{1}^{+\infty} \frac{\sin x}{\sqrt{x}} dx$ сходится условно\\
		$\displaystyle \int_{1}^{+\infty} \sqrt{x} \sin x dx$ расходится	
		\item Нельзя пользоваться эквивалентностью в случае знакопеременной функции.\\
		$\displaystyle \int_{1}^{\infty} \frac{\sin x}{\sqrt{x} - \sin x} dx$ -- расходится\\
		$\displaystyle f(x) \sim \frac{\sin x}{\sqrt{x}}$ при $\displaystyle x \to \infty$ 
		$\displaystyle\int_{1}^{\infty} \frac{\sin x}{\sqrt{x}} $ сходится.\\ 
		Выделим главную часть: $\displaystyle \frac{\sin x}{\sqrt{x}} \left( \frac{1}{1 - \frac{\sin x}{\sqrt{x}}}  \right) = 
		\frac{\sin x}{\sqrt{x}} (1 + \frac{\sin x}{\sqrt{x}} + \frac{\sin^2 x}{x} + \frac{\sin^3 x}{x\sqrt{x}} + r(x)) =
		\underset{\text{сх-ся}}{\frac{\sin x}{\sqrt{x}}} + \underset{\text{расходится}}{\frac{\sin^2 x}{x}} +
		\underset{\text{сх-ся абс-но}}{\frac{\sin^3 x}{x\sqrt{x}}} + \underset{\text{сх-ся абс.}}{\frac{\sin^4 x}{x^2}} + |q(x)|, \ \ \ |q(x)| \leqslant \frac{c}{x^2}$\\
		$\displaystyle \left(\frac{1}{1-t} = 1 + t + t^2 + t^3 + r(t), t \to 0\right)$
		\item $\displaystyle \int_{0}^{+\infty} \frac{\sin x}{x^{\alpha}}dx = \int_{0}^{1} + \int_{1}^{+\infty} $\\
		При $x \to 0 \sin x \sim x$  и $\sin x > 0$ на $(0;1)$\\
		$\displaystyle \int_{0}^{1} \frac{dx}{x^{\alpha -1}}$
		\item $\displaystyle \int_{-1}^{1} \frac{dx}{x} = \int_{-1}^{0} \frac{dx}{x} + \int_{0}^{1} \frac{dx}{x}$ расходится.
		Но сходится в смысле главного значения.
	\end{MyList}
\end{Example}

\begin{Rem}
	$\displaystyle \int_{1}^{+\infty} f \cdot g, f$ -- периодична c периодом $T > 0, g$ -- монотонна $\underset{x \to + \infty}{\to} 0$\\
	Тогда 
	\begin{MyList}
		\item Если $\displaystyle \int_{1}^{+\infty} g$ сходится $\displaystyle \Rightarrow \int_{1}^{+\infty} fg$
		\item Если $\displaystyle \int_{1}^{+\infty} g$ расходится, то 
		$\displaystyle \left(\int_{1}^{+\infty} fg \text{ сходится } \Leftrightarrow \int_{1}^{1+T} f = 0 \right)$
	\end{MyList}
\end{Rem}

\begin{proof}
	Упражнение.
\end{proof}

\begin{Cons}
	$\displaystyle \int_{1}^{+\infty} \frac{\sin^2 x}{x} dx$ расходится\\
	$\displaystyle \int_{1}^{+\infty} \frac{\sin^3 x}{x} dx$ сходится
\end{Cons}

\Subsection{Длина, площадь и  объём}\\
\Subsubsection{Площадь}
\begin{Def} 
	$||x||, x \in \R^n$ -- длина вектора. \\
	$$\displaystyle ||A-B|| = \sqrt{\sum_{i=1}^{n} (A_i - B_i)^2}$$
\end{Def} 

\begin{Def} 
	Движение -- отображение $U: \R^n \to \R^n$, сохраняющее расстояния.\\
	$$||A-B|| = ||U(A) - U(B)|| \ \ \ \forall A, B \in \R^n$$
\end{Def} 

\begin{Def} 
	Площадь -- функционал $S: {P} \to [0; +\infty),$ где $\{P\}$ -- множество квадрируемых фигур из $\R^2$
\end{Def} 

\begin{Thm}[Свойства площади]
	\begin{MyList}
		\item Аддитивность: $P_1, P_2$ -- квадрируемы и $P_1 \bigcap P_2 = \varnothing$. Тогда
		$P_1 \bigcup P_2$ -- квадрируемая и \\$S(P_1 \bigcup P_2) = S(P_1) + S(P_2)$
		\item Нормированность на прямоугольниках: площадь прямоугольника со сторонами $a$ и $b$ равна $ab$
		\item Инвариантность относительно движений: $S(U(P)) = S(P)$
		\item Монотонность: $P, P_2$ -- квадрируемые, $P_1 \subset P$, тогда $S(P_1) \leqslant S(P)$
		\begin{proof}
			$P = P_1 \cup (P \backslash P_1), \ P_1 \cap (P \backslash P_1) = \varnothing$. Тогда по аддитивности
			площади: $S(P) = S(P_1) + S(P \backslash P_1) \geqslant S(P_1)$
		\end{proof}
		\item Если $P$ содержится в некотором отрезке, то $S(p) = 0$
		\begin{proof}
			$P$ можно поместить в прямоугольник сколь угодно малой площади. 
		\end{proof}
		\item Усиленная аддитивность: $P_1$ и $P_2$ пересекаются по множеству нулевой площади. Тогда
		$S(P_1 \cup P_2) = S(P_1) + S(P_2)$
		\begin{proof}
			Возьмем $P = P_1 \cap P_2 \Rightarrow S(P_1) = S(P) + S(P_1 \backslash P) = S(P_1 \backslash P)\\
			S(P_1 \cup P_2) = S(P_1 \backslash P)  +S(P_2) = S(P_1) + S(P_2)$
		\end{proof}
 	\end{MyList}
\end{Thm}

\Subsubsection{Объём}
\begin{Def} 
	Объём -- функционал $V: \{T\} \to [0; +\infty)$, где $\{T\}$ -- класс кубируемых тел
\end{Def} 

\begin{Thm}[Свойства объёма] 
	\begin{enumerate}
		\item Аддитивность: $T_1, T_2$ -- кубируемые, $T_1 \cap T_2 = \varnothing$, тогда $T_1 \cup T_2$ --
		кубируемое, $V(T_1 \cup T_2) = V(T_1) + V(T_2)$
		\item Нормированность на прямоугольных параллелепипедах. Объём параллелепипеда:\\ 
		$a \times b \times c = abc$
		\item  Инвариантность относительно движения: $V(U(T)) = V(T)$
		\item Монотонность: $T_1, T$ -- кубируемые, $T_1 \subset T$, тогда $V(T_1) \leqslant V(T)$
		\item Если тело T содержится в некотором прямоугольнике, то его объём равен нулю.
		\item Усиленная аддитивность. $T_1, T_2$ -- кубируемые, $T_1 \cap T_2$ нулевого объёма, тогда\\
		$V(T_1 \cup T_2) = V(T_1) + V(T_2)$
	\end{enumerate}
\end{Thm} 

\begin{Def} 
	$P \subset \R^2, h \geqslant 0$. Множество $Q = P \times [0;h]$ называется прямым цилиндром с
	основанием $P$ и высотой $h$. 
\end{Def} 

\begin{Def} 
	$T \subset \R^3, x \in \R\\ T(x) = \{(y, z) \in \R^2: (x, y, z) \in T\}$ -- сечение
\end{Def} 

\Subsubsection{Длина пути}
\begin{Def} 
	$\gamma: [a; b] \to R^m, \gamma - \text{ непрерывное отображение }\\
	\gamma_i, \ \ i = 1, ..., m$ -- $i$-тая координатная функция.\\
	Если все $\gamma_i$ непрерывны, то отображение $\gamma$ непрерывно. 
\end{Def} 

\begin{Def} 
	Путь в $R^m$ -- $\gamma = (\gamma_1, \gamma_2, ..., \gamma_m): [a,b] \to R^m$\\
	$\gamma(a)$ -- начало пути\\
	$\gamma(b)$ -- конец пути\\
	$\gamma* = \gamma([a,b])$ -- носитель пути. В каком-то смысле можно считать, что это изображение пути.
\end{Def} 

\begin{Example}
	Полуокружность:\\
	$\gamma^1(t) = (t, \sqrt{1-t^2}), t \in [-1, 1]$, пробегаем дугу слева направо.\\
	$\gamma^2(t) = (- \cos t, \sin t), t \in [0, \pi]$\\
	$\gamma^3(t) = (\cos t, \sin t), t \in [0,\pi]$\\
	$\gamma^4(t) = (\cos t, |\sin t|), t \in [-\pi, \pi]$. пробежали дугу туда и обратно.\\
	Все четыре отображения разные, но носитель пути у всех одинаковый. 
\end{Example}

\begin{Def} 
	$\gamma(a) = \gamma(b)$ -- замкнутый путь
\end{Def} 

\begin{Def} 
	Если $\gamma(t_1) = \gamma(t_2)$ только при $t_1 = t_2$ или $t_1, t_2 \in \{a; b\},$ то путь 
	несамопересекающийся (простой)
\end{Def} 

\begin{Def} 
	Если $\gamma_i \in C^r[a;b], i = 1, ..., m$, то путь $\gamma$ гладкости $r$, $r \in \N \cup \{\infty\}$
\end{Def} 

\begin{Def} 
	Если $\gamma^-(t) = \gamma(a+b-t)$ -- противоположный путь.
\end{Def} 

\begin{Ex}
	Посмотреть на кривые Пеано. 
\end{Ex}

\begin{Def} 
	$\gamma: [a; b] \to \R^m$, $\widetilde{\gamma}: [\alpha; \beta] \to \R^m$ -- эквивалентные,
	если существует строго возрастающая функция и $[a;b] \overset{\text{на}}{\to} [\alpha; \beta]:
	\gamma = \widetilde{\gamma} \circ u$.\\ 
	Это отношение эквивалентности:
	\begin{enumerate}
		\item $\gamma \sim \gamma, \ \ u = id [a;b]$
		\item $\gamma \sim \widetilde{\gamma} \Leftrightarrow \gamma \sim \widetilde{\gamma} \ \ \ u^{-1}$ -- обратное отображение
		\item $\gamma_1 \sim \gamma_2, \gamma_2 \sim \gamma_3 \Rightarrow \gamma_1 \sim \gamma_3 \ \ \ u_1 \circ u_2$
	\end{enumerate}
\end{Def} 

\begin{Def} 
	Класс эквивалентных путей -- кривая\\
	Каждый представитель класса -- параметризация кривой\\
	Кривая называется r-гладкой, если у неё найдется гладкая параметризация
\end{Def} 

\Subsubsection{Длина кривой}
\begin{Def} 
	$\gamma \in C([a;b] \to \R^m) -$ путь в $R^m$
	\begin{enumerate}
		\item Длина кривой, соединяющей точки $A$ и $B$ не меньше $||AB||$
		\item Нужна аддитивность: $a < c < b$, $\gamma^1 = \gamma \big|_{[a;c]},
		\gamma^2=\gamma \big|_{[c;b]} \Rightarrow S_{\gamma} = S_{\gamma^1} + S_{\gamma^2}$
	\end{enumerate}
\end{Def} 

\begin{Example}
	$\tau = \{t_0, t_1, t_2, ..., t_n \}$ -- дробление $[a, b]\\
	l_{\tau}$ -- вписанная ломаная. 
\end{Example}

\begin{figure*}[h]
	\centering
	\def\svgwidth{0.4\columnwidth}
	\input{img/path_length.pdf_tex}
\end{figure*}

\begin{Def} 
	$\gamma$ -- путь в $\R^m$. Длиной пути $\gamma$ называется $S_{\gamma} = \underset{\tau}{\sup} \, l_{\tau}$ 
\end{Def} 

\begin{Def} 
	Путь с $S_{\gamma} < + \infty$ -- спрямляемый. 
\end{Def} 

\begin{Lm}
	Длины эквивалентных путей равны.	
\end{Lm}
\begin{proof}
	$\gamma \sim \widetilde{\gamma} \circ u, \ \ u: [a;b] \overset{\text{на}}{\to}{[\alpha; \beta]}$ строго возрастает\\
	$\tau = \{t_k\}^n_{k=1}$ -- дробление $[a;b]$\\
	$\widetilde{t}_k = u(t_k), \widetilde{\tau} = \{\widetilde{t}_k\}$ -- дробление $[\alpha, \beta]$\\
	$\displaystyle l_{\tau} = \sum_{k=0}^{n-1} \underbrace{||\gamma(t_{k+1}) - \gamma(t_k)||}_{\text{{ длина отрезка }}} = 
	\sum_{k=0}^{n-1} || \widetilde{\gamma} (\widetilde{t}_{k+1}) - \widetilde{t}_k|| = l_{\widetilde{\tau}}$\\
	$l_{\tau} = l_{\widetilde{\tau}} \leqslant S_{\widetilde{\gamma}} \Rightarrow S_{\gamma} \leqslant S_{\widetilde{\gamma}}$\\
	Поменяем: $\gamma$ и $\widetilde{\gamma}$ местами $\Rightarrow S_{\widetilde{\gamma}} \leqslant S_{\gamma}$
\end{proof}

\begin{Rem}
	Противоположные пути имеют одинаковую длину. 
\end{Rem}

\begin{Lm}[Аддитивность длины пути]
	$\gamma: [a;b] \to \R, a < c < b$\\
	$\gamma^1 = \gamma \big|_{[a;c]}, \gamma^2 = \gamma \big|_{[c;b]}$\\
	$S_{\gamma} = S_{\gamma^1} + S_{\gamma^2}$
\end{Lm}

\begin{proof}
	Обозначим $S_1 = S_{\gamma^1}, S_2 = S_{\gamma^2}$. Возьмём дробления $\tau_1$ и $\tau_2$ отрезков $[a, c]$ и $[c, b]$; тогда $\tau = \tau_1 \cup \tau_2$ -- дробление $[a, b]$. Потроим по $\tau_1$ и $\tau_2$ ломаные, вписанные в $\gamma^1$ и $\gamma^2$, и обозачим через $l_1$ и $l_2$ их длины.
    Тогда $l_1 + l_2  = l_{\tau} \leqslant s_{\gamma}$. Последовательно переходя в левой части к супремуму по всевозможным дроблениям $\tau_1$ и $\tau_2$, получаем $$s_1 + l_2 \leqslant s_{\gamma},$$ $$ s_1 + s_2 \leqslant s_{\gamma}.$$
    Докажем противоположное неравенство $$s_{\gamma} \leqslant s_1 + s_2.$$
    Возьмём дробление $\tau = \{ t_k \}^n_{k = 0}$ отрезка $[a, b]$ и докажем, что $l_{\tau} \leqslant s_1 + s_2$; отсюда и будет следовать требуемое. Если $c \in \tau$, то $\tau = \tau_1 \cup \tau_2$, где $\tau_1$ и $\tau_2$ -- дробления $[a, c]$ и $[c, b]$. Поэтому $$l_{\tau} = l_1 + l_2 \leqslant s_1 + s_2.$$
    Если $c \notin \tau$, то добавим $c$ в число точек дробления, то есть положим $\tau^* = \tau \cup \{ c \}$. Пусть $c \in (t_{\nu}, t_{\nu + 1})$. По неравенству треугольника        
    $$ l_{\tau} = \sum_{k = 0}^{\nu - 1} |\gamma (t_{k + 1}) - \gamma (t_k) | + |\gamma (t_{\nu + 1}) - \gamma(t_{\nu}) | + \sum_{k = \nu + 1}^{n - 1}|\gamma (t_{k+1}) - \gamma (t_k) | \leqslant $$
    $$ \leqslant \sum_{k = 0}^{\nu = 1} |\gamma (t_{k+1}) -\gamma (t_k)| + |\gamma(c) - \gamma(t_{\nu})| + |\gamma(t_{\nu + 1}) - \gamma(c)| + \sum_{k = \nu + 1}^{n - 1}|\gamma (t_{k + 1}) - \gamma(t_k)| = l_{\tau^*}$$
    По доказанному $$l_{\tau} \leqslant l_{\tau^*} \leqslant s_1 + s_2$$
\end{proof}

\begin{Def} 
	Длина кривой -- длина какой-то из её параметризаций
\end{Def} 

\begin{Example}
	Пример ограниченной, но неспрямляемой кривой: кривая Коха.
	Длины:
	\begin{enumerate}
		\item $\displaystyle n = 1: \ \frac{1}{3} \cdot 4$\\
		\item $\displaystyle n = 2: \ \left(\frac{4}{3}\right)^2$
		\item $\displaystyle n = 3: \ \left(\frac{4}{3}\right)^3$
	\end{enumerate}
\end{Example}

\begin{figure*}[h]
	\centering
	\def\svgwidth{0.4\columnwidth}
	\input{img/Koch_curve.pdf_tex}
  \end{figure*}

\Subsubsection{Приложения интеграла Римана}

\begin{Def} 
	$f: [a;b] \to \R$\\ $Q_f\{(x, y) \in \R^2: x\in [a;b], y \in [0; f(x)]\}$ -- подграфик\\
	Если $f \in C[a;b]$, то $Q_f$ называют криволинейной трапеция\\
\end{Def} 

\begin{Thm} 
	Пусть $f \in R[a;b].$ Тогда $Q_f$ квадрируема 
\end{Thm} 

\begin{proof}
	Без доказательства
\end{proof}

\begin{Rem}
	Суммы Дарбу $s_{\tau}, S_{\tau}\\
	\forall \tau \ \ \ s_{\tau} \leqslant S(Q_f) \leqslant S_{\tau}$\\
	Вспомним, что $\underset{\tau}{\sup} S_{\tau} = \underset{\tau}{\inf} S_{\tau}$\\
	$\Rightarrow S(Q_f) = \displaystyle \int_{a}^{b} f dx$
\end{Rem}

\begin{Rem}
	$\displaystyle S(Q_f) = - \int_{a}^{b} f$
\end{Rem}

\begin{Example}
	Площадь эллипса: $\displaystyle E = \{(x, y): \frac{x^2}{a^2} + \frac{y^2}{b^2} \leqslant 1\}, a, b > 0$\\
	$\displaystyle y = b \sqrt{1 - \frac{x^2}{a^2}}, x\in [0;a]$\\
	$\displaystyle S_E = 4 \int_{0}^{a} b	\sqrt{1 - \frac{x^2}{a^2}} dx = 4b \int_{0}^{\frac{\pi}{2}} a \cos^2 t dt = 4ba \cdot \frac{\pi}{4} = \pi ba$
\end{Example}
\end{document}