\documentclass[12pt]{article}

% Автор: Илья Дудников
% Автор стиля: Сергей Копелиович

\usepackage{cmap}
\usepackage[T2A]{fontenc}
\usepackage[utf8]{inputenc}
\usepackage[russian]{babel}
\usepackage{graphicx}
\usepackage{amsthm,amsmath,amssymb}
\usepackage{listings}
\usepackage{color}
\usepackage{xcolor}
\usepackage{array}
\usepackage{epigraph}
\usepackage{multicol}

\usepackage[russian,colorlinks=true,urlcolor=red,linkcolor=blue]{hyperref}
\usepackage{enumerate}
\usepackage{datetime}
\usepackage{fancyhdr}
\usepackage{lastpage}
\usepackage{verbatim}
\usepackage{tikz}
\usepackage{MnSymbol}
\usetikzlibrary{arrows,decorations.markings,decorations.pathmorphing}
\usepackage{pgfplots}

\usepackage{ifthen}
\usepackage{mathtools}

%\usepackage{tabls}
%\usepackage{tabularx}
%\usepackage{xifthen}
%\listfiles

\def\NAME{Лекции}
\def\SEASON{Конспект лекций по матанализу, ПИ, 1 семестр}

\sloppy
\voffset=-20mm
\textheight=235mm
\hoffset=-22mm
\textwidth=180mm
\headsep=12pt
\footskip=20pt

\parskip=0em
\parindent=0em

\setlength\epigraphwidth{.8\textwidth}

\newlength{\tmplen}
\newlength{\tmpwidth}
\newcounter{listcounter}

% Список с маленькими отступами
\newenvironment{MyList}[1][4pt]{
  \begin{enumerate}[1.]
  \setlength{\parskip}{0pt}
  \setlength{\itemsep}{#1}
}{       
  \end{enumerate}
}
% Вложенный список с маленькими отступами
\newenvironment{InnerMyList}[1][0pt]{
  \vspace*{-0.5em}
  \begin{enumerate}[(a)]
  \setlength{\parskip}{-0pt}
  \setlength{\itemsep}{#1}
}{       
  \end{enumerate}
  \vspace*{-0.5em}
}
% Список с маленькими отступами
\newenvironment{MyItemize}[1][4pt]{
  \begin{itemize}
  \setlength{\parskip}{0pt}
  \setlength{\itemsep}{#1}
}{       
  \end{itemize}
}

% Основные математические символы
\def\TODO{{\color{red}\bf TODO}}
\def\C{\mathbb{C}}       %
\def\Q{\mathbb{Q}}       %
\def\N{\mathbb{N}}       %
\def\R{\mathbb{R}}       %
\def\F2{\mathbb{F}_2}    %
\def\Z{\mathbb{Z}}       %
\def\INF{\t{+}\infty}    % +inf
\def\EPS{\varepsilon}    %
\def\EMPTY{\varnothing}  %
\def\PHI{\varphi}        %
\def\SO{\Rightarrow}     % =>
\def\EQ{\Leftrightarrow} % <=>
\def\t{\texttt}          % mono font
\def\c#1{{\rm\sc{#1}}}   % font for classes NP, SAT, etc
\def\O{\mathcal{O}}      %
\def\NO{\t{\#}}          % #
\def\XOR{\text{ {\raisebox{-2pt}{\ensuremath{\Hat{}}}} }}
\renewcommand{\le}{\leqslant}
\renewcommand{\ge}{\geqslant}
\newcommand{\q}[1]{\langle #1 \rangle}               % <x>
\newcommand\URL[1]{{\footnotesize{\url{#1}}}}        %
% \newcommand{\sfrac}[2]{{\scriptscriptstyle\frac{#1}{#2}}}  % Очень маленькая дробь
% \newcommand{\mfrac}[2]{{\scriptstyle\frac{#1}{#2}}}    % Небольшая дробь
\newcommand{\sfrac}[2]{{\scriptstyle\frac{#1}{#2}}}  % Очень маленькая дробь
\newcommand{\mfrac}[2]{{\textstyle\frac{#1}{#2}}}    % Небольшая дробь

\newcommand{\fix}[1]{{\color{fixcolor}{#1}}} % \underline
\def\bonus{\t{\red{(*)}}}
\def\ifbonus#1{\ifthenelse{\equal{#1}{}}{}{\bonus}}
\def\smallsquare{$\scalebox{0.5}{$\square$}$}

\newlength{\myItemLength}
\setlength{\myItemLength}{0.3em}
\def\ItemSymbol{\smallsquare}
\def\Item{\vspace*{\myItemLength}\ItemSymbol \ \ }

\newcommand{\LET}{%
  % [line width=0.6pt]
  \begin{tikzpicture}%
  \draw(0.8ex,0) -- (0.8ex,1.6ex);%
  \draw(0,1.6ex) -- (0.8ex,1.6ex);%
  \end{tikzpicture}%
  \hspace*{0.1em}%
}

% Отступы
\def\makeparindent{\hspace*{\parindent}\unskip}
\def\up{\vspace*{-0.5em}}%{\vspace*{-\baselineskip}}
\def\down{\vspace*{0.5em}}
\def\LINE{\vspace*{-1em}\noindent \underline{\hbox to 1\textwidth{{ } \hfil{ } \hfil{ } }}}
\def\BOX#1{\mbox{\fbox{\bf{#1}}}}
\def\Pagebreak{\pagebreak\vspace*{-1.5em}}

% Мелкий заголовок
\newcommand{\THEE}[1]{
  \vspace*{0.5em}
  \noindent{\bf \underline{#1}}%\hspace{0.5em}
  \vspace*{0.2em}
}
% Другой тип мелкого заголовка
\newcommand{\THE}[1]{
  \vspace*{0.5em} $\bullet$
  \noindent{\bf #1}%\hspace{0.5em}
  \vspace*{0.2em}
}

\newenvironment{MyTabbing}{
  \t\bgroup
  \vspace*{-\baselineskip}
  \begin{tabbing}
    aaaa\=aaaa\=aaaa\=aaaa\=aaaa\=aaaa\kill
}{
  \end{tabbing}
  \t\egroup
}

% Код с правильными отступами
\lstnewenvironment{code}{
  \lstset{}
%  \vspace*{-0.2em}
}%
{
%  \vspace*{-0.2em}
}
\lstnewenvironment{codep}{
  \lstset{language=python}
}%
{
}

% Формулы с правильными отступами
\newenvironment{smallformula}{
 
  \vspace*{-0.8em}
}{
  \vspace*{-1.2em}
  
}
\newenvironment{formula}{
 
  \vspace*{-0.4em}
}{
  \vspace*{-0.6em}
  
}

% Большая квадратная скобка
\makeatletter
\newenvironment{sqcases}{%
  \matrix@check\sqcases\env@sqcases
}{%
  \endarray\right.%
}
\def\env@sqcases{%
  \let\@ifnextchar\new@ifnextchar
  \left\lbrack
  \def\arraystretch{1.2}%
  \array{@{}l@{\quad}l@{}}%
}
\makeatother

% Определяем основные секции: \begin{Lm}, \begin{Thm}, \begin{Def}, \begin{Rem}
\renewcommand{\qedsymbol}{$\blacksquare$}
\theoremstyle{definition} % жирный заголовок, плоский текст
\newtheorem{Thm}{\underline{Теорема}}[subsection] % нумерация будет "<номер subsection>.<номер теоремы>"
\newtheorem{Lm}[Thm]{\underline{Lm}} % Нумерация такая же, как и у теорем
\newtheorem{Ex}[Thm]{Упражнение} % Нумерация такая же, как и у теорем
\newtheorem{Example}[Thm]{Пример} % Нумерация такая же, как и у теорем
\newtheorem{Code}[Thm]{Код} % Нумерация такая же, как и у теорем
\theoremstyle{plain} % жирный заголовок, курсивный текст
\newtheorem{Def}[Thm]{Def} % Нумерация такая же, как и у теорем
\theoremstyle{remark} % курсивный заголовок, плоский текст
\newtheorem{Cons}[Thm]{Следствие} % Нумерация такая же, как и у теорем
\newtheorem{Conj}[Thm]{Гипотеза} % Нумерация такая же, как и у теорем
\newtheorem{Prop}[Thm]{Утверждение} % Нумерация такая же, как и у теорем
\newtheorem{Rem}[Thm]{Замечание} % Нумерация такая же, как и у теорем
\newtheorem{Remark}[Thm]{Замечание} % Нумерация такая же, как и у теорем
\newtheorem{Algo}[Thm]{Алгоритм} % Нумерация такая же, как и у теорем

% Определяем ЗАГОЛОВКИ
\def\SectionName{unknown}
\def\AuthorName{unknown}

\newlength{\sectionvskip}
\setlength{\sectionvskip}{0.5em}
\newcommand{\Section}[4][]{
  % Заголовок
  \pagebreak
%  \ifthenelse{\isempty{#1}}{
    \refstepcounter{section}
%  }{}
  \vspace{0.5em}
%  \ifthenelse{\isempty{#1}}{
%    \addtocontents{toc}{\protect\addvspace{-5pt}}%
    \addcontentsline{toc}{section}{\arabic{section}. #2}
%  }{}
  \begin{center}
    {\Large \bf Раздел \NO{\arabic{section}}: #2} \\ 
    \vspace{\sectionvskip}
    \ifthenelse{\equal{#3}{}}{}{{\large #3}\\}
  \end{center}

  \LINE

  % Запомнили название и автора главы
  \gdef\SectionName{#2}
  \gdef\AuthorName{#4}

  % Заголовок страницы
  \lhead{\SEASON}
  \chead{}
  \rhead{\SectionName}
  \renewcommand{\headrulewidth}{0.4pt}

  \lfoot{Глава \NO{\arabic{section}}.}
  \cfoot{\thepage\t{/}\pageref*{LastPage}}
  \rfoot{Автор: \AuthorName}
  \renewcommand{\footrulewidth}{0.4pt}
}

\newcommand{\Subsection}[2][]{
  \refstepcounter{subsection}
  \vspace*{1em}
  \ifthenelse{\equal{#1}{}}
    {\addcontentsline{toc}{subsection}{\arabic{section}.\arabic{subsection}. #2}}
    {\addcontentsline{toc}{subsection}{\arabic{section}.\arabic{subsection}. \bonus\,#2}}
  {\color{blue}\bf\large \arabic{section}.\arabic{subsection}. \ifbonus{#1}\,{#2}} 
  \vspace*{0.5em}
  \makeparindent
}
\newcommand{\Subsubsection}[2][]{
  \refstepcounter{subsubsection}
  \vspace*{1em}
  \ifthenelse{\equal{#1}{}}
    {\addcontentsline{toc}{subsubsection}{\arabic{section}.\arabic{subsection}.\arabic{subsubsection}. #2}}
    {\addcontentsline{toc}{subsubsection}{\arabic{section}.\arabic{subsection}.\arabic{subsubsection}. \bonus\,#2}}
  {\color{blue}\bf\large \arabic{section}.\arabic{subsection}.\arabic{subsubsection}. \ifbonus{#1}\,#2}
  \vspace*{0.5em}
  \makeparindent
}

\newcommand{\Header}{
  \pagestyle{empty}
  \renewcommand{\dateseparator}{--}
  \begin{center}
    {\Large\bf 
     Матанализ 2 семестр ПИ,\\
    \vspace{0.3em}
    \NAME}\\
    \vspace{0.7em}
    {Собрано {\today} в {\currenttime}}
  \end{center}

  \LINE
  \vspace{0em}

  \renewcommand{\baselinestretch}{0.98}\normalsize
  \tableofcontents
  \renewcommand{\baselinestretch}{1.0}\normalsize
  \pagebreak
}

\newcommand{\BeginConspect}{
  \pagestyle{fancy}
  \setcounter{page}{1}
}

\definecolor{mygray}{rgb}{0.7,0.7,0.7}
\definecolor{ltgray}{rgb}{0.9,0.9,0.9}
\definecolor{fixcolor}{rgb}{0.7,0,0}
\definecolor{red2}{rgb}{0.7,0,0}
\definecolor{dkred}{rgb}{0.4,0,0}
\definecolor{dkblue}{rgb}{0,0,0.6}
\definecolor{dkgreen}{rgb}{0,0.6,0}
\definecolor{brown}{rgb}{0.5,0.5,0}

\newcommand{\green}[1]{{\color{green}{#1}}}
\newcommand{\black}[1]{{\color{black}{#1}}}
\newcommand{\red}[1]{{\color{red}{#1}}}
\newcommand{\dkred}[1]{{\color{dkred}{#1}}}
\newcommand{\blue}[1]{{\color{blue}{#1}}}
\newcommand{\dkgreen}[1]{{\color{dkgreen}{#1}}}

\newcommand{\Mod}[1]{\ (\mathrm{mod}\ #1)}

\DeclareMathOperator{\Real}{Re}
\DeclareMathOperator{\Imag}{Im}
\DeclareMathOperator{\lcm}{lcm}
\DeclareMathOperator{\sign}{sign}
\DeclareMathOperator{\Si}{Si}
\DeclareMathOperator{\const}{const}

\begin{document}
	\Header
	\BeginConspect

	\Section{Системы линейных уравнений}{}{Илья Дудников}

	\[ (*)\begin{cases}
		a_{11} x_1 + a_{12} x_2 + ... + a_{1n} x_n = b_1 \\ 
		a_{21} x_1 + a_{22} x_2 + ... + a_{2n} x_n = b_2 \\ 
		\cdots\\
		a_{m1} x_1 + a_{m2} x_2 + ... + a_{mn} x_n = b_n
	\end{cases}\]
	$A = (a_{ij})$ -- матрица коэффициентов, $X = \begin{pmatrix}
	x_1 \\ 
	x_2 \\ 
	\vdots \\ 
	x_n
	\end{pmatrix}$, $B = \begin{pmatrix}
	b_1 \\ 
	b_2 \\ 
	\vdots \\ 
	b_n
	\end{pmatrix}$.  

	\begin{Def}
		Решение СЛУ $(*)$ называется $\alpha_1, ..., \alpha_n \in K : $ при $x_i = \alpha_i$ все уравнения становятся верными.
	\end{Def}

	\begin{Def}
		СЛУ $(*)$ совместна, если $\exists$ хотя бы одно решение. Иначе - несовместна.
	\end{Def}

	\Subsection{Ранг матрицы}

	$A - m \times n, A = (A_1, A_2, ..., A_m), A_i$ -- строки. \\
	$A = (A^1, A^2, ..., A^n), A^j$ -- столбцы.
	
	\begin{Def}
		Строчным (столбцовым) рангом матрицы $A$ называется максимальное число ЛНЗ строк (столбцов). \\
		Иначе, количество элементов в базисе $\langle A_1, ..., A_m\rangle (\langle A^1, ..., A^n\rangle)$. 
	\end{Def}

	\begin{Thm}
		Строчный и столбцовый ранги совпадают.
	\end{Thm}

	Обозначение: $\rank A$.

	\begin{Def}
		Минором матрицы $A - m \times n$ $k$-го порядка называется определитель, 
		составленный из элементов матрицы $A$, стоящих на $k$ выбранных строках и на $k$ выбранных столбцов.  
	\end{Def}

	\begin{Example}
		$\left(\begin{array}{cccc}
		1 & 4 & 8 & -3 \\ 
		2 & 5 & 9 & -4 \\ 
		3 & 6 & -2 & -5
		\end{array}\right)$. Если вы выберем вторую и третью строку, а также первый и последний столбец, то минор второго порядка:
		\[\left|\begin{array}{cc}
		2 & -4 \\ 
		3 & -5
		\end{array}\right|\]
	\end{Example}

	\begin{Thm}
		Ранг матрицы $A$ равен наибольшему порядку минора, отличного от нуля.
	\end{Thm}

	\begin{Thm}[Связь определителя с рангом матрицы]
		$A - n \times n$. Тогда $\rank A < n \EQ \det A = 0$.
	\end{Thm}

	\begin{proof}
		$\SO$. $\rank A < n \SO$ строки $A_1, ..., A_n$ ЛЗ, т.е.
		$\exists \alpha_1, ..., \alpha_n \in K : \alpha_1 A_1 + \alpha_2 A_2 + ... + \alpha_n A_n = 0$ ( $\alpha_i$ не все равны нулю). 
		Пусть $\alpha_1 \neq 0 \SO A_1 = - \frac{\alpha_2}{\alpha_1} A_2 - ... - \frac{\alpha_n}{\alpha_1} A_n$.
		Обнулим первую строку: прибавим к ней $A_2$, умноженную на $-\frac{\alpha_2}{\alpha_1}$, $A_3$, умноженную на $-\frac{\alpha_3}{\alpha_1}$ и т.д.
		Поскольку теперь первая строка целиком нулевая, то $\det A = 0$. \\

		$\Leftarrow$. Индукция $n = 1 \SO a_{11} = 0$. $n - 1 \to n$.
		\[\left|\begin{array}{cccc}
		a_{11} & a_{12} & \cdots & a_{1n} \\ 
		a_{21} & a_{22} & \cdots & a_{2n} \\ 
		\vdots & \vdots & \ddots & \vdots \\ 
		a_{n1} & a_{n2} & \cdots & a_{nn}
		\end{array}\right| = \]
		Можем считать, что $A^1 \neq 0, a_{11} \neq 0$. Домножим первую строку на $- \frac{a_{21}}{a_{11}}$ и прибавляем ко второй строке.
		Затем домножаем первую строку на $-\frac{a_{31}}{a_{11}}$ и прибавляем ко третьей строке и т.д.  
		\[= \left|\begin{array}{cccc}
		a_{11} & a_{12} & \cdots & a_{1n} \\ 
		0 & a_{22}' & \cdots & a_{2n}' \\ 
		\vdots & \vdots & \ddots & \vdots \\ 
		0 & a_{n2}' & \cdots & a_{nn}'
		\end{array}\right| = a_{11} \cdot \left|\begin{array}{ccc}
		a_{22}' & \cdots & a_{2n}' \\ 
		\vdots & \ddots & \vdots \\ 
		a_{n2}' & \cdots & a_{nn}'
		\end{array}\right|\]
		По предположению $A_2', ..., A_n'$ -- ЛЗ. $\begin{cases}
			A_2' = A_2 - \frac{a_{21}}{a_{11}} \cdot A_1 \\
			\cdots \\
			A_n' = A_n - \frac{a_{n1}}{a_{11}} \cdot A_1
		\end{cases}$. \\  
		$0 = \alpha_2 A_2' + ... + \alpha_n A_n' = (...) A_1 + \alpha_2 \cdot A_2 + ... + \alpha_n A_n \SO A_1, ..., A_n$ -- ЛЗ $\SO \rank A < n$.
	\end{proof}

	\begin{Def}
		Элементарными преобразованиями над строками (столбцами) называется
		\begin{MyList}
			\item Перестановка строк (столбцов).
			\item Умножение строки (столбца) на $\lambda \neq 0$.
			\item Прибавление к одной строке (столбцу) другой строки (столбца), умноженной на $\lambda \neq 0$.
		\end{MyList}
	\end{Def}

	\begin{Thm}
		При элементарных преобразованиях ранг матрицы не меняется.
	\end{Thm}

	\begin{proof}
		$1, 2$ -- очевидно.
		$(A_1, ..., A_i, ..., A_j, ..., A_n) \to (A_1, ..., A_i + \lambda A_j, ..., A_j, ..., A_n)$ 
	\end{proof}

	\begin{Def}
		Матрица называется трапецевидной, если у неё в $\forall$ ненулевой строке число нулей слева различно.
	\end{Def}

	\begin{Rem}
		$\rank$ трапецевидной матрицы равен числу ненулевых строк.
	\end{Rem}

	\begin{Thm}[О вычислении ранга]
		Любую матрицу с помощью элементарных преобразований можно привести к трапецевидной.
	\end{Thm}

	\Subsection{Структура решений СЛУ}

	\begin{Def}
		СЛУ (*) называется однородной, если все свободные члены равны нулю.
	\end{Def}

	\begin{Def}
		Нулевое решение однородной СЛУ называется тривиальным. Любое другое решение -- нетривиальным.
	\end{Def}

	\begin{Lm}
		Пусть $Y, Z$ -- решения $AX = 0 \SO \alpha Y + \beta Z$ -- тоже решение, $\alpha, \beta \in K$.
	\end{Lm}

	\begin{proof}
		\[AY = 0, AZ = 0 \SO A(\alpha Y + \beta Z) = \alpha AY + \beta AZ = 0\]
	\end{proof}

	\begin{Thm}[Структура решений однородной СЛУ]
		$AX = 0, A - m \times n, n$ -- число неизвестных, $r = \rank A \SO$
		$\exists n - r$ ЛНЗ решений $X_1, ..., X_{n - r} : \forall$ решение $X = \alpha_1 X_1 + ... + \alpha_{n - r} X_{n - r}$.  
	\end{Thm}

	\begin{proof}
		$A = (A^1, ..., A^n)$, $A^1, ..., A^r$ -- ЛНЗ столбцы $\SO $
		\[
		\begin{cases}
			A_{r + 1} = \beta_{r + 1 \ 1}A^1 + ... + \beta_{r + 1 \ n} A^{r} \\
			\cdots \\
			A^n = \beta_{n \ 1}A^1 + ... + \beta_{n \ r}A^r
		\end{cases}
		\]
		$AX = 0 \EQ x_1 A^1 + x_2 A^2 + ... + x_n A^n = 0$. \\
		$X_1 = \begin{pmatrix}
		\beta_{r + 1 \ 1} \\ 
		\vdots \\ 
		\beta_{r + 1 \ r} \\ 
		-1 \\ 
		0 \\ 
		\vdots \\ 
		0
		\end{pmatrix}, X_2 = \begin{pmatrix}
		\beta_{r + 2 \ 1} \\ 
		\vdots \\ 
		\beta_{r + 2 \ r} \\ 
		0 \\ 
		-1 \\
		0 \\ 
		\vdots \\ 
		0
		\end{pmatrix}, ..., X_{n + r} = \begin{pmatrix}
		\beta_{n \ 1} \\ 
		\vdots \\ 
		\beta_{n \ r} \\ 
		0 \\ 
		\vdots \\ 
		-1 \\ 
		\end{pmatrix}$ -- решения. \\
		Пусть $Z = \begin{pmatrix}
		x_1^* \\ 
		\vdots \\ 
		x_r^* \\ 
		\vdots \\ 
		x_n^*
		\end{pmatrix}$ -- решение. Рассмотрим $Y = Z + x_{r + 1}^* X_1 + x_{r + 2}^* X_2 + ... + x_n^* X_{n - r}$. 
		$Y = \begin{pmatrix}
		y_1 \\ 
		\vdots \\ 
		y_r \\ 
		0 \\ 
		\vdots \\
		0
		\end{pmatrix}$ -- решение~$\{y_1 A_1 + ... + y_r A_r = 0\}$. Но $A_1, ..., A_r$ -- ЛНЗ $\SO Y = \begin{pmatrix}
		0 \\ 
		\vdots \\ 
		0
		\end{pmatrix} \SO 0 = Z + x_{r + 1}^* X_1 + x_{r + 2}^* X_2 + ... + x_n^* X_{n - r}$.
	\end{proof}

	\begin{Def}
		$\forall n - r$ ЛНЗ решений однородной системы линейных уравнений называется
		\textbf{фундаментальной системой решений},
		решение вида $X = \alpha_1 X_1 + ... + \alpha_{n - r} X_{n - r}$ -- \textbf{общее решение}.
	\end{Def}

	\Subsection{Неоднородные СЛУ}

	$AX = B, A - m \times n, X = \begin{pmatrix}
	x_1 \\ 
	\vdots \\ 
	x_n
	\end{pmatrix}, B = \begin{pmatrix}
	b_1 \\ 
	\vdots \\ 
	b_n
	\end{pmatrix}$. \\
	$\overline{A} = \left(\begin{array}{c|c}
	A & B
	\end{array}\right)$ -- расширенная матрица $m \times (n + 1)$.

	\begin{Thm}[Кронекера-Капелли]
		$(*)$ -- совместна $\EQ \rank A = \rank \overline{A}$. 
	\end{Thm}

	\begin{proof}
		$\SO$. $AX = B$ -- совместна $\SO \exists$ решение $x_1 A^1 + ... + x_n A^n = B \SO$
		$B$ -- линейная комбинация $A^1, ..., A^n \SO \rank A = \rank \overline{A}$.

		$\Leftarrow$. $\rank A = \rank \overline{A} = r \SO \exists A^1, ..., A^r$ -- ЛНЗ $\SO A^1, ..., A^r, B$ -- ЛЗ $\SO B = \alpha_1 A^1 + ... + \alpha_r A^r$, не все $\alpha_i = 0 \SO (\alpha_1, ..., \alpha_r, 0, ..., 0)$ -- решение системы.   
	\end{proof}

	\begin{Thm}[О структуре решений неоднородной СЛУ]
		$AX = B, \rank A = r, n$ -- число неизвестных, система совместна. 
		$X_*$ -- какое-то решение СЛУ, $X_1, ..., X_{n - r}$ -- фундаментальные решения $AX = 0$.
		Тогда любое решение $(*)$ имеет вид $X = \alpha_1 X_1 + ... + \alpha_{n - r} X_{n - r} + X_*, \alpha_1, ..., \alpha_{n - r} \in K$. 
	\end{Thm}

	\begin{proof}
		$AX_* = B \SO AX = AX_* \SO A (X - X_*) = 0 \SO X - X* = \alpha_1 X_1 + ... + \alpha_{n - r} X_{n - r}$.
	\end{proof}

	\begin{Example}[Решение СЛУ методом Гаусса]
		\begin{gather*}
			\begin{cases}
				x_1 + x_2 + x_3 + x_4 = 4 \\
				x_1 + x_2 + 2x_3 + 2x_4 = 2 \\
				2x_1 + 2x_2 + 3x_3 + 3x_4 = 6
			\end{cases} \thicksim \left(\begin{array}{cccc|c}
			1 & 1 & 1 & 1 & 4 \\ 
			1 & 1 & 2 & 2 & 2 \\ 
			2 & 2 & 3 & 3 & 6
			\end{array}\right) \thicksim \\
			\thicksim \left(\begin{array}{cccc|c}
			1 & 1 & 1 & 1 & 4 \\ 
			0 & 0 & 1 & 1 & -2 \\ 
			0 & 0 & 1 & 1 & -2
			\end{array}\right) \thicksim \left(\begin{array}{cccc|c}
			1 & 1 & 1 & 1 & 4 \\ 
			0 & 0 & 1 & 1 & -2 \\ 
			\end{array}\right) \thicksim \left(\begin{array}{cccc|c}
			1 & 1 & 1 & 1 & 4 \\ 
			0 & 1 & 0 & 0 & \alpha \\ 
			0 & 0 & 1 & 1 & -2 \\ 
			0 & 0 & 0 & 1 & \beta
			\end{array}\right) \thicksim \left(\begin{array}{cccc|c}
			1 & 1 & 1 & 0 & 4 - \beta \\ 
			0 & 1 & 0 & 0 & \alpha \\ 
			0 & 0 & 1 & 0 & -2 - \beta \\ 
			0 & 0 & 0 & 1 & \beta
			\end{array}\right) \thicksim \\
			\thicksim \left(\begin{array}{cccc|c}
			1 & 1 & 0 & 0 & 6 \\ 
			0 & 1 & 0 & 0 & \alpha \\ 
			0 & 0 & 1 & 0 & -2 - \beta \\ 
			0 & 0 & 0 & 1 & \beta
			\end{array}\right) \thicksim \left(\begin{array}{cccc|c}
			1 & 0 & 0 & 0 & 6 - \alpha \\ 
			0 & 1 & 0 & 0 & \alpha \\ 
			0 & 0 & 1 & 0 & -2 - \beta \\ 
			0 & 0 & 0 & 1 & \beta
			\end{array}\right) \SO \begin{pmatrix}
				x_1 \\ 
				x_2 \\ 
				x_3 \\ 
				x_4
				\end{pmatrix} = \begin{pmatrix}
				6 \\ 
				0 \\ 
				-2 \\ 
				0
				\end{pmatrix} + \alpha \begin{pmatrix}
				-1 \\ 
				1 \\ 
				0 \\ 
				0
				\end{pmatrix} + \beta \begin{pmatrix}
				0 \\ 
				0 \\ 
				-1 \\ 
				1
				\end{pmatrix}
		\end{gather*}
	\end{Example}

\Section{Линейные отображения векторных пространств}{}{Илья Дудников}

\begin{Def}
	$V, W$ -- векторные пространства над $K$. Отображение $f : V \to W$ называется линейным,
	если:
	\begin{MyList}
		\item $f(x + y) = f(x) + f(y) \ \forall x, y \in V$
		\item $f(\alpha x) = \alpha f(x) \ \forall x \in V, \alpha \in K$ 
	\end{MyList}
\end{Def}

\begin{Rem}
	$1, 2 \thicksim f(\alpha x + \beta y) = \alpha f(x) + \beta f(y) \ \forall x, y \in V, \alpha, \beta \in K$. 
\end{Rem}

\begin{Def}
	$\Hom (V, W) = \{f : V \to W \text{ -- линейные}\}$
\end{Def}

\begin{Lm}
	$\Hom (V, W)$ -- векторное пространство над $K$.
\end{Lm}

\begin{proof}
	$f, g \in \Hom (V, W), (f + g)(x) = f(x) + g(x), (\alpha f)(x) = \alpha f(x) \SO f + g, \alpha f \in \Hom (V, W)$. 
\end{proof}

\begin{Def}
	$f \in \Hom (V, W), \ker f = \{x \in V : f(x) = 0\}$ -- ядро отображения $f$, $\Image f = \{f(x), x \in V\}$ -- образ $f$. 
\end{Def}

\begin{Lm}
	$\ker f \subset V, \Image f \subset W$ -- подпространства. 
\end{Lm}

\begin{proof}
	$x, y \in \ker f, f(x + y) = f(x) + f(y) = 0 + 0 = 0 \SO x + y \in \ker f$.
	Аналогично, $f(\alpha x) = \alpha f(x) = 0 \SO \alpha x \in \ker f \ \forall \alpha \in K \SO \ker f$ -- подпространство.  
\end{proof}

\begin{Ex}
	$\Image f$ -- подпространство. 
\end{Ex}

\begin{Thm}
	$f \in \Hom (V, W)$.
	\begin{MyList}
		\item $f$  -- инъективно $\EQ \ker f = \{0\}$.
		\item $f$ -- сюръективно $\EQ \Image f = W$. 
	\end{MyList} 
\end{Thm} 

\begin{proof}
	$\Leftarrow.$ $x_1 \neq x_2$, если $f(x_1) = f(x_2) \SO f(x_1 - x_2) = 0 \SO x_1 - x_2 \in \ker f \SO x_1~-~x_2~=~0~!? $  \\
	$\SO$. Пусть $x \in \ker f, x \neq 0 \SO f(x) = f(0) = 0 !?$.  
\end{proof}

\Subsection{Матрица линейного отображения}

$e_1, ..., e_n$ -- базис $V, e_1', ..., e_m'$ -- базис $W, f \in \Hom (V, W)$ \\
$x \in V, x = x_1 e_1 + ... + x_n e_n, x_i \in K, f(x) = x_1 f(e_1) + ... + x_n f(e_n) \EQ$ задать $f$ значит задать $f(e_i),~i~=~1, ..., n$. \\

% $f(e_1) = a_{11} e_1' + a_{21} e_2' + ... + a_{m1} e_m'$

\[\begin{cases}
	f(e_1) = a_{11} e_1' + a_{21} e_2' + ... + a_{m1} e_m' \\
	\cdots \\
	f(e_n) = a_{1n} e_1' + a_{2n} e_2' + ... + a_{mn} e_m'
\end{cases}\]

\begin{Def}
	Матрицей $f \in \Hom(V, W)$ в базисе $e_1, ..., e_n$ и $e_1', ..., e_m'$ назыается 
	\[A=\left(\begin{array}{cccc}
	a_{11} & a_{12} & \cdots & a_{1n} \\ 
	\vdots & \vdots & \ddots & \vdots \\ 
	a_{m1} & a_{m2} & \cdots & a_{mn}
	\end{array}\right) = \left(\begin{array}{cccc}
	f(e_1) & f(e_2) & \cdots & f(e_n)
	\end{array}\right)\]
\end{Def}

\begin{Thm}
	\begin{MyList}
		\item $\Hom(V, W)$ взаимно-однозначно соответствует $M(m, n, K)$.
		\item $e_1, ..., e_n$ -- базис $V$, $e_1', ..., e_m'$ -- базис $W$, $x \in V \to X = \begin{pmatrix}
		x_1 \\ 
		\vdots \\ 
		x_n
		\end{pmatrix}, f(x) \in W \to Y = \begin{pmatrix}
		y_1 \\ 
		\vdots \\ 
		y_m
		\end{pmatrix}, f \to A$
		$\SO AX = Y$.  
	\end{MyList}
\end{Thm}

\begin{proof}
	\begin{MyList}
		\item $f \to A$ отображение однозначно определяется $f(e_i) \SO A$ определена однозначно.
		С другой стороны, взяв произвольную матрицу $B$, можем построить по ней отображение $g$.

		\item $f \to A = (a_{ij}), 1 \leqslant i \leqslant n, 1 \leqslant j \leqslant m$.
		\begin{gather*}
			f(x) = f(x_1 e_1 + ... x_n e_n) = x_1 f(e_1) + ... + x_n f(e_n) = \\
			= x_1 (a_{11} e_1' + a_{21} e_2' + ... + a_{m1} e_m') + ... + x_n (a_{1n} e_1' + a_{2n} e_2' + ... + a_{mn} e_m') = \\
			= \underbrace{(a_{11} x_1 + a_{12} x_2 + ... + a_{1n} x_n)}_{y_1} e_1' + ... + \underbrace{(a_{m1} x_1 + a_{m2} x_2 + ... + a_{mn} x_n)}_{y_m} e_m' \SO Y = AX
		\end{gather*}
	\end{MyList}
\end{proof}

\begin{Cons}
	\begin{MyList}
		\item $\dim \Hom(V, W) = \dim V \cdot \dim W$
		\item $\alpha, \beta \in K, f, g \in \Hom(V, W), f \to A, g \to B$ в фиксированных базисах $\SO \alpha f + \beta g \to \alpha A + \beta B$.
		\item $f : V \to W, g : W \to U \SO g \circ f : V \to U, g \circ f(x) = g(f(x))$.
		Фиксируем базисы, $f \to A, g \to B \SO g \circ f \to BA$  
	\end{MyList}
\end{Cons}

\gdef\AuthorName{Дарья Гольденберг}

\begin{proof}
	\begin{MyList}
		\item Соответствие матриц.
		\item $(\alpha f  + \beta g) (e_i) = \alpha f (e_i) + \beta g (e_i) \in \alpha A + \beta B$.
		\item $ V \to n, W \to l, U \to m, A \to l \times n, B \to m\times l\\ g \circ f (e_i) = g (\sum_{k=1}^l a_{ki}e_k) = \sum_{k=1}^n a_{ki} g(e_k') = \sum_{k=1}^l a_{ki} \sum_{j =1}^m b_{jk} e_j'' = \sum_{j = 1}^m \sum_{k = 1}^l b_{jk} a_{ki} e_j''$, где $b_{jk} a_{ki} \to BA$.
	\end{MyList}
\end{proof}

\begin{Thm}
	$f: V \to W, \ \dim V, \dim W < \inf$ 
	$$ \Rightarrow \dim \ker f + \dim \Image f = \dim V$$
\end{Thm}

\begin{proof}
	$\ker f \subset V, e_1,...,e_k$ - базис $\ker f$. Дополним до базиса $V: e_1, ..., e_k, e_{k+1},..., e_n $ -- базис $V$. \\
	$x \in V, f(x)\in \Image f \ f(x) = x_{k+1} f(e_{k+1}) + ... + x_n f(e_n)  \in \Image f \\
	f(e_1) = ... = f(e_k) = 0 \Rightarrow \Image f = \langle f(e_{k+1}), ..., f(e_n) \rangle $. Надо доказать, что $f(e_{k+1}), ..., f(e_n)$ -- ЛНЗ. \\
	Предположим обратное. $\alpha_{k+1} f(e_{k+1}) + ... + \alpha_n  f(e_n) = 0 \Rightarrow f(\alpha_{k+1}e_{k+1} + ... + \alpha_n e_n) = 0 \Rightarrow\alpha_{k+1} e_{k+1} + ... + \alpha_n e_n \in \ker f = \langle e_1, ..., e_k\rangle$ -- невозможно.\\
	$\dim \ker f = k,\dim V = n, \dim \Image f = n - k$.
\end{proof}

\Subsection{Линейные операторы}

\begin{Def}
	Линейным оператором называется линейное отображение $a: V \to V$, т.е. $a \in \Hom(V, V)$. \\
	Обозначается $\End V = \Hom(V, V)$.
\end{Def}

\begin{Def}
	Тождественным отображением называется отображение  
\end{Def}

\begin{Def}

\end{Def}

\begin{Example}
	\begin{MyList}
		\item Нулевой оператор. $0 \in \End V. \ \ 0(x) = 0$. $0 \to 
			\left(\begin{array}{cccc}
			0 & \cdots & 0 \\ 
			\vdots & \ddots & \vdots \\
			0 &  \cdots & 0
			\end{array}\right) = 0 $
		\item Оператор подобия. $\forall x \in V \ ax = \lambda x \to 
		    \left(\begin{array}{cccc}
			\lambda & \cdots & 0 \\ 
			\vdots & \ddots & \vdots \\
			0 &  \cdots & \lambda
			\end{array}\right) $
		\item Оператор поворота в $\R^2$. $z \to z e^{i\varphi}$ -- поворот на $\varphi$. Зафиксируем базис -- $1, i \Rightarrow a(1) = \cos \varphi + i\sin \varphi, a(i) = i(\cos \varphi + i\sin \varphi) = -\sin \varphi + i\cos \varphi  \ \to$
			$\left(\begin{array}{cccc}
				\cos \varphi & -\sin \varphi \\ 
				\sin \varphi & \cos \varphi
				\end{array}\right) $
		\item Оператор дифференцирования. $V = \R[x]$. $\frac{\,d}{\,dx} f \to f'$, зафиксируем базис -- $1, x, x^2, x^3$. \\ 
		$\frac{\,d}{\,dx}(1) = 0, \frac{\,d}{\,dx} (x) = 1, \frac{\,d}{\,dx}(x^2) = 2x, \frac{\,d}{\,dx} (x^3) = 3x^2$. Тогда матрица имеет вид: 
			$\left(\begin{array}{cccc}
			0 & 1 & 0 & 0\\ 
			0 & 0 & 2 & 0\\
			0 & 0 & 0 & 3 \\
			0 & 0 & 0 & 0
			\end{array}\right)$.\\
		Возьмём другой базис -- $1, x+1, x^2 + x + 1, x^3 + x^2 + x + 1$. \\
		Посчитаем значения: $\frac{\,d}{\,dx} (1) = 0, \frac{\,d}{\,dx}(x + 1) = 1, \frac{\,d}{\,dx}(x^2 + x + 1) = 2x + 1, \frac{\,d}{\,dx}(x^3 +x^2 + x + 1) = 3x^2 + 2x + 1$. \\
		Матрица имеет вид: 
		$\left(\begin{array}{cccc}
			0 & 1 & 1 & 1\\ 
			0 & 0 & 2 & 2\\
			0 & 0 & 0 & 3 \\
			0 & 0 & 0 & 0
			\end{array}\right)$.
	\end{MyList}
\end{Example}

\begin{Def}
	$(e_i), (e_i')$ -- базисы V, $\dim V = n$. Разложим 
	$\begin{cases}
		e_1' = c_{11}e_1 + c_{21}e_2 + ... + c_{n1}e_n \\ 
		\cdots\\
		e_n' = c_{1n}e_1 + c_{2n}e_2 + ... + c_{nn}e_n
	\end{cases}$. Тогда матрица вида $C = 
	\left(\begin{array}{cccc}
		c_{11} & c_{12} & \cdots & c_{1n} \\ 
		c_{21} & c_{22} & \cdots & c_{2n} \\
		\vdots & \ddots & \vdots \\
		c_{n1} & c_{n2} & \cdots & c_{nn}
		\end{array}\right)$ называется матрицой перехода от базиса $(e_i)$ к $(e_i')$.
\end{Def}

\begin{Thm}[Преобразование координат вектора при переходе к другому базису]
	V -- векторное пространство над полем K, $(e_i), (e_i')$ -- базисы V, $x\in V, \ x \to X = 
	\left(\begin{array}{c}
		x_1 \\ 
		x_2 \\ 
		\vdots \\ 
		x_n
		\end{array}\right) $ -- координаты вектора в базисе $(e_i)$. $x \to X' = 
		\left(\begin{array}{c}
		x_1' \\ 
		x_2' \\ 
		\vdots \\ 
		x_n'
		\end{array}\right)$ -- координаты вектора в базисе $(e_i'), C$ -- матрица перехода от $(e_i)$ к $(e_i')$. 
	\begin{MyList}
		\item $X = CX'$
		\item $C - \text{обратима} \ (\det C \neq 0)$		
	\end{MyList}
\end{Thm}

\begin{proof}
	\begin{MyList}
		\item $x = x_1' e_1' + ... + x_n'e_n' = x_1'(c_{11}e_1 + c_{21}e_2 +... + c_{n1}e_n) + ... + x_n'(c_{1n}e_1 + c_{2n}e_2 + ... + c_{nn}e_n) = \underbrace{(c_{11}x_1' + c_{12}x_2' + ... + c_{1n}x_n')}_{x_1} e_1 + ... + \underbrace{(c_{n1}x_1' + c_{n2}x_2' + ... + c_{nn}x_n')}_{x_n} e_n$\\
		$\Rightarrow \left(\begin{array}{c}
			x_1 \\ 
			x_2 \\ 
			\vdots \\ 
			x_n
			\end{array}\right) = C \left(\begin{array}{c}
			x_1' \\ 
			x_2' \\ 
			\vdots \\ 
			x_n'
			\end{array}\right) $
		\item $ \forall X \ X = C X'$ по доказанному, тогда $ X = C X' = C D X \Rightarrow CD = E \Rightarrow \det C \neq 0$.
	\end{MyList}
\end{proof}

\begin{Thm}[Изменение матрицы линейного оператора при переходе к другому базису]
	$V$ -- векторное пространство, $\dim V = n, a \in \End V$, фиксируем базисы $(e_i), (e_i'), \ A$ -- матрица оператора в базисе $(e_i), \ A'$ -- в базисе $(e_i')$, C -- матрица перехода от $(e_i) \text{ к } (e_i')$.
	$$\Rightarrow A' = C^{-1} A C$$ 
\end{Thm}

\gdef\AuthorName{Ксения Кузьмина}

\begin{Lm}
	$U \subset V, a \in \End \, V\\
	U -$ а-инвариантно $\Leftrightarrow$ $\exists$ базис $V: A = 
	\left(\begin{array}{cc}
		B & C\\
		0 & D
		\end{array}\right), B = \dim U \times \dim U$
\end{Lm}

\begin{proof}
	$U$ -- а-инвариантно. Выберем базис $U$: $e_1, ... , e_k$ и дополним его до базиса $V$ матрицы $a$
	$ae_i = b_{1i}e_1+\dots+b_{ki}e_k \Leftrightarrow \left(\begin{array}{cc}
		b_{1i} & \cdot\\
		b_{ki} & \cdot\\
		0 & \cdot \\
		0 & \cdot
	\end{array}
	\right)$
\end{proof}

\begin{Lm}
	$U, W \subset V, a \in \End \, V\\
	V = U \oplus W, U, W$ -- а-инвариантны $\Leftrightarrow \exists$ базис $A = \left(\begin{array}{cc}
		B & 0\\
		0 & C
	\end{array}\right)\\
	B = \dim U \times \dim U, C = \dim W \times \dim W$
\end{Lm}

\begin{proof}
	$V$ = $U \oplus V$, выберем $U: e_1, ..., e_k, W: e_{k+1}, ..., e_n\\
	a(e_i) \in U, i = 1, ..., k, a(e_j) \in W, j = k+1, ..., n \Leftrightarrow A = \left(\begin{array}{cc}
		B & 0\\
		0 & C
	\end{array}\right)$
\end{proof}

\begin{Example}
	\begin{enumerate}
		\item $V = M(2, \R) \ \ a: X \to X^T, X \in M(2, \R)$\\
		$E_{11} = \left(\begin{array}{cc}
			1 & 0\\
			0 & 0
		\end{array}\right), \ \ 
		E_{12} = \left(\begin{array}{cc}
			0 & 1\\
			0 & 0
		\end{array}\right), \ \  
		E_{21} = \left(\begin{array}{cc}
			0 & 0\\
			1 & 0
		\end{array}\right), \ \ 
		E_{22} = \left(\begin{array}{cc}
			0 & 0\\
			0 & 1
		\end{array}\right)\\\\
		a(E_{11}) = E_{11}, \ \ a(E_{12}) = E_{21}, \ \ a(E_{21}) = E_{12} \ \ a(E_{22}) = E_{22}\\\\
		A = \left(\begin{array}{cccc}
			1 & 0 & 0 & 0\\
			0 & 0 & 1 & 0\\
			0 & 1 & 0 & 0\\
			0 & 0 & 0 & 1
		\end{array}\right) \ \ \ \ \  <E_{11}> \oplus <E_{12}, E_{21}> \oplus <E_{22}> \ = V$ инвариантны
		\item $V = K[x]_3 \ \ \ a: \frac{d}{dx} (f \to f') \ \ \ 1, x, x^2, x^3\\
		\left(\begin{array}{cccc}
			0 & 1 & 0 & 0\\
			0 & 0 & 2 & 0\\
			0 & 0 & 0 & 3\\
			0 & 0 & 0 & 1
		\end{array}\right) \ \ \ \frac{d}{dx} : <1, x, x^2> \to <1, x> \subset <1, x, x^2>$
	\end{enumerate}
\end{Example}

\Subsection{Собственные векторы и числа}
\begin{Def} 
	Собственным вектором оператора a называется $\forall$ ненулевой вектор одномерного инвариантного подпространства.
\end{Def} 

\begin{Def} 
	x - собственный вектор, $ax = \lambda x$, тогда $\lambda$ = собственное число, ассоциированное вектору $x$\\
	$a \to A, x \to X \ \ \ AX = \lambda X \Rightarrow (A - \lambda E)X = 0$
\end{Def} 

\begin{Def} 
	Характеристическим многочленом оператора a (матрицы A) назыается $\chi_a(t) = \det (A-tE)$
\end{Def} 

\begin{Thm}[О собственных числах]
	Все собственные числа оператора a и только они являются корнями характеристического многочлена. 
\end{Thm}

\begin{proof}
	$AX = \lambda X \Leftrightarrow (A - \lambda E)X = 0$ -- имеет ненулевое решение $\Leftrightarrow 
	\det (A- \lambda E)X = 0 \Leftrightarrow$ все собственные числа корни $\chi_a(t)$
\end{proof}

\begin{Lm}[Независимость собственных чисел от выбора базиса]
	Характеристические многочлены оператора a в разных базисах совпадают. 
\end{Lm}

\begin{proof}
	$a(e_i) \to A \ \ \ (e_i') \to A' \ \ \ C $ -- матрица перехода от $(e_i)$ к $(e_i')\\
	A' = C^{-1}AC \ \ \chi_a(t) = \det (A' - tE) = \det(C^{-1}AC) = \det (C^{-1}AC - t \cdot C^{-1}C) =
	\det(C^{-1} (A-tE)C) = \det C^{-1} \cdot \det(A - tE) \cdot \det C = \det(A-tE) = \chi_a(t)e_i$
\end{proof}

\begin{Thm}[Линейная независимость собственных векторов]
	Собственные векторы, соответствующие различным собственным числам, линейно независимы. 
\end{Thm} 

\begin{proof}
	$n = 1$ -- очевидно.
	Пусть доказали при $n-1$. Индукционный переход: $n-1 \to n: V_1, V_2, ..., V_n$ -- собственные векторы\\
	$aV_i = \lambda_i V_i, \ \ \ \lambda_1, ..., \lambda_n$ -- различны\\
	Пусть $V_1, V_2, ..., V_n$ -- линейно зависимы. Тогда $\alpha_1 V_1 + \alpha_2 V_2 + ... + \alpha_n V_n = 0, \alpha_i \in K \Rightarrow$
	под действием a: $\alpha_1 \lambda_1 V_1 + \alpha_2 \lambda_2 V_2 + ... + \alpha_n \lambda_n V_n = 0$\\
	Будем считать, что $\lambda_1 \neq 0 \Rightarrow \alpha_1 \lambda_1 V_1 + \alpha_2 \lambda_2 V_2 + ... + \alpha_n \lambda_n V_n - \lambda_1(\alpha_1 V_1 + ... + \alpha_n V_n) =
	\alpha_2(\lambda_2 - \lambda-1)V_2 + ... + \alpha_n(\lambda_n - \lambda_1)V_n = 0 \Rightarrow$ по предположению индукции $\alpha_2 = ... = \alpha_n = 0$ 
\end{proof}

\begin{Def} 
	Оператор a называется диагонализируемым, если существует базис такой, что $A = \left(\begin{array}{cccc}
		\lambda_1 & 0 & 0 & 0\\
		0 & \lambda_2 & 0 & 0\\
		0 & 0 & \ddots & 0\\
		0 & 0 & 0 & \lambda_n
	\end{array}
	\right)$
\end{Def} 

\begin{Thm}[Критерий диагонализируемости] 
	Если $\chi_a(t)$ имеет $n$ различных корней $(n = \dim V)$ над рассматриваемым полем, то оператор a -- диагонализируем. 
\end{Thm} 

\begin{proof}
	В качестве базиса берём собственные векторы. 
\end{proof}

\begin{Example}
	Оператор поворота $A = \left(\begin{array}{cc}
		\cos \varphi & - \sin \varphi\\
		\sin \varphi & \cos \varphi
	\end{array}\right)$ -- недиагонализируем над $\R$
\end{Example}

\begin{Lm}
	Над полем $\C$ любой оператор имеет одномерное инвариантное подпространство.
\end{Lm}

\begin{Def} 
	Кратность $\lambda$ как кратность корня $\chi_a (t) = 0$ называется алгебраической кратностью собственного числа. 
\end{Def} 

\begin{Def}
	$\lambda$ -- собственное число, $V^{\lambda} = \{x \in V: ax = \lambda x\}$\\
	$\dim V^{\lambda}$ -- геометрическая кратность собственного числа $\lambda$
\end{Def}

\begin{Example}
	$A = \left(\begin{array}{cc}
		\lambda & 0\\
		0 & \lambda\\
	\end{array}
	\right) \ \ \chi_a(t) = \left|\begin{array}{cc}
		\lambda - t & 0\\
		0 & \lambda -t
	\end{array}\right| = (A-tE) = (\lambda - t)^2 \Rightarrow \lambda$ собственное число алгебраической кратности 2.\\
	$(A - \lambda E)X = 0$\\
	$\left( \left(\begin{array}{cc}
		\lambda & 0\\
		0 & \lambda
	\end{array}\right) - \lambda \left(\begin{array}{cc}
		1 & 0\\
		0 & 1
	\end{array}\right)\right) \left( \begin{array}{c}
		X_1\\
		X_2
	\end{array}\right) = \left( \begin{array}{c}
		0\\
		0
	\end{array} \right)\\
	\dim V^{\lambda} = 2$ -- геометрическая кратность
\end{Example}

\begin{Example}
	$A = \left(\begin{array}{cc}
		\lambda & 1\\
		0 & \lambda
	\end{array} \right)$
	
	$\chi_a(t) = \left|\begin{array}{cc}
	\lambda - t & 1\\
	0 & \lambda - t
	\end{array} \right| = (\lambda - t)^2 \Rightarrow$ алгебраическая кратность $\lambda$ = 2

	$\left( \left(\begin{array}{cc}
		\lambda & 1
		\\
		0 & \lambda
	\end{array}\right) - \lambda \left(\begin{array}{cc}
		1 & 0\\
		0 & 1
	\end{array}\right)\right) \left( \begin{array}{c}
		X_1\\
		X_2
	\end{array}\right) = \left( \begin{array}{c}
		0\\
		0
	\end{array} \right)$\\

	$\left(\begin{array}{cc}
		0 & 1\\
		0 & 0
	\end{array}\right)
	\left(\begin{array}{c}
		X_1\\
		X_2
	\end{array}\right)
	\left(\begin{array}{c}
		0\\
		0
	\end{array}\right) \ \ \ V^{\lambda} = <\left(\begin{array}{c}
		1\\
		0
	\end{array}\right)> \\ \dim V^{\lambda} = 1$ -- геометрическая кратность
\end{Example}

\begin{Lm}
	Геометрическая кратность собственного числа $\lambda \leqslant$ алгебраической кратности
\end{Lm}

\begin{proof}
	$V^{\lambda}$ -- инвариантно относительно a, $V^{\lambda} = \{x: ax = \lambda x\}$\\
	По лемме: $a \to \left(\begin{array}{cc}
		B & C\\
		0 & D
	\end{array}\right) \ \ \ B - m \times m, \dim V^{\lambda} = m\\
	a \big|_{V^{\lambda}} = \chi_a \big|_V^{\lambda} = (t - \lambda)^m\\
	\chi_a = \det \left( \left(\begin{array}{cc}
		B & C\\
		0 & d
	\end{array}\right) - tE \right) = (t - \lambda)^m p(t) \Rightarrow$ алгебраическая кратность $\lambda \geqslant m$ 
\end{proof}

\begin{Thm}[Критерий диагонализируемости] 
	$a \in \End \, V$ -- диагонализируема $\Leftrightarrow$ 1. Все собственные числа $\in K$ 2. $\forall$ собственных чисел $\lambda$ алгебраическая кратность = геометрическая кратность 
\end{Thm} 

\Subsection{Жорданова нормальная форма}
\begin{Def} 
	Жордановой клеткой порядка m соответствующей собственному числу $\lambda$ называется\\
	$J_m(\lambda) = \left( \begin{array}{cccc}
		\lambda & 1 & \cdots & 0\\
		\vdots & \lambda &  & \vdots\\
		0 & & \ddots &  1\\
		0 & \dots & & \lambda
	\end{array} \right)$
\end{Def} 

\begin{Example}
	\begin{enumerate}
		\item $\left(\begin{array}{cc}
			\lambda & 1\\
			0 & \lambda
		\end{array}\right)$
		\item $\left(\begin{array}{ccc}
			\lambda & 1 & 0\\
			0 & \lambda & 1\\
			0 & 0 & \lambda
		\end{array}\right)$
	\end{enumerate}
\end{Example}

\begin{Def} 
	ЖНФ оператора $a \in \End V$ называется $$\left(\begin{array}{cccc}
		J_{k_1}()\lambda_1) &  &  & \\
		 & J_{k_2}()\lambda_2) &  &\\
		 & & \ddots &\\
		 & & & J_{k_n}()\lambda_n)
	\end{array}\right)$$
\end{Def} 

\begin{Def} 
	Базис, в котором оператор a имеет ЖНФ называется жордановом
\end{Def} 

\begin{Thm}[ЖНФ] 
	\begin{enumerate}
		\item Над алгебраическим замкнутым полем $\forall a \in \End V$ имеет ЖНФ
		\item ЖНФ определена с точностью до перестановки клеток
	\end{enumerate}
\end{Thm} 

\begin{Thm} 
	$a \in \End V$ имеет ЖНФ над произвольным полем $\Leftrightarrow$ характеристический многочлен раскладывается на линейные множители
\end{Thm} 

\Subsection{Теорема Гамильтона-Кэли}	

\begin{Def} 
	$f(x) = a_nx^n+...+a_1x+a_0 \in K[x]$\\
	$A -- m\times m$\\
	$f(A) = a_nA^N+...+a_iA+a_0E$, $E -- m \times m$ -- многочлен от матрицы
\end{Def} 

\begin{Def} 
	$a \in \End V\\
	f(a) = a_n \cdot a^n + ... + a_1 \cdot a + a_0 \cdot \id$ -- многочлен от оператора
\end{Def} 

\begin{Thm}[Гамильтона-Кэли] 
	$\chi_a(A)=0$, $a \in \End V$, $a \to A \in M(m, K)$
\end{Thm} 

\begin{proof}
	$\chi_a(t) = \det (tE-A)\\
	B = tE-A, \widetilde{B} = (B_{ij})^T$ -- взаимная матрица\\
	$\widetilde{B} = \left(
		\begin{array}{ccc}	
		c_{11}^{m-1} t^{m-1} + ... + c_{11}^0 & ... \\
		c_{12}^{m-1}t^{m-1} + ... + c_{12}^0 & ...
	\end{array}\right) = t^{m-1}B_{m-1} + t^{m-2} B_{m-2} + ... + B_0\\
	\widetilde{B} \cdot B = \det (tE - A) \cdot E\\
	(t^{m-1}B_{m-1} + ... + B_0) \cdot (tE - A) = (a_m t^m + ... + a_0) \cdot E$\\
	Приравняем коэффициенты, домножим:
	$t^m: B_m-1 = a_m E \ \ | \cdot A^m\\
	t^{m-1}: B_{m-2} - B_{m-1} A = a_{m-1} E \ \ |\cdot A^{m-1}\\
	...\\
	t: B_0 - B_1A = a_1 E \ \ |\cdot A\\
	1: -B_0A = a_0E \ \ |\cdot E\\$
	Вычтем строки: $\Rightarrow 0 = \chi_a(A)$
\end{proof}

\Subsection{Билинейные формы}
\begin{Def} 
	$f: V \times V \to K$ линейное по каждому аргументу называется билинейным отображением, то есть выполняется\\
	\begin{enumerate}
		\item $f(\alpha x + \beta y, z) = \alpha f(x,z) + \beta f(y,z)$
		\item $f(x, \alpha y + \beta z) = \alpha f(x, y) + \beta f(x, z)$
	\end{enumerate}

	$(e_i)_{i=1}^n$ -- базис $V$, $x = \sum_{i=1}^{n} x_ie_i, y = \sum_{j=1}^{n} y_je_j\\
	f(x, y) = f(\sum_{i} x_ie_i, \sum_{j} y_je_j) = \sum_{i, j = 1}^{n} x_i, y_j f(e_i, e_j) $
\end{Def} 

\begin{Def} 
	$B = (b_{ij}), b_{ij} = f(e_i, e_j), 1 \leqslant i, j, \leqslant n$\\
	B -- матрица билинейной формы $f$
	$X = \left(
		\begin{array}{c}	
		x_1\\
		\vdots\\
		x_n
	\end{array}\right), Y = \left(
	\begin{array}{ccc}	
		y_1\\
		\vdots\\
		y_n
	\end{array}\right) \Rightarrow f(x, y) = X^T B Y$
\end{Def} 

\begin{Example}
	\begin{enumerate}
		\item Скалярное произведение $(x, y) = x_1y_1 + ... + x_ny_n = (x_1, ..., x_n) \left(
			\begin{array}{ccc}	
				1 & &\\
				& \ddots &\\
				& & 1
			\end{array}\right) \left(
				\begin{array}{ccc}	
					y_1\\
					\vdots\\
					y_n
				\end{array}\right)$
		\item $f, g \in C[a,b], (f, g) = \int_{a}^{b} f(x)g(x)dx$
	\end{enumerate}	
\end{Example}

\begin{Def} 
	Билинейные формы f
	\begin{enumerate}
		\item Симметрические: $f(x,y) = f(y,x) \forall x, y \in V \ \ B = B^T$ (симметрическая матрица)
		\item Кососимметрические: $f(x, y) = -f (y,x) \forall x, y \in V \ \ -B = B^T$ (кососимметрическая матрица)
	\end{enumerate}
\end{Def} 

\Subsubsection{Замена базиса}

\begin{Thm}[Преобразование матрицы билинейной формы при изменении базиса] 
	$f: V \times V \to K$ в базисе $(e_i): B$, в базисе $(e_i'): B'$\\
	Тогда $B' = C^TBC$
\end{Thm}  

\begin{proof}
	$x \to X$ в базисе $(e_i), X'$ в базисе $(e_i') X = C X',\\
	y \to Y$ в $(e_i)$, $Y'$ в $(e_i') \ \ Y = CY'$\\
	$f(x,y) = X^T B Y = (CX')^T B (CY') = X'^TC^TBCY' = X'^TB'Y' \Rightarrow B' = C^TBC$
\end{proof}

\Subsection{Квадратичные формы}
\begin{Def} 
	Квадратичной формой $Q: V \to K$, ассоциированной с некоторой симметрической билинейной формой $f: V \times V \to K$, называется $q(x)=f(x,x)$\\
	Матрица квадратичной формы $A \ \ \ (A = A^T)$ \ \ \ $q(x) = X^TAX, X = \left(
		\begin{array}{ccc}	
			x_1\\
			\vdots\\
			x_n
		\end{array}\right)\\
		q(x) = (x_1, ..., x_n)\left(
			\begin{array}{ccc}	
				a_{11} & & \\
				& \ddots & \\
				& &a_nn
			\end{array}\right)\left(
			\begin{array}{ccc}	
				y_1\\
				\vdots\\
				y_n
			\end{array}\right) = \displaystyle \sum_{i,h = 1}^{n} a_{ij}x_iy_j$ -- однородный многочлен 2 степени от $n$ переменных\\
			$a_{ij} = a_{ji} \ \ \ a_{ij}x_ix_j+a_{ji}x_ix_j = 2a_{ij}x_ix_j\\
			\displaystyle q(x) = \sum_{i=1}^{n} a_{ii} x_i^2 + 2 \sum_{1 \leqslant u < j \leqslant n} a_{ij} x_ix_j$
\end{Def} 

\begin{Def} 
	Каноническим видом квадратичной формы называется $\displaystyle \sum_{i=1}^{n} \lambda_i x^2_i $
\end{Def} 

\begin{Def} 
	Базис, в котором квадратичная форма имеет канонический вид, называется каноническим.
\end{Def} 

\begin{Rem}
	Замена переменной $\leftrightarrow$ переход к другому базису
\end{Rem}

\begin{Thm}[Преобразование Лагранжа] 
	$V$ -- векторное пространство над полем $K, \\ char K \neq 2 \Rightarrow
	\forall q: V \to K$ может быть приведена к каноническому виду $(\exists$ базис: $q$ имеет канонический вид$)$
\end{Thm} 

\begin{proof}
	$q(x) = \sum_{i=1}^{n} a_{ii}x^2_i + 2 \sum_{i<j}^{a_ij} x_i x_j$\\
	$q = 0$ -- доказывать нечего, будем считать $q \neq 0$. 
	\begin{enumerate}
		\item Пусть $a_{11} = 0, \exists i > 1: a_{ii} \neq 0 \Rightarrow$ сделаем замену $y_i = x_1, x_i = y_1 \Rightarrow a_11 y_1^2 + ... $, где $a_{11} \neq 0$
		\item $a_{ii} = 0 \forall i = 1, ..., n \Rightarrow \exists a_{ij} \neq 0, i < j \Rightarrow x_i = y_i+y_j, x_j = y_i - y_j$\\
		$a_{ij} x_i x_j \to a_{ij}(y_i+y_j)(y_i-y_j) = a_{ij} \cdot y_i^2 - a_{ij} y^2_j \Rightarrow$ по п.1 можно считать, что $a_{11} \neq 0$
		\item Индукция\\
		База: $n=1 \ \ q(x) = a_{11} x_1^2$\\
		Индукционный переход: $n-1 \to n, a_{11} \neq 0$ в силу пункта один\\
		$\displaystyle q(x) = a_{11}\left(x_1^2 + \frac{2a_{12}}{a_{11}} x_1x_2 + \frac{2a_{13}}{a_11} x_1x_3 +... + \frac{2a_{1n}}{a_{11}} x_1x_n\right) + \varphi(x_2, ..., x_n) = 
		a_{11}\left(x_1^2 + \frac{2a_{12}}{a_{11}} x_1x_2 + ... + \frac{2a_{1n}}{a_{11}} x_1x_n\right) + a_{11}(\left( \frac{a_{12}}{a_{11}} x_2\right)^2  + ... + ...) -
		(...) + \varphi (x_2, ..., x_n) = a_{11}\left(x_1 + \frac{a_{12}}{a_{11}} x_2 + ... + \frac{a_{1n}}{a_{11}} x_n\right)^2 - \psi (x_2, ..., x_n) = 
		a_{11} y^2_1 + b_{22}z_1^2 + ... + b_{nn}z_n^2$
	\end{enumerate}
\end{proof}

\Subsubsection{Квадратичная форма над $\R$}

$\lambda_1 x_1^2 + ... + \lambda_nx_n^2$\\
Если находимся над полем $\C$, тогда $y_i = \sqrt{\lambda_i}x_i \Rightarrow y_1^2+...+y_r^2, r$ -- ранг формы, $r \leqslant n$\\
Над $\R$ ситуация иная. $\lambda_i > 0 \ \ y_i = \sqrt{\lambda_i}x_i, \ \ \lambda_j < 0 \ \ y_i = \sqrt{-\lambda_j}x_j \Rightarrow\\
y^2_1 + ... + y^2_s - y^2_{s+1} - ... - y^2_r$ 

\begin{Def} 
	Говорят, что квадратичная форма приведена к нормальному виду, если она представляет собой сумму чистых квадратов $(y^2_1 + ... + y^2_s - y^2_{s+1} - ... - y^2_r)$. 
\end{Def} 

\begin{Def} 
	Ранг квадратичной формы равен рангу соответствующей матрицы.\\
	$\rank q = \rank A$
\end{Def} 

\begin{Thm}[Закон инерции квадратичных форм] 
	$q: V \to \R$ -- квадратичная форма.\\ $\dim V = n, \rank q = r$\\
	Параметры $s$ и $r - s$ при приведении квадратичной формы к нормальному виду не зависят от базиса. 
\end{Thm} 

\begin{proof}
	$A$ -- матрица квадратичной формы в базисе $(e_i) \Rightarrow C^TAC$ -- матрица квадратичной формы в базисе $(e_i')$, где $C$ -- матрица перехода от $e_i$ к $(e_i'), \det C \neq 0$
	Несложно показать, что количество линейно независимых строк одинаково у $A$ и $C^TAC$. $\rank A = \rank C^TAC = r$ (было доказано)\\
	Предположим, что в базисе $(e_i)$ квадратичная форма имеет следующий вид: $q = x_1^2 + ... + x_s^2 - x^2_{s+1} - ... - x^2_r$\\
	А в базисе $(e_i'): q = x_q^2 + ... + x^2_t - x^2_{t+1} - ... - x^2_r$\\
	Предположим, что $t<s$
	Рассмотрим два подпространства пространства $V: U_1 = <e_1, ..., e_s> \ \ U_2 = <e'_{t+1}, ..., e_n'>$\\
	Рассмотрим размерность подпространства $U_1 + U_2:\\ \dim (U_1 + U_2) \leqslant n$\\
	С другой стороны $\dim (U_1 + U_2) = \dim U_1 + \dim U_2 - \dim (U_1 \cap U_2) = s + n- t - \dim (U_1 \cap U_2) \Rightarrow \dim (U_1 \cap U_2) \geqslant
	s + n - t - n = s - t > 0 \Rightarrow x \in U_1 \cap U_2, \ \ q(x) > 0, x \in U_1; \ \ q(x) < 0, x \in U_2 \Rightarrow$ противоречие
\end{proof}

\begin{Def} 
	Предположим, что квадратичная форма приведена. Тогда числа $s$ и $r-s$ называются индексами инерции или положительным и отрицательным индексами инерции. 
	А пары чисел $(s, r-s)$ -- сигнатура квадратичной формы. 
\end{Def} 

\begin{Rem}[Мотивация изучения квадратичных форм]
	Квадратичные формы нужны, чтобы исследовать экстремумы функций. $f(x) - f(x_0) = \sum f'_{x_i} \Delta x_i + \sum f''_{x_ix_j} \Delta x_i \Delta x_j + ...$ 
\end{Rem}

\begin{Def} 
	Всё рассматриваем над $\R$\\
	$q: V \to \R$ называется 
	\begin{enumerate}
		\item Положительно определенной, если $q(x) > 0 \ \forall x \neq 0, x \in V$
		\item Отрицательно определенной, если $q(x) < 0 \ \forall x \neq 0$
		\item Положительно полуопределенной, если $q(x) \geqslant 0 \ \forall x$
		\item Отрицательной полуопределенной, если $q(x) \leqslant 0 \ \forall x$
		\item Неопределенной, если $q(x) \cdot q(y) < 0 \ \ \exists x, y \in V$
	\end{enumerate}
\end{Def} 

\begin{Example}
	$n = 2$
	\begin{enumerate}
		\item $x^2 + y^2$
		\item $-x^2 - y^2$
		\item $x^2 - 2xy + y^2$
		\item --
		\item $x^2 - y^2$
	\end{enumerate}	
\end{Example}

\Subsubsection{Теорема Якоби}

\begin{Def} 
	$A = \left(
		\begin{array}{cccc}
			a_{11} & ... &  a_{1n}\\
			\vdots & \ddots & \vdots\\
			a_{n1} & ... & a_{\nu}
		\end{array}
	\right)$

	$\Delta_1 = a_{11}, \Delta_2 = \left| \begin{array}{cc}
		a_{11} & a_{12}\\
		a_{21} & a_{22}
	\end{array} \right|, ..., \Delta N = \left| \begin{array}{ccc}
		a_{11} & ... & a_{1n}\\
		\vdots & \ddots & \vdots\\
		a_{n1} & ... & a_{nn}
	\end{array} \right| -$ Главные миноры\\
	$\Delta_0 = 1$
\end{Def} 

\begin{Def} 
	$char K \neq 2\\
	f(x,y) = \frac{1}{2} (q(x+y) - q(x) - q(y))$ называется билинейной формой, полученной поляризацией квадратичной формы $q$
\end{Def} 

\begin{Ex}
	Показать, что $f(x, y)$ -- билинейная форма\\
	$q(x) = f(x,x) \ \ q(ax) = a^2 q(x)$
\end{Ex}

\begin{Def} 
	Если $q$ положительно/отрицательно определена/полуопределена, то её поляризация $f(x, y)$ называется положительно/отрицательно определенной/полуопределенной 
\end{Def} 

\begin{Def}
	Матрица $A$ называется положительно определенной, если соответствующая ей билинейная форма положительно определена. 
\end{Def}

\begin{Thm} 
	Матрица $A$ (Над $\R$) -- положительно определенная $\Leftrightarrow \exists$ невырожденная $C: A = C^T \cdot C$
\end{Thm} 

\begin{proof}
	A -- положительно определена $\Leftrightarrow$ соответствующая ей билинейная форма $f(x,y)$ положительно определена,  $q(x)$ -- положительно определена.
	$\Leftrightarrow \exists$ базис: матрица квадратичной формы $q$ -- E $(x^2_1 + ... + x^2_n) \Leftrightarrow \exists$ матрица перехода $C: E = C^TAC \Leftrightarrow A = (C^T)^{-1} \cdot C^{-1}$ 
\end{proof}

\begin{Thm}[Якоби, критерий положительной определенности] 
	Теорема верна для любого поля, но в основном мы находимся над $\R$\\
	$q: V \to K, char K \neq 2, q \to A, \Delta_i \neq 0, i = 1, ..., n  \Rightarrow \exists$ базис $(e_i'): q(x) = \frac{\Delta_0}{\Delta_1} x_1^2 + \frac{\Delta_1}{\Delta_2} x_2^2 + ... + \frac{\Delta_{n-1}}{\Delta_n} x^2_n$\\
\end{Thm} 

\begin{proof}
	По индукции.\\
	$n=1 q(x) = a_{11} x^2_1 = \frac{1}{a_{11}} (a_{11}x_1)^2$\\
	Индукционный переход: $n-1 \to n$\\
	$(e_i), i = 1, ..., n$ -- исходный произвольный базис $U = <e_1, ..., e_{n-1}> \subset V$\\
	$\overline{q} = q \big|_U, \overline{A}$ -- матрица $A$, в которой вычеркнули последнюю строчку и столбец.\\
	Для $\overline{q}$ утверждение верно. 
	Заметим, что $\overline{\Delta_1} = \Delta_1, ..., \overline{\Delta_{n-1}} = \Delta_{n-1} (\overline{\Delta_i})$ -- главные миноры для $\overline{A} \Rightarrow
	\exists (e_i'), i = 1, ..., n-1: \overline{q} = \frac{\Delta_0}{\Delta_1} x_1'^2 + ... + \frac{\Delta_{n-2}}{\Delta_{n-1}} x_{n-1}'^2$\\
	Возвращаемся в пространство $V$, ищем вектор x $\{ f(x, e_i') = 0, i = 1,..,n-1$ -- система линейных уравнений, n-1 уравнение, n неизвестных.
	$\rank$ СЛУ < n $\Rightarrow \exists$ нетривиальное решение $\Rightarrow \exists \widetilde{e_n}$ -- решение СЛУ. 
	На данный момент имеем, что $q$ почти нужна к нужному виду, но последний коэффициент неизвестный. 
	$q = \frac{\Delta_0}{\Delta_1} x_1'^2 + ... + \frac{\Delta_{n-2}}{\Delta_{n-1}} x_{n-1}'^2 + ? x_n'^2$\\
	$e_n' = \lambda \widetilde{e_n} \ \ \lambda$ выберем: $C$ -- матрица перехода от $(e_i))\\
	K(e_i', \widetilde{e_n}) \ \ \det C$ -- линейно зависит от $\lambda\\ 
	\begin{cases}
		e_1 = ...\\
		...\\
		\lambda \widetilde{e_n} = \lambda ... \lambda	
	\end{cases}\\
	\lambda: \det C = \frac{1}{\det A} = \frac{1}{\Delta_n}$\\
	В базисе $(e_i'), i = 1, ..., n \ \ A'$ -- матрица квадратичной формы $q$ -- диагональная\\
	$\frac{f(e_n', e_n')}{\Delta_{n-1}} = \frac{\Delta_0}{\Delta_1} \cdot \frac{\Delta_1}{\Delta_2} \cdot ... \cdot \frac{\Delta_{n-1}}{\Delta_{n-1}} \cdot f(e_n', e_n') = 
	\det A' = \det (C^TAC) = (\det C)^2 \cdot \det A = \frac{1}{\Delta^2_n} \cdot \Delta_n = \frac{1}{\Delta_n} \Rightarrow f(e_n', e_n') = \frac{\Delta_n-1}{\Delta_n} $
\end{proof}

\begin{Cons}
	$q: V \to \R$ -- квадратичная форма\\
	Тогда отрицательный индекс инерации $q$ равен числу перемен знака в последовательности $\Delta_0, \Delta_1, ..., \Delta_n$ 
\end{Cons}

\begin{Thm}[Критерий Сильвестра] 
	$q: V \to \R$\\
	$q$ -- положительно  определена $\Leftrightarrow \Delta_i > 0, i = 1,...,n$
\end{Thm} 

\begin{proof}
	$\Leftarrow.$ Очевидно.\\
	$\Rightarrow$. По индукции.\\
	$n=1 \ q = a_{11}x^2_1, a_{11} > 0$\\
	Индукционный переход: $n-1 \to n \ \ U = <e_1, ..., e_{n-1}>$\\
	$\overline{q} = q \big|_U$ -- положительно определена $\Rightarrow \Delta_i > 0, i = 1, ..., n-1$\\
	$q$ -- положительно определена $\Rightarrow A$ --положительно определена $\Rightarrow A = C^TC \Rightarrow \Delta_n = \det A = (det C)^2 > 0$ 
\end{proof}

\Subsubsection{Ортогональные преобразования}
\begin{Def} 
	$X = CY$ -- замена переменных. Соответствует переходу от одного базиса к другому. 
\end{Def} 

\begin{Def} 
	Матрица $C$ называется ортогональной, если она обладает следующим свойством $C^TC=E \Leftrightarrow \sum_{k=1}^{n} c_{ik} c_{jk} = \begin{cases}
		1, i = j\\
		0, i \neq j
	\end{cases} \Leftrightarrow \sum_{k=1}^{n} c_{ki}c_{kj} = \begin{cases}
		1, i = j\\
		0, i \neq j
	\end{cases}$
\end{Def} 

\begin{Example}
	$\left(\begin{array}{cc}
		\frac{1}{2} & -\frac{\sqrt{3}}{2}\\
		\frac{\sqrt{3}}{2} & \frac{1}{2}
	\end{array} \right), \ \
	\frac{1}{3} \left(
	\begin{array}{ccc}
		-1 & 2 & 2\\
		2 & -1 & 2\\ 
		2 & 2 & -1
	\end{array} \right)$
\end{Example}

\begin{Thm} 
	$\forall q: V \to \R \ \exists$ ортогональное преобразование $C:$\\
	$q = \lambda_1 x_1^2 + ... + \lambda_n x_n^2, \lambda_i$ -- собственные числа матрицы $A$
\end{Thm} 

\end{document}