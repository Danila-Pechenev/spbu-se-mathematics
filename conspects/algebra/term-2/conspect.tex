\documentclass[12pt]{article}

% Автор: Илья Дудников
% Автор стиля: Сергей Копелиович

\usepackage{cmap}
\usepackage[T2A]{fontenc}
\usepackage[utf8]{inputenc}
\usepackage[russian]{babel}
\usepackage{graphicx}
\usepackage{amsthm,amsmath,amssymb}
\usepackage{listings}
\usepackage{color}
\usepackage{xcolor}
\usepackage{array}
\usepackage{epigraph}
\usepackage{multicol}

\usepackage[russian,colorlinks=true,urlcolor=red,linkcolor=blue]{hyperref}
\usepackage{enumerate}
\usepackage{datetime}
\usepackage{fancyhdr}
\usepackage{lastpage}
\usepackage{verbatim}
\usepackage{tikz}
\usepackage{MnSymbol}
\usetikzlibrary{arrows,decorations.markings,decorations.pathmorphing}
\usepackage{pgfplots}

\usepackage{ifthen}
\usepackage{mathtools}

%\usepackage{tabls}
%\usepackage{tabularx}
%\usepackage{xifthen}
%\listfiles

\def\NAME{Лекции}
\def\SEASON{Конспект лекций по матанализу, ПИ, 1 семестр}

\sloppy
\voffset=-20mm
\textheight=235mm
\hoffset=-22mm
\textwidth=180mm
\headsep=12pt
\footskip=20pt

\parskip=0em
\parindent=0em

\setlength\epigraphwidth{.8\textwidth}

\newlength{\tmplen}
\newlength{\tmpwidth}
\newcounter{listcounter}

% Список с маленькими отступами
\newenvironment{MyList}[1][4pt]{
  \begin{enumerate}[1.]
  \setlength{\parskip}{0pt}
  \setlength{\itemsep}{#1}
}{       
  \end{enumerate}
}
% Вложенный список с маленькими отступами
\newenvironment{InnerMyList}[1][0pt]{
  \vspace*{-0.5em}
  \begin{enumerate}[(a)]
  \setlength{\parskip}{-0pt}
  \setlength{\itemsep}{#1}
}{       
  \end{enumerate}
  \vspace*{-0.5em}
}
% Список с маленькими отступами
\newenvironment{MyItemize}[1][4pt]{
  \begin{itemize}
  \setlength{\parskip}{0pt}
  \setlength{\itemsep}{#1}
}{       
  \end{itemize}
}

% Основные математические символы
\def\TODO{{\color{red}\bf TODO}}
\def\C{\mathbb{C}}       %
\def\Q{\mathbb{Q}}       %
\def\N{\mathbb{N}}       %
\def\R{\mathbb{R}}       %
\def\F2{\mathbb{F}_2}    %
\def\Z{\mathbb{Z}}       %
\def\INF{\t{+}\infty}    % +inf
\def\EPS{\varepsilon}    %
\def\EMPTY{\varnothing}  %
\def\PHI{\varphi}        %
\def\SO{\Rightarrow}     % =>
\def\EQ{\Leftrightarrow} % <=>
\def\t{\texttt}          % mono font
\def\c#1{{\rm\sc{#1}}}   % font for classes NP, SAT, etc
\def\O{\mathcal{O}}      %
\def\NO{\t{\#}}          % #
\def\XOR{\text{ {\raisebox{-2pt}{\ensuremath{\Hat{}}}} }}
\renewcommand{\le}{\leqslant}
\renewcommand{\ge}{\geqslant}
\newcommand{\q}[1]{\langle #1 \rangle}               % <x>
\newcommand\URL[1]{{\footnotesize{\url{#1}}}}        %
% \newcommand{\sfrac}[2]{{\scriptscriptstyle\frac{#1}{#2}}}  % Очень маленькая дробь
% \newcommand{\mfrac}[2]{{\scriptstyle\frac{#1}{#2}}}    % Небольшая дробь
\newcommand{\sfrac}[2]{{\scriptstyle\frac{#1}{#2}}}  % Очень маленькая дробь
\newcommand{\mfrac}[2]{{\textstyle\frac{#1}{#2}}}    % Небольшая дробь

\newcommand{\fix}[1]{{\color{fixcolor}{#1}}} % \underline
\def\bonus{\t{\red{(*)}}}
\def\ifbonus#1{\ifthenelse{\equal{#1}{}}{}{\bonus}}
\def\smallsquare{$\scalebox{0.5}{$\square$}$}

\newlength{\myItemLength}
\setlength{\myItemLength}{0.3em}
\def\ItemSymbol{\smallsquare}
\def\Item{\vspace*{\myItemLength}\ItemSymbol \ \ }

\newcommand{\LET}{%
  % [line width=0.6pt]
  \begin{tikzpicture}%
  \draw(0.8ex,0) -- (0.8ex,1.6ex);%
  \draw(0,1.6ex) -- (0.8ex,1.6ex);%
  \end{tikzpicture}%
  \hspace*{0.1em}%
}

% Отступы
\def\makeparindent{\hspace*{\parindent}\unskip}
\def\up{\vspace*{-0.5em}}%{\vspace*{-\baselineskip}}
\def\down{\vspace*{0.5em}}
\def\LINE{\vspace*{-1em}\noindent \underline{\hbox to 1\textwidth{{ } \hfil{ } \hfil{ } }}}
\def\BOX#1{\mbox{\fbox{\bf{#1}}}}
\def\Pagebreak{\pagebreak\vspace*{-1.5em}}

% Мелкий заголовок
\newcommand{\THEE}[1]{
  \vspace*{0.5em}
  \noindent{\bf \underline{#1}}%\hspace{0.5em}
  \vspace*{0.2em}
}
% Другой тип мелкого заголовка
\newcommand{\THE}[1]{
  \vspace*{0.5em} $\bullet$
  \noindent{\bf #1}%\hspace{0.5em}
  \vspace*{0.2em}
}

\newenvironment{MyTabbing}{
  \t\bgroup
  \vspace*{-\baselineskip}
  \begin{tabbing}
    aaaa\=aaaa\=aaaa\=aaaa\=aaaa\=aaaa\kill
}{
  \end{tabbing}
  \t\egroup
}

% Код с правильными отступами
\lstnewenvironment{code}{
  \lstset{}
%  \vspace*{-0.2em}
}%
{
%  \vspace*{-0.2em}
}
\lstnewenvironment{codep}{
  \lstset{language=python}
}%
{
}

% Формулы с правильными отступами
\newenvironment{smallformula}{
 
  \vspace*{-0.8em}
}{
  \vspace*{-1.2em}
  
}
\newenvironment{formula}{
 
  \vspace*{-0.4em}
}{
  \vspace*{-0.6em}
  
}

% Большая квадратная скобка
\makeatletter
\newenvironment{sqcases}{%
  \matrix@check\sqcases\env@sqcases
}{%
  \endarray\right.%
}
\def\env@sqcases{%
  \let\@ifnextchar\new@ifnextchar
  \left\lbrack
  \def\arraystretch{1.2}%
  \array{@{}l@{\quad}l@{}}%
}
\makeatother

% Определяем основные секции: \begin{Lm}, \begin{Thm}, \begin{Def}, \begin{Rem}
\renewcommand{\qedsymbol}{$\blacksquare$}
\theoremstyle{definition} % жирный заголовок, плоский текст
\newtheorem{Thm}{\underline{Теорема}}[subsection] % нумерация будет "<номер subsection>.<номер теоремы>"
\newtheorem{Lm}[Thm]{\underline{Lm}} % Нумерация такая же, как и у теорем
\newtheorem{Ex}[Thm]{Упражнение} % Нумерация такая же, как и у теорем
\newtheorem{Example}[Thm]{Пример} % Нумерация такая же, как и у теорем
\newtheorem{Code}[Thm]{Код} % Нумерация такая же, как и у теорем
\theoremstyle{plain} % жирный заголовок, курсивный текст
\newtheorem{Def}[Thm]{Def} % Нумерация такая же, как и у теорем
\theoremstyle{remark} % курсивный заголовок, плоский текст
\newtheorem{Cons}[Thm]{Следствие} % Нумерация такая же, как и у теорем
\newtheorem{Conj}[Thm]{Гипотеза} % Нумерация такая же, как и у теорем
\newtheorem{Prop}[Thm]{Утверждение} % Нумерация такая же, как и у теорем
\newtheorem{Rem}[Thm]{Замечание} % Нумерация такая же, как и у теорем
\newtheorem{Remark}[Thm]{Замечание} % Нумерация такая же, как и у теорем
\newtheorem{Algo}[Thm]{Алгоритм} % Нумерация такая же, как и у теорем

% Определяем ЗАГОЛОВКИ
\def\SectionName{unknown}
\def\AuthorName{unknown}

\newlength{\sectionvskip}
\setlength{\sectionvskip}{0.5em}
\newcommand{\Section}[4][]{
  % Заголовок
  \pagebreak
%  \ifthenelse{\isempty{#1}}{
    \refstepcounter{section}
%  }{}
  \vspace{0.5em}
%  \ifthenelse{\isempty{#1}}{
%    \addtocontents{toc}{\protect\addvspace{-5pt}}%
    \addcontentsline{toc}{section}{\arabic{section}. #2}
%  }{}
  \begin{center}
    {\Large \bf Раздел \NO{\arabic{section}}: #2} \\ 
    \vspace{\sectionvskip}
    \ifthenelse{\equal{#3}{}}{}{{\large #3}\\}
  \end{center}

  \LINE

  % Запомнили название и автора главы
  \gdef\SectionName{#2}
  \gdef\AuthorName{#4}

  % Заголовок страницы
  \lhead{\SEASON}
  \chead{}
  \rhead{\SectionName}
  \renewcommand{\headrulewidth}{0.4pt}

  \lfoot{Глава \NO{\arabic{section}}.}
  \cfoot{\thepage\t{/}\pageref*{LastPage}}
  \rfoot{Автор: \AuthorName}
  \renewcommand{\footrulewidth}{0.4pt}
}

\newcommand{\Subsection}[2][]{
  \refstepcounter{subsection}
  \vspace*{1em}
  \ifthenelse{\equal{#1}{}}
    {\addcontentsline{toc}{subsection}{\arabic{section}.\arabic{subsection}. #2}}
    {\addcontentsline{toc}{subsection}{\arabic{section}.\arabic{subsection}. \bonus\,#2}}
  {\color{blue}\bf\large \arabic{section}.\arabic{subsection}. \ifbonus{#1}\,{#2}} 
  \vspace*{0.5em}
  \makeparindent
}
\newcommand{\Subsubsection}[2][]{
  \refstepcounter{subsubsection}
  \vspace*{1em}
  \ifthenelse{\equal{#1}{}}
    {\addcontentsline{toc}{subsubsection}{\arabic{section}.\arabic{subsection}.\arabic{subsubsection}. #2}}
    {\addcontentsline{toc}{subsubsection}{\arabic{section}.\arabic{subsection}.\arabic{subsubsection}. \bonus\,#2}}
  {\color{blue}\bf\large \arabic{section}.\arabic{subsection}.\arabic{subsubsection}. \ifbonus{#1}\,#2}
  \vspace*{0.5em}
  \makeparindent
}

\newcommand{\Header}{
  \pagestyle{empty}
  \renewcommand{\dateseparator}{--}
  \begin{center}
    {\Large\bf 
     Матанализ 2 семестр ПИ,\\
    \vspace{0.3em}
    \NAME}\\
    \vspace{0.7em}
    {Собрано {\today} в {\currenttime}}
  \end{center}

  \LINE
  \vspace{0em}

  \renewcommand{\baselinestretch}{0.98}\normalsize
  \tableofcontents
  \renewcommand{\baselinestretch}{1.0}\normalsize
  \pagebreak
}

\newcommand{\BeginConspect}{
  \pagestyle{fancy}
  \setcounter{page}{1}
}

\definecolor{mygray}{rgb}{0.7,0.7,0.7}
\definecolor{ltgray}{rgb}{0.9,0.9,0.9}
\definecolor{fixcolor}{rgb}{0.7,0,0}
\definecolor{red2}{rgb}{0.7,0,0}
\definecolor{dkred}{rgb}{0.4,0,0}
\definecolor{dkblue}{rgb}{0,0,0.6}
\definecolor{dkgreen}{rgb}{0,0.6,0}
\definecolor{brown}{rgb}{0.5,0.5,0}

\newcommand{\green}[1]{{\color{green}{#1}}}
\newcommand{\black}[1]{{\color{black}{#1}}}
\newcommand{\red}[1]{{\color{red}{#1}}}
\newcommand{\dkred}[1]{{\color{dkred}{#1}}}
\newcommand{\blue}[1]{{\color{blue}{#1}}}
\newcommand{\dkgreen}[1]{{\color{dkgreen}{#1}}}

\newcommand{\Mod}[1]{\ (\mathrm{mod}\ #1)}

\DeclareMathOperator{\Real}{Re}
\DeclareMathOperator{\Imag}{Im}
\DeclareMathOperator{\lcm}{lcm}
\DeclareMathOperator{\sign}{sign}
\DeclareMathOperator{\Si}{Si}
\DeclareMathOperator{\const}{const}

\begin{document}
	\Header
	\BeginConspect

	\Section{Системы линейных уравнений}{}{Илья Дудников}

	\[ (*)\begin{cases}
		a_{11} x_1 + a_{12} x_2 + ... + a_{1n} x_n = b_1 \\ 
		a_{21} x_1 + a_{22} x_2 + ... + a_{2n} x_n = b_2 \\ 
		\cdots\\
		a_{m1} x_1 + a_{m2} x_2 + ... + a_{mn} x_n = b_n
	\end{cases}\]
	$A = (a_{ij})$ -- матрица коэффициентов, $X = \begin{pmatrix}
	x_1 \\ 
	x_2 \\ 
	\vdots \\ 
	x_n
	\end{pmatrix}$, $B = \begin{pmatrix}
	b_1 \\ 
	b_2 \\ 
	\vdots \\ 
	b_n
	\end{pmatrix}$.  

	\begin{Def}
		Решение СЛУ $(*)$ называется $\alpha_1, ..., \alpha_n \in K : $ при $x_i = \alpha_i$ все уравнения становятся верными.
	\end{Def}

	\begin{Def}
		СЛУ $(*)$ совместна, если $\exists$ хотя бы одно решение. Иначе - несовместна.
	\end{Def}

	\Subsection{Ранг матрицы}

	$A - m \times n, A = (A_1, A_2, ..., A_m), A_i$ -- строки. \\
	$A = (A^1, A^2, ..., A^n), A^j$ -- столбцы.
	
	\begin{Def}
		Строчным (столбцовым) рангом матрицы $A$ называется максимальное число ЛНЗ строк (столбцов). \\
		Иначе, количество элементов в базисе $\langle A_1, ..., A_m\rangle (\langle A^1, ..., A^n\rangle)$. 
	\end{Def}

	\begin{Thm}
		Строчный и столбцовый ранги совпадают.
	\end{Thm}

	Обозначение: $\rank A$.

	\begin{Def}
		Минором матрицы $A - m \times n$ $k$-го порядка называется определитель, 
		составленный из элементов матрицы $A$, стоящих на $k$ выбранных строках и на $k$ выбранных столбцов.  
	\end{Def}

	\begin{Example}
		$\left(\begin{array}{cccc}
		1 & 4 & 8 & -3 \\ 
		2 & 5 & 9 & -4 \\ 
		3 & 6 & -2 & -5
		\end{array}\right)$. Если вы выберем вторую и третью строку, а также первый и последний столбец, то минор второго порядка:
		\[\left|\begin{array}{cc}
		2 & -4 \\ 
		3 & -5
		\end{array}\right|\]
	\end{Example}

	\begin{Thm}
		Ранг матрицы $A$ равен наибольшему порядку минора, отличного от нуля.
	\end{Thm}

	\begin{Thm}[Связь определителя с рангом матрицы]
		$A - n \times n$. Тогда $\rank A < n \EQ \det A = 0$.
	\end{Thm}

	\begin{proof}
		$\SO$. $\rank A < n \SO$ строки $A_1, ..., A_n$ ЛЗ, т.е.
		$\exists \alpha_1, ..., \alpha_n \in K : \alpha_1 A_1 + \alpha_2 A_2 + ... + \alpha_n A_n = 0$ ( $\alpha_i$ не все равны нулю). 
		Пусть $\alpha_1 \neq 0 \SO A_1 = - \frac{\alpha_2}{\alpha_1} A_2 - ... - \frac{\alpha_n}{\alpha_1} A_n$.
		Обнулим первую строку: прибавим к ней $A_2$, умноженную на $-\frac{\alpha_2}{\alpha_1}$, $A_3$, умноженную на $-\frac{\alpha_3}{\alpha_1}$ и т.д.
		Поскольку теперь первая строка целиком нулевая, то $\det A = 0$. \\

		$\Leftarrow$. Индукция $n = 1 \SO a_{11} = 0$. $n - 1 \to n$.
		\[\left|\begin{array}{cccc}
		a_{11} & a_{12} & \cdots & a_{1n} \\ 
		a_{21} & a_{22} & \cdots & a_{2n} \\ 
		\vdots & \vdots & \ddots & \vdots \\ 
		a_{n1} & a_{n2} & \cdots & a_{nn}
		\end{array}\right| = \]
		Можем считать, что $A^1 \neq 0, a_{11} \neq 0$. Домножим первую строку на $- \frac{a_{21}}{a_{11}}$ и прибавляем ко второй строке.
		Затем домножаем первую строку на $-\frac{a_{31}}{a_{11}}$ и прибавляем ко третьей строке и т.д.  
		\[= \left|\begin{array}{cccc}
		a_{11} & a_{12} & \cdots & a_{1n} \\ 
		0 & a_{22}' & \cdots & a_{2n}' \\ 
		\vdots & \vdots & \ddots & \vdots \\ 
		0 & a_{n2}' & \cdots & a_{nn}'
		\end{array}\right| = a_{11} \cdot \left|\begin{array}{ccc}
		a_{22}' & \cdots & a_{2n}' \\ 
		\vdots & \ddots & \vdots \\ 
		a_{n2}' & \cdots & a_{nn}'
		\end{array}\right|\]
		По предположению $A_2', ..., A_n'$ -- ЛЗ. $\begin{cases}
			A_2' = A_2 - \frac{a_{21}}{a_{11}} \cdot A_1 \\
			\cdots \\
			A_n' = A_n - \frac{a_{n1}}{a_{11}} \cdot A_1
		\end{cases}$. \\  
		$0 = \alpha_2 A_2' + ... + \alpha_n A_n' = (...) A_1 + \alpha_2 \cdot A_2 + ... + \alpha_n A_n \SO A_1, ..., A_n$ -- ЛЗ $\SO \rank A < n$.
	\end{proof}

	\begin{Def}
		Элементарными преобразованиями над строками (столбцами) называется
		\begin{MyList}
			\item Перестановка строк (столбцов).
			\item Умножение строки (столбца) на $\lambda \neq 0$.
			\item Прибавление к одной строке (столбцу) другой строки (столбца), умноженной на $\lambda \neq 0$.
		\end{MyList}
	\end{Def}

	\begin{Thm}
		При элементарных преобразованиях ранг матрицы не меняется.
	\end{Thm}

	\begin{proof}
		$1, 2$ -- очевидно.
		$(A_1, ..., A_i, ..., A_j, ..., A_n) \to (A_1, ..., A_i + \lambda A_j, ..., A_j, ..., A_n)$ 
	\end{proof}

	\begin{Def}
		Матрица называется трапецевидной, если у неё в $\forall$ ненулевой строке число нулей слева различно.
	\end{Def}

	\begin{Rem}
		$\rank$ трапецевидной матрицы равен числу ненулевых строк.
	\end{Rem}

	\begin{Thm}[О вычислении ранга]
		Любую матрицу с помощью элементарных преобразований можно привести к трапецевидной.
	\end{Thm}

	\Subsection{Структура решений СЛУ}

	\begin{Def}
		СЛУ (*) называется однородной, если все свободные члены равны нулю.
	\end{Def}

	\begin{Def}
		Нулевое решение однородной СЛУ называется тривиальным. Любое другое решение -- нетривиальным.
	\end{Def}

	\begin{Lm}
		Пусть $Y, Z$ -- решения $AX = 0 \SO \alpha Y + \beta Z$ -- тоже решение, $\alpha, \beta \in K$.
	\end{Lm}

	\begin{proof}
		\[AY = 0, AZ = 0 \SO A(\alpha Y + \beta Z) = \alpha AY + \beta AZ = 0\]
	\end{proof}

	\begin{Thm}[Структура решений однородной СЛУ]
		$AX = 0, A - m \times n, n$ -- число неизвестных, $r = \rank A \SO$
		$\exists n - r$ ЛНЗ решений $X_1, ..., X_{n - r} : \forall$ решение $Y = \alpha_1 X_1 + ... + \alpha_{n - r} X_{n - r}$.  
	\end{Thm}

	\begin{proof}
		$A = (A^1, ..., A^n)$, $A^1, ..., A^r$ -- ЛНЗ столбцы $\SO $
		\[
		\begin{cases}
			A_{r + 1} = \beta_{r + 1 \ 1}A^1 + ... + \beta_{r + 1 \ n} A^{r} \\
			\cdots \\
			A^n = \beta_{n \ 1}A^1 + ... + \beta_{n \ r}A^r
		\end{cases}
		\]
		$AX = 0 \EQ x_1 A^1 + x_2 A^2 + ... + x_n A^n = 0$. \\
		$X_1 = \begin{pmatrix}
		\beta_{r + 1 \ 1} \\ 
		\vdots \\ 
		\beta_{r + 1 \ r} \\ 
		-1 \\ 
		0 \\ 
		\vdots \\ 
		0
		\end{pmatrix}, X_2 = \begin{pmatrix}
		\beta_{r + 2 \ 1} \\ 
		\vdots \\ 
		\beta_{r + 2 \ r} \\ 
		0 \\ 
		-1 \\
		0 \\ 
		\vdots \\ 
		0
		\end{pmatrix}, ..., X_{n + r} = \begin{pmatrix}
		\beta_{n \ 1} \\ 
		\vdots \\ 
		\beta_{n \ r} \\ 
		0 \\ 
		\vdots \\ 
		-1 \\ 
		\end{pmatrix}$ -- решения. \\
		Пусть $Z = \begin{pmatrix}
		x_1^* \\ 
		\vdots \\ 
		x_r^* \\ 
		\vdots \\ 
		x_n^*
		\end{pmatrix}$ -- решение. Рассмотрим $Y = Z + x_{r + 1}^* X_1 + x_{r + 2}^* X_2 + ... + x_n^* X_{n - r}$. 
		$Y = \begin{pmatrix}
		y_1 \\ 
		\vdots \\ 
		y_r \\ 
		0 \\ 
		\vdots \\
		0
		\end{pmatrix}$ -- решение~$\{y_1 A_1 + ... + y_r A_r = 0\}$. Но $A_1, ..., A_r$ -- ЛНЗ $\SO Y = \begin{pmatrix}
		0 \\ 
		\vdots \\ 
		0
		\end{pmatrix} \SO 0 = Z + x_{r + 1}^* X_1 + x_{r + 2}^* X_2 + ... + x_n^* X_{n - r}$.
	\end{proof}
\end{document}