\documentclass[12pt]{article}

% Автор стиля: Сергей Копелиович
% Автор конспекта: Илья Дудников

\usepackage{cmap}
\usepackage[T2A]{fontenc}
\usepackage[utf8]{inputenc}
\usepackage[russian]{babel}
\usepackage{graphicx}
\usepackage{amsthm,amsmath,amssymb}
\usepackage{listings}
\usepackage{color}
\usepackage{xcolor}
\usepackage{array}
\usepackage{epigraph}

\usepackage[russian,colorlinks=true,urlcolor=red,linkcolor=blue]{hyperref}
\usepackage{enumerate}
\usepackage{datetime}
\usepackage{fancyhdr}
\usepackage{lastpage}
\usepackage{verbatim}
\usepackage{tikz}
\usetikzlibrary{arrows,decorations.markings,decorations.pathmorphing}
\usepackage{pgfplots}

\usepackage{ifthen}
\usepackage{mathtools}

%\usepackage{tabls}
%\usepackage{tabularx}
%\usepackage{xifthen}
%\listfiles

\def\NAME{Лекция, 10/01/21}

\sloppy
\voffset=-20mm
\textheight=235mm
\hoffset=-22mm
\textwidth=180mm
\headsep=12pt
\footskip=20pt

\parskip=0em
\parindent=0em

\setlength\epigraphwidth{.8\textwidth}

\newlength{\tmplen}
\newlength{\tmpwidth}
\newcounter{listcounter}

% Список с маленькими отступами
\newenvironment{MyList}[1][4pt]{
  \begin{enumerate}[1.]
  \setlength{\parskip}{0pt}
  \setlength{\itemsep}{#1}
}{       
  \end{enumerate}
}
% Вложенный список с маленькими отступами
\newenvironment{InnerMyList}[1][0pt]{
  \vspace*{-0.5em}
  \begin{enumerate}[(a)]
  \setlength{\parskip}{-0pt}
  \setlength{\itemsep}{#1}
}{       
  \end{enumerate}
  \vspace*{-0.5em}
}
% Список с маленькими отступами
\newenvironment{MyItemize}[1][4pt]{
  \begin{itemize}
  \setlength{\parskip}{0pt}
  \setlength{\itemsep}{#1}
}{       
  \end{itemize}
}

\DeclareMathOperator{\lcm}{lcm}

% Основные математические символы
\def\TODO{{\color{red}\bf TODO}}
\def\N{\mathbb{N}}       %
\def\R{\mathbb{R}}       %
\def\F2{\mathbb{F}_2}    %
\def\Z{\mathbb{Z}}       %
\def\INF{\t{+}\infty}    % +inf
\def\EPS{\varepsilon}    %
\def\EMPTY{\varnothing}  %
\def\PHI{\varphi}        %
\def\SO{\Rightarrow}     % =>
\def\EQ{\Leftrightarrow} % <=>
\def\t{\texttt}          % mono font
\def\c#1{{\rm\sc{#1}}}   % font for classes NP, SAT, etc
\def\O{\mathcal{O}}      %
\def\NO{\t{\#}}          % #
\def\XOR{\text{ {\raisebox{-2pt}{\ensuremath{\Hat{}}}} }}
\renewcommand{\le}{\leqslant}
\renewcommand{\ge}{\geqslant}
\newcommand{\q}[1]{\langle #1 \rangle}               % <x>
\newcommand\URL[1]{{\footnotesize{\url{#1}}}}        %
% \newcommand{\sfrac}[2]{{\scriptscriptstyle\frac{#1}{#2}}}  % Очень маленькая дробь
% \newcommand{\mfrac}[2]{{\scriptstyle\frac{#1}{#2}}}    % Небольшая дробь
\newcommand{\sfrac}[2]{{\scriptstyle\frac{#1}{#2}}}  % Очень маленькая дробь
\newcommand{\mfrac}[2]{{\textstyle\frac{#1}{#2}}}    % Небольшая дробь

\newcommand{\fix}[1]{{\color{fixcolor}{#1}}} % \underline
\def\bonus{\t{\red{(*)}}}
\def\ifbonus#1{\ifthenelse{\equal{#1}{}}{}{\bonus}}
\def\smallsquare{$\scalebox{0.5}{$\square$}$}

\newlength{\myItemLength}
\setlength{\myItemLength}{0.3em}
\def\ItemSymbol{\smallsquare}
\def\Item{\vspace*{\myItemLength}\ItemSymbol \ \ }

\newcommand{\LET}{%
  % [line width=0.6pt]
  \begin{tikzpicture}%
  \draw(0.8ex,0) -- (0.8ex,1.6ex);%
  \draw(0,1.6ex) -- (0.8ex,1.6ex);%
  \end{tikzpicture}%
  \hspace*{0.1em}%
}

% Отступы
\def\makeparindent{\hspace*{\parindent}\unskip}
\def\up{\vspace*{-0.5em}}%{\vspace*{-\baselineskip}}
\def\down{\vspace*{0.5em}}
\def\LINE{\vspace*{-1em}\noindent \underline{\hbox to 1\textwidth{{ } \hfil{ } \hfil{ } }}}
\def\BOX#1{\mbox{\fbox{\bf{#1}}}}
\def\Pagebreak{\pagebreak\vspace*{-1.5em}}

% Мелкий заголовок
\newcommand{\THEE}[1]{
  \vspace*{0.5em}
  \noindent{\bf \underline{#1}}%\hspace{0.5em}
  \vspace*{0.2em}
}
% Другой тип мелкого заголовка
\newcommand{\THE}[1]{
  \vspace*{0.5em} $\bullet$
  \noindent{\bf #1}%\hspace{0.5em}
  \vspace*{0.2em}
}

\newenvironment{MyTabbing}{
  \t\bgroup
  \vspace*{-\baselineskip}
  \begin{tabbing}
    aaaa\=aaaa\=aaaa\=aaaa\=aaaa\=aaaa\kill
}{
  \end{tabbing}
  \t\egroup
}

% Код с правильными отступами
\lstnewenvironment{code}{
  \lstset{}
%  \vspace*{-0.2em}
}%
{
%  \vspace*{-0.2em}
}
\lstnewenvironment{codep}{
  \lstset{language=python}
}%
{
}

% Формулы с правильными отступами
\newenvironment{smallformula}{
 
  \vspace*{-0.8em}
}{
  \vspace*{-1.2em}
  
}
\newenvironment{formula}{
 
  \vspace*{-0.4em}
}{
  \vspace*{-0.6em}
  
}

% Большая квадратная скобка
\makeatletter
\newenvironment{sqcases}{%
  \matrix@check\sqcases\env@sqcases
}{%
  \endarray\right.%
}
\def\env@sqcases{%
  \let\@ifnextchar\new@ifnextchar
  \left\lbrack
  \def\arraystretch{1.2}%
  \array{@{}l@{\quad}l@{}}%
}
\makeatother

% Определяем основные секции: \begin{Lm}, \begin{Thm}, \begin{Def}, \begin{Rem}
\renewcommand{\qedsymbol}{$\blacksquare$}
\theoremstyle{definition} % жирный заголовок, плоский текст
\newtheorem{Thm}{\underline{Теорема}}[subsection] % нумерация будет "<номер subsection>.<номер теоремы>"
\newtheorem{Lm}[Thm]{\underline{Lm}} % Нумерация такая же, как и у теорем
\newtheorem{Ex}[Thm]{Упражнение} % Нумерация такая же, как и у теорем
\newtheorem{Example}[Thm]{Пример} % Нумерация такая же, как и у теорем
\newtheorem{Code}[Thm]{Код} % Нумерация такая же, как и у теорем
\theoremstyle{plain} % жирный заголовок, курсивный текст
\newtheorem{Def}[Thm]{Def.} % Нумерация такая же, как и у теорем
\theoremstyle{remark} % курсивный заголовок, плоский текст
\newtheorem{Cons}[Thm]{Следствие} % Нумерация такая же, как и у теорем
\newtheorem{Conj}[Thm]{Гипотеза} % Нумерация такая же, как и у теорем
\newtheorem{Prop}[Thm]{Утверждение} % Нумерация такая же, как и у теорем
\newtheorem{Rem}[Thm]{Замечание} % Нумерация такая же, как и у теорем
\newtheorem{Remark}[Thm]{Замечание} % Нумерация такая же, как и у теорем
\newtheorem{Algo}[Thm]{Алгоритм} % Нумерация такая же, как и у теорем

% Определяем ЗАГОЛОВКИ
\def\SectionName{Алгебра}
\def\AuthorName{Илья Дудников}

\newlength{\sectionvskip}
\setlength{\sectionvskip}{0.5em}
\newcommand{\Section}[4][]{
  % Заголовок
  \pagebreak
%  \ifthenelse{\isempty{#1}}{
    \refstepcounter{section}
%  }{}
  \vspace{0.5em}
%  \ifthenelse{\isempty{#1}}{
%    \addtocontents{toc}{\protect\addvspace{-5pt}}%
    \addcontentsline{toc}{section}{\arabic{section}. #2}
%  }{}
  


  % Запомнили название и автора главы
  \gdef\SectionName{#2}
  \gdef\AuthorName{#4}

  % Заголовок страницы
  \lhead{Конспект, \NAME}
  \chead{}
  \rhead{\SectionName}
  \renewcommand{\headrulewidth}{0.4pt}
    
  \lfoot{}
  \cfoot{\thepage\t{/}\pageref*{LastPage}}
  \rfoot{Автор: \AuthorName}
  \renewcommand{\footrulewidth}{0.4pt}
}

\newcommand{\Subsection}[2][]{
  \refstepcounter{subsection}
  \vspace*{1em}
  \ifthenelse{\equal{#1}{}}
    {\addcontentsline{toc}{subsection}{\arabic{section}.\arabic{subsection}. #2}}
    {\addcontentsline{toc}{subsection}{\arabic{section}.\arabic{subsection}. \bonus\,#2}}
  {\color{blue}\bf\large \arabic{section}.\arabic{subsection}. \ifbonus{#1}\,{#2}} 
  \vspace*{0.5em}
  \makeparindent
}
\newcommand{\Subsubsection}[2][]{
  \refstepcounter{subsubsection}
  \vspace*{1em}
  \ifthenelse{\equal{#1}{}}
    {\addcontentsline{toc}{subsubsection}{\arabic{section}.\arabic{subsection}.\arabic{subsubsection}. #2}}
    {\addcontentsline{toc}{subsubsection}{\arabic{section}.\arabic{subsection}.\arabic{subsubsection}. \bonus\,#2}}
  {\color{blue}\bf\large \arabic{section}.\arabic{subsection}.\arabic{subsubsection}. \ifbonus{#1}\,#2}
  \vspace*{0.5em}
  \makeparindent
}

\newcommand{\Header}{
  \pagestyle{empty}
  \renewcommand{\dateseparator}{--}
  \begin{center}
    {\Large\bf 
     Алгебра 1 семестр ПИ,\\
    \vspace{0.3em}
     \NAME}\\
    \vspace{0.7em}
    {Собрано {\today} в {\currenttime}}
  \end{center}

  \LINE
  \vspace{0em}

  \renewcommand{\baselinestretch}{0.98}\normalsize
  \tableofcontents
  \renewcommand{\baselinestretch}{1.0}\normalsize
  \pagebreak
}

\newcommand{\BeginConspect}{
  \pagestyle{fancy}
  \setcounter{page}{1}
}

\definecolor{mygray}{rgb}{0.7,0.7,0.7}
\definecolor{ltgray}{rgb}{0.9,0.9,0.9}
\definecolor{fixcolor}{rgb}{0.7,0,0}
\definecolor{red2}{rgb}{0.7,0,0}
\definecolor{dkred}{rgb}{0.4,0,0}
\definecolor{dkblue}{rgb}{0,0,0.6}
\definecolor{dkgreen}{rgb}{0,0.6,0}
\definecolor{brown}{rgb}{0.5,0.5,0}

\newcommand{\green}[1]{{\color{green}{#1}}}
\newcommand{\black}[1]{{\color{black}{#1}}}
\newcommand{\red}[1]{{\color{red}{#1}}}
\newcommand{\dkred}[1]{{\color{dkred}{#1}}}
\newcommand{\blue}[1]{{\color{blue}{#1}}}
\newcommand{\dkgreen}[1]{{\color{dkgreen}{#1}}}

\begin{document}

\Header

\BeginConspect

\Section{Основы теории чисел}{Илья Дудников}

\Subsection{Наименьшее общее кратное}

\begin{Def}
    Общим кратным $a_1, a_2, ..., a_n$ называется число $M > 0 : a_i | M \ \forall i = 1, ..., n$ 
    
    Наименьшее из общих кратных -- НОК.
\end{Def}

\begin{Thm}
    $\lcm (a, b) = \frac{ab}{\gcd (a, b)}$ 
\end{Thm}

\begin{proof}
    $a = a_1 d, b = b_1 d, (a_1, b_1) = 1$
    \[M = at = bs \SO \frac{M}{b} = \frac{at}{b} = \frac{a_1 dt}{b_1 d} = \frac{a_1 t}{b_1} \SO M = \frac{b \cdot a_1 t}{b_1}\]
    $t$ делится на $b_1$, т.е. $t = b_1 k$
    \[M = \frac{b a_1 b_1 k}{b_1} = b a_1 k \text{ - минимально при } k = 1 \SO M = b a_1 = \frac{b a_1 d}{d} = \frac{ab}{\gcd(a, b)}\]  
\end{proof}

\Subsection{Математическая индукция}
\begin{MyList}
    \item Аксоима. $\forall$ подмножество $\N$ имеет наши элементы $\SO$ ММИ.
    \item Аксиома. $A_1, A_n \SO A_{n + 1} \SO \forall A_n$ 
\end{MyList}

\begin{Cons}
    Пусть $a_1, a_2, ..., a_n$ -- попарно взаимно-простые $\SO \lcm(a_1, a_2, ..., a_n) = a_1 \cdot a_2 \cdot ... \cdot a_n$  
\end{Cons}

\begin{proof}
    $n = 2$. $\lcm (a_1, a_2) = \frac{a_1 a_2}{\gcd(a_1, a_2)} = a_1 \cdot a_2$
    
    Пусть верно для $n$. Тогда для $n + 1$
    \[(a_i, a_n a_{n + 1}) = (a_i, a_{n + 1}) = 1 \SO a_1, a_2, ..., a_{n - 1}, a_n a_{n + 1} \SO\] 
    \[\SO \lcm (a_1, ..., a_{n - 1}, a_n \cdot a_{n + 1}) = a_1 \cdot a_2 \cdot ... \cdot a_{n - 1} \cdot a_n \cdot a_{n + 1}\]  
\end{proof}

\Subsection{Простые числа}

\begin{Def}
    Число $p > 1$ называется простым, если оно делится только на $1$ и на $p$. Иначе число называется составным.
\end{Def}

\begin{Thm}[о наименьшем делителе]
    Наименьший делитель $a > 1$ -- простое число
\end{Thm}

\begin{proof}
    $M = \left\{d | d > 1, d | a\right\} \neq \varnothing$ Пусть $p$ - наименьший элемент $M$.
    Предположим, что $p$ -- составное, т.е. $p = bq, q < p, q | p, p | a \SO q | a$ -- противоречие. 
\end{proof}
\Pagebreak

\begin{Thm}
    $p$ - наименьший делитель $ > 1$ числа $n \SO p \leqslant \sqrt{n}$  
\end{Thm}

\begin{proof}
    \[n = mp, p \leqslant m \SO np \leqslant nm \SO mp \cdot p \leqslant nm \SO p^2 \leqslant n \SO p \leqslant \sqrt{n}\]
\end{proof}

\begin{Thm}[Теорема Евклида]
    Простых чисел бесконечно много
\end{Thm}

\begin{proof}
    Пусть $p_1, p_2, ..., p_n$ -- все простые числа, $a = p_1 \cdot p_2 \cdot ... \cdot p_n + 1$.
    Если $a \vdots p_i$, то $1 \vdots p_i \SO a$ -- новое простое число.  
\end{proof}

\Subsection{Основная теорема арифметики}
\begin{Lm}
    $p$ -- простое $\SO \forall a > 1 \to p | a \vee (p, a) = 1$ 
\end{Lm}

\begin{proof}
    \[(p, a) | p \SO (p, a) = 1 \vee (p, a) = p\] 
\end{proof}

\begin{Lm}
    $p$ -- простое, $p | a_1 \cdot a_2 \cdot ... \cdot a_n \SO \exists i = 1, ..., n : p | a_i$ 
\end{Lm}

\begin{proof}
    Если $(p, a_i) = 1, i = 1, ..., n \SO 1 = (p, a_1) = (p, a_1 a_2) = (p, a_1 a_2 a_3) = (p, a_1 \cdot ... \cdot a_n) = 1 \SO \exists a_i : p | a_i$ 
\end{proof}

\begin{Thm}[Основная теорема арифметики]
    \begin{MyList}
        \item $\forall a > 1 \to a = p_1^{\alpha_1} \cdot p_2^{\alpha_2} \cdot ... \cdot p_k^{\alpha_k}, p_1, p_2, ..., p_k$ -- различные простые, $\alpha_1, \alpha_2, ..., \alpha_k \geqslant 1$
        \item с точностью до перестановки множителей это представление единственно  
    \end{MyList}
\end{Thm}

\begin{proof}
    \begin{MyList}
        \item из всех делителей $a$ выбираем наименьший -- $p_1$ - простое $\SO a = p_1 \cdot a_1$.
        Рассмотрим $a_1$ -- наименьший делитель - $p_2 \SO a_1 = p_1 \cdot p_2 \cdot a_2$ и т.д. 
        \[a_1 > a_2 > a_3 > ... \SO \exists a_n = 1 \SO a = \text{ разложение на простые}\]
        
        \item Предположим, что представление не одно, то есть 
        \[a = p_1 \cdot p_2 \cdot ... \cdot p_s = q_1 \cdot q_2 \cdot ... \cdot q_n\]
        Не умаляя общности, пусть $n \geqslant s \SO p_1 | q_1 ... q_n$. Тогда, по лемме 2 $p_1 | q_i \SO p_1 = q_i$.
        Перенумеруем $i = 1 \SO p_2 p_3 ... p_s = q_2 q_3 ... q_n \SO$ все $p_s$ сократятся, т.е. $1 = q_{s + 1} ... q_n \SO s = n$     
    \end{MyList}
\end{proof}

\begin{Def}
    $a = p_1^{\alpha_1} \cdot p_2^{\alpha_2} \cdot p_k^{\alpha_k}$ --- каноническое разложение числа $a$ 
\end{Def}

\begin{Cons}
    Любой делитель $a = p_a^{\alpha_1} ... p_k^{\alpha_k}$ имеет вид $b = p_1^{\beta_1} \cdot ... \cdot p_k^{\beta_k}, 0 \leqslant \beta_i \leqslant \alpha_i$  
\end{Cons}

\begin{proof}
    $b | a \SO b$ содержит в разложении $p_i$  
\end{proof}

\Pagebreak
\begin{Cons}
    $\gcd (a_1, ..., a_n)$ имеет вид $p_1^{\alpha_1} p_2^{\alpha_2} ... p_k^{\alpha_k}$, где \\
    $a_i = \min \left\{\text{показатель степени $p_i$, с которым $p_i$ входит в разложение } a_1, a_2, ..., a_n\right\}$   
\end{Cons}

\begin{Cons}
    $\lcm (a_1, ..., a_n)$ имеет вид $p_1^{\alpha_1} p_2^{\alpha_2} ... p_k^{\alpha_k}$, где \\
    $a_i = \max \left\{\text{показатель степени $p_i$, с которым $p_i$ входит в разложение } a_1, a_2, ..., a_n\right\}$   
\end{Cons}

\Subsection{Непрерывные дроби (Цепные дроби)}

\begin{Def}
    Выражение вида 
    \[a_0 + \frac{1}{a_1 + \frac{1}{a_2 + \frac{1}{a_3 + ...}}}\] 
    называется непрерывной дробью. Обозначение: $[a_0, a_1, a_2, ...]$ 
\end{Def}

\begin{Thm}
    Любое вещественное число может быть представлено в виде непрерывной дроби.

    Если число иррационально -- в виде бесконечной дроби, если рациональное -- в виде конечной.
\end{Thm}

\begin{proof}
    $a > b$
    \[\frac{a}{b} = a_0 + \frac{r_1}{b} = a_0 + \frac{1}{\frac{b}{r_1}} = a_0 + \frac{1}{a_1 + \frac{r_2}{r_1}} = a_0 + \frac{1}{a_1 + \frac{1}{a_2 + \frac{r_3}{r_2}}} \text { и т.д.}\]  

    где
    \begin{align*}
        a &= b \cdot a_0 + r_1 \\
        b &= r_1 \cdot a_1 + r_2 \\
        r_1 &= r_2 \cdot a_2 + r_3 \\
    \end{align*} 
\end{proof}

\begin{Def}
    Для $\frac{a}{b} \ \delta_0 = \frac{a_0}{1}, \delta_1 = a_0 + \frac{1}{a_1}, \delta_2 = a_0 + \frac{1}{a_1 + \frac{1}{a_2}}$ и т.д. называются подходящими дробями. 
\end{Def}

\begin{Thm}[Формулы подходящих дробей]
    $\delta_k = \frac{p_k}{q_k}, p_{-1} = 1, q_{-1} = 0, p_0 = a_0, q = 1$
    \[\SO \begin{cases}
        p_k = a_k \cdot p_{k - 1} + p_{k - 2} \\
        q_k = a_k \cdot q_{k - 1} + q_{k - 2}
    \end{cases}\] 
\end{Thm}

\begin{proof}
    $\delta_1 = a_0 + \frac{1}{a_1} = \frac{a_0 a_1 + 1}{a_1 \cdot 1 + 0} = \frac{a_1 \cdot p_0 + p_{-1}}{a_1 \cdot q_0 + q_{-1}}$ 

    Предположим, что для $k$ верно. Тогда для $k + 1$ 
    \[\delta_{k + 1} = a_0 + \frac{1}{a_1 + \frac{1}{a_2 + ... \frac{}{a_k + \frac{1}{a_{k + 1}}}}} = \frac{(a_k + \frac{1}{a_{k + 1}}) \cdot p_{k - 1} + p_{k - 2}}{(a_k + \frac{1}{a_{k + 1}})\cdot q_{k - 1} + q_{k - 2}} = \]
    \[= \frac{(a_{k + 1} \cdot a_k + 1) \cdot p_{k - 1} + p_{k - 2} \cdot a_{k + 1}}{(a_{k + 1} \cdot a_k + 1) \cdot q_{k - 1} + q_{k - 2} \cdot a_{k + 1}} = \frac{a_{k + 1}(a_k \cdot p_{k - 1} + p_{k - 2}) + p_{k - 1}}{a_{k + 1}(a_k q_{k - 1} + q_{k - 2}) + q_{k - 1}} = \frac{a_{k + 1} \cdot p_k + p_{k - 1}}{a_{k + 1} \cdot q_k + q_{k - 1}}\]
\end{proof}

\end{document}