\documentclass[12pt]{article}

% Автор стиля: Сергей Копелиович
% Автор конспекта: Илья Дудников

\usepackage{cmap}
\usepackage[T2A]{fontenc}
\usepackage[utf8]{inputenc}
\usepackage[russian]{babel}
\usepackage{graphicx}
\usepackage{amsthm,amsmath,amssymb}
\usepackage{listings}
\usepackage{color}
\usepackage{xcolor}
\usepackage{array}
\usepackage{mathrsfs}
\usepackage{epigraph}

\usepackage[russian,colorlinks=true,urlcolor=red,linkcolor=blue]{hyperref}
\usepackage{enumerate}
\usepackage{datetime}
\usepackage{fancyhdr}
\usepackage{lastpage}
\usepackage{verbatim}
\usepackage{tikz}
\usetikzlibrary{arrows,decorations.markings,decorations.pathmorphing}
\usepackage{pgfplots}

\usepackage{ifthen}
\usepackage{mathtools}

%\usepackage{tabls}
%\usepackage{tabularx}
%\usepackage{xifthen}
%\listfiles

\usepackage{import}
\usepackage{xifthen}
\usepackage{pdfpages}
\usepackage{transparent}

\def\NAME{Лекция, 10/08/21}

\sloppy
\voffset=-20mm
\textheight=235mm
\hoffset=-22mm
\textwidth=180mm
\headsep=12pt
\footskip=20pt

\parskip=0em
\parindent=0em

\setlength\epigraphwidth{.8\textwidth}

\newlength{\tmplen}
\newlength{\tmpwidth}
\newcounter{listcounter}

% Список с маленькими отступами
\newenvironment{MyList}[1][4pt]{
  \begin{enumerate}[1.]
  \setlength{\parskip}{0pt}
  \setlength{\itemsep}{#1}
}{       
  \end{enumerate}
}
% Вложенный список с маленькими отступами
\newenvironment{InnerMyList}[1][0pt]{
  \vspace*{-0.5em}
  \begin{enumerate}[(a)]
  \setlength{\parskip}{-0pt}
  \setlength{\itemsep}{#1}
}{       
  \end{enumerate}
  \vspace*{-0.5em}
}
% Список с маленькими отступами
\newenvironment{MyItemize}[1][4pt]{
  \begin{itemize}
  \setlength{\parskip}{0pt}
  \setlength{\itemsep}{#1}
}{       
  \end{itemize}
}

% Основные математические символы
\def\TODO{{\color{red}\bf TODO}}
\def\N{\mathbb{N}}       %
\def\R{\mathbb{R}}       %
\def\F2{\mathbb{F}_2}    %
\def\Z{\mathbb{Z}}       %
\def\INF{\t{+}\infty}    % +inf
\def\EPS{\varepsilon}    %
\def\EMPTY{\varnothing}  %
\def\PHI{\varphi}        %
\def\SO{\Rightarrow}     % =>
\def\EQ{\Leftrightarrow} % <=>
\def\t{\texttt}          % mono font
\def\c#1{{\rm\sc{#1}}}   % font for classes NP, SAT, etc
\def\O{\mathcal{O}}      %
\def\NO{\t{\#}}          % #
\def\XOR{\text{ {\raisebox{-2pt}{\ensuremath{\Hat{}}}} }}
\renewcommand{\le}{\leqslant}
\renewcommand{\ge}{\geqslant}
\newcommand{\q}[1]{\langle #1 \rangle}               % <x>
\newcommand\URL[1]{{\footnotesize{\url{#1}}}}        %
% \newcommand{\sfrac}[2]{{\scriptscriptstyle\frac{#1}{#2}}}  % Очень маленькая дробь
% \newcommand{\mfrac}[2]{{\scriptstyle\frac{#1}{#2}}}    % Небольшая дробь
\newcommand{\sfrac}[2]{{\scriptstyle\frac{#1}{#2}}}  % Очень маленькая дробь
\newcommand{\mfrac}[2]{{\textstyle\frac{#1}{#2}}}    % Небольшая дробь

\newcommand{\fix}[1]{{\color{fixcolor}{#1}}} % \underline
\def\bonus{\t{\red{(*)}}}
\def\ifbonus#1{\ifthenelse{\equal{#1}{}}{}{\bonus}}
\def\smallsquare{$\scalebox{0.5}{$\square$}$}

\newlength{\myItemLength}
\setlength{\myItemLength}{0.3em}
\def\ItemSymbol{\smallsquare}
\def\Item{\vspace*{\myItemLength}\ItemSymbol \ \ }

\newcommand{\LET}{%
  % [line width=0.6pt]
  \begin{tikzpicture}%
  \draw(0.8ex,0) -- (0.8ex,1.6ex);%
  \draw(0,1.6ex) -- (0.8ex,1.6ex);%
  \end{tikzpicture}%
  \hspace*{0.1em}%
}

% Отступы
\def\makeparindent{\hspace*{\parindent}\unskip}
\def\up{\vspace*{-0.5em}}%{\vspace*{-\baselineskip}}
\def\down{\vspace*{0.5em}}
\def\LINE{\vspace*{-1em}\noindent \underline{\hbox to 1\textwidth{{ } \hfil{ } \hfil{ } }}}
\def\BOX#1{\mbox{\fbox{\bf{#1}}}}
\def\Pagebreak{\pagebreak\vspace*{-1.5em}}

% Мелкий заголовок
\newcommand{\THEE}[1]{
  \vspace*{0.5em}
  \noindent{\bf \underline{#1}}%\hspace{0.5em}
  \vspace*{0.2em}
}
% Другой тип мелкого заголовка
\newcommand{\THE}[1]{
  \vspace*{0.5em} $\bullet$
  \noindent{\bf #1}%\hspace{0.5em}
  \vspace*{0.2em}
}

\newenvironment{MyTabbing}{
  \t\bgroup
  \vspace*{-\baselineskip}
  \begin{tabbing}
    aaaa\=aaaa\=aaaa\=aaaa\=aaaa\=aaaa\kill
}{
  \end{tabbing}
  \t\egroup
}

% Код с правильными отступами
\lstnewenvironment{code}{
  \lstset{}
%  \vspace*{-0.2em}
}%
{
%  \vspace*{-0.2em}
}
\lstnewenvironment{codep}{
  \lstset{language=python}
}%
{
}

% Формулы с правильными отступами
\newenvironment{smallformula}{
 
  \vspace*{-0.8em}
}{
  \vspace*{-1.2em}
  
}
\newenvironment{formula}{
 
  \vspace*{-0.4em}
}{
  \vspace*{-0.6em}
  
}

% Большая квадратная скобка
\makeatletter
\newenvironment{sqcases}{%
  \matrix@check\sqcases\env@sqcases
}{%
  \endarray\right.%
}
\def\env@sqcases{%
  \let\@ifnextchar\new@ifnextchar
  \left\lbrack
  \def\arraystretch{1.2}%
  \array{@{}l@{\quad}l@{}}%
}
\makeatother

\DeclareMathOperator{\sort}{sort}

% Определяем основные секции: \begin{Lm}, \begin{Thm}, \begin{Def}, \begin{Rem}
\renewcommand{\qedsymbol}{$\blacksquare$}
\theoremstyle{definition} % жирный заголовок, плоский текст
\newtheorem{Thm}{\underline{Теорема}}[subsection] % нумерация будет "<номер subsection>.<номер теоремы>"
\newtheorem{Lm}[Thm]{\underline{Lm}} % Нумерация такая же, как и у теорем
\newtheorem{Ex}[Thm]{Упражнение} % Нумерация такая же, как и у теорем
\newtheorem{Example}[Thm]{Пример} % Нумерация такая же, как и у теорем
\newtheorem{Solution}[Thm]{Решение}
\newtheorem{Code}[Thm]{Код} % Нумерация такая же, как и у теорем
\theoremstyle{plain} % жирный заголовок, курсивный текст
\newtheorem{Def}[Thm]{Def.} % Нумерация такая же, как и у теорем
\theoremstyle{remark} % курсивный заголовок, плоский текст
\newtheorem{Cons}[Thm]{Следствие} % Нумерация такая же, как и у теорем
\newtheorem{Conj}[Thm]{Гипотеза} % Нумерация такая же, как и у теорем
\newtheorem{Prop}[Thm]{Утверждение} % Нумерация такая же, как и у теорем
\newtheorem{Rem}[Thm]{Замечание} % Нумерация такая же, как и у теорем
\newtheorem{Remark}[Thm]{Замечание} % Нумерация такая же, как и у теорем
\newtheorem{Algo}[Thm]{Алгоритм} % Нумерация такая же, как и у теорем

% Определяем ЗАГОЛОВКИ
\def\SectionName{Алгебра}
\def\AuthorName{Илья Дудников}

\newlength{\sectionvskip}
\setlength{\sectionvskip}{0.5em}
\newcommand{\Section}[4][]{
  % Заголовок
  \pagebreak
%  \ifthenelse{\isempty{#1}}{
    \refstepcounter{section}
%  }{}
  \vspace{0.5em}
%  \ifthenelse{\isempty{#1}}{
%    \addtocontents{toc}{\protect\addvspace{-5pt}}%
    \addcontentsline{toc}{section}{\arabic{section}. #2}
%  }{}
  


  % Запомнили название и автора главы
  \gdef\SectionName{#2}
  \gdef\AuthorName{#4}

  % Заголовок страницы
  \lhead{Конспект, \NAME}
  \chead{}
  \rhead{\SectionName}
  \renewcommand{\headrulewidth}{0.4pt}
    
  \lfoot{}
  \cfoot{\thepage\t{/}\pageref*{LastPage}}
  \rfoot{Автор: \AuthorName}
  \renewcommand{\footrulewidth}{0.4pt}
}

\newcommand{\Subsection}[2][]{
  \refstepcounter{subsection}
  \vspace*{1em}
  \ifthenelse{\equal{#1}{}}
    {\addcontentsline{toc}{subsection}{\arabic{section}.\arabic{subsection}. #2}}
    {\addcontentsline{toc}{subsection}{\arabic{section}.\arabic{subsection}. \bonus\,#2}}
  {\color{blue}\bf\large \arabic{section}.\arabic{subsection}. \ifbonus{#1}\,{#2}} 
  \vspace*{0.5em}
  \makeparindent
}
\newcommand{\Subsubsection}[2][]{
  \refstepcounter{subsubsection}
  \vspace*{1em}
  \ifthenelse{\equal{#1}{}}
    {\addcontentsline{toc}{subsubsection}{\arabic{section}.\arabic{subsection}.\arabic{subsubsection}. #2}}
    {\addcontentsline{toc}{subsubsection}{\arabic{section}.\arabic{subsection}.\arabic{subsubsection}. \bonus\,#2}}
  {\color{blue}\bf\large \arabic{section}.\arabic{subsection}.\arabic{subsubsection}. \ifbonus{#1}\,#2}
  \vspace*{0.5em}
  \makeparindent
}

\newcommand{\Header}{
  \pagestyle{empty}
  \renewcommand{\dateseparator}{--}
  \begin{center}
    {\Large\bf 
    Алгебра 1 семестр ПИ,\\
    \vspace{0.3em}
     \NAME}\\
    \vspace{0.7em}
    {Собрано {\today} в {\currenttime}}
  \end{center}

  \LINE
  \vspace{0em}

  \renewcommand{\baselinestretch}{0.98}\normalsize
  \tableofcontents
  \renewcommand{\baselinestretch}{1.0}\normalsize
  \pagebreak
}

\newcommand{\BeginConspect}{
  \pagestyle{fancy}
  \setcounter{page}{1}
}

\definecolor{mygray}{rgb}{0.7,0.7,0.7}
\definecolor{ltgray}{rgb}{0.9,0.9,0.9}
\definecolor{fixcolor}{rgb}{0.7,0,0}
\definecolor{red2}{rgb}{0.7,0,0}
\definecolor{dkred}{rgb}{0.4,0,0}
\definecolor{dkblue}{rgb}{0,0,0.6}
\definecolor{dkgreen}{rgb}{0,0.6,0}
\definecolor{brown}{rgb}{0.5,0.5,0}

\newcommand{\green}[1]{{\color{green}{#1}}}
\newcommand{\black}[1]{{\color{black}{#1}}}
\newcommand{\red}[1]{{\color{red}{#1}}}
\newcommand{\dkred}[1]{{\color{dkred}{#1}}}
\newcommand{\blue}[1]{{\color{blue}{#1}}}
\newcommand{\dkgreen}[1]{{\color{dkgreen}{#1}}}

\newcommand{\Mod}[1]{\ (\mathrm{mod}\ #1)}

\begin{document}

\Header

\BeginConspect

\Section{Теория сравнений}{}{Илья Дудников}

\Subsection{Начала теории сравнений}

\begin{Def}
    $a$ и $b$ называются сравнимыми по модулю $m > 0$, если они имеют одинаковые остатки при делении на $m$
    \[a \equiv b \Mod m, a \equiv b(m), a \stackrel{m}{\equiv} b\]
\end{Def}

\begin{Prop}
    \[\EQ \begin{cases}
        a \equiv b \Mod m \\
        a - b \vdots m \\
        a \equiv b + mt
    \end{cases}\]
\end{Prop}

\begin{proof}
    $1) \SO 2)$
    \[a = mq_1 + r, b = mq_2 + r \SO a - b = m(q_1 - q_2) \vdots m\]
    $2) \SO 3)$
    \[a - b \vdots m \SO a - b = mt \SO a = b + mt\]
    $3) \SO 1)$. Поделим $a$ и $b$ на $m$:
    \[a = mq_1 + r_1, b = mq_2 + r_2\]
    \begin{align*}
    &3): a = b + mt \SO mq_1 + r_1 = mq_2 + r_2 + mt \SO \\
    &\SO m(q_1 - q_2 - t) = r_2 - r_1 \SO m | r_2 - r_1 \SO r_2 - r_1 = 0
    \end{align*}
\end{proof}

Свойства:
\begin{MyList}
    \item Рефлексивность. $a \equiv a \Mod m$
    \item Симметричность. $a \equiv b \Mod m \SO b \equiv a \Mod m$
    \item Транзитивность. $a \equiv b \Mod m \SO b \equiv c \Mod m \SO a \equiv c \Mod m$
    \begin{proof}
        \[a - c = a - b + b - c \vdots m\]
    \end{proof}
    \item $a \equiv b \Mod m, c \equiv d \Mod m \SO a + c \equiv b + d \Mod m$
    \item $a \equiv b \Mod m, c \equiv d \Mod m \SO ac \equiv bd \Mod m$
    \begin{proof}
        \[ac - bd = ac - bc + bc - bd = c(a - b) + b(c - d) \vdots m\]
    \end{proof}
    \Pagebreak
    \item $d | a, d | b, d | m, a \equiv b \Mod m \SO \frac{a}{d} \equiv \frac{b}{d} \Mod{\frac{m}{d}}$
    \begin{proof}
        \[a - b = a_1 d - b_1 d = my = m_1 d t \SO a_1 - b_1 = m_1 t\]
    \end{proof}
    \item $a \equiv b \Mod m \SO ka \equiv kb \Mod m$ 
    \item $d | a, d | b, (m, d) = 1, a \equiv b \Mod m \SO \frac{a}{d} \equiv \frac{b}{d} \Mod m$
    \begin{proof}
        \[a = a_1 d, b = b_1 d, a - b \vdots m \SO (a_1 - b_1) \cdot d \vdots m \SO a_1 - b_1 \vdots m\]
    \end{proof}
    \item $d | m, a \equiv b \Mod m \SO a \equiv b \Mod d$ 
    \item $a \equiv b \Mod m \SO (a, m) = (b, m)$ 
    \begin{proof}
        \[a \equiv b \Mod m \SO a = b + mt \SO (a, m) = (b, m)\]
    \end{proof}
\end{MyList}

\Subsection{Классы вычетов}

\begin{Def}
    Классом вычетов по $\Mod m$ называется множество чисел, сравнимых с $a$ по модулю $m$
    \[m = 7, \overline{1} = \{-6, 8, 1, 15, ...\}\]
    \[\overline{a} = \{x | x \equiv a \Mod m\}\]
    Элементы классов вычетов -- \textbf{вычеты}. Обычно рассматривают наименьший неотрицательный вычет. 
\end{Def}

\begin{Def}
    Множество вычетов, взятых по одному из разных классов образуют полную систему вычетов. Например
    \[\{0, 1, 2, ..., m - 1\}\]
\end{Def}

\begin{Lm}
    Множество из $m$ чисел, попарно несравнимых по модулю $m$, образуют полную систему вычетов.
\end{Lm}

\begin{Thm}
    $(a, m) = 1$. Если $x$ пробегает полную систему вычетов по $\Mod m \SO \forall b \to ax + b$ тоже пробегает полную систему вычетов по $\Mod m$  
\end{Thm}

\begin{proof}
    $x$ принадлежит $m$ значений $\SO ax + b$ принадлежит $m$ значений. \\
    Пусть $x_1 \not\equiv x_2 \Mod m$. Предположим, что $ax_1 + b \equiv ax_2 + b \Mod m \SO ax_1 \equiv ax_2 \Mod m \SO x_1 \equiv x_2 \Mod m$ 
\end{proof}

\Pagebreak
\Subsection{Кольцо классов вычетов}
\begin{Def}
    Определим сложение и умножение вычетов по фиксированному модулю $m$.
    \[\overline{a} + \overline{b} = \overline{a + b}, \overline{a} \cdot \overline{b} = \overline{ab}\]
\end{Def}

\begin{Lm}
    Сложение и умножение определены корректно
\end{Lm}

\begin{proof}
    $a \equiv a_1 \Mod m, b \equiv b_1 \Mod m$
    \[\SO a + b = a_1 + b_1 \Mod m, a \cdot b = a_1 \cdot b_1 \Mod m \SO \overline{a} + \overline{b} = \overline{a}_1 + \overline{b}_1, \overline{a} \cdot \overline{b} = \overline{a}_1 \cdot \overline{b}_1\] 
\end{proof}

\begin{Def}
    Группа $G$ называется коммутативной (абелевой), Если
    \[\forall x, y \in G \to xy = yx\] 
\end{Def}

\begin{Thm}
    $\Z_m$ образует коммутативную группу относительно сложения 
\end{Thm}

\begin{proof}
    $\overline{a} + \overline{b} = \overline{a + b} \in \Z_m$
    \begin{MyList}
        \item $(\overline{a} + \overline{b}) + \overline{c} = \overline{a + b} + \overline{c} = \overline{a + b + c}$ \\
        $\overline{a} + (\overline{b} + \overline{c}) = \overline{a} + \overline{b + c} = \overline{a + b + c}$
        \item $\overline{0}$. $\overline{a} + \overline{0} = \overline{a + 0} = \overline{a}$
        \item $-\overline{a} = \overline{m - a} \SO \overline{a} - \overline{a} = \overline{a + m - a} = \overline{0}$
        \item $\overline{a} + \overline{b} = \overline{b} + \overline{a}$      
    \end{MyList} 
\end{proof}

\begin{Def}
    (Ассоциативным) кольцом называется множество $R$, на котором заданы бинарные операции:
    \begin{MyList}
        \item $\forall x, y, z \to (x + y) + z = x + (y + z)$ 
        \item $\exists 0 \in R : \forall x \in R \to x + 0 = x$ 
        \item $\forall x \in R \ \exists (-x) \in R : x + (-x) = 0$ 
        \item $\forall x, y \in R \to x + y = y + x$ 
        \item $\forall x, y, z \in R \to (y + z) = xy + xz, (x + y)z = xz + yz$ 
        \item $\forall x, y, z \in R \to (xy)z = x(yz)$ 
    \end{MyList}
\end{Def}

\begin{Rem}
    $\exists 1 \in R : \forall x \in R \to x \cdot 1 = 1 \cdot x = x$ -- кольцо с единицей \\
    $\forall x, y \in R \to xy = yx$ -- коммутативное кольцо 
\end{Rem}

\begin{Thm}
    $\Z_m$ -- коммутативное кольцо с единицей.
\end{Thm}

\begin{proof}
    \[\overline{a}(\overline{b} + \overline{z}) = \overline{a} \cdot \overline{b + c} = \overline{a(b + c)} = \overline{ab + ac}\]
    и т.д.
\end{proof}

\begin{Def}
    Кольца $R$, в котором $\forall a, b \to (ab = 0 \SO a = 0 \vee b = 0)$ называется кольцом без делителей нуля.
    
    Если $ab = 0$ и $a, b \neq 0$, то $a, b$ -- делители нуля 
\end{Def}

\begin{Def}
    Коммутативное кольцо без делителей нуля -- область целостности.
\end{Def}

\Pagebreak
\begin{Thm}
    \begin{MyList}
        \item $\Z_m$ имеет делители нуля $\EQ m$ -- составное число
        \item $\Z_p, p$ - простое -- область целостности.
    \end{MyList}
\end{Thm}

\begin{proof}
    "$\SO$". $m = n \cdot k, \overline{n} \cdot \overline{k} = \overline{0}$ в $\Z_m$
    
    "$\Leftarrow$". $\overline{n} \cdot \overline{k} = \overline{0} \SO n \cdot k \equiv 0 \Mod m$
    
    Предположим, что $m$ -- простое $\SO m | n \vee m | k \SO \overline{n} = \overline{0} \vee \overline{k} = \overline{0}$. Но $\overline{n}$ и $\overline{k}$ -- делители нуля, т.е. $\overline{n}, \overline{k} \neq 0 \SO m$ -- составное.   

    $1) \SO 2)$ 
\end{proof}

\Subsection{Приведенная система вычетов}

\begin{Def}
    Вычеты, выбранные из полной системы вычетов и взаимно-простые с модулем $m$ обрузуют приведенную систему вычетов
\end{Def}

\begin{Def}
    Количество вычетов в приведенной системе вычетов обозначается $\varphi (m)$ -- функция Эйлера. 
\end{Def}

\begin{Lm}
    Если $p$ -- простое, то 
    \[\varphi(p) = p - 1\]
\end{Lm}

\begin{Thm}
    $(a, m) = 1, x$ пробегает приведенную систему вычетов $\SO ax$ тоже пробегает приведенную систему вычетов по $\Mod m$  
\end{Thm}

\begin{proof}
    $x \to \varphi(m), ax \to \varphi(m)$
    
    $(ax,m) = (a, m) = 1 \SO ax$ набор чисел из $\varphi(m)$, взаимно-простых с $m \SO \{ax\}$ -- приведенная система вычетов. 
\end{proof}

\Subsection{Функция Эйлера}
\begin{Lm}
    $p$ -- простое, $\alpha > 0$
    \[\varphi(p^\alpha) = p^\alpha - p^{\alpha - 1}\] 
\end{Lm}

\begin{proof}
    $1, 2, 3, ..., p, 2p, 3p, ..., p \cdot p, ..., p^\alpha - 1$. Выбросим из этого множества числа, делящиеся на $p$.
    Таких чисел будет ровно количество коэффициентов при $p$ до $p^\alpha$, т.е. $p^{\alpha - 1}$  
\end{proof}

\begin{Def}
    Функия $\Theta: \N \to \N$ называется мультипликативной, если 
    \[(a, b) = 1 \SO \Theta(ab) = \Theta(a) \cdot \Theta(b)\] 
\end{Def}

\begin{Thm}[Мультипликативность функции Эйлера]
    $\varphi$ мультипликативна     
\end{Thm}

\begin{proof}
    $(a, b) = 1$
    \[\begin{array}{ccccc}
    1 & 2 & 3 & ... & b \\ 
    b + 1 & b + 2 & b + 3 & ... & 2b \\ 
    ... & ... & ... & ... & ... \\ 
    (a - 1)b + 1 & (a - 1)b + 2 & (a - 1)b + 3 & ... & ab
    \end{array}\]
    Количество чисел, взаимно-простых с $b : \forall$ строка $: kb + r, k = 0, ..., a - 1, 1 \leqslant r \leqslant b$.
    Рассмотрим $k$-ю строку: $(kb + r, b) = 1 \SO (r, b) = 1$. Количество чисел $kb + r : (kb + r, b) = 1 = \varphi (b) \SO$
    есть $\varphi(b)$ столбцов, в которых числа $(kb + r, b) = 1$. Найдем в этих столбцах числа, взаимно-простые с $a$.
    $\forall$ столбец $: xb + r, x = 0, ..., a - 1 \SO xb + r$ -- полная система вычетов по $\Mod a \SO$ 
    среди $\{xb + r\}$ чисел, взаимно-простых с $a = \varphi(a) \SO$ всего чисел, взаимно-простых с $ab = \varphi(a) \cdot \varphi(b)$          
\end{proof}

\begin{Cons}
    $n = p_1^{\alpha_1} \cdot p_2^{\alpha_2} \cdot ... \cdot p_k^{\alpha_k}$ -- каноническое разложение
    $\SO \varphi(n) = (p_1^{\alpha_1} - p_1^{\alpha_1 - 1})(p_2^{\alpha_2} - p_2^{\alpha_2 - 1}) \cdot ... \cdot (p_k^{\alpha_k} - p_k^{\alpha_k - 1})$  
\end{Cons}

\begin{Rem}
    $\varphi(n) = n(1 - \frac{1}{p_1})(1 - \frac{1}{p_2}) \cdot ... \cdot (1 - \frac{1}{p_k})$ 
\end{Rem}

\begin{Thm}[Теорема Эйлера]
    $(m, a) = 1 \SO a^{\varphi(m)} \equiv 1 \Mod m$ 
\end{Thm}

\begin{proof}
    $r_1, r_2, ..., r_{\varphi(m)}$ -- приведенная система вычетов по $\Mod m$
    
    $\SO ar_1, ar_2, ..., ar_{\varphi(m)}$ -- приведенная система вычетов по $\Mod m$. Пусть $ar_i = \rho_i$
    \[\SO ar_1 \cdot ar_2 \cdot ... ar_{\varphi(m)} = \rho_1 \rho_2 \cdot ... \cdot \rho_{\varphi(m)}\]
    \[a^{\varphi(m)} r_1 \cdot r_2 \cdot ... \cdot r_{\varphi(m)} = \rho_1 \cdot \rho_2 \cdot ... \cdot \rho_{\varphi(m)} \SO a^{\varphi(m)} \equiv 1 \Mod m\]   
\end{proof}

\begin{Thm}[Теорема Ферма]
    $p$ -- простое, $(a, p) = 1 \SO a^{p - 1} \equiv 1 \Mod p$ 
\end{Thm}

\begin{proof}
    $\varphi(p) = p - 1$ 
\end{proof}

\begin{Def}
    $\Z_m^* = \{r : 0 \leqslant r < m, (r, m) = 1\}$ -- приведенная система вычетов по $\Mod m$ 
\end{Def}

\begin{Thm}
    $\Z_m^*$ -- коммутативная группа по умножению
\end{Thm}

\begin{proof}
    $r_1, r_2 \in \Z_m^*$. $(r_1, m) = (r_2, m) = 1 \SO (r_1 \cdot r_2, m) = 1 \SO r_1 \cdot r_2 \in \Z_m^*, 1 \in \Z_m^*$
    
    $r \in \Z_m^*$, то $r^{-1} = r^{\varphi(m) - 1} \SO r^{\varphi(m) - 1} \cdot r = r^{\varphi(m)} \equiv 1 \Mod m$  
\end{proof}

\Subsection{Сравнения с одним неизвестным}
\begin{Def}
    $f(x) \equiv 0 \Mod m$. Решением этого сравнения называется $x_0 : f(x_0) \equiv 0 \Mod m$. Решения $x_1$ и $x_2$ называются эквивалентными, если $x_1 \equiv x_2 \Mod m$

    Решить сравнение -- найти решений из полной системы вычетов.
\end{Def}

\begin{Thm}[Решение линейного сравнения]
    $ax \equiv b \Mod m , (a, m) = d$ 
    \begin{MyList}
        \item $d \not| \ b \SO$ решений нет.
        \item $d | b \SO \exists d$ решений $ : x = x_0 + m_1 t, t = 0, 1, ..., d - 1, m_1 = \frac{m}{d}, x_0$ -- какое-то решение
    \end{MyList}
\end{Thm}

\begin{proof}
    \begin{MyList}
        \item Очевидно
        \item Если $(a, m) = 1, x$ пробегает полную систему вычетов по $\Mod m \SO ax$ -- полная система вычетов по $\Mod m \SO \exists x_0 : ax_0 \equiv b \Mod m$
        
        Если $(a, m) = d, a = a_1d, m = m_1d, b = b_1d, (a_1, m_1) = 1$ \\
        $a_1x \equiv b_1 \Mod {m_1}$ -- $\exists$ решение $x_0 : a_1x_0 \equiv b_1 \Mod {m_1}$. $x = x_0 + m_1 t$ -- решение $ax \equiv b \Mod m$
        \[a(x_0 + m_1t) = a_1dx_0 + a_1dm_1t \equiv b_1d \Mod m\]
        Посмотрим, какие решения принадлежат полной системе вычетов, т.е. $0 \leqslant x_0 + m_1 t < m$.
        Ясно, что такие решения будут при $t = 0, 1, ..., d - 1$.       
    \end{MyList}
\end{proof}

\begin{Thm}[Методы решения $ax \equiv b \Mod m, (a, m) = 1$]
    \begin{MyList}
        \item $ax \equiv b \Mod m \SO x \equiv a^{\varphi(m) - 1} \cdot b \Mod m$ 
        \item $ax \equiv b \Mod m \SO x \equiv (-1)^n p_{n - 1} \cdot b \Mod m$ \\
        $\frac{m}{a}$ -- непрерывная дробь, $p_{n - 1}$ -- числитель $(n - 1)$-й подходящей дроби, $ \frac{p_n}{q_n} = \frac{m}{a}$  
    \end{MyList}
\end{Thm}

\begin{proof}
    $p_k \cdot q_{k - 1} - p_{k - 1} \cdot q_k = (-1)^{k - 1}$. $k = n$
    \[m \cdot q_{n - 1} - p_{n - 1} \cdot a = (-1)^{n - 1} \SO -p_{n - 1} \cdot a \equiv (-1)^{n - 1} \Mod m\]
    \[ap_{n - 1} b = (-1)^n b \Mod m \SO a \cdot (-1)^n p_{n - 1} \cdot b \equiv b \Mod m \SO x \equiv (-1)^n p_{n - 1} \cdot b \Mod m\]
\end{proof}

\Subsection{Диофантовы уравнения}

\begin{Def}
    Уравнение вида 
    \[a_1 x_1 + a_2 x_2 + ... + a_n x_n = b\]
    где $a_i \in \Z, x_i$ -- переменные и $\exists i : a_i \neq 0$, называется диофантовым.  
\end{Def}

\begin{Lm}
    $ax + by = c, a, b \neq 0$. Если $x_0 : ax_0 \equiv c \Mod b \SO (x_0, \frac{ax_0 - c}{b})$ -- решения уравнения 
\end{Lm}

\begin{proof}
    $by \equiv c - ax$ при $x = x_0$ и $c - ax \ \vdots \ b \SO \frac{c - ax_0}{b} \in \Z$   
\end{proof}

\begin{Thm}
    $ax + by = c, d = (a, b), d | c$. Пусть $(x_0, y_0)$ -- какое-то решение $\SO$ все решения:
    \[\begin{cases}
        x = x_0 - \frac{b}{d}t \\
        y = y_0 + \frac{a}{d}t
    \end{cases}, t \in \Z\]  
\end{Thm}

\Subsection{Системы сравнений}

\[\begin{cases}
    x \equiv b_1 \Mod m_1 \\
    x \equiv b_2 \Mod m_2 \\
    ... \\
    x \equiv b_k \Mod m_k
\end{cases}\]

\begin{Thm}[Китайская теорема об остатках]
    $(m_i, m_j) = 1, i \neq j$. Тогда
    \begin{MyList}
        \item Решение системы существует:
        \[x \equiv \frac{M}{m_1} \cdot M_1' b_1 + \frac{M}{m_2} \cdot M_2' b_2 + ... + \frac{M}{m_k}M_k' b_k \Mod M\]
        $M = m_1 \cdot m_2 \cdot ... \cdot m_k, M_i' : M_i' \cdot \frac{M}{m_i} \equiv 1 \Mod m_i$
        \item Решение единственно 
    \end{MyList}
\end{Thm}

\begin{proof}
    \begin{MyList}
        \item Подставим в $i$-е уравнение: 
        \[x \equiv \frac{M}{m_i} M_i' b_i \Mod m_i \SO x \equiv b_i \Mod m_i\]
        \item Без доказательства.
    \end{MyList}
\end{proof}

\begin{Thm}[Теорема Вильсона]
    $p$ -- простое $\EQ$ $(p - 1)! \equiv -1 \Mod p$  
\end{Thm}

\begin{proof}
    "$\SO$". $\Z_p^* = \{1, 2, ..., p - 1\}$ -- группа, $a \in \Z_p^*$
    \[a^2 = 1 \SO a = \pm 1, 1 \cdot 2 \cdot ... \cdot (p - 1) \SO 1 \cdot (p - 1) \equiv -1 \Mod p\]
    \[a \neq a^{-1} \SO 2 \cdot ... \cdot (p - 2) = 1\]
    
    "$\Leftarrow$". Предположим, что $$k | p, k > 1, k \neq p \SO k < p \SO 1 \cdot 2 \cdot ... \cdot (p - 1) \ \vdots \ k \SO (p - 1)! \equiv 0 \Mod p$$ 
\end{proof}

\begin{Algo}[Алгоритм RSA]
    \begin{MyList}
        \item Выбираем $p, q$ -- простые
        \item $n = p \cdot q, \varphi(n) = (p - 1)(q - 1)$
        \item Выбираем $e : (e, \varphi(n)) = 1$
        \item Решаем $e \cdot d \equiv 1 \Mod {\varphi(n)} \SO$ находим $d$  
    \end{MyList}

    Шифрование:
    \begin{MyList}
        \item $m$ -- текст (в виде цифрового кода)
        \item $c \equiv m^e \Mod n \SO c$ -- шифр
    \end{MyList}

    Ключи:
    \begin{MyItemize}
        \item $(e, n)$ -- открытый ключ
        \item $(d, n)$ -- закрытый ключ
    \end{MyItemize}

    Дешифрование:
    \[c^d \equiv m^{ed} \equiv m \Mod n\]

    Трудность $n = p \cdot q$. 
\end{Algo}

\end{document}