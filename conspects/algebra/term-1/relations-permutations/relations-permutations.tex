\documentclass[12pt]{article}

% Автор стиля: Сергей Копелиович
% Автор конспекта: Вячеслав Бучин

\usepackage{cmap}
\usepackage[T2A]{fontenc}
\usepackage[utf8]{inputenc}
\usepackage[russian]{babel}
\usepackage{graphicx}
\usepackage{amsthm,amsmath,amssymb}
\usepackage{listings}
\usepackage{color}
\usepackage{xcolor}
\usepackage{array}
\usepackage{epigraph}

\usepackage[russian,colorlinks=true,urlcolor=red,linkcolor=blue]{hyperref}
\usepackage{enumerate}
\usepackage{datetime}
\usepackage{fancyhdr}
\usepackage{lastpage}
\usepackage{verbatim}
\usepackage{tikz}
\usetikzlibrary{arrows,decorations.markings,decorations.pathmorphing}
\usepackage{pgfplots}

\usepackage{ifthen}
\usepackage{mathtools}

%\usepackage{tabls}
%\usepackage{tabularx}
%\usepackage{xifthen}
%\listfiles

\def\NAME{Лекция, 09/17/21}

\sloppy
\voffset=-20mm
\textheight=235mm
\hoffset=-22mm
\textwidth=180mm
\headsep=12pt
\footskip=20pt

\parskip=0em
\parindent=0em

\setlength\epigraphwidth{.8\textwidth}

\newlength{\tmplen}
\newlength{\tmpwidth}
\newcounter{listcounter}

% Список с маленькими отступами
\newenvironment{MyList}[1][4pt]{
  \begin{enumerate}[1.]
  \setlength{\parskip}{0pt}
  \setlength{\itemsep}{#1}
}{       
  \end{enumerate}
}
% Вложенный список с маленькими отступами
\newenvironment{InnerMyList}[1][0pt]{
  \vspace*{-0.5em}
  \begin{enumerate}[(a)]
  \setlength{\parskip}{-0pt}
  \setlength{\itemsep}{#1}
}{       
  \end{enumerate}
  \vspace*{-0.5em}
}
% Список с маленькими отступами
\newenvironment{MyItemize}[1][4pt]{
  \begin{itemize}
  \setlength{\parskip}{0pt}
  \setlength{\itemsep}{#1}
}{       
  \end{itemize}
}

% Основные математические символы
\def\TODO{{\color{red}\bf TODO}}
\def\N{\mathbb{N}}       %
\def\R{\mathbb{R}}       %
\def\F2{\mathbb{F}_2}    %
\def\Z{\mathbb{Z}}       %
\def\INF{\t{+}\infty}    % +inf
\def\EPS{\varepsilon}    %
\def\EMPTY{\varnothing}  %
\def\PHI{\varphi}        %
\def\SO{\Rightarrow}     % =>
\def\EQ{\Leftrightarrow} % <=>
\def\t{\texttt}          % mono font
\def\c#1{{\rm\sc{#1}}}   % font for classes NP, SAT, etc
\def\O{\mathcal{O}}      %
\def\NO{\t{\#}}          % #
\def\XOR{\text{ {\raisebox{-2pt}{\ensuremath{\Hat{}}}} }}
\renewcommand{\le}{\leqslant}
\renewcommand{\ge}{\geqslant}
\newcommand{\q}[1]{\langle #1 \rangle}               % <x>
\newcommand\URL[1]{{\footnotesize{\url{#1}}}}        %
% \newcommand{\sfrac}[2]{{\scriptscriptstyle\frac{#1}{#2}}}  % Очень маленькая дробь
% \newcommand{\mfrac}[2]{{\scriptstyle\frac{#1}{#2}}}    % Небольшая дробь
\newcommand{\sfrac}[2]{{\scriptstyle\frac{#1}{#2}}}  % Очень маленькая дробь
\newcommand{\mfrac}[2]{{\textstyle\frac{#1}{#2}}}    % Небольшая дробь

\newcommand{\fix}[1]{{\color{fixcolor}{#1}}} % \underline
\def\bonus{\t{\red{(*)}}}
\def\ifbonus#1{\ifthenelse{\equal{#1}{}}{}{\bonus}}
\def\smallsquare{$\scalebox{0.5}{$\square$}$}

\newlength{\myItemLength}
\setlength{\myItemLength}{0.3em}
\def\ItemSymbol{\smallsquare}
\def\Item{\vspace*{\myItemLength}\ItemSymbol \ \ }

\newcommand{\LET}{%
  % [line width=0.6pt]
  \begin{tikzpicture}%
  \draw(0.8ex,0) -- (0.8ex,1.6ex);%
  \draw(0,1.6ex) -- (0.8ex,1.6ex);%
  \end{tikzpicture}%
  \hspace*{0.1em}%
}

% Отступы
\def\makeparindent{\hspace*{\parindent}\unskip}
\def\up{\vspace*{-0.5em}}%{\vspace*{-\baselineskip}}
\def\down{\vspace*{0.5em}}
\def\LINE{\vspace*{-1em}\noindent \underline{\hbox to 1\textwidth{{ } \hfil{ } \hfil{ } }}}
\def\BOX#1{\mbox{\fbox{\bf{#1}}}}
\def\Pagebreak{\pagebreak\vspace*{-1.5em}}

% Мелкий заголовок
\newcommand{\THEE}[1]{
  \vspace*{0.5em}
  \noindent{\bf \underline{#1}}%\hspace{0.5em}
  \vspace*{0.2em}
}
% Другой тип мелкого заголовка
\newcommand{\THE}[1]{
  \vspace*{0.5em} $\bullet$
  \noindent{\bf #1}%\hspace{0.5em}
  \vspace*{0.2em}
}

\newenvironment{MyTabbing}{
  \t\bgroup
  \vspace*{-\baselineskip}
  \begin{tabbing}
    aaaa\=aaaa\=aaaa\=aaaa\=aaaa\=aaaa\kill
}{
  \end{tabbing}
  \t\egroup
}

% Код с правильными отступами
\lstnewenvironment{code}{
  \lstset{}
%  \vspace*{-0.2em}
}%
{
%  \vspace*{-0.2em}
}
\lstnewenvironment{codep}{
  \lstset{language=python}
}%
{
}

% Формулы с правильными отступами
\newenvironment{smallformula}{
 
  \vspace*{-0.8em}
}{
  \vspace*{-1.2em}
  
}
\newenvironment{formula}{
 
  \vspace*{-0.4em}
}{
  \vspace*{-0.6em}
  
}

% Большая квадратная скобка
\makeatletter
\newenvironment{sqcases}{%
  \matrix@check\sqcases\env@sqcases
}{%
  \endarray\right.%
}
\def\env@sqcases{%
  \let\@ifnextchar\new@ifnextchar
  \left\lbrack
  \def\arraystretch{1.2}%
  \array{@{}l@{\quad}l@{}}%
}
\makeatother

% Определяем основные секции: \begin{Lm}, \begin{Thm}, \begin{Def}, \begin{Rem}
\renewcommand{\qedsymbol}{$\blacksquare$}
\theoremstyle{definition} % жирный заголовок, плоский текст
\newtheorem{Thm}{\underline{Теорема}}[subsection] % нумерация будет "<номер subsection>.<номер теоремы>"
\newtheorem{Lm}[Thm]{\underline{Lm}} % Нумерация такая же, как и у теорем
\newtheorem{Ex}[Thm]{Упражнение} % Нумерация такая же, как и у теорем
\newtheorem{Example}[Thm]{Пример} % Нумерация такая же, как и у теорем
\newtheorem{Code}[Thm]{Код} % Нумерация такая же, как и у теорем
\theoremstyle{plain} % жирный заголовок, курсивный текст
\newtheorem{Def}[Thm]{Def.} % Нумерация такая же, как и у теорем
\theoremstyle{remark} % курсивный заголовок, плоский текст
\newtheorem{Cons}[Thm]{Следствие} % Нумерация такая же, как и у теорем
\newtheorem{Conj}[Thm]{Гипотеза} % Нумерация такая же, как и у теорем
\newtheorem{Prop}[Thm]{Утверждение} % Нумерация такая же, как и у теорем
\newtheorem{Rem}[Thm]{Замечание} % Нумерация такая же, как и у теорем
\newtheorem{Remark}[Thm]{Замечание} % Нумерация такая же, как и у теорем
\newtheorem{Algo}[Thm]{Алгоритм} % Нумерация такая же, как и у теорем

% Определяем ЗАГОЛОВКИ
\def\SectionName{Алгебра}
\def\AuthorName{Вячеслав Бучин, Илья Дудников}

\newlength{\sectionvskip}
\setlength{\sectionvskip}{0.5em}
\newcommand{\Section}[4][]{
  % Заголовок
  \pagebreak
%  \ifthenelse{\isempty{#1}}{
    \refstepcounter{section}
%  }{}
  \vspace{0.5em}
%  \ifthenelse{\isempty{#1}}{
%    \addtocontents{toc}{\protect\addvspace{-5pt}}%
    \addcontentsline{toc}{section}{\arabic{section}. #2}
%  }{}
  


  % Запомнили название и автора главы
  \gdef\SectionName{#2}
  \gdef\AuthorName{#4}

  % Заголовок страницы
  \lhead{Конспект, \NAME}
  \chead{}
  \rhead{\SectionName}
  \renewcommand{\headrulewidth}{0.4pt}
    
  \lfoot{}
  \cfoot{\thepage\t{/}\pageref*{LastPage}}
  \rfoot{Автор: \AuthorName}
  \renewcommand{\footrulewidth}{0.4pt}
}

\newcommand{\Subsection}[2][]{
  \refstepcounter{subsection}
  \vspace*{1em}
  \ifthenelse{\equal{#1}{}}
    {\addcontentsline{toc}{subsection}{\arabic{section}.\arabic{subsection}. #2}}
    {\addcontentsline{toc}{subsection}{\arabic{section}.\arabic{subsection}. \bonus\,#2}}
  {\color{blue}\bf\large \arabic{section}.\arabic{subsection}. \ifbonus{#1}\,{#2}} 
  \vspace*{0.5em}
  \makeparindent
}
\newcommand{\Subsubsection}[2][]{
  \refstepcounter{subsubsection}
  \vspace*{1em}
  \ifthenelse{\equal{#1}{}}
    {\addcontentsline{toc}{subsubsection}{\arabic{section}.\arabic{subsection}.\arabic{subsubsection}. #2}}
    {\addcontentsline{toc}{subsubsection}{\arabic{section}.\arabic{subsection}.\arabic{subsubsection}. \bonus\,#2}}
  {\color{blue}\bf\large \arabic{section}.\arabic{subsection}.\arabic{subsubsection}. \ifbonus{#1}\,#2}
  \vspace*{0.5em}
  \makeparindent
}

\newcommand{\Header}{
  \pagestyle{empty}
  \renewcommand{\dateseparator}{--}
  \begin{center}
    {\Large\bf 
     Алгебра 1 семестр ПИ,\\
    \vspace{0.3em}
     \NAME}\\
    \vspace{0.7em}
    {Собрано {\today} в {\currenttime}}
  \end{center}

  \LINE
  \vspace{0em}

  \renewcommand{\baselinestretch}{0.98}\normalsize
  \tableofcontents
  \renewcommand{\baselinestretch}{1.0}\normalsize
  \pagebreak
}

\newcommand{\BeginConspect}{
  \pagestyle{fancy}
  \setcounter{page}{1}
}

\definecolor{mygray}{rgb}{0.7,0.7,0.7}
\definecolor{ltgray}{rgb}{0.9,0.9,0.9}
\definecolor{fixcolor}{rgb}{0.7,0,0}
\definecolor{red2}{rgb}{0.7,0,0}
\definecolor{dkred}{rgb}{0.4,0,0}
\definecolor{dkblue}{rgb}{0,0,0.6}
\definecolor{dkgreen}{rgb}{0,0.6,0}
\definecolor{brown}{rgb}{0.5,0.5,0}

\newcommand{\green}[1]{{\color{green}{#1}}}
\newcommand{\black}[1]{{\color{black}{#1}}}
\newcommand{\red}[1]{{\color{red}{#1}}}
\newcommand{\dkred}[1]{{\color{dkred}{#1}}}
\newcommand{\blue}[1]{{\color{blue}{#1}}}
\newcommand{\dkgreen}[1]{{\color{dkgreen}{#1}}}

\begin{document}

\Header

\BeginConspect

\Section{Отношения и перестановки}{}{Вячеслав Бучин, Илья Дудников}

\Subsection{Отношения}

\begin{Def} Отношением $\omega$ на $X \times Y$ называется любое подмножество $X \times Y$.
\end{Def}
Если $X = Y$, то говорят про отношение на $X$.

Отношение на $X$ называется:
\begin{MyList}
    \item рефлексивным, если $\forall x \in X (x,x) \in \omega$
    \item антирефлексивным, если $(x, y) \in \omega \SO x \neq y$
    \item симметричным, если $(x, y) \in \omega \SO (y, x) \in \omega$
    \item антисимметричным, если $(x, y), (y, x) \in \omega \SO y = x$
    \item транзитивным, если $(x, y), (y, z) \in \omega \SO (x, z) \in \omega$
\end{MyList}

\Subsection{Отношение эквивалентности}{}

\begin{Def}
    Отношение на $X$, которое является рефлексивным, симметричным, транзитивным, называется эквивалентностью и обозначается $x \sim y$
\end{Def}

\begin{Example}
    $X = \Z$ $x \omega y \EQ x - y \vdots 5$
    \begin{MyList}
        \item $x - x \vdots 5$ --- рефлексивно
        \item $x - y \vdots 5 \SO y - x \vdots 5$ --- симметрично
        \item $x - y \vdots 5$, $y - z \vdots 5$ $x - z = (x - y) + (y - z) \vdots 5 \SO x - z \vdots 5$ --- транзитивно
    \end{MyList}
    $\SO \omega$ --- отношение эвивалентности
\end{Example}

\Subsection{Класс эквивалентности}{}

\begin{Def}
    Классом эквивалентности, содержащим $a \in X$, называется $[a] = \{x: x \in X, x \sim a\}$
\end{Def}
\begin{Def}
    Разбиением множества $X$ называется  $\pi(X) = \{ X_i \}$:
    \begin{MyList}
        \item $X_1 \cup X_2 \cdots = X$
        \item $\forall i, j: i \neq j, X_i \cap X_j = \EMPTY$
    \end{MyList}
\end{Def}

\begin{Thm}
    Связь эквивалентности и разбиения множества

    \begin{MyList}
        \item Отношения эквивалентности на $X$ задаёт разбиение множества $\pi(X)$, $X_i$ --- классы эквивалентности
        \item Разбиение $\pi(X)$ задаёт эквивалентность на $X$
    \end{MyList}
    \begin{proof}

        \begin{MyList}
            \item $X_i = [x] = \{y \in X: y \sim x\}$ --- перебираем все $x \in X \SO X = X_1 \cup X_2 \cup \cdots X_n \cup \cdots$

            $X_i, X_j : X_i = [x_i], X_j = [x_j]$ предположим, что $a \in X_i \cap X_j \SO a \sim x_i, a \sim x_j \SO x_i \sim x_j \SO X_i = X_j \SO [x_i]$ задают разбиения
            \item $\sim : x \sim y \EQ x, y \in X_i$, проверить, что $\sim$ --- эквивалентность:
            \begin{MyList}
                \item $x, x \in X_i \SO x \sim x$
                \item $x, y \in X_i \SO y, x \in X_i$
                \item $x, y, y, z \in X_i \SO x, z \in X_i$
            \end{MyList}
            $\SO \sim$ --- эквивалентность
        \end{MyList}
    \end{proof}
\end{Thm}

\begin{Def}
    $\sim$ на $X$, тогда фактормножество $(X/\sim)$ --- множество, состоящее из классов эквивалентности
\end{Def}


\Subsection{Перестановка}{--- биективное отображение $X = \{1, 2,  \cdots, n\}$ в $X$

Запись перестановки: $ \left(\begin{array}{cccc}
            1 & 2 & 3 & 4 \\
            2 & 4 & 1 & 3
        \end{array}\right)$

\begin{Def}
    Композиция перестановок. ( $\sigma, \tau$  ) \\
    $\sigma, \tau \Rightarrow \sigma \circ \tau = \sigma \tau$ --- выполняется справа налево.
\end{Def}

\begin{Def}
    $e = \left(\begin{array}{cccc}
            1 & 2 & ... & n \\
            1 & 2 & ... & n
        \end{array}\right)$ --- тождественная перестановка
\end{Def}

\begin{Prop} $\forall \sigma \to \exists \sigma^{-1}$ \end{Prop}

Множество всех перестановок $X = \{1, 2, ..., n\}$ обозначается $S_n$

\begin{Def}
    Группой называется некоторое множество $G$, на котором определена бинарная операция: $\forall x, y \in G \to xy \in G$. При этом выполняются следующие аксиомы
    \begin{MyList}
        \item $\forall x, y, z \in G \to (xy)z = x(yz)$ - ассоциативность.
        \item $\forall x \in G \to \exists e \in G : xe = ex = x$ - нейтральный элемент
        \item $\forall x \in G \to \exists x^{-1} \in G : x x^{-1} = x^{-1} x = e$
    \end{MyList}
\end{Def}

\begin{Thm}
    $S_n$ относительно композиции является группой.
\end{Thm}

\begin{Def}
    Порядком группы $G$ называется количество элементов в $G$ \\
    Обозначается $|G|$
\end{Def}

\begin{Def}
    $\left(\begin{array}{ccccc}
            1   & i_1 & i_2 & ... & i_k \\
            i_1 & i_2 & i_3 & ... & 1
        \end{array}\right)$ -- $k$-цикл \\
    $\left(\begin{array}{cc}
            i & j \\
            j & i
        \end{array}\right) = (ij)$ -- транспозиция.
\end{Def}

\begin{Example}

    $\sigma = \left(\begin{array}{ccccc}
            1 & 5 & 3 & 4 & 2
        \end{array}\right) = \left(\begin{array}{ccccc}
            1 & 2 & 3 & 4 & 5 \\
            5 & 1 & 4 & 2 & 3
        \end{array}\right)$  \\
    $\sigma^2 = \left(\begin{array}{ccccc}
            1 & 5 & 3 & 4 & 2
        \end{array}\right)  \left(\begin{array}{ccccc}
            1 & 5 & 3 & 4 & 2
        \end{array}\right)  = \left(\begin{array}{ccccc}
            1 & 3 & 2 & 5 & 4
        \end{array}\right)$
\end{Example}
\begin{Thm} $\forall \sigma \in S_n $ может быть разложена в произведение независимых циклов.
\end{Thm}
\begin{proof}
    $1 \leqslant i, j \leqslant n$. $i \thicksim j \Leftrightarrow \exists p \in \mathbb{Z} : \sigma^p (i) = j$

    \begin{MyList}
        \item $\sigma^0 (i) = i$ - рефлексивность
        \item $\sigma^p (i) = j \Rightarrow \sigma^{-p} (j) = i$ - симметричность
        \item $\sigma^p (i) = j, \sigma^q (j) = k \Rightarrow \sigma^{p + q} (i) = k$ - транзитивность
    \end{MyList}
    $\Rightarrow $ по теореме о разбиении множества $\Rightarrow X = X_1 \cup ... \cup X_s \Rightarrow \forall X_i $ соответствует цикл, длина которого равна $|X_i|$ \\
    Пусть $j \in X_i$, тогда $\left(\begin{array}{ccccc}
                j         & \sigma(j)   & \sigma^2(j) & ... & \sigma^p(j) \\
                \sigma(j) & \sigma^2(j) & \sigma^3(j) & ... & j
            \end{array}\right) \Rightarrow $ все такие циклы независимы.

    \begin{Rem}Можно доказать, что это разложение единственно с точностью до порядка.
    \end{Rem}

\end{proof}

\begin{Cons}
    $\forall \sigma \in S_n$ раскладывается в произведение транспозиций
\end{Cons}

\begin{proof}
    Рассмотрим какой-то $k$-цикл. \\
    $$\left(\begin{array}{ccccc}
            i_1 & i_2 & i_3 & ... & i_k \\
            i_2 & i_3 & i_4 & ... & i_1
        \end{array}\right) = \left(\begin{array}{cc}
            i_1 & i_k \\
            i_k & i_1
        \end{array}\right) \left(\begin{array}{cc}
            i_1       & i_{k - 1} \\
            i_{k - 1} & i_1
        \end{array}\right) ... \left(\begin{array}{cc}
            i_1 & i_3 \\
            i_3 & i_1
        \end{array}\right) \left(\begin{array}{cc}
            i_1 & i_2 \\
            i_2 & i_1
        \end{array}\right)$$
\end{proof}

\begin{Rem}Разложение перестановки в произведение транспозиций не является единственным.\end{Rem}

\Subsection{Знак перестановки}
\begin{Def}
    $\sigma = \tau_1, \tau_2, ..., \tau_k, \tau_i, 1 \leqslant i \leqslant k$ - транспозиции. \\
    Знаком перестановки $\sigma $ называется $\varepsilon_\sigma = (-1)^k$ \\
    \begin{Rem}Если $\tau = (ij) \Rightarrow \tau^2 = (ij)^2 = e$ \end{Rem}
\end{Def}

\begin{Thm} О знаке перестановки
    \begin{MyList}
        \item $\varepsilon_\sigma$ не зависит от способа разложения $\sigma$ на произведение транспозиций
        \item $\varepsilon_\sigma \varepsilon_\tau = \varepsilon_{\sigma \tau}$
    \end{MyList}
\end{Thm}

\begin{proof}
    \begin{MyList}
        \item $\sigma = \tau_1 \tau_2 ... \tau_k = \tau_1 ' \tau_2 ' ... \tau_s ', \tau_i, \tau_j '$ - транспозиции. \\
        $\Rightarrow \tau_1 \tau_2 ... \tau_k \tau_s ' = \tau_1 ' \tau_2 ' ... \tau_{s - 1} ' \Rightarrow  \tau_1 \tau_2 ... \tau_k \tau_s ' \tau_{s - 1} ' = \tau_1 ' \tau_2 ' ... \tau_{s - 2} '
            \Rightarrow e = \tau_1 \tau_2 ... \tau_k \tau_s ' ... \tau_1 '$. \\ Если $k, s$ одной четности $\Rightarrow e $ раскладывается в четное число транспозиций \\
        $k, s$ разной четности $\Rightarrow e$ раскладывается в нечетное число транспозиций.

        Докажем, что $e$ не может быть разложена в нечётное число транспозиций. Найдем транспозицию, содержащую $i$ и будем двигать её влево\\
        $e = \tau_1 \tau_2 ... (ij) ...$ \newpage Смотрим транспозицию слева от $(ij)$:
        $$(ij)(ij) = e \Rightarrow$$ число транспозиций уменьшилось на $2$
        $$(ik)(ij) = (ij)(jk)$$
        $$(jk)(ij) = (ik)(jk)$$
        $$(kl)(ij) = (ij)(kl)$$

        $\Rightarrow$ если не будет пункта $1 \Rightarrow e = (it)...$ \\
        $e(i) = i$. Однако правая часть $i \to t$, что невозможно. $\Rightarrow$ обязательно будет 1 $\Rightarrow$ число транспозиций уменьшится на 2

        Было $k + s$ транспозиций. $k + s - 2, k + s - 4, ... = 0 \Rightarrow k + s$ - чётное.

        \item $\varepsilon_\sigma \varepsilon_\tau = \varepsilon_{\sigma \tau}$ \\
        $\sigma = \tau_1 ... \tau_k, \tau = \tau_1 ' ... \tau_s ' $ \\
        $\varepsilon_\sigma \varepsilon = (-1)^k \cdot (-1)^s = (-1)^{k + s}$ \\
        $\varepsilon_{\sigma \tau} = (-1)^{k + s}$
    \end{MyList}
\end{proof}

\begin{Def}
    Если $\varepsilon = +1$, то перестановка называется четной
\end{Def}

\Subsection{Чётные перестановки}

$A_n = \{\text{чётные перестановки в } S_n\}$

$\overline{A_n} = S_n \setminus A_n$
\begin{Prop}
    $|A_n| = |\overline{A_n}| = \frac{n!}{2}$
\end{Prop}

\begin{proof}
    Пусть $\tau = (ij), \sigma \in A_n, \varphi : A_n \to \overline{A_n}, \varphi (\sigma) = \tau \sigma \in \overline{A_n}$

    Инъективность: $\sigma_1 \neq \sigma_2 \in A_n, \varphi(\sigma_1) = \tau \sigma_1, \varphi(\sigma_2) = \tau \sigma_2$

    Если $\tau \sigma_1 = \tau \sigma_2 \Rightarrow \sigma_1 = \sigma_2$ - противоречие \\
    Сюръективность: Пусть $\rho \in \overline{A_n} \Rightarrow \tau \rho \in A_n \Rightarrow \varphi (\tau \rho) = \tau(\tau \rho) = \rho \Rightarrow \varphi$ - биективно $\Rightarrow |A_n| = |\overline{A_n}|$
\end{proof}

\begin{Rem}
    $e \in A_n, \sigma, \rho \in A_n \Rightarrow \sigma \rho \in A_n$.
    \[\sigma = \tau_1 \tau_2 ... \tau_k, \sigma^{-1} = \tau_k \tau_{k - 1} ... \tau_1 \in A_n\]
    Значит $A_n$ - группа относительно композиции.
\end{Rem}

\begin{Def}
    $G$ - группа. Множество $H \subseteq G$ называется подгруппой $G$, если оно также образует группу.
    Обозначение: $H \leqslant G$
\end{Def}

\begin{Thm}
    $A_n \leqslant S_n \Rightarrow |A_n| = \frac{n!}{2}$
\end{Thm}

\begin{Def}
    $A_n$ - знакопеременная группа (alternating)
\end{Def}
\Pagebreak

\Subsection{Инверсии}
\begin{Def}
    $\sigma = \left(\begin{array}{ccccccc}
    1 & ... & s & ... & t & ... & n \\ 
    i & ... & i_s & ... & i_t & ... & i_n
    \end{array}\right)$. Говорят, что $(s, t)$ образуют инверсию, если $s < t \wedge i_s > i_t$. Количество всех инверсий равно $inv(\sigma)$  
\end{Def}

\begin{Thm}[Инверсии и четность и перестановки]
    $\sigma$ -- четная (нечетная) $\Leftrightarrow inv(\sigma)$ четно (нечетно) 
\end{Thm}

\begin{proof}
    \begin{MyList}
        \item Пусть $\sigma = \left(\begin{array}{cccc}
        ... & s & t & ... \\ 
        ... & i & j & ...
        \end{array}\right), \tau = \left(\begin{array}{cc}
        i & i + 1 \\ 
        i + 1 & i
        \end{array}\right), j = i + 1$ 
        Хотим узнать, как меняется количество инверсий при умножении на $\tau$.
        \[\tau \sigma = \left(\begin{array}{cc}
        i & i + 1 \\ 
        i + 1 & i
        \end{array}\right) \left(\begin{array}{ccccc}
        ... & s & ... & t & ... \\ 
        ... & i & ... & i + 1 & ...
        \end{array}\right) \Rightarrow\] 
        количество инверсий изменится на 1. \\
        Число инверсий в парах без $s$ и $t$ не поменялось. $(k, s), (m, t)$ - тоже не поменялось. $(s, t)$ - изменилось на 1.
        \item $\tau = (ij)$ - произовальная транспозиция. $\sigma$ - произовальная перестановка.
        
        $\tau = \left(\begin{array}{cc}
        i & i + 1
        \end{array}\right) \left(\begin{array}{cc}
        i + 1 & i + 2
        \end{array}\right) ... \left(\begin{array}{cc}
        i + k - 1 & j
        \end{array}\right) \left(\begin{array}{cc}
        i + k - 2 & i + k - 1
        \end{array}\right) ... \left(\begin{array}{cc}
        i + 1 & i + 2
        \end{array}\right) \left(\begin{array}{cc}
        i & i + 1
        \end{array}\right)$ 

        $\Rightarrow \tau$ раскладывается в $2(k - 1) + 1$ транспозицию соседних элементов $\Rightarrow$ число инверсий $\tau \sigma$ изменится на нечётное число.
        
        \item $\sigma = \sigma_1 \sigma_2 ... \sigma_l e$, где $\sigma_i$ -- независимые циклы. 
        
        Если $\sigma_l$ раскладывается в чётное число транспозиций, то в $\sigma_l e$ чётное число инверсий (т.к. каждая транспозиция меняет $inv(\sigma_l)$ на нечетное число). \\
        Если $\sigma_l$ раскладывается в нечётное число транспозиций, то в $\sigma_l e$ нечётное число инверсий. 
    \end{MyList}
\end{proof}
\end{document}