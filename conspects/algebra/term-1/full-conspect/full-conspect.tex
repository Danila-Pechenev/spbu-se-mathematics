\documentclass[12pt]{article}

% Автор: Игорь Смирнов
% Автор стиля: Илья Дудников

\usepackage{cmap}
\usepackage[T2A]{fontenc}
\usepackage[utf8]{inputenc}
\usepackage[russian]{babel}
\usepackage{graphicx}
\usepackage{amsthm,amsmath,amssymb}
\usepackage{listings}
\usepackage{color}
\usepackage{xcolor}
\usepackage{array}
\usepackage{epigraph}

\usepackage[russian,colorlinks=true,urlcolor=red,linkcolor=blue]{hyperref}
\usepackage{enumerate}
\usepackage{datetime}
\usepackage{fancyhdr}
\usepackage{lastpage}
\usepackage{verbatim}
\usepackage{tikz}
\usetikzlibrary{arrows,decorations.markings,decorations.pathmorphing}
\usepackage{pgfplots}

\usepackage{ifthen}
\usepackage{mathtools}

%\usepackage{tabls}
%\usepackage{tabularx}
%\usepackage{xifthen}
%\listfiles

\def\NAME{Лекции}
\def\SEASON{Конспект лекций по алгебре, ПИ, 1 семестр}

\sloppy
\voffset=-20mm
\textheight=235mm
\hoffset=-22mm
\textwidth=180mm
\headsep=12pt
\footskip=20pt

\parskip=0em
\parindent=0em

\setlength\epigraphwidth{.8\textwidth}

\newlength{\tmplen}
\newlength{\tmpwidth}
\newcounter{listcounter}

% Список с маленькими отступами
\newenvironment{MyList}[1][4pt]{
  \begin{enumerate}[1.]
  \setlength{\parskip}{0pt}
  \setlength{\itemsep}{#1}
}{       
  \end{enumerate}
}
% Вложенный список с маленькими отступами
\newenvironment{InnerMyList}[1][0pt]{
  \vspace*{-0.5em}
  \begin{enumerate}[(a)]
  \setlength{\parskip}{-0pt}
  \setlength{\itemsep}{#1}
}{       
  \end{enumerate}
  \vspace*{-0.5em}
}
% Список с маленькими отступами
\newenvironment{MyItemize}[1][4pt]{
  \begin{itemize}
  \setlength{\parskip}{0pt}
  \setlength{\itemsep}{#1}
}{       
  \end{itemize}
}

% Основные математические символы
\def\TODO{{\color{red}\bf TODO}}
\def\C{\mathbb{C}}       %
\def\Q{\mathbb{Q}}       %
\def\N{\mathbb{N}}       %
\def\R{\mathbb{R}}       %
\def\F2{\mathbb{F}_2}    %
\def\Z{\mathbb{Z}}       %
\def\INF{\t{+}\infty}    % +inf
\def\EPS{\varepsilon}    %
\def\EMPTY{\varnothing}  %
\def\PHI{\varphi}        %
\def\SO{\Rightarrow}     % =>
\def\EQ{\Leftrightarrow} % <=>
\def\t{\texttt}          % mono font
\def\c#1{{\rm\sc{#1}}}   % font for classes NP, SAT, etc
\def\O{\mathcal{O}}      %
\def\NO{\t{\#}}          % #
\def\XOR{\text{ {\raisebox{-2pt}{\ensuremath{\Hat{}}}} }}
\renewcommand{\le}{\leqslant}
\renewcommand{\ge}{\geqslant}
\newcommand{\q}[1]{\langle #1 \rangle}               % <x>
\newcommand\URL[1]{{\footnotesize{\url{#1}}}}        %
% \newcommand{\sfrac}[2]{{\scriptscriptstyle\frac{#1}{#2}}}  % Очень маленькая дробь
% \newcommand{\mfrac}[2]{{\scriptstyle\frac{#1}{#2}}}    % Небольшая дробь
\newcommand{\sfrac}[2]{{\scriptstyle\frac{#1}{#2}}}  % Очень маленькая дробь
\newcommand{\mfrac}[2]{{\textstyle\frac{#1}{#2}}}    % Небольшая дробь

\newcommand{\fix}[1]{{\color{fixcolor}{#1}}} % \underline
\def\bonus{\t{\red{(*)}}}
\def\ifbonus#1{\ifthenelse{\equal{#1}{}}{}{\bonus}}
\def\smallsquare{$\scalebox{0.5}{$\square$}$}

\newlength{\myItemLength}
\setlength{\myItemLength}{0.3em}
\def\ItemSymbol{\smallsquare}
\def\Item{\vspace*{\myItemLength}\ItemSymbol \ \ }

\newcommand{\LET}{%
  % [line width=0.6pt]
  \begin{tikzpicture}%
  \draw(0.8ex,0) -- (0.8ex,1.6ex);%
  \draw(0,1.6ex) -- (0.8ex,1.6ex);%
  \end{tikzpicture}%
  \hspace*{0.1em}%
}

% Отступы
\def\makeparindent{\hspace*{\parindent}\unskip}
\def\up{\vspace*{-0.5em}}%{\vspace*{-\baselineskip}}
\def\down{\vspace*{0.5em}}
\def\LINE{\vspace*{-1em}\noindent \underline{\hbox to 1\textwidth{{ } \hfil{ } \hfil{ } }}}
\def\BOX#1{\mbox{\fbox{\bf{#1}}}}
\def\Pagebreak{\pagebreak\vspace*{-1.5em}}

% Мелкий заголовок
\newcommand{\THEE}[1]{
  \vspace*{0.5em}
  \noindent{\bf \underline{#1}}%\hspace{0.5em}
  \vspace*{0.2em}
}
% Другой тип мелкого заголовка
\newcommand{\THE}[1]{
  \vspace*{0.5em} $\bullet$
  \noindent{\bf #1}%\hspace{0.5em}
  \vspace*{0.2em}
}

\newenvironment{MyTabbing}{
  \t\bgroup
  \vspace*{-\baselineskip}
  \begin{tabbing}
    aaaa\=aaaa\=aaaa\=aaaa\=aaaa\=aaaa\kill
}{
  \end{tabbing}
  \t\egroup
}

% Код с правильными отступами
\lstnewenvironment{code}{
  \lstset{}
%  \vspace*{-0.2em}
}%
{
%  \vspace*{-0.2em}
}
\lstnewenvironment{codep}{
  \lstset{language=python}
}%
{
}

% Формулы с правильными отступами
\newenvironment{smallformula}{
 
  \vspace*{-0.8em}
}{
  \vspace*{-1.2em}
  
}
\newenvironment{formula}{
 
  \vspace*{-0.4em}
}{
  \vspace*{-0.6em}
  
}

% Большая квадратная скобка
\makeatletter
\newenvironment{sqcases}{%
  \matrix@check\sqcases\env@sqcases
}{%
  \endarray\right.%
}
\def\env@sqcases{%
  \let\@ifnextchar\new@ifnextchar
  \left\lbrack
  \def\arraystretch{1.2}%
  \array{@{}l@{\quad}l@{}}%
}
\makeatother

% Определяем основные секции: \begin{Lm}, \begin{Thm}, \begin{Def}, \begin{Rem}
\renewcommand{\qedsymbol}{$\blacksquare$}
\theoremstyle{definition} % жирный заголовок, плоский текст
\newtheorem{Thm}{\underline{Теорема}}[subsection] % нумерация будет "<номер subsection>.<номер теоремы>"
\newtheorem{Lm}[Thm]{\underline{Lm}} % Нумерация такая же, как и у теорем
\newtheorem{Ex}[Thm]{Упражнение} % Нумерация такая же, как и у теорем
\newtheorem{Example}[Thm]{Пример} % Нумерация такая же, как и у теорем
\newtheorem{Code}[Thm]{Код} % Нумерация такая же, как и у теорем
\theoremstyle{plain} % жирный заголовок, курсивный текст
\newtheorem{Def}[Thm]{Def} % Нумерация такая же, как и у теорем
\theoremstyle{remark} % курсивный заголовок, плоский текст
\newtheorem{Cons}[Thm]{Следствие} % Нумерация такая же, как и у теорем
\newtheorem{Conj}[Thm]{Гипотеза} % Нумерация такая же, как и у теорем
\newtheorem{Prop}[Thm]{Утверждение} % Нумерация такая же, как и у теорем
\newtheorem{Rem}[Thm]{Замечание} % Нумерация такая же, как и у теорем
\newtheorem{Remark}[Thm]{Замечание} % Нумерация такая же, как и у теорем
\newtheorem{Algo}[Thm]{Алгоритм} % Нумерация такая же, как и у теорем

% Определяем ЗАГОЛОВКИ
\def\SectionName{unknown}
\def\AuthorName{unknown}

\newlength{\sectionvskip}
\setlength{\sectionvskip}{0.5em}
\newcommand{\Section}[4][]{
  % Заголовок
  \pagebreak
%  \ifthenelse{\isempty{#1}}{
    \refstepcounter{section}
%  }{}
  \vspace{0.5em}
%  \ifthenelse{\isempty{#1}}{
%    \addtocontents{toc}{\protect\addvspace{-5pt}}%
    \addcontentsline{toc}{section}{\arabic{section}. #2}
%  }{}
  \begin{center}
    {\Large \bf Раздел \NO{\arabic{section}}: #2} \\ 
    \vspace{\sectionvskip}
    \ifthenelse{\equal{#3}{}}{}{{\large #3}\\}
  \end{center}

  \LINE

  % Запомнили название и автора главы
  \gdef\SectionName{#2}
  \gdef\AuthorName{#4}

  % Заголовок страницы
  \lhead{\SEASON}
  \chead{}
  \rhead{\SectionName}
  \renewcommand{\headrulewidth}{0.4pt}

  \lfoot{Глава \NO{\arabic{section}}.}
  \cfoot{\thepage\t{/}\pageref*{LastPage}}
  \rfoot{Автор: \AuthorName}
  \renewcommand{\footrulewidth}{0.4pt}
}

\newcommand{\Subsection}[2][]{
  \refstepcounter{subsection}
  \vspace*{1em}
  \ifthenelse{\equal{#1}{}}
    {\addcontentsline{toc}{subsection}{\arabic{section}.\arabic{subsection}. #2}}
    {\addcontentsline{toc}{subsection}{\arabic{section}.\arabic{subsection}. \bonus\,#2}}
  {\color{blue}\bf\large \arabic{section}.\arabic{subsection}. \ifbonus{#1}\,{#2}} 
  \vspace*{0.5em}
  \makeparindent
}
\newcommand{\Subsubsection}[2][]{
  \refstepcounter{subsubsection}
  \vspace*{1em}
  \ifthenelse{\equal{#1}{}}
    {\addcontentsline{toc}{subsubsection}{\arabic{section}.\arabic{subsection}.\arabic{subsubsection}. #2}}
    {\addcontentsline{toc}{subsubsection}{\arabic{section}.\arabic{subsection}.\arabic{subsubsection}. \bonus\,#2}}
  {\color{blue}\bf\large \arabic{section}.\arabic{subsection}.\arabic{subsubsection}. \ifbonus{#1}\,#2}
  \vspace*{0.5em}
  \makeparindent
}

\newcommand{\Header}{
  \pagestyle{empty}
  \renewcommand{\dateseparator}{--}
  \begin{center}
    {\Large\bf 
     Алгебра 1 семестр ПИ,\\
    \vspace{0.3em}
    \NAME}\\
    \vspace{0.7em}
    {Собрано {\today} в {\currenttime}}
  \end{center}

  \LINE
  \vspace{0em}

  \renewcommand{\baselinestretch}{0.98}\normalsize
  \tableofcontents
  \renewcommand{\baselinestretch}{1.0}\normalsize
  \pagebreak
}

\newcommand{\BeginConspect}{
  \pagestyle{fancy}
  \setcounter{page}{1}
}

\definecolor{mygray}{rgb}{0.7,0.7,0.7}
\definecolor{ltgray}{rgb}{0.9,0.9,0.9}
\definecolor{fixcolor}{rgb}{0.7,0,0}
\definecolor{red2}{rgb}{0.7,0,0}
\definecolor{dkred}{rgb}{0.4,0,0}
\definecolor{dkblue}{rgb}{0,0,0.6}
\definecolor{dkgreen}{rgb}{0,0.6,0}
\definecolor{brown}{rgb}{0.5,0.5,0}

\newcommand{\green}[1]{{\color{green}{#1}}}
\newcommand{\black}[1]{{\color{black}{#1}}}
\newcommand{\red}[1]{{\color{red}{#1}}}
\newcommand{\dkred}[1]{{\color{dkred}{#1}}}
\newcommand{\blue}[1]{{\color{blue}{#1}}}
\newcommand{\dkgreen}[1]{{\color{dkgreen}{#1}}}

\newcommand{\Mod}[1]{\ (\mathrm{mod}\ #1)}

\DeclareMathOperator{\Real}{Re}
\DeclareMathOperator{\Imag}{Im}
\DeclareMathOperator{\lcm}{lcm}

\begin{document}

\Header

\BeginConspect

\Section{Отношения и перестановки}{}{Вячеслав Бучин, Илья Дудников}

\Subsection{Отношения}

\begin{Def} Отношением $\omega$ на $X \times Y$ называется любое подмножество $X \times Y$.
\end{Def}
Если $X = Y$, то говорят про отношение на $X$.

Отношение на $X$ называется:
\begin{MyList}
    \item рефлексивным, если $\forall x \in X (x,x) \in \omega$
    \item антирефлексивным, если $(x, y) \in \omega \SO x \neq y$
    \item симметричным, если $(x, y) \in \omega \SO (y, x) \in \omega$
    \item антисимметричным, если $(x, y), (y, x) \in \omega \SO y = x$
    \item транзитивным, если $(x, y), (y, z) \in \omega \SO (x, z) \in \omega$
\end{MyList}

\Subsection{Отношение эквивалентности}{}

\begin{Def}
    Отношение на $X$, которое является рефлексивным, симметричным, транзитивным, называется эквивалентностью и обозначается $x \sim y$
\end{Def}

\begin{Example}
    $X = \Z$ $x \omega y \EQ x - y \vdots 5$
    \begin{MyList}
        \item $x - x \vdots 5$ --- рефлексивно
        \item $x - y \vdots 5 \SO y - x \vdots 5$ --- симметрично
        \item $x - y \vdots 5$, $y - z \vdots 5$ $x - z = (x - y) + (y - z) \vdots 5 \SO x - z \vdots 5$ --- транзитивно
    \end{MyList}
    $\SO \omega$ --- отношение эвивалентности
\end{Example}

\Subsection{Класс эквивалентности}{}

\begin{Def}
    Классом эквивалентности, содержащим $a \in X$, называется $[a] = \{x: x \in X, x \sim a\}$
\end{Def}
\begin{Def}
    Разбиением множества $X$ называется  $\pi(X) = \{ X_i \}$:
    \begin{MyList}
        \item $X_1 \cup X_2 \cdots = X$
        \item $\forall i, j: i \neq j, X_i \cap X_j = \EMPTY$
    \end{MyList}
\end{Def}

\begin{Thm}
    Связь эквивалентности и разбиения множества

    \begin{MyList}
        \item Отношения эквивалентности на $X$ задаёт разбиение множества $\pi(X)$, $X_i$ --- классы эквивалентности
        \item Разбиение $\pi(X)$ задаёт эквивалентность на $X$
    \end{MyList}
    \begin{proof}

        \begin{MyList}
            \item $X_i = [x] = \{y \in X: y \sim x\}$ --- перебираем все $x \in X \SO X = X_1 \cup X_2 \cup \cdots X_n \cup \cdots$

            $X_i, X_j : X_i = [x_i], X_j = [x_j]$ предположим, что $a \in X_i \cap X_j \SO a \sim x_i, a \sim x_j \SO x_i \sim x_j \SO X_i = X_j \SO [x_i]$ задают разбиения
            \item $\sim : x \sim y \EQ x, y \in X_i$, проверить, что $\sim$ --- эквивалентность:
            \begin{MyList}
                \item $x, x \in X_i \SO x \sim x$
                \item $x, y \in X_i \SO y, x \in X_i$
                \item $x, y, y, z \in X_i \SO x, z \in X_i$
            \end{MyList}
            $\SO \sim$ --- эквивалентность
        \end{MyList}
    \end{proof}
\end{Thm}

\begin{Def}
    $\sim$ на $X$, тогда фактормножество $(X/\sim)$ --- множество, состоящее из классов эквивалентности
\end{Def}


\Subsection{Перестановка}{--- биективное отображение $X = \{1, 2,  \cdots, n\}$ в $X$

Запись перестановки: $ \left(\begin{array}{cccc}
            1 & 2 & 3 & 4 \\
            2 & 4 & 1 & 3
        \end{array}\right)$

\begin{Def}
    Композиция перестановок. ( $\sigma, \tau$  ) \\
    $\sigma, \tau \Rightarrow \sigma \circ \tau = \sigma \tau$ --- выполняется справа налево.
\end{Def}

\begin{Def}
    $e = \left(\begin{array}{cccc}
            1 & 2 & ... & n \\
            1 & 2 & ... & n
        \end{array}\right)$ --- тождественная перестановка
\end{Def}

\begin{Prop} $\forall \sigma \to \exists \sigma^{-1}$ \end{Prop}

Множество всех перестановок $X = \{1, 2, ..., n\}$ обозначается $S_n$

\begin{Def}
    Группой называется некоторое множество $G$, на котором определена бинарная операция: $\forall x, y \in G \to xy \in G$. При этом выполняются следующие аксиомы
    \begin{MyList}
        \item $\forall x, y, z \in G \to (xy)z = x(yz)$ - ассоциативность.
        \item $\forall x \in G \to \exists e \in G : xe = ex = x$ - нейтральный элемент
        \item $\forall x \in G \to \exists x^{-1} \in G : x x^{-1} = x^{-1} x = e$
    \end{MyList}
\end{Def}

\begin{Thm}
    $S_n$ относительно композиции является группой.
\end{Thm}

\begin{Def}
    Порядком группы $G$ называется количество элементов в $G$ \\
    Обозначается $|G|$
\end{Def}

\begin{Def}
    $\left(\begin{array}{ccccc}
            1   & i_1 & i_2 & ... & i_k \\
            i_1 & i_2 & i_3 & ... & 1
        \end{array}\right)$ -- $k$-цикл \\
    $\left(\begin{array}{cc}
            i & j \\
            j & i
        \end{array}\right) = (ij)$ -- транспозиция.
\end{Def}

\begin{Example}

    $\sigma = \left(\begin{array}{ccccc}
            1 & 5 & 3 & 4 & 2
        \end{array}\right) = \left(\begin{array}{ccccc}
            1 & 2 & 3 & 4 & 5 \\
            5 & 1 & 4 & 2 & 3
        \end{array}\right)$  \\
    $\sigma^2 = \left(\begin{array}{ccccc}
            1 & 5 & 3 & 4 & 2
        \end{array}\right)  \left(\begin{array}{ccccc}
            1 & 5 & 3 & 4 & 2
        \end{array}\right)  = \left(\begin{array}{ccccc}
            1 & 3 & 2 & 5 & 4
        \end{array}\right)$
\end{Example}
\begin{Thm} $\forall \sigma \in S_n $ может быть разложена в произведение независимых циклов.
\end{Thm}
\begin{proof}
    $1 \leqslant i, j \leqslant n$. $i \thicksim j \Leftrightarrow \exists p \in \mathbb{Z} : \sigma^p (i) = j$

    \begin{MyList}
        \item $\sigma^0 (i) = i$ - рефлексивность
        \item $\sigma^p (i) = j \Rightarrow \sigma^{-p} (j) = i$ - симметричность
        \item $\sigma^p (i) = j, \sigma^q (j) = k \Rightarrow \sigma^{p + q} (i) = k$ - транзитивность
    \end{MyList}
    $\Rightarrow $ по теореме о разбиении множества $\Rightarrow X = X_1 \cup ... \cup X_s \Rightarrow \forall X_i $ соответствует цикл, длина которого равна $|X_i|$ \\
    Пусть $j \in X_i$, тогда $\left(\begin{array}{ccccc}
                j         & \sigma(j)   & \sigma^2(j) & ... & \sigma^p(j) \\
                \sigma(j) & \sigma^2(j) & \sigma^3(j) & ... & j
            \end{array}\right) \Rightarrow $ все такие циклы независимы.

    \begin{Rem}Можно доказать, что это разложение единственно с точностью до порядка.
    \end{Rem}

\end{proof}

\begin{Cons}
    $\forall \sigma \in S_n$ раскладывается в произведение транспозиций
\end{Cons}

\begin{proof}
    Рассмотрим какой-то $k$-цикл. \\
    $$\left(\begin{array}{ccccc}
            i_1 & i_2 & i_3 & ... & i_k \\
            i_2 & i_3 & i_4 & ... & i_1
        \end{array}\right) = \left(\begin{array}{cc}
            i_1 & i_k \\
            i_k & i_1
        \end{array}\right) \left(\begin{array}{cc}
            i_1       & i_{k - 1} \\
            i_{k - 1} & i_1
        \end{array}\right) ... \left(\begin{array}{cc}
            i_1 & i_3 \\
            i_3 & i_1
        \end{array}\right) \left(\begin{array}{cc}
            i_1 & i_2 \\
            i_2 & i_1
        \end{array}\right)$$
\end{proof}

\begin{Rem}Разложение перестановки в произведение транспозиций не является единственным.\end{Rem}

\Subsection{Знак перестановки}
\begin{Def}
    $\sigma = \tau_1, \tau_2, ..., \tau_k, \tau_i, 1 \leqslant i \leqslant k$ - транспозиции. \\
    Знаком перестановки $\sigma $ называется $\varepsilon_\sigma = (-1)^k$ \\
    \begin{Rem}Если $\tau = (ij) \Rightarrow \tau^2 = (ij)^2 = e$ \end{Rem}
\end{Def}

\begin{Thm} О знаке перестановки
    \begin{MyList}
        \item $\varepsilon_\sigma$ не зависит от способа разложения $\sigma$ на произведение транспозиций
        \item $\varepsilon_\sigma \varepsilon_\tau = \varepsilon_{\sigma \tau}$
    \end{MyList}
\end{Thm}

\begin{proof}
    \begin{MyList}
        \item $\sigma = \tau_1 \tau_2 ... \tau_k = \tau_1 ' \tau_2 ' ... \tau_s ', \tau_i, \tau_j '$ - транспозиции. \\
        $\Rightarrow \tau_1 \tau_2 ... \tau_k \tau_s ' = \tau_1 ' \tau_2 ' ... \tau_{s - 1} ' \Rightarrow  \tau_1 \tau_2 ... \tau_k \tau_s ' \tau_{s - 1} ' = \tau_1 ' \tau_2 ' ... \tau_{s - 2} '
            \Rightarrow e = \tau_1 \tau_2 ... \tau_k \tau_s ' ... \tau_1 '$. \\ Если $k, s$ одной четности $\Rightarrow e $ раскладывается в четное число транспозиций \\
        $k, s$ разной четности $\Rightarrow e$ раскладывается в нечетное число транспозиций.

        Докажем, что $e$ не может быть разложена в нечётное число транспозиций. Найдем транспозицию, содержащую $i$ и будем двигать её влево\\
        $e = \tau_1 \tau_2 ... (ij) ...$ \newpage Смотрим транспозицию слева от $(ij)$:
        $$(ij)(ij) = e \Rightarrow$$ число транспозиций уменьшилось на $2$
        $$(ik)(ij) = (ij)(jk)$$
        $$(jk)(ij) = (ik)(jk)$$
        $$(kl)(ij) = (ij)(kl)$$

        $\Rightarrow$ если не будет пункта $1 \Rightarrow e = (it)...$ \\
        $e(i) = i$. Однако правая часть $i \to t$, что невозможно. $\Rightarrow$ обязательно будет 1 $\Rightarrow$ число транспозиций уменьшится на 2

        Было $k + s$ транспозиций. $k + s - 2, k + s - 4, ... = 0 \Rightarrow k + s$ - чётное.

        \item $\varepsilon_\sigma \varepsilon_\tau = \varepsilon_{\sigma \tau}$ \\
        $\sigma = \tau_1 ... \tau_k, \tau = \tau_1 ' ... \tau_s ' $ \\
        $\varepsilon_\sigma \varepsilon = (-1)^k \cdot (-1)^s = (-1)^{k + s}$ \\
        $\varepsilon_{\sigma \tau} = (-1)^{k + s}$
    \end{MyList}
\end{proof}

\begin{Def}
    Если $\varepsilon = +1$, то перестановка называется четной
\end{Def}

\Subsection{Чётные перестановки}

$A_n = \{\text{чётные перестановки в } S_n\}$

$\overline{A_n} = S_n \setminus A_n$
\begin{Prop}
    $|A_n| = |\overline{A_n}| = \frac{n!}{2}$
\end{Prop}

\begin{proof}
    Пусть $\tau = (ij), \sigma \in A_n, \varphi : A_n \to \overline{A_n}, \varphi (\sigma) = \tau \sigma \in \overline{A_n}$

    Инъективность: $\sigma_1 \neq \sigma_2 \in A_n, \varphi(\sigma_1) = \tau \sigma_1, \varphi(\sigma_2) = \tau \sigma_2$

    Если $\tau \sigma_1 = \tau \sigma_2 \Rightarrow \sigma_1 = \sigma_2$ - противоречие \\
    Сюръективность: Пусть $\rho \in \overline{A_n} \Rightarrow \tau \rho \in A_n \Rightarrow \varphi (\tau \rho) = \tau(\tau \rho) = \rho \Rightarrow \varphi$ - биективно $\Rightarrow |A_n| = |\overline{A_n}|$
\end{proof}

\begin{Rem}
    $e \in A_n, \sigma, \rho \in A_n \Rightarrow \sigma \rho \in A_n$.
    \[\sigma = \tau_1 \tau_2 ... \tau_k, \sigma^{-1} = \tau_k \tau_{k - 1} ... \tau_1 \in A_n\]
    Значит $A_n$ - группа относительно композиции.
\end{Rem}

\begin{Def}
    $G$ - группа. Множество $H \subseteq G$ называется подгруппой $G$, если оно также образует группу.
    Обозначение: $H \leqslant G$
\end{Def}

\begin{Thm}
    $A_n \leqslant S_n \Rightarrow |A_n| = \frac{n!}{2}$
\end{Thm}

\begin{Def}
    $A_n$ - знакопеременная группа (alternating)
\end{Def}
\Pagebreak

\Subsection{Инверсии}
\begin{Def}
    $\sigma = \left(\begin{array}{ccccccc}
    1 & ... & s & ... & t & ... & n \\ 
    i & ... & i_s & ... & i_t & ... & i_n
    \end{array}\right)$. Говорят, что $(s, t)$ образуют инверсию, если $s < t \wedge i_s > i_t$. Количество всех инверсий равно $inv(\sigma)$  
\end{Def}

\begin{Thm}[Инверсии и четность и перестановки]
    $\sigma$ -- четная (нечетная) $\Leftrightarrow inv(\sigma)$ четно (нечетно) 
\end{Thm}

\begin{proof}
    \begin{MyList}
        \item Пусть $\sigma = \left(\begin{array}{cccc}
        ... & s & t & ... \\ 
        ... & i & j & ...
        \end{array}\right), \tau = \left(\begin{array}{cc}
        i & i + 1 \\ 
        i + 1 & i
        \end{array}\right), j = i + 1$ 
        Хотим узнать, как меняется количество инверсий при умножении на $\tau$.
        \[\tau \sigma = \left(\begin{array}{cc}
        i & i + 1 \\ 
        i + 1 & i
        \end{array}\right) \left(\begin{array}{ccccc}
        ... & s & ... & t & ... \\ 
        ... & i & ... & i + 1 & ...
        \end{array}\right) \Rightarrow\] 
        количество инверсий изменится на 1. \\
        Число инверсий в парах без $s$ и $t$ не поменялось. $(k, s), (m, t)$ - тоже не поменялось. $(s, t)$ - изменилось на 1.
        \item $\tau = (ij)$ - произовальная транспозиция. $\sigma$ - произовальная перестановка.
        
        $\tau = \left(\begin{array}{cc}
        i & i + 1
        \end{array}\right) \left(\begin{array}{cc}
        i + 1 & i + 2
        \end{array}\right) ... \left(\begin{array}{cc}
        i + k - 1 & j
        \end{array}\right) \left(\begin{array}{cc}
        i + k - 2 & i + k - 1
        \end{array}\right) ... \left(\begin{array}{cc}
        i + 1 & i + 2
        \end{array}\right) \left(\begin{array}{cc}
        i & i + 1
        \end{array}\right)$ 

        $\Rightarrow \tau$ раскладывается в $2(k - 1) + 1$ транспозицию соседних элементов $\Rightarrow$ число инверсий $\tau \sigma$ изменится на нечётное число.
        
        \item $\sigma = \sigma_1 \sigma_2 ... \sigma_l e$, где $\sigma_i$ -- независимые циклы. 
        
        Если $\sigma_l$ раскладывается в чётное число транспозиций, то в $\sigma_l e$ чётное число инверсий (т.к. каждая транспозиция меняет $inv(\sigma_l)$ на нечетное число). \\
        Если $\sigma_l$ раскладывается в нечётное число транспозиций, то в $\sigma_l e$ нечётное число инверсий. 
    \end{MyList}
\end{proof}

\Section{Теория чисел}{}{Илья Дудников}
\Subsection{Делимость}
\begin{Def}
    $a \vdots b$ или $b | a \Leftrightarrow \exists q : a = b \cdot q, b \neq 0$ 
\end{Def}

Свойства:
\begin{MyList}
    \item Рефлексивность. $a \vdots a, a \neq 0$ 
    \item Антисимметричность на $\N$. $a \vdots b, b \vdots a \Rightarrow a = b$
    \item Транзитивность. $a \vdots b, \vdots c \Rightarrow a \vdots c$ 
    \item $a | b, a | c \Rightarrow a | (b \pm c)$.
    \begin{proof}
        $b = a \cdot q_1, c = a q_2 \Rightarrow b \pm c = a q_1 \pm a q_2 = a(q_1 \pm q_2)$ 
    \end{proof}
    \item $a | b \Rightarrow \forall c \to a | bc$ 
    \item Пусть $a | b_i, i = 1, ..., n, a | (b_1 + ... + b_n + c) \Rightarrow a | c$
    \begin{proof}
        $b_1 + ... + b_n + c = aq, a q_1 + a q_2 + ... + a q_n + c = aq \Rightarrow c = a(q - q_1 - ... - q_n) \\ \Rightarrow a | c$ 
    \end{proof}
    \item $a | b \Rightarrow \forall k \neq 0 \to ka | kb$ 
    \item $ka | kb \Rightarrow a | b$ 
\end{MyList}

\begin{Thm}[О делении с остатком]
    $$\forall a \wedge \forall b > 0 \ \exists ! q, r, 0 \leqslant r < b : a = bq + r$$
\end{Thm}

\begin{Def}
    $a$ - делимое, $b$ - делитель, $q$ - частное (неполное частное), $r$ - остаток
\end{Def}

\begin{proof}
    $\exists$-ние. Рассмотрим $a - bq$. Выберем $q$ так, чтобы $a - bq > 0$ было наименьшим. 
    Положим $r = a - bq \geqslant 0 \Rightarrow a = bq + r$. По выбору $q \to a - b(q + 1) < 0 \Rightarrow a < b(q + 1) \Rightarrow \\ 
    r = a - bq < b(q + 1) - bq = b$.

    Единственность. Преположим, что $a = b q_1  + r_1 = b q_2 + r_2, 0 \leqslant r_1, r_2 < b$ 
    \[|r_1 - r_2| < b, b q_1 + r_1 = b q_2 + r_2 \Rightarrow b(q_1 - q_2) = r_2 - r_1 \Rightarrow |b(q_1 - q_2)| \geqslant b\]
    Но $|r_1 - r_2| < b$ - противоречие.
\end{proof} 
\Pagebreak

\Subsection{Наибольший общий делитель}
\begin{Def}
    Общим делителем $a_1, a_2, ..., a_n$ называется $d: d | a_i, i = 1, ..., n$.
\end{Def}

\begin{Def}
    Наибольший общий делитель $a_1, a_2, ..., a_n$ называется $d$ такое, что 
    \begin{MyList}
        \item $d > 0$ 
        \item $d | a_i, i = 1, ..., n$ 
        \item если $d' | a_i, i = 1, ..., n$, то $d' | d$ 
    \end{MyList} 
    Обозначается $\gcd(a_1, a_2, ..., a_n) = (a_1, a_2, ..., a_n)$ 

    \begin{Rem}
        По определению $\gcd(0, 0) = 0$. $a \neq 0$, то $\gcd(a, 0) = 0$   
    \end{Rem}
\end{Def}

Свойства:
\begin{MyList}
    \item $b | a \Rightarrow (a, b) = b$ 
    \begin{proof}
        Докажем, что множество делителей $(a, b)$ совпадает с множество делителей $b$. \\
        $$d | (a, b) \Rightarrow d | b$$
        $$d | b \Rightarrow d | a \text{(по транзитивности)}\Rightarrow d | (a, b)$$    
    \end{proof}
    \item $a = bq + c \Rightarrow (a, b) = (b, c)$
    \item \begin{Algo}[Алгоритм Евклида]
        \[a = b q_1 + r_1, 0 \leqslant r_1 < b\]
        \[b = r_1 q_1 + r_2, 0 \leqslant r_2 < r_1\]
        \[r_1 = r_2 q_3 + r_3, 0 \leqslant r_3 < r_2\]
        \[...\]
        \[r_{n - 2} = r_{n - 1} q_n + r_n, 0 \leqslant r_n < r_{n - 1}\]
        \[r_{n - 1} = r_n q_{n + 1}\] 
    \end{Algo} 
    \begin{Thm}
        $r_n = \gcd (a, b)$ 
    \end{Thm}

    \begin{proof}
        $r_1 > r_2 > r_3 > ... \geqslant 0 \Rightarrow \exists r_{n + 1} = 0$. $r_n | r_{n - 1}$.  
        \[r_n = (r_n, r_{n - 1}) = (r_{n - 1}, r_{n - 2}) = ... = (r_2, r_1) = (b, r_1) = (a, b)\] 
    \end{proof}
    \Pagebreak
    \item $(ma, mb) = m \cdot (a, b)$ 
    \item $d | a, d | b \Rightarrow \left(\frac{a}{d}, \frac{b}{d}\right) = \frac{(a, b)}{d}$
    \begin{proof}
        $(a, b) = \left(d \cdot \frac{a}{d}, d \cdot \frac{b}{d}\right) = d \cdot \left(\frac{a}{d}, \frac{b}{d}\right)$ 
    \end{proof}
    \item $(a, b) = 1 \Rightarrow (a, bc) = (a, c)$ 
    \begin{proof}
        Докажем, что $(a, bc) | (a, c)$ 
        \[(a, bc) | a, (a, bc) | ac, (a, bc) | bc \Rightarrow (a, bc) | (ac, bc) \Rightarrow (a, bc) | (a, b) \cdot c = c\Rightarrow (a, bc) | (a, c)\]

        Теперь докажем, что $(a, c) | (a, bc)$
        \[(a, c) | a, (a, c) | c \Rightarrow (a, c) | bc \Rightarrow (a, c) | (a, bc) \Rightarrow (a, bc) = (a, c)\] 
    \end{proof}
    \item $(a, b) = 1, b | ac \Rightarrow b | c$
    \begin{proof}
        \[b | bc, b | ac \Rightarrow b | (bc, ac) = c\] 
    \end{proof}
    \item $(a, b) = (a - b, b)$
\end{MyList}

\begin{Thm}[Линейное представление НОД]
    $$(a, b) = d \Rightarrow \exists u, v : u \cdot a + v \cdot b = d$$
\end{Thm}

\begin{proof}
    Из алгоритма Евклида:
    \[r_{n - 2} = r_{n - 1} \cdot q_n + r_n \Rightarrow d = r_n = r_{n - 2} - r_{n - 1} \cdot q_n\]
    \[r_{n - 3} = r_{n - 2} \cdot q_{n - 1} + r_{n - 1} \Rightarrow d = r_{n - 2} - (r_{n - 3} - r_{n - 2} \cdot q_{n - 1}) q_n\]
    Из следующей строки выражаем $r_{n - 2}$ и т.д. $\Rightarrow$ останутся $a$ и $b$ $\Rightarrow d = u \cdot a + v \cdot b$  
\end{proof}


\Subsection{Наименьшее общее кратное}

\begin{Def}
    Общим кратным $a_1, a_2, ..., a_n$ называется число $M > 0 : a_i | M \ \forall i = 1, ..., n$ 
    
    Наименьшее из общих кратных -- НОК.
\end{Def}

\begin{Thm}
    $\lcm (a, b) = \frac{ab}{\gcd (a, b)}$ 
\end{Thm}

\begin{proof}
    $a = a_1 d, b = b_1 d, (a_1, b_1) = 1$
    \[M = at = bs \SO \frac{M}{b} = \frac{at}{b} = \frac{a_1 dt}{b_1 d} = \frac{a_1 t}{b_1} \SO M = \frac{b \cdot a_1 t}{b_1}\]
    $t$ делится на $b_1$, т.е. $t = b_1 k$
    \[M = \frac{b a_1 b_1 k}{b_1} = b a_1 k \text{ - минимально при } k = 1 \SO M = b a_1 = \frac{b a_1 d}{d} = \frac{ab}{\gcd(a, b)}\]  
\end{proof}

\Subsection{Математическая индукция}
\begin{MyList}
    \item Аксоима. $\forall$ подмножество $\N$ имеет наши элементы $\SO$ ММИ.
    \item Аксиома. $A_1, A_n \SO A_{n + 1} \SO \forall A_n$ 
\end{MyList}

\begin{Cons}
    Пусть $a_1, a_2, ..., a_n$ -- попарно взаимно-простые $\SO \lcm(a_1, a_2, ..., a_n) = a_1 \cdot a_2 \cdot ... \cdot a_n$  
\end{Cons}

\begin{proof}
    $n = 2$. $\lcm (a_1, a_2) = \frac{a_1 a_2}{\gcd(a_1, a_2)} = a_1 \cdot a_2$
    
    Пусть верно для $n$. Тогда для $n + 1$
    \[(a_i, a_n a_{n + 1}) = (a_i, a_{n + 1}) = 1 \SO a_1, a_2, ..., a_{n - 1}, a_n a_{n + 1} \SO\] 
    \[\SO \lcm (a_1, ..., a_{n - 1}, a_n \cdot a_{n + 1}) = a_1 \cdot a_2 \cdot ... \cdot a_{n - 1} \cdot a_n \cdot a_{n + 1}\]  
\end{proof}

\Subsection{Простые числа}

\begin{Def}
    Число $p > 1$ называется простым, если оно делится только на $1$ и на $p$. Иначе число называется составным.
\end{Def}

\begin{Thm}[о наименьшем делителе]
    Наименьший делитель $a > 1$ -- простое число
\end{Thm}

\begin{proof}
    $M = \left\{d | d > 1, d | a\right\} \neq \varnothing$ Пусть $p$ - наименьший элемент $M$.
    Предположим, что $p$ -- составное, т.е. $p = bq, q < p, q | p, p | a \SO q | a$ -- противоречие. 
\end{proof}
\Pagebreak

\begin{Thm}
    $p$ - наименьший делитель $ > 1$ числа $n \SO p \leqslant \sqrt{n}$  
\end{Thm}

\begin{proof}
    \[n = mp, p \leqslant m \SO np \leqslant nm \SO mp \cdot p \leqslant nm \SO p^2 \leqslant n \SO p \leqslant \sqrt{n}\]
\end{proof}

\begin{Thm}[Теорема Евклида]
    Простых чисел бесконечно много
\end{Thm}

\begin{proof}
    Пусть $p_1, p_2, ..., p_n$ -- все простые числа, $a = p_1 \cdot p_2 \cdot ... \cdot p_n + 1$.
    Если $a \vdots p_i$, то $1 \vdots p_i \SO a$ -- новое простое число.  
\end{proof}

\Subsection{Основная теорема арифметики}
\begin{Lm}
    $p$ -- простое $\SO \forall a > 1 \to p | a \vee (p, a) = 1$ 
\end{Lm}

\begin{proof}
    \[(p, a) | p \SO (p, a) = 1 \vee (p, a) = p\] 
\end{proof}

\begin{Lm}
    $p$ -- простое, $p | a_1 \cdot a_2 \cdot ... \cdot a_n \SO \exists i = 1, ..., n : p | a_i$ 
\end{Lm}

\begin{proof}
    Если $(p, a_i) = 1, i = 1, ..., n \SO 1 = (p, a_1) = (p, a_1 a_2) = (p, a_1 a_2 a_3) = (p, a_1 \cdot ... \cdot a_n) = 1 \SO \exists a_i : p | a_i$ 
\end{proof}

\begin{Thm}[Основная теорема арифметики]
    \begin{MyList}
        \item $\forall a > 1 \to a = p_1^{\alpha_1} \cdot p_2^{\alpha_2} \cdot ... \cdot p_k^{\alpha_k}, p_1, p_2, ..., p_k$ -- различные простые, $\alpha_1, \alpha_2, ..., \alpha_k \geqslant 1$
        \item с точностью до перестановки множителей это представление единственно  
    \end{MyList}
\end{Thm}

\begin{proof}
    \begin{MyList}
        \item из всех делителей $a$ выбираем наименьший -- $p_1$ - простое $\SO a = p_1 \cdot a_1$.
        Рассмотрим $a_1$ -- наименьший делитель - $p_2 \SO a_1 = p_1 \cdot p_2 \cdot a_2$ и т.д. 
        \[a_1 > a_2 > a_3 > ... \SO \exists a_n = 1 \SO a = \text{ разложение на простые}\]
        
        \item Предположим, что представление не одно, то есть 
        \[a = p_1 \cdot p_2 \cdot ... \cdot p_s = q_1 \cdot q_2 \cdot ... \cdot q_n\]
        Не умаляя общности, пусть $n \geqslant s \SO p_1 | q_1 ... q_n$. Тогда, по лемме 2 $p_1 | q_i \SO p_1 = q_i$.
        Перенумеруем $i = 1 \SO p_2 p_3 ... p_s = q_2 q_3 ... q_n \SO$ все $p_s$ сократятся, т.е. $1 = q_{s + 1} ... q_n \SO s = n$     
    \end{MyList}
\end{proof}

\begin{Def}
    $a = p_1^{\alpha_1} \cdot p_2^{\alpha_2} \cdot p_k^{\alpha_k}$ --- каноническое разложение числа $a$ 
\end{Def}

\begin{Cons}
    Любой делитель $a = p_a^{\alpha_1} ... p_k^{\alpha_k}$ имеет вид $b = p_1^{\beta_1} \cdot ... \cdot p_k^{\beta_k}, 0 \leqslant \beta_i \leqslant \alpha_i$  
\end{Cons}

\begin{proof}
    $b | a \SO b$ содержит в разложении $p_i$  
\end{proof}

\Pagebreak
\begin{Cons}
    $\gcd (a_1, ..., a_n)$ имеет вид $p_1^{\alpha_1} p_2^{\alpha_2} ... p_k^{\alpha_k}$, где \\
    $a_i = \min \left\{\text{показатель степени $p_i$, с которым $p_i$ входит в разложение } a_1, a_2, ..., a_n\right\}$   
\end{Cons}

\begin{Cons}
    $\lcm (a_1, ..., a_n)$ имеет вид $p_1^{\alpha_1} p_2^{\alpha_2} ... p_k^{\alpha_k}$, где \\
    $a_i = \max \left\{\text{показатель степени $p_i$, с которым $p_i$ входит в разложение } a_1, a_2, ..., a_n\right\}$   
\end{Cons}

\Subsection{Непрерывные дроби (Цепные дроби)}

\begin{Def}
    Выражение вида 
    \[a_0 + \frac{1}{a_1 + \frac{1}{a_2 + \frac{1}{a_3 + ...}}}\] 
    называется непрерывной дробью. Обозначение: $[a_0, a_1, a_2, ...]$ 
\end{Def}

\begin{Thm}
    Любое вещественное число может быть представлено в виде непрерывной дроби.

    Если число иррационально -- в виде бесконечной дроби, если рациональное -- в виде конечной.
\end{Thm}

\begin{proof}
    $a > b$
    \[\frac{a}{b} = a_0 + \frac{r_1}{b} = a_0 + \frac{1}{\frac{b}{r_1}} = a_0 + \frac{1}{a_1 + \frac{r_2}{r_1}} = a_0 + \frac{1}{a_1 + \frac{1}{a_2 + \frac{r_3}{r_2}}} \text { и т.д.}\]  

    где
    \begin{align*}
        a &= b \cdot a_0 + r_1 \\
        b &= r_1 \cdot a_1 + r_2 \\
        r_1 &= r_2 \cdot a_2 + r_3 \\
    \end{align*} 
\end{proof}

\begin{Def}
    Для $\frac{a}{b} \ \delta_0 = \frac{a_0}{1}, \delta_1 = a_0 + \frac{1}{a_1}, \delta_2 = a_0 + \frac{1}{a_1 + \frac{1}{a_2}}$ и т.д. называются подходящими дробями. 
\end{Def}

\begin{Thm}[Формулы подходящих дробей]
    $\delta_k = \frac{p_k}{q_k}, p_{-1} = 1, q_{-1} = 0, p_0 = a_0, q = 1$
    \[\SO \begin{cases}
        p_k = a_k \cdot p_{k - 1} + p_{k - 2} \\
        q_k = a_k \cdot q_{k - 1} + q_{k - 2}
    \end{cases}\] 
\end{Thm}

\begin{proof}
    $\delta_1 = a_0 + \frac{1}{a_1} = \frac{a_0 a_1 + 1}{a_1 \cdot 1 + 0} = \frac{a_1 \cdot p_0 + p_{-1}}{a_1 \cdot q_0 + q_{-1}}$ 

    Предположим, что для $k$ верно. Тогда для $k + 1$ 
    \[\delta_{k + 1} = a_0 + \frac{1}{a_1 + \frac{1}{a_2 + ... \frac{}{a_k + \frac{1}{a_{k + 1}}}}} = \frac{(a_k + \frac{1}{a_{k + 1}}) \cdot p_{k - 1} + p_{k - 2}}{(a_k + \frac{1}{a_{k + 1}})\cdot q_{k - 1} + q_{k - 2}} = \]
    \[= \frac{(a_{k + 1} \cdot a_k + 1) \cdot p_{k - 1} + p_{k - 2} \cdot a_{k + 1}}{(a_{k + 1} \cdot a_k + 1) \cdot q_{k - 1} + q_{k - 2} \cdot a_{k + 1}} = \frac{a_{k + 1}(a_k \cdot p_{k - 1} + p_{k - 2}) + p_{k - 1}}{a_{k + 1}(a_k q_{k - 1} + q_{k - 2}) + q_{k - 1}} = \frac{a_{k + 1} \cdot p_k + p_{k - 1}}{a_{k + 1} \cdot q_k + q_{k - 1}}\]
\end{proof}

\Section{Теория сравнений}{}{Илья Дудников}

\Subsection{Начала теории сравнений}

\begin{Def}
    $a$ и $b$ называются сравнимыми по модулю $m > 0$, если они имеют одинаковые остатки при делении на $m$
    \[a \equiv b \Mod m, a \equiv b(m), a \stackrel{m}{\equiv} b\]
\end{Def}

\begin{Prop}
    \[\EQ \begin{cases}
        a \equiv b \Mod m \\
        a - b \vdots m \\
        a \equiv b + mt
    \end{cases}\]
\end{Prop}

\begin{proof}
    $1) \SO 2)$
    \[a = mq_1 + r, b = mq_2 + r \SO a - b = m(q_1 - q_2) \vdots m\]
    $2) \SO 3)$
    \[a - b \vdots m \SO a - b = mt \SO a = b + mt\]
    $3) \SO 1)$. Поделим $a$ и $b$ на $m$:
    \[a = mq_1 + r_1, b = mq_2 + r_2\]
    \begin{align*}
    &3): a = b + mt \SO mq_1 + r_1 = mq_2 + r_2 + mt \SO \\
    &\SO m(q_1 - q_2 - t) = r_2 - r_1 \SO m | r_2 - r_1 \SO r_2 - r_1 = 0
    \end{align*}
\end{proof}

Свойства:
\begin{MyList}
    \item Рефлексивность. $a \equiv a \Mod m$
    \item Симметричность. $a \equiv b \Mod m \SO b \equiv a \Mod m$
    \item Транзитивность. $a \equiv b \Mod m \SO b \equiv c \Mod m \SO a \equiv c \Mod m$
    \begin{proof}
        \[a - c = a - b + b - c \vdots m\]
    \end{proof}
    \item $a \equiv b \Mod m, c \equiv d \Mod m \SO a + c \equiv b + d \Mod m$
    \item $a \equiv b \Mod m, c \equiv d \Mod m \SO ac \equiv bd \Mod m$
    \begin{proof}
        \[ac - bd = ac - bc + bc - bd = c(a - b) + b(c - d) \vdots m\]
    \end{proof}
    \Pagebreak
    \item $d | a, d | b, d | m, a \equiv b \Mod m \SO \frac{a}{d} \equiv \frac{b}{d} \Mod{\frac{m}{d}}$
    \begin{proof}
        \[a - b = a_1 d - b_1 d = my = m_1 d t \SO a_1 - b_1 = m_1 t\]
    \end{proof}
    \item $a \equiv b \Mod m \SO ka \equiv kb \Mod m$ 
    \item $d | a, d | b, (m, d) = 1, a \equiv b \Mod m \SO \frac{a}{d} \equiv \frac{b}{d} \Mod m$
    \begin{proof}
        \[a = a_1 d, b = b_1 d, a - b \vdots m \SO (a_1 - b_1) \cdot d \vdots m \SO a_1 - b_1 \vdots m\]
    \end{proof}
    \item $d | m, a \equiv b \Mod m \SO a \equiv b \Mod d$ 
    \item $a \equiv b \Mod m \SO (a, m) = (b, m)$ 
    \begin{proof}
        \[a \equiv b \Mod m \SO a = b + mt \SO (a, m) = (b, m)\]
    \end{proof}
\end{MyList}

\Subsection{Классы вычетов}

\begin{Def}
    Классом вычетов по $\Mod m$ называется множество чисел, сравнимых с $a$ по модулю $m$
    \[m = 7, \overline{1} = \{-6, 8, 1, 15, ...\}\]
    \[\overline{a} = \{x | x \equiv a \Mod m\}\]
    Элементы классов вычетов -- \textbf{вычеты}. Обычно рассматривают наименьший неотрицательный вычет. 
\end{Def}

\begin{Def}
    Множество вычетов, взятых по одному из разных классов образуют полную систему вычетов. Например
    \[\{0, 1, 2, ..., m - 1\}\]
\end{Def}

\begin{Lm}
    Множество из $m$ чисел, попарно несравнимых по модулю $m$, образуют полную систему вычетов.
\end{Lm}

\begin{Thm}
    $(a, m) = 1$. Если $x$ пробегает полную систему вычетов по $\Mod m \SO \forall b \to ax + b$ тоже пробегает полную систему вычетов по $\Mod m$  
\end{Thm}

\begin{proof}
    $x$ принадлежит $m$ значений $\SO ax + b$ принадлежит $m$ значений. \\
    Пусть $x_1 \not\equiv x_2 \Mod m$. Предположим, что $ax_1 + b \equiv ax_2 + b \Mod m \SO ax_1 \equiv ax_2 \Mod m \SO x_1 \equiv x_2 \Mod m$ 
\end{proof}

\Pagebreak
\Subsection{Кольцо классов вычетов}
\begin{Def}
    Определим сложение и умножение вычетов по фиксированному модулю $m$.
    \[\overline{a} + \overline{b} = \overline{a + b}, \overline{a} \cdot \overline{b} = \overline{ab}\]
\end{Def}

\begin{Lm}
    Сложение и умножение определены корректно
\end{Lm}

\begin{proof}
    $a \equiv a_1 \Mod m, b \equiv b_1 \Mod m$
    \[\SO a + b = a_1 + b_1 \Mod m, a \cdot b = a_1 \cdot b_1 \Mod m \SO \overline{a} + \overline{b} = \overline{a}_1 + \overline{b}_1, \overline{a} \cdot \overline{b} = \overline{a}_1 \cdot \overline{b}_1\] 
\end{proof}

\begin{Def}
    Группа $G$ называется коммутативной (абелевой), Если
    \[\forall x, y \in G \to xy = yx\] 
\end{Def}

\begin{Thm}
    $\Z_m$ образует коммутативную группу относительно сложения 
\end{Thm}

\begin{proof}
    $\overline{a} + \overline{b} = \overline{a + b} \in \Z_m$
    \begin{MyList}
        \item $(\overline{a} + \overline{b}) + \overline{c} = \overline{a + b} + \overline{c} = \overline{a + b + c}$ \\
        $\overline{a} + (\overline{b} + \overline{c}) = \overline{a} + \overline{b + c} = \overline{a + b + c}$
        \item $\overline{0}$. $\overline{a} + \overline{0} = \overline{a + 0} = \overline{a}$
        \item $-\overline{a} = \overline{m - a} \SO \overline{a} - \overline{a} = \overline{a + m - a} = \overline{0}$
        \item $\overline{a} + \overline{b} = \overline{b} + \overline{a}$      
    \end{MyList} 
\end{proof}

\begin{Def}
    (Ассоциативным) кольцом называется множество $R$, на котором заданы бинарные операции:
    \begin{MyList}
        \item $\forall x, y, z \to (x + y) + z = x + (y + z)$ 
        \item $\exists 0 \in R : \forall x \in R \to x + 0 = x$ 
        \item $\forall x \in R \ \exists (-x) \in R : x + (-x) = 0$ 
        \item $\forall x, y \in R \to x + y = y + x$ 
        \item $\forall x, y, z \in R \to (y + z) = xy + xz, (x + y)z = xz + yz$ 
        \item $\forall x, y, z \in R \to (xy)z = x(yz)$ 
    \end{MyList}
\end{Def}

\begin{Rem}
    $\exists 1 \in R : \forall x \in R \to x \cdot 1 = 1 \cdot x = x$ -- кольцо с единицей \\
    $\forall x, y \in R \to xy = yx$ -- коммутативное кольцо 
\end{Rem}

\begin{Thm}
    $\Z_m$ -- коммутативное кольцо с единицей.
\end{Thm}

\begin{proof}
    \[\overline{a}(\overline{b} + \overline{z}) = \overline{a} \cdot \overline{b + c} = \overline{a(b + c)} = \overline{ab + ac}\]
    и т.д.
\end{proof}

\begin{Def}
    Кольца $R$, в котором $\forall a, b \to (ab = 0 \SO a = 0 \vee b = 0)$ называется кольцом без делителей нуля.
    
    Если $ab = 0$ и $a, b \neq 0$, то $a, b$ -- делители нуля 
\end{Def}

\begin{Def}
    Коммутативное кольцо без делителей нуля -- область целостности.
\end{Def}

\Pagebreak
\begin{Thm}
    \begin{MyList}
        \item $\Z_m$ имеет делители нуля $\EQ m$ -- составное число
        \item $\Z_p, p$ - простое -- область целостности.
    \end{MyList}
\end{Thm}

\begin{proof}
    "$\SO$". $m = n \cdot k, \overline{n} \cdot \overline{k} = \overline{0}$ в $\Z_m$
    
    "$\Leftarrow$". $\overline{n} \cdot \overline{k} = \overline{0} \SO n \cdot k \equiv 0 \Mod m$
    
    Предположим, что $m$ -- простое $\SO m | n \vee m | k \SO \overline{n} = \overline{0} \vee \overline{k} = \overline{0}$. Но $\overline{n}$ и $\overline{k}$ -- делители нуля, т.е. $\overline{n}, \overline{k} \neq 0 \SO m$ -- составное.   

    $1) \SO 2)$ 
\end{proof}

\Subsection{Приведенная система вычетов}

\begin{Def}
    Вычеты, выбранные из полной системы вычетов и взаимно-простые с модулем $m$ обрузуют приведенную систему вычетов
\end{Def}

\begin{Def}
    Количество вычетов в приведенной системе вычетов обозначается $\varphi (m)$ -- функция Эйлера. 
\end{Def}

\begin{Lm}
    Если $p$ -- простое, то 
    \[\varphi(p) = p - 1\]
\end{Lm}

\begin{Thm}
    $(a, m) = 1, x$ пробегает приведенную систему вычетов $\SO ax$ тоже пробегает приведенную систему вычетов по $\Mod m$  
\end{Thm}

\begin{proof}
    $x \to \varphi(m), ax \to \varphi(m)$
    
    $(ax,m) = (a, m) = 1 \SO ax$ набор чисел из $\varphi(m)$, взаимно-простых с $m \SO \{ax\}$ -- приведенная система вычетов. 
\end{proof}

\Subsection{Функция Эйлера}
\begin{Lm}
    $p$ -- простое, $\alpha > 0$
    \[\varphi(p^\alpha) = p^\alpha - p^{\alpha - 1}\] 
\end{Lm}

\begin{proof}
    $1, 2, 3, ..., p, 2p, 3p, ..., p \cdot p, ..., p^\alpha - 1$. Выбросим из этого множества числа, делящиеся на $p$.
    Таких чисел будет ровно количество коэффициентов при $p$ до $p^\alpha$, т.е. $p^{\alpha - 1}$  
\end{proof}

\begin{Def}
    Функия $\Theta: \N \to \N$ называется мультипликативной, если 
    \[(a, b) = 1 \SO \Theta(ab) = \Theta(a) \cdot \Theta(b)\] 
\end{Def}

\begin{Thm}[Мультипликативность функции Эйлера]
    $\varphi$ мультипликативна     
\end{Thm}

\begin{proof}
    $(a, b) = 1$
    \[\begin{array}{ccccc}
    1 & 2 & 3 & ... & b \\ 
    b + 1 & b + 2 & b + 3 & ... & 2b \\ 
    ... & ... & ... & ... & ... \\ 
    (a - 1)b + 1 & (a - 1)b + 2 & (a - 1)b + 3 & ... & ab
    \end{array}\]
    Количество чисел, взаимно-простых с $b : \forall$ строка $: kb + r, k = 0, ..., a - 1, 1 \leqslant r \leqslant b$.
    Рассмотрим $k$-ю строку: $(kb + r, b) = 1 \SO (r, b) = 1$. Количество чисел $kb + r : (kb + r, b) = 1 = \varphi (b) \SO$
    есть $\varphi(b)$ столбцов, в которых числа $(kb + r, b) = 1$. Найдем в этих столбцах числа, взаимно-простые с $a$.
    $\forall$ столбец $: xb + r, x = 0, ..., a - 1 \SO xb + r$ -- полная система вычетов по $\Mod a \SO$ 
    среди $\{xb + r\}$ чисел, взаимно-простых с $a = \varphi(a) \SO$ всего чисел, взаимно-простых с $ab = \varphi(a) \cdot \varphi(b)$          
\end{proof}

\begin{Cons}
    $n = p_1^{\alpha_1} \cdot p_2^{\alpha_2} \cdot ... \cdot p_k^{\alpha_k}$ -- каноническое разложение
    $\SO \varphi(n) = (p_1^{\alpha_1} - p_1^{\alpha_1 - 1})(p_2^{\alpha_2} - p_2^{\alpha_2 - 1}) \cdot ... \cdot (p_k^{\alpha_k} - p_k^{\alpha_k - 1})$  
\end{Cons}

\begin{Rem}
    $\varphi(n) = n(1 - \frac{1}{p_1})(1 - \frac{1}{p_2}) \cdot ... \cdot (1 - \frac{1}{p_k})$ 
\end{Rem}

\begin{Thm}[Теорема Эйлера]
    $(m, a) = 1 \SO a^{\varphi(m)} \equiv 1 \Mod m$ 
\end{Thm}

\begin{proof}
    $r_1, r_2, ..., r_{\varphi(m)}$ -- приведенная система вычетов по $\Mod m$
    
    $\SO ar_1, ar_2, ..., ar_{\varphi(m)}$ -- приведенная система вычетов по $\Mod m$. Пусть $ar_i = \rho_i$
    \[\SO ar_1 \cdot ar_2 \cdot ... ar_{\varphi(m)} = \rho_1 \rho_2 \cdot ... \cdot \rho_{\varphi(m)}\]
    \[a^{\varphi(m)} r_1 \cdot r_2 \cdot ... \cdot r_{\varphi(m)} = \rho_1 \cdot \rho_2 \cdot ... \cdot \rho_{\varphi(m)} \SO a^{\varphi(m)} \equiv 1 \Mod m\]   
\end{proof}

\begin{Thm}[Теорема Ферма]
    $p$ -- простое, $(a, p) = 1 \SO a^{p - 1} \equiv 1 \Mod p$ 
\end{Thm}

\begin{proof}
    $\varphi(p) = p - 1$ 
\end{proof}

\begin{Def}
    $\Z_m^* = \{r : 0 \leqslant r < m, (r, m) = 1\}$ -- приведенная система вычетов по $\Mod m$ 
\end{Def}

\begin{Thm}
    $\Z_m^*$ -- коммутативная группа по умножению
\end{Thm}

\begin{proof}
    $r_1, r_2 \in \Z_m^*$. $(r_1, m) = (r_2, m) = 1 \SO (r_1 \cdot r_2, m) = 1 \SO r_1 \cdot r_2 \in \Z_m^*, 1 \in \Z_m^*$
    
    $r \in \Z_m^*$, то $r^{-1} = r^{\varphi(m) - 1} \SO r^{\varphi(m) - 1} \cdot r = r^{\varphi(m)} \equiv 1 \Mod m$  
\end{proof}

\Subsection{Сравнения с одним неизвестным}
\begin{Def}
    $f(x) \equiv 0 \Mod m$. Решением этого сравнения называется $x_0 : f(x_0) \equiv 0 \Mod m$. Решения $x_1$ и $x_2$ называются эквивалентными, если $x_1 \equiv x_2 \Mod m$

    Решить сравнение -- найти решений из полной системы вычетов.
\end{Def}

\begin{Thm}[Решение линейного сравнения]
    $ax \equiv b \Mod m , (a, m) = d$ 
    \begin{MyList}
        \item $d \not| \ b \SO$ решений нет.
        \item $d | b \SO \exists d$ решений $ : x = x_0 + m_1 t, t = 0, 1, ..., d - 1, m_1 = \frac{m}{d}, x_0$ -- какое-то решение
    \end{MyList}
\end{Thm}

\begin{proof}
    \begin{MyList}
        \item Очевидно
        \item Если $(a, m) = 1, x$ пробегает полную систему вычетов по $\Mod m \SO ax$ -- полная система вычетов по $\Mod m \SO \exists x_0 : ax_0 \equiv b \Mod m$
        
        Если $(a, m) = d, a = a_1d, m = m_1d, b = b_1d, (a_1, m_1) = 1$ \\
        $a_1x \equiv b_1 \Mod {m_1}$ -- $\exists$ решение $x_0 : a_1x_0 \equiv b_1 \Mod {m_1}$. $x = x_0 + m_1 t$ -- решение $ax \equiv b \Mod m$
        \[a(x_0 + m_1t) = a_1dx_0 + a_1dm_1t \equiv b_1d \Mod m\]
        Посмотрим, какие решения принадлежат полной системе вычетов, т.е. $0 \leqslant x_0 + m_1 t < m$.
        Ясно, что такие решения будут при $t = 0, 1, ..., d - 1$.       
    \end{MyList}
\end{proof}

\begin{Thm}[Методы решения $ax \equiv b \Mod m, (a, m) = 1$]
    \begin{MyList}
        \item $ax \equiv b \Mod m \SO x \equiv a^{\varphi(m) - 1} \cdot b \Mod m$ 
        \item $ax \equiv b \Mod m \SO x \equiv (-1)^n p_{n - 1} \cdot b \Mod m$ \\
        $\frac{m}{a}$ -- непрерывная дробь, $p_{n - 1}$ -- числитель $(n - 1)$-й подходящей дроби, $ \frac{p_n}{q_n} = \frac{m}{a}$  
    \end{MyList}
\end{Thm}

\begin{proof}
    $p_k \cdot q_{k - 1} - p_{k - 1} \cdot q_k = (-1)^{k - 1}$. $k = n$
    \[m \cdot q_{n - 1} - p_{n - 1} \cdot a = (-1)^{n - 1} \SO -p_{n - 1} \cdot a \equiv (-1)^{n - 1} \Mod m\]
    \[ap_{n - 1} b = (-1)^n b \Mod m \SO a \cdot (-1)^n p_{n - 1} \cdot b \equiv b \Mod m \SO x \equiv (-1)^n p_{n - 1} \cdot b \Mod m\]
\end{proof}

\Subsection{Диофантовы уравнения}

\begin{Def}
    Уравнение вида 
    \[a_1 x_1 + a_2 x_2 + ... + a_n x_n = b\]
    где $a_i \in \Z, x_i$ -- переменные и $\exists i : a_i \neq 0$, называется диофантовым.  
\end{Def}

\begin{Lm}
    $ax + by = c, a, b \neq 0$. Если $x_0 : ax_0 \equiv c \Mod b \SO (x_0, \frac{ax_0 - c}{b})$ -- решения уравнения 
\end{Lm}

\begin{proof}
    $by \equiv c - ax$ при $x = x_0$ и $c - ax \ \vdots \ b \SO \frac{c - ax_0}{b} \in \Z$   
\end{proof}

\begin{Thm}
    $ax + by = c, d = (a, b), d | c$. Пусть $(x_0, y_0)$ -- какое-то решение $\SO$ все решения:
    \[\begin{cases}
        x = x_0 - \frac{b}{d}t \\
        y = y_0 + \frac{a}{d}t
    \end{cases}, t \in \Z\]  
\end{Thm}

\Subsection{Системы сравнений}

\[\begin{cases}
    x \equiv b_1 \Mod m_1 \\
    x \equiv b_2 \Mod m_2 \\
    ... \\
    x \equiv b_k \Mod m_k
\end{cases}\]

\begin{Thm}[Китайская теорема об остатках]
    $(m_i, m_j) = 1, i \neq j$. Тогда
    \begin{MyList}
        \item Решение системы существует:
        \[x \equiv \frac{M}{m_1} \cdot M_1' b_1 + \frac{M}{m_2} \cdot M_2' b_2 + ... + \frac{M}{m_k}M_k' b_k \Mod M\]
        $M = m_1 \cdot m_2 \cdot ... \cdot m_k, M_i' : M_i' \cdot \frac{M}{m_i} \equiv 1 \Mod m_i$
        \item Решение единственно 
    \end{MyList}
\end{Thm}

\begin{proof}
    \begin{MyList}
        \item Подставим в $i$-е уравнение: 
        \[x \equiv \frac{M}{m_i} M_i' b_i \Mod m_i \SO x \equiv b_i \Mod m_i\]
        \item Без доказательства.
    \end{MyList}
\end{proof}

\begin{Thm}[Теорема Вильсона]
    $p$ -- простое $\EQ$ $(p - 1)! \equiv -1 \Mod p$  
\end{Thm}

\begin{proof}
    "$\SO$". $\Z_p^* = \{1, 2, ..., p - 1\}$ -- группа, $a \in \Z_p^*$
    \[a^2 = 1 \SO a = \pm 1, 1 \cdot 2 \cdot ... \cdot (p - 1) \SO 1 \cdot (p - 1) \equiv -1 \Mod p\]
    \[a \neq a^{-1} \SO 2 \cdot ... \cdot (p - 2) = 1\]
    
    "$\Leftarrow$". Предположим, что $$k | p, k > 1, k \neq p \SO k < p \SO 1 \cdot 2 \cdot ... \cdot (p - 1) \ \vdots \ k \SO (p - 1)! \equiv 0 \Mod p$$ 
\end{proof}

\begin{Algo}[Алгоритм RSA]
    \begin{MyList}
        \item Выбираем $p, q$ -- простые
        \item $n = p \cdot q, \varphi(n) = (p - 1)(q - 1)$
        \item Выбираем $e : (e, \varphi(n)) = 1$
        \item Решаем $e \cdot d \equiv 1 \Mod {\varphi(n)} \SO$ находим $d$  
    \end{MyList}

    Шифрование:
    \begin{MyList}
        \item $m$ -- текст (в виде цифрового кода)
        \item $c \equiv m^e \Mod n \SO c$ -- шифр
    \end{MyList}

    Ключи:
    \begin{MyItemize}
        \item $(e, n)$ -- открытый ключ
        \item $(d, n)$ -- закрытый ключ
    \end{MyItemize}

    Дешифрование:
    \[c^d \equiv m^{ed} \equiv m \Mod n\]

    Трудность $n = p \cdot q$. 
\end{Algo}

\Section{Комплексные числа}{}{Илья Дудников}

\begin{Def}
    Множество $\{(a, b) | a, b \in \R\}$ называется множество комплексных чисел, если:
    \begin{MyList}
        \item $(a, b) = (c, d) \EQ a = c, b = d$
        \item $(a, b) + (c, d) = (a + c, b + d)$
        \item $(a, b) \cdot (c, d) = (ac - bd, ad + bc)$
        \item $a = (a, 0)$    
    \end{MyList} 

    Проверим корректность:
    \begin{MyItemize}
        \item 1 и 4: $a = b \EQ (a, 0) = (b, 0)$
        \item 2 и 4: $a + b = (a, 0) + (b, 0) = (a + b, 0) = a + b$
        \item 3 и 4: $a \cdot b = (a, 0) \cdot (b, 0) = (ab, 0) = ab$   
    \end{MyItemize}
\end{Def}

\begin{Thm}
    $\C$ образует коммутативное кольцо с единицей.
\end{Thm}

\begin{proof}
    $(0, 0)$ -- нейтральный элемент по сложению. $(a, b) : -(a, b) = (-a, -b)$ -- обратный элемент по сложению. Остальные свойства несложно проверяются.
\end{proof}

\begin{Def}
    Множество $K$ называется полем, если $K$ является коммутативным кольцом с единицей и 
    \[\forall x \in K^* = K \setminus \{0\} \ \exists x^{-1} \in K : x \cdot x^{-1} = 1\]
\end{Def}

\begin{Thm}
    $\Z_p (p \text{ -- простое}), \Q, \R, \C$ -- поля.   
\end{Thm}

\begin{proof}
    $\Q, \R$ -- поля.
    
    $\Z_p$ -- коммутативное кольцо с единицей, $\Z_p^*$ -- мультипликативная группа $\SO \Z_p$ -- поле.
    
    $(a, b) \in \C^*, (a, b)^{-1} = \frac{(a, -b)}{a^2 + b^2} = \left(\frac{a}{a^2 + b^2}, \frac{-b}{a^2 + b^2}\right)$ 
    \[(a, b) \cdot \frac{(a, -b)}{a^2 + b^2} = \frac{(a^2 + b^2, 0)}{a^2 + b^2} = (1, 0)\]
\end{proof}

\begin{Def}
    $(a, b)$ и $(a, -b)$ -- комплексно-сопряженные числа. \\
    $|(a, b)| = \sqrt{a^2 + b^2}$ -- модуль комплексного числа. Заметим, что $|(a, 0)| = \sqrt{a^2 + 0} = \sqrt{a^2} = |a|$ \\
    $(a, b) \cdot (a, -b) = a^2 + b^2 = |(a, b)|^2$ 
\end{Def}

\Subsection{Алгебраическая форма записи комплексного числа}

\begin{Def}
    Положим $i = (0, 1)$. Тогда
    \[(a, b) = (a, 0) + (0, b) = (a, 0) \cdot (1, 0) + (b, 0) \cdot (0, 1) = a + bi\]
\end{Def}
\Pagebreak

\Subsection{Геометрическое представление комплексных чисел}

\begin{Def}
    $z = a + bi$. $\Real z = a$ -- вещественная часть числа $z$, $\Imag z = b$ -- мнимая часть.   \\
    $z = a + bi \mapsto $ точка на комплексной плоскости. $(a, b)$ -- радиус-вектор $OM$.  \\
    $\rho = |z| = \sqrt{a^2 + b^2}$ -- длина вектора $OM$. $\PHI = \widehat{(\Real, OM)}$ -- аргумент комплексного числа. \\
    $\arg z = \PHI, \PHI = \PHI_0 + 2\pi k, \PHI_0 \in [0; 2\pi)$ или $\PHI_0 \in (-\pi; pi]$.

    \begin{figure}[h]
        \centering
        \input{images/pic1.pdf_tex}
    \end{figure}
\end{Def}

\Subsection{Тригонометрическая форма записи комплексного числа}

\begin{Def}
    $a = \rho \cos \PHI, b = \rho \sin \PHI \SO z = a + bi = \rho(\cos \PHI + i\sin \PHI)$
    \[\tg \PHI = \frac{b}{a} \SO \PHI = \begin{cases}
        \arctg \frac{b}{a}, z \in \text{ I и II четверти} \\
        \arctg \frac{b}{a} + \pi, z \in \text{III и IV четверти} 
    \end{cases}\] 

\end{Def} 

\begin{Def}[Неравенство треугольника]
    $z_1, z_2 \in \C$
    \begin{MyList}
        \item $|z_1 + z_2| \leqslant |z_1| + |z_2|$
        \item $|z_1 - z_2| \geqslant ||z_1| - |z_2||$  
    \end{MyList} 
\end{Def}

\begin{proof}
    \begin{MyList}
        \item $z_1 = \rho_1(\cos \PHI_1 + i\sin \PHI_1), z_2 = \rho_2(\cos \PHI_2 + i\sin \PHI_2)$ \\
        \begin{gather*}
            |z_1 + z_2|^2 = |\rho_1 \cos \PHI_1 + \rho_2 \cos \PHI_2 + i(\rho_1 \sin \PHI_1 + \rho_2 \sin \PHI_2)|^2 = \\
            = \rho_1^2 \cos^2 \PHI_1 + 2 \rho_1 \rho_2 \cos \PHI_1 \cos \PHI_2 + \rho_2^2 \cos^2 \PHI_2 + \rho_1^2 \sin^2 \PHI_1 + 2 \rho_1 \rho_2 \sin \PHI_1 \sin \PHI_2 + \rho_2^2 \sin^2 \PHI_2 = \\
            = \rho_1^2 + 2\rho_1 \rho_2 \cos(\PHI_1 - \PHI_2) = \rho_2^2 \leqslant \rho_1^2 + 2\rho_1 \rho_2 + \rho_2^2 = (\rho_1 + \rho_2)^2 = (|z_1| + |z_2|)^2
        \end{gather*}

        \item \[
            |z_1| = |z_1 - z_2 + z_2| \leqslant |z_1 - z_2| + |z_2| \SO |z_1| - |z_2| \leqslant |z_1 - z_2| \SO ||z_1| - |z_2|| \leqslant |z_1 - z_2|
        \]
    \end{MyList}
\end{proof}

\begin{Rem}
    $|z_1 + z_2| = |z_1| + z_2| \EQ z_1 \parallel z_2$ 
\end{Rem}

\Pagebreak
\begin{Thm}[Умножение комплексных чисел в тригонометрической форме]
    $z_1 = \rho_1 (\cos \PHI_1 + i\sin \PHI_1), z_2 = \rho_2 (\cos \PHI_2 + i\sin \PHI_2)$. Тогда 
    \[z_1 \cdot z_2 = \rho_1 \cdot \rho_2 (\cos (\PHI_1 + \PHI_2) + i\sin (\PHI_1 + \PHI_2))\] 
\end{Thm}

\begin{proof}
    Достаточно перемножить, заметить формулу косинуса суммы и синуса суммы
\end{proof}

\begin{Cons}[Формула Муавра]
    $z = \rho (\cos \PHI + i\sin \PHI) \SO z^n = \rho^n(\cos n\PHI + i\sin n\PHI)$ 
\end{Cons}

\begin{proof}
    \begin{MyList}
        \item $n \geqslant 0$. По индукции: $n = 1$ очевидно. \\
        $n - 1 \to n:$
        \[z^n = z^{n - 1} \cdot z = \rho^{n - 1}(\cos (n - 1)\PHI + i\sin(n - 1)\PHI) \cdot \rho(\cos \PHI + i\sin\PHI) = \rho^n(\cos n\PHI + i\sin n\PHI)\] 

        \item $n < 0$. Пусть $n = -m, m > 0$. Тогда
        \[z^n = \frac{1}{z^m} = \frac{1}{\rho^m(\cos m\PHI + i\sin m\PHI)} = \rho^{-m} \frac{\cos m\PHI - i\sin m\PHI}{1} = \rho^n (\cos n\PHI + i\sin n\PHI)\]
    \end{MyList}
\end{proof}

\Subsection{Извлечение корней из комплексных чисел}

\begin{Def}
    Корнем $n$-й степени из комплексного числа $z$ называется $w \in \C : w^n = z$ 
\end{Def}

\begin{Thm}
    $\forall z \in \C^* \ \exists n$ корней $n$-й степени $z_k, k = 0, 1, ..., n - 1$
    \[z_k = \sqrt[n]{\rho}(\cos \frac{\PHI + 2\pi k}{n} + i\sin \frac{\PHI + 2\pi k}{n}), z = \rho(\cos \PHI + i\sin \PHI)\]  
\end{Thm}

\begin{proof}
    $w^n = z, w = R(\cos \Theta + i \sin \Theta)$ 
    \begin{gather*}
        \SO (w^n = z): R^n (\cos (n \Theta) + i\sin (n \Theta)) = \rho(\cos \PHI + i\sin \PHI) \SO \\
        \SO R = \sqrt[n]{\rho}, \cos (n \Theta) = \cos \PHI, \sin(n \Theta) = \sin \PHI \SO \\
        \SO n \Theta = \PHI + 2\pi k \SO \Theta = \frac{\PHI + 2\pi k}{n} \SO \text{ любой корень имеет вид } z_k
    \end{gather*}
\end{proof}

\Subsection{Корни из единицы}

\begin{Def}
    Корень из 1 $n$-й степени $\varepsilon_k$ называется первообразным, если он принадлежит показателю, т.е. $\forall m : 0 < m < n \to \varepsilon^m \neq 1$ 
\end{Def}

\begin{Thm}[О первообразном корне]
    Корень из 1 $n$-й степени является первообразным $\EQ (k, n) = 1$. 
\end{Thm}

\begin{proof}
    "$\SO$". $\varepsilon_k$ -- первообразный корень. Предположим, что $(k, n) = d > 1, k = k_1 d, n = n_1 d, n_1 < n$. Тогда 
    \[\varepsilon_k^{n_1} = \cos \frac{2\pi k n_1}{n} + i \sin \frac{2\pi k n_1}{n} = \cos \frac{2\pi k_1 d n_1}{n} + i\sin \frac{2\pi k_1 d n_1}{n} = 1 ?! \SO d = 1\]
    
    "$\Leftarrow$". $(k, n) = 1$. Предположим, что $\varepsilon_k^m = 1 \SO \cos \frac{2\pi k m}{n} = 1, \sin \frac{2\pi k m}{n} = 0$
    \[\EQ \frac{2\pi k m}{n} = 2\pi s \SO \frac{k m}{n} \in \Z \SO n | m \SO m \geqslant n\]  
\end{proof}

Свойства:

\begin{MyList}
    \item $\alpha$ -- корень из 1 степени $n, \beta$ -- корень из 1 степени $m \SO \alpha \cdot \beta$ -- тоже корень из 1.
    \begin{proof}
        $(\alpha \beta)^{\lcm (m, n)} = 1$ 
    \end{proof}
    \item Если $\alpha$ -- корень из 1 степени $n$, то $\alpha^{-1}$ -- корень из 1 степени $n$ 
    \begin{proof}
        $(\alpha^{-1})^n = \frac{1}{\alpha^n} = 1$ 
    \end{proof}
    \item $u_n = \{z \in \mathbb{C} : z^n = 1\}$ -- мультипликативная коммутативная группа. \\
    $u_n = \{\varepsilon_k, \varepsilon_k^2, ..., \varepsilon_k^{n - 1}, 1\}, \varepsilon_k$ -- первообразный корень. 
\end{MyList}

\begin{Def}
    Группа $G$ называется циклической, если $G = \{a, a^2, a^3, ...\}$. Пишут $G = \langle a\rangle$ -- группа $G$ порождается элементом $a$. 
\end{Def}

\begin{Def}
    $G_1$ -- группа с операцией $*_1, G_2$ -- группа с операцией $*_2$. Говорят, что группы $G_1$ и $G_2$ \textbf{изоморфны},
    если $\exists \PHI : G_1 \to G_2 : $
    \begin{MyItemize}
        \item $\PHI$ -- биективно
        \item $\forall x, y \in G_1 \to \PHI(x *_1 y) = \PHI(x) *_2 \PHI(y)$ 
    \end{MyItemize}    
\end{Def}

\begin{Thm}
    $u_n \simeq \Z_n$ 
\end{Thm}

\begin{proof}
    $\varepsilon$ -- первообразный корень, т.е. $u_n = \{\varepsilon^k\}, k = 0, ..., n - 1$ \\
    $\PHI : \Z_n \to u_n, \PHI(k) = \varepsilon^k, \PHI$ -- биекция
    \[\PHI(k + m) = \varepsilon^k \cdot \varepsilon^m = \PHI(k) \cdot \PHI(m)\]  
\end{proof}

\Subsection{Показательная форма записи комплексного числа}

\begin{Def}
    $e^{i\PHI} = \cos \PHI + i\sin \PHI$ -- показательная форма записи комплексного числа. 
\end{Def}

\begin{Def}[Формула Эйлера]
    $e^{i\PHI} = \cos \PHI + i\sin \PHI, e^{-i\PHI} = \cos \PHI - i\sin \PHI$. Тогда
    \begin{align*}
        &\cos \PHI = \frac{e^{i\PHI} + e^{-i\PHI}}{2} \\
        &\sin \PHI = \frac{e^{i\PHI} - e^{-i\PHI}}{2i}
    \end{align*} 
\end{Def}

Свойства комплексных чисел:
\begin{MyList}
    \item $|z_1 \cdot z_2| = |z_1| \cdot |z_2|$ 
    \item $\overline{z_1 \cdot z_2} = \overline{z_1} \cdot \overline{z_2}$
    \item $\overline{z_1 \pm z_2} = \overline{z_1} \pm \overline{z_2}$
    \item $z + \overline{z} \in \R$
    \item $i(z - \overline{z}) \in \R$     
\end{MyList}

\Section{Многочлены}{}{Илья Дудников}

\begin{Def}
    $R$ -- коммутативное кольцо с 1. Множество $\{(a_0, a_1, ..., a_n, ...), \exists n : \forall m > n \to a_m = 0\}$
    \begin{MyList}
        \item $\alpha \in R \to \alpha(a_0, a_1, ..., a_n, ...) = (\alpha a_0, \alpha a_1, ..., \alpha a_n, ...)$
        \item $(a_0, a_1, ..., a_n, ...) + (b_0, b_1, ..., b_n) = (a_0 + b_0, a_1 + b_1, ..., a_n + b_n, ...)$ 
        \item $(a_0, a_1, ..., a_n) \cdot (b_0, b_1, ..., b_n, ...) = (c_0, c_1, ..., c_n, ...)$, где
        \[c_k = \sum_{s + t = k} a_s b_t\]
        \item $\forall a \in R \to a = (a, 0, 0, ...)$ 
    \end{MyList} 
    Это множество называется многочленами над $R$.
\end{Def}

Корректность определения:
\begin{MyItemize}
    \item все действия 1, 2, 3 не выводят из множества. 
    \item Согласование 1 и 4, 2 и 4, 3 и 4.
\end{MyItemize}

\begin{Thm}
    Множество многочленов над $R$ -- коммутативное кольцо с 1
\end{Thm}

\begin{proof}
    $(0, 0, ...)$ -- нулевой элемент, $(1, 0, 0, ...)$ -- единица. Ассоциативность несложно доказывается.
\end{proof}

\begin{Def}
    Введём $x = (0, 1, 0, ...)$. Тогда $x^2 = (0, 1, 0, 0, ...) \cdot (0, 1, 0, 0, ...) (0, 0 \cdot 1 + 1 \cdot 0, 0 \cdot 0 + 1 \cdot 1, 0, ...) = (0, 0, 1, 0, ...)$  
    \[\SO x^k = (\underset{0}{0} , \underset{1}{0} , ..., \underset{k}{1} , \underset{k + 1}{0}, ... )\]
    Тогда 
    \[(a_0, a_1, ..., a_n, ...) = (a_0, 0, ...) + (0, a_1, 0, ...) + ... + (0, ..., a_k, ...) = a_0 + a_1 x + a_2 x^2 + ... + a_n x^n\]
    Обозначение: $R[x] = \{a_0 + a_1 x + ... + a_n x^n\}$ -- кольцо многочленов над $R$ от переменной $x$. 
\end{Def}

\begin{Def}
    Коэффициент $a_n \neq 0 : a_m = 0, m > n$ называется старшим коэффициентом. Если $n \geqslant 1$, то $n$ -- степень многочлена.
    \[\deg (f) = n\]
    Если $a_0$ -- старший коэффициент. Если $a_0 \neq 0$, то $\deg (f) = 0$. Если $a_0 = 0$, то $\deg (f) = -\infty$  
\end{Def}

\begin{Thm}
    $f, g \in R[x]$ 
    \begin{MyList}
        \item $\deg(f + g) \leqslant \max \{\deg f, \deg g\}$
        \item $\deg (f \cdot g) \leqslant \deg f + \deg g$  
    \end{MyList}
\end{Thm}

\begin{proof}
    $f = a_0 + a_1 x + ... + a_n x^n, g = b_0 + b_1 x + ... b_m x^m$.
    Тогда если $n > m$, то 
    \[f + g = a_n x^n + ... \SO \deg (f + g) \leqslant \deg n\]

    \[f \cdot g = a_n \cdot b_m x^{n + m} + ... \SO \deg (f \cdot g) \leqslant n + m\]
\end{proof}

\begin{Example}
    $\Z_6 [x]$. $f = 2x^2 + 1, g = 3x + 2$ 
    \[f \cdot g = 6x^3 + 4x^2 + 3x + 2 = 4x^2 + 3x + 2 \SO \deg (f \cdot g) = 2 < \deg f + \deg g = 3\]
\end{Example}

\begin{Def}
    Коммутативное кольцо с 1 без делителей нуля называется областью целостности.
\end{Def}

\begin{Thm}
    $R$ -- область целостности $\SO R[x]$ -- область целостности.
\end{Thm}

\begin{proof}
    $f = a_n x^n + ..., g = b_m x^m + ... \SO f \cdot g = a_n b_m x^{n + m} + ..., a_n \cdot b_m \neq 0$ т.к. $R$ -- область целостности
    $\SO R[x]$ -- область целостности. 
\end{proof}

\begin{Lm}[О сокращении]
    $R$ -- область целостности, $a, b, c \in R, a \neq 0$. Тогда $ab = ac \SO b = c$. 
\end{Lm}

\begin{proof}
    $ab = ac \SO a(b - c) = 0 \SO b - c = 0$ 
\end{proof}

\begin{Thm}[О делении с остатком]
    $R$ -- область целостности, $\forall f \in R[x], \forall g \in R[x]$ с обратимым старшим коэффициентом 
    $\exists ! \ q, r \in R[x] : f = g \cdot q + r, \deg r < \deg g$ 
\end{Thm}

\begin{proof}
    Существование. $\deg f = n, \deg g = m, n < m \SO f = g \cdot 0 + f$ \\
    $n \geqslant m$. Индукция по $n$. $n = 0 \SO a_0 = b_0(b_0^{-1} a_0) + 0$. \\
    $n - 1 \mapsto n$. $f = a_n x^n + ... + a_0$, $g = b_m x^m + ... + b_0$. Рассмотрим 
    \[\overline{f} = f - a_n b_m^{-1} x^{n - m} \cdot g \SO \deg \overline{f} < n\]
    $\SO$ по предположению индукции $\overline{f} = g \cdot q + r \SO f - a_n b_m^{-1} x^{n - m} g = g \cdot q + r$ 
    \[\SO f = (g + a_n b_m^{-1} x^{n - m})g + r, \deg r \leqslant \deg g\]

    Единственность. Предположим, что 
    \[f = g \cdot q_1 + r_1, f = g \cdot q_2 + r_2\]
    $\SO g(q_1 - q_2) = r_2 - r_1$. Если $q_1 \neq q_2, r_1 \neq r_2$, то $\deg (g(q_1 - q_2)) \geqslant m, \deg (r_2 - r_1) < m$ -- противоречие.  
\end{proof}

\begin{Def}
    $f = a_n x^n + ... + a_0, c \in R$. Тогда $f(c) = a_n c^n + ... + a_1 c + a_0$ -- значение многочлена в точке $c$. \\
    Если $f(c) = 0$, то $c$ -- корень многочлена.
\end{Def}

\begin{Thm}[Теорема Безу]
    $R$ -- область целостности, $a \in R, f \in R[x]$ 
    \[\SO f = (x - a) \cdot q(x) + f(a)\]
\end{Thm}

\begin{proof}
    $f = (x - a) q + r, \exists ! q, r, \deg r < 1 \SO r \in R$. $x = a \SO f(a) = 0 + r \SO r = f(a)$. 
\end{proof}

\begin{Cons}
    $f \ \vdots \ x - a \EQ f(a) = 0$ 
\end{Cons}

\begin{Def}
    Многочлен $f \in R[x]$ со старшим коэффициентом 1 называется нормализованным.
\end{Def}

\Subsection{Корни многочлена}

\begin{Def}
    Если многочлен $f = (x - c)^k q, q(c) \neq 0$, то $c$ -- корень кратности $k$. \\
    Иначе, если $(x - c)^k | f, (x - c)^{k + 1} \not| f$, то $c$ -- корень кратности $k$.
\end{Def}

\begin{Thm}[О количестве корней многочлена]
    Число корней многочлена $f$ с учетом их кратности $\leqslant \deg f$
\end{Thm}

\begin{proof}
    Индукция по $\deg f$ \\
    $\deg f = 0 \SO $ всё верно. Если нет корней $\SO$ всё верно. \\
    Пусть для для $\deg f < n$ доказано. Докажем для $\deg f = n$. \\
    $c_1$ -- корень $f \SO f = (x - c_1)^k g(x)$, если $c_2 \neq c_1 $ -- другой корень $\SO f(c_2) = (c_2 - c_1)^k g(c_1) = 0 \SO c_2$ -- корень $g$. 
    $\deg g = n - k < n \SO$ корней $g \leqslant n - k \SO$ корней $f \leqslant k + n - k = n$.
\end{proof}
 
\begin{Cons}
    $f, g \in R[x]$. Пусть $\max \{\deg f, \deg g\} = n$. Предположим, что $\exists c_1, ..., c_{n + 1} \in R : f(c_i) = g(c_i) \SO f$ совпадает с $g$   
\end{Cons}

\begin{proof}
    $h = f - g, \deg h \leqslant n, h(c_i) = 0, i = 1, ..., n + 1 \SO h \equiv 0$ 
\end{proof}

Формальное равенство многочленов.
$$f = g = a_n x^n + ... + a_1 x + a_0$$ 

Функциональное равенство многочленов.
\[f(x) = g(x) \ \forall x \in R\]

\begin{Example}
    $\Z_5$. $f = x^5 + x^4, g = x^4 + x$. \\ 
    $f - g = x^5 - x, x^5 \equiv x \Mod 5 \SO x^5 - x = 0 \ \forall x \in \Z_5 \SO f(x) = g(x)$ 
\end{Example}

\begin{Rem}
    $R$ -- бесконечно, то $\forall x \in R \ f(x) = g(x) \EQ f = g$
\end{Rem}

\begin{Ex}
    Доказать утверждение выше
\end{Ex}

\Pagebreak
\Subsection{Наибольший общий делитель}

Пусть $R = K$ -- поле.

\begin{Def}
    $R$ -- область целостности, $a, b \in R$. $d = \gcd (a, b)$, если
    \begin{MyList}
        \item $d | a, d | b$ 
        \item $c | a, c | b \SO c | d$ 
    \end{MyList} 
\end{Def}

\begin{Thm}[О НОД для многочленов]
    $\forall f, g \in K[x]$
    \begin{MyList}
        \item $\exists d = \gcd (f, g)$ определен однозначно с точностью до ассоциированного элемента
        \item $\exists h_1, h_2 \in K[x] : h_1 f + h_2 g = d$ 
    \end{MyList} 
\end{Thm}

\begin{proof}
    Полностью аналогично доказательству о существовании нод над $\Z$
\end{proof}

\begin{Def}
    $R$ -- область целостности. $R^*$ -- обратимые элементы. Если $a, b \in R, a = \varepsilon b, \varepsilon \in R^*$, то $a, b$ -- ассоциированные.
\end{Def}

\begin{Example}
    $K$ -- поле, $K^* = K \setminus \{0\} \SO$ все элементы ассоциированны. 
\end{Example}

\begin{Example}
    $\Z, \Z^* = \{-1, 1\}, a = -b$ 
\end{Example}

\begin{Example}
    $\Z_n, \Z_n^* = \{m \in \Z : (m, n) = 1\}$ 
    \begin{Ex}
        Доказать.
    \end{Ex}
\end{Example}

\begin{Example}
    $K[x], (K[x])^* = K^*$ 
\end{Example}

\begin{Def}
    $f, g \in K[x]$ -- взаимно-простые, если $(f, g) = 1$ 
\end{Def}

\Subsection{Факториальность кольца многочленов}

\begin{Def}
    $R$ -- область целостности. $p \in R :$ 
    \begin{MyList}
        \item $p \notin R^*$ 
        \item $p = ab \SO a $ или $b \in R^*$ 
    \end{MyList}

    Тогда $p$ -- простой (неразложимый) элемент $R$.
\end{Def}

\begin{Def}
    Если в области целостности $R \ \forall a \in R \setminus \{0\}$
    $\exists a = \varepsilon \cdot p_1 \cdot ... \cdot p_s, \varepsilon \in R^*, p_1, ... p_s$ -- простые, определенное с точностью до порядка и умноженное на $\varepsilon$,
    то $R$ -- факториальное кольцо.
\end{Def}

\begin{Def}
    Простой элемент кольца $K[x]$ называется неприводимым многочленом.
\end{Def}

\begin{Example}
    $x^2 - 2$ -- неприводим над $\mathbb{Q}$ 
\end{Example}

\begin{Example}
    $\forall$ многочлен $x - a$ неприводим. 
\end{Example}

\begin{Ex}
    Над бесконечным полем $K$ существует бесконечно много неприводимых многочленов.
\end{Ex}

Свойства:

\begin{MyList}
    \item $f \in K[x], \PHI$ -- неприводим над $K \SO$ либо $\PHI | f$, либо $(f, \PHI) = 1$.
    \begin{proof}
        $d = (f, \PHI)$. Если $d \neq 1 \SO d | f, d | \PHI \SO d = \varepsilon \cdot \PHI..., \varepsilon \in K^* \SO \PHI | f$ 
    \end{proof}
    \item $\PHI_1, \PHI_2 \in K[x]$ -- неприводимы. \TODO
    \item $f, g, \PHI \in K[x], \PHI$ -- неприводим, $\PHI | fg \SO \PHI | f \vee \PHI | g$ 
    \begin{proof}
        $\PHI \not| f \SO (\PHI, f) = 1 \SO \exists h_1, h_2 \in K[x] : h_1 \PHI + h_2 f = 1 \SO h_1 \PHI g + h_2 f g = g \SO \PHI | (h_1 \PHI g + h_2 f g) \SO \PHI | g$ 
    \end{proof}
    \item $\PHI \in K[x]$ -- неприводим, $f_1, ..., f_s \in K[x]$, $\PHI | f_1 \cdot ... \cdot f_s \SO \exists i, 1 \leqslant i \leqslant s : \PHI | f_i$  
\end{MyList}

\begin{Lm}
    $\forall f \in K[x] : \deg f \geqslant 1$ делится хотя бы на один неприводимый многочлен.
\end{Lm}

\begin{proof}
    $\deg f = n$. $n = 1$ -- верно. Предположим, что для $m < n$ тоже всё верно.

    $f$ -- приводим $\SO f = f_1 \cdot g, \deg f_1 < n \SO \exists $ неприводимый многочлен $\PHI$, который делит $f_1 \SO f_1 | f$.
\end{proof}

\begin{Thm}
    $K$ -- поле. $K[x]$ -- факториальное кольцо.
\end{Thm}

\begin{proof}
    Существование. $f \in K[x]$. Если $f$ -- неприводим, то очевидно.
    Если $f = \PHI_1 \cdot f_1, \PHI$ -- неприводим; $f_1 = \PHI_2 \cdot f_2$ и т.д.
    $\SO f = \PHI_1 \cdot \PHI_2 \cdot ... \cdot \PHI_k$
    
    Единственность. $f = \varepsilon \PHI_1 \PHI_2 ... \PHI_k = \eta \psi_1 \psi_2, ..., \psi_s, \varepsilon, \eta \in K^*, \PHI_i, \psi_j$ -- неприводимы.
    $k \leqslant s \SO \PHI_1 | \eta \psi_1 \psi_2 ... \psi_s \SO \PHI_1 | \psi_j \SO \PHI_1 = \psi_j \SO \varepsilon^{-1} \eta \psi_{s - k}  ... \psi_s = 1 \SO \varepsilon = \eta, \psi_{s - k} ... \psi_s = 1$  
\end{proof}

\begin{Def}
    $f = \varepsilon \cdot \PHI_1^{k_1} \cdot ... \cdot \PHI_s^{k_s}$, где $\varepsilon \in K^*, \PHI_1, \PHI_2, ..., \PHI_s$ -- различные неприводимые многочлены над $K$, $k_1, k_2, ..., k_s \geqslant 1 \in \Z$ -- каноническое разложение многочлена $f$.
\end{Def}

\Subsection{Каноническое разложение многочлена над $\C$}

\begin{Def}
    Поле $K$ называется алгебраически замкнутым, если любой многочлен $\deg \geqslant 1$ имеет хотя бы один корень.
\end{Def}

\begin{Example}
    $\Q(x^2 - 2), \R (x^2 + 1), \Z_p$ не являются алгебраически замкнутыми.
\end{Example}

\begin{Thm}[Основная теорема высшей алгебры]
    $\C$ -- алгебраически замкнуто.
\end{Thm}

\begin{Cons}
    Каноническое разложение многочлена над $\C$ 
    \[f = a_n (x - z_1)^{k_1} (x - z_2)^{k_2} ... (x - z_s)^{k_s}\]
    $z_1, ..., z_s \in \C, a_n \in \C$ 
\end{Cons}

\Pagebreak
\Subsection{Каноническое разложение многочлена над $\R$}

\begin{Def}
    $R_1, R_2$ -- кольца с 1. Гомоморфизмом колец $R_1$ и $R_2$ называется $\PHI: R_1 \to R_2 : $
    \begin{MyList}
        \item $\forall x, y \in R_1 \ \PHI(x + y) = \PHI(x) + \PHI(y)$
        \item $\forall x, y \in R_1 \ \PHI(x \cdot y) = \PHI(x) \cdot \PHI(y)$
        \item $\PHI(1) = 1$   
    \end{MyList}  
\end{Def}

\begin{Def}
    Биективный гомоморфизм колец -- \textbf{изоморфизм}, при этом сами кольца называются \textbf{изоморфными}.
\end{Def}

\begin{Example}
    \begin{MyList}
        \item $id: R \to R$ -- изоморфизм.
        \item $\PHI : \Z \to \Z_n, \PHI(a) = a \Mod n$ -- гомоморфизм. 
        \begin{proof}
            \[\PHI(a + b) = a + b \Mod n = \begin{cases}
                a + b, 0 \leqslant a + b < n \\
                a_0 + b_0 \equiv a + b \Mod n
            \end{cases} = a \Mod n + b \Mod n = \PHI(a) + \PHI(b)\]
            \[ab \equiv a_0 b_0 \Mod n \SO \PHI(ab) = ab \Mod n = (a \Mod n) \cdot (b \Mod n) = \PHI(a) \cdot \PHI(b)\]
            \[\PHI(1) = 1\]
            $\SO \PHI$ -- гомоморфизм. \\
            $\PHI(kn) = 0 \SO$ не инъективно $\SO$ не изоморфизм. 
        \end{proof}
        
        \item $\PHI: R \to R[x], \PHI(r) = r$ -- вложение - гомоморфизм.
    \end{MyList}
\end{Example}

Свойства:
\begin{MyList}
    \item $\PHI(0) = 0$ 
    \begin{proof}
        $\PHI(x) = \PHI(x + 0) = \PHI(x) + \PHI(0) \SO \PHI(0) = \PHI(x) - \PHI(x) = 0$
    \end{proof}

    \item $\PHI(-x) = -\PHI(x)$ 
    \begin{proof}
        $0 = \PHI(x - x) = \PHI(x) + \PHI(-x) \SO \PHI(-x) = -\PHI(x)$ 
    \end{proof}

    \begin{Rem}
        Противоположный элемент в кольце определен однозначно.

        \begin{proof}
            Пусть $x$ имеет $y$ и $z$ -- противоположные элементы. Тогда
            \[z = 0 + z = y + x + z = y + 0 = y\]
        \end{proof}
    \end{Rem}
\end{MyList}

\begin{Def}
    Поле $K_1$ и $K_2$ называются изоморфными, если они изоморфны как кольца.
\end{Def}

\begin{Rem}
    При этом $\forall x \in K, \PHI(x^{-1}) = \PHI(x)^{-1}$
    $$1 = \PHI(x \cdot x^{-1}) = \PHI(x) \cdot \PHI(x^{-1}) \SO \PHI(x^{-1}) = \PHI(x)^{-1}$$

    Обратный элемент определен однозначно: если $y, z$ -- обратные к $x$ 
    \[(yx = zx = 1) \SO y = yxz = z\]
\end{Rem}

\begin{Def}
    Изоморфизм $\PHI: K \to K$ ( $K$ -- поле) называется автоморфизмом.
\end{Def}

\begin{Example}
    \begin{MyList}
        \item Найдем все автоморфизмы $\Q \to \Q$. $\PHI: \Q \to Q$ -- произвольный автоморфизм. $\PHI(1) = 1 \SO n \in \Z, n > 0$
        \[\PHI(n) = \PHI(\underbrace{1 + 1 + ... + 1}_n) = \underbrace{\PHI(1) + ... + \PHI(1)}_n = n\]
        $\SO \PHI(-n) = -n \SO\PHI|_\Z = id$ \\
        $q \cdot \PHI \left(\frac{1}{q}\right) = \underbrace{\PHI \left(\frac{1}{q}\right) + ... + \PHI \left(\frac{1}{q}\right)}_q = \PHI\left(\frac{1}{q} + ... + \frac{1}{q}\right) = \PHI(1) = 1 \SO \PHI\left(\frac{1}{q}\right) = \frac{1}{q} \SO \PHI \left(\frac{p}{q}\right) = p \cdot \PHI \left(\frac{1}{q}\right) = \frac{p}{q} $
        $\SO$ автоморфизмы $\Q = \{id\}$    

        \item автоморфизмы $\R = \{id\}$
        \item $\C$, автоморфизмы: $id, \PHI(z) = \overline{z}$   
    \end{MyList}
\end{Example}

\begin{Def}
    $K$ -- поле, тогда $k$ -- подполе поля $K$, если $k \subset K$ и $k$ -- поле.
\end{Def}

\begin{Lm}
    $f \in k[x], c \in K, k \subset K, f(c) = 0, \PHI : K \to K$ -- автоморфизм, $\PHI |_k = id \SO \PHI(c) $ -- корень $f$.
\end{Lm}

\begin{proof}
    $f = a_n x^n + ... + a_0, a_i = k, i = 0, ..., n$
    \[\PHI(f) = \PHI(a_n x^n + ... + a_0) = \PHI(a_n) \cdot \PHI(x^n) + ... + \PHI(a_1) \PHI(x) + \PHI(a_0) =\]
    \[= a_n \PHI(x)^n + ... + a_1 \PHI(x) + a_0 = f(\PHI(x))\]
    $x = c : \PHI(f(c)) = \PHI(0) = 0$. С другой стороны $\PHI(f(c)) = f(\PHI(c)) \SO \PHI(c)$ корень $f$. 
\end{proof}

\begin{Thm}[Неприводимые многочлены над $\R$]
    Неприводимые многочлены над $\R$ 

    \begin{MyList}
        \item $x - \alpha, \alpha \in \R$ 
        \item $ax^2 + bx = c, b^2 - 4ac < 0$ 
    \end{MyList}
\end{Thm}

\begin{proof}
    $f$ -- неприводимый многочлен над $\R, \deg f \geqslant 2$, корней из $\R$ нет.
    Пусть $\alpha + \beta i$ -- комплексный корень $f$ ( $\beta \neq 0$ ). По лемме $a - \beta i $ -- тоже корень $\SO f \ \vdots \ (x - \alpha - \beta i)(x - \alpha + \beta i)$.
    \[(x - \alpha - \beta i)(x - \alpha + \beta i) = (x - \alpha)^2 + \beta^2 = x^2 - 2\alpha x + \alpha^2 + \beta^2 \in \R[x] \text{ -- неприводим}\]  
    $\SO f = a(x^2 - 2 \alpha x + \alpha^2 + \beta^2), a \in \R^*$ 
\end{proof}

\begin{Cons}
    Каноническое разложение над $\R$. 
    \[f = a(x - \alpha_1)^{k_1} ... (x - \alpha_s)^{k_s} (x^2 + p_1 x + q_1)^{r_1} ... (x^2 + p_t x + q_t)^{r_t}\]
    где $p_i^2 - 4q_i < 0, i = 1, ..., t$.  
\end{Cons}

\Pagebreak

\Subsection{Уравнения 3-й степени}

$ax^3 + bx^2 + cx + d = 0$
\begin{Thm}[Метод Кардано]
    \begin{MyList}
        \item $x^3 + b'x^2 + c'x + d' = 0$ (поделили на $a$)
        \item $x = y - \frac{b'}{3} \SO y^3 + py + q = 0$
        \item $y = u + v : (u + v)^3 + p(u + v) + q = 0$
        \[u^3 + 3u^2v + 3uv^2 + v^3 + p(u + v) + q = 0 \EQ (u + v)(3uv + p) + u^3 +v^3 + q = 0\]
        \[\SO \begin{cases}
            3uv + p = 0 \\
            u^3 + v^3 + q = 0
        \end{cases} \EQ \begin{cases}
            v = -\frac{p}{3u} \\
            u^3 - \frac{p}{27u^3} + q = 0
        \end{cases} \EQ \begin{cases}
            v = -\frac{p}{3u} \\
            27u^6 + 27qu^3 - p^3 = 0
        \end{cases}\] 
        $\SO$ получим три несимметричных решения $(u, v)$
        
        \item Находим $y$ и $x$.
    \end{MyList} 
\end{Thm} 

\Subsection{Уравнения 4-й степени}

$ax^4 + bx^3 + cx^2 + dx + e = 0$ 

\begin{Thm}[Метод Феррари]
    \begin{MyList}
        \item $x^4 + b'x^3 + c'x^2 + d'x + e' = 0$ (поделим на $a$)
        \item $x = y - \frac{b'}{4} \SO y^4 + \alpha y^2 + \beta y + \gamma$ 
        \item Введем $u : (y^2 + \alpha + u)^2 - (\alpha y^2 + uy^2 + 2 \alpha u - \beta y - \gamma + u^2 + \alpha^2) = 0$ \\
        Выбираем $u : (\alpha + u)y^2 + \beta y + (u + \alpha)^2 - \gamma$ -- полный квадрат $\EQ \beta^2 - u(\alpha + u)((u + \alpha)^2 - \gamma) = 0$ -- уравнение третьей степени. Находим $u$.
        
        \item Раскладываем разность квадратов: 2 квадратных уравнения относительно $y$
        \item Находим $x$.
    \end{MyList}
\end{Thm}

\Subsection{Отделение кратных корней}

\begin{Def}
    $f \in K[x], f = a_n x^n + ... + a_1 x + a_0$.
    Производной многочлена $f$ называется $f' = n a_n x^{n - 1} + ... + a_1$ 
\end{Def}

Свойства:

\begin{MyList}
    \item $\deg f = 0 \SO f' = 0$
    \item $(f + g)' = f' + g'$
    \item $(c \cdot f)' = c \cdot f', c \in K$ 
    \item $(f \cdot g)' = f'g + fg'$  
    \item $(f_1 \cdot f_2 \cdot ... \cdot f_k)' = f_1'f_2...f_k + f_1f_2'...f_k + f_1f_2...f_k'$
    \item $(f^k)' = k \cdot f^{k - 1} \cdot f'$  
\end{MyList}

\Pagebreak
\begin{proof}
    4. $f = a_n x^n, g = b_m x^m$ 
    \[(a_n x^n \cdot b_m x^m)' = (n + m)a_n b_m x^{n + m - 1}\]
    \[(a_n x^n)' \cdot b_m x^m + a_n x^n \cdot (b_m x^m)' = a_n b_m (n + m) x^{n + m - 1}\]
    $f = a_n x^n, g = b_m x^m + ... + b_1 x + b_0$ 
    \[(fg)' = \left(\sum \right)' = \sum ()' = f'g + fg'\]
    $f = a_n x^n + ... + a_1 x + a_0, g = b_m x^m + ... + b_1 x + b_0$ 
    $a_k x^k \cdot g$ -- верно $\SO$ верно и для $f \cdot g$

    5. Индукция \\
    5. $\SO$ 6.
\end{proof}

\begin{Def}
    Корни кратности 1 -- простые корни, корни кратности $> 1$ -- кратные корни.
\end{Def}

\begin{Thm}[Критерий кратности корня]
    $f \in K[x]$. $c$ -- кратный корень $f \EQ f(c) = f'(c) = 0$ 
\end{Thm}

\begin{proof}
    "$\SO$". $f = (x - c)^2 \cdot g, f(c) = 0$.
    $f'(c) = 2(x - c)g + (x - c)^2 \cdot g' \SO f'(c) = 0$. \\
    "$\Leftarrow$". $\deg f \geqslant 2, f$ делим на $(x - c)^2 \SO f = (x - c)^2 \cdot g + ax + b$
    \[\SO f = (x - c)^2 g + a(x - c) + r\]
    $f(c) = 0 \SO r = 0$
    \[f' = 2(x - c)g + (x - c)^2g' + a, f'(c) = 0 \SO a = 0 \SO f = (x - c)^2 g\]   
\end{proof}

\end{document}