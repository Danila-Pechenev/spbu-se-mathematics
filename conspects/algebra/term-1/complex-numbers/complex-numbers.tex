\documentclass[12pt]{article}

% Автор стиля: Сергей Копелиович
% Автор конспекта: Илья Дудников

\usepackage{cmap}
\usepackage[T2A]{fontenc}
\usepackage[utf8]{inputenc}
\usepackage[russian]{babel}
\usepackage{graphicx}
\usepackage{amsthm,amsmath,amssymb}
\usepackage{listings}
\usepackage{color}
\usepackage{xcolor}
\usepackage{array}
\usepackage{mathrsfs}
\usepackage{epigraph}

\usepackage[russian,colorlinks=true,urlcolor=red,linkcolor=blue]{hyperref}
\usepackage{enumerate}
\usepackage{datetime}
\usepackage{fancyhdr}
\usepackage{lastpage}
\usepackage{verbatim}
\usepackage{tikz}
\usetikzlibrary{arrows,decorations.markings,decorations.pathmorphing}
\usepackage{pgfplots}

\usepackage{ifthen}
\usepackage{mathtools}

%\usepackage{tabls}
%\usepackage{tabularx}
%\usepackage{xifthen}
%\listfiles

\usepackage{import}
\usepackage{xifthen}
\usepackage{pdfpages}
\usepackage{transparent}

\def\NAME{Лекция, 10/22/21}

\sloppy
\voffset=-20mm
\textheight=235mm
\hoffset=-22mm
\textwidth=180mm
\headsep=12pt
\footskip=20pt

\parskip=0em
\parindent=0em

\setlength\epigraphwidth{.8\textwidth}

\newlength{\tmplen}
\newlength{\tmpwidth}
\newcounter{listcounter}

% Список с маленькими отступами
\newenvironment{MyList}[1][4pt]{
  \begin{enumerate}[1.]
  \setlength{\parskip}{0pt}
  \setlength{\itemsep}{#1}
}{       
  \end{enumerate}
}
% Вложенный список с маленькими отступами
\newenvironment{InnerMyList}[1][0pt]{
  \vspace*{-0.5em}
  \begin{enumerate}[(a)]
  \setlength{\parskip}{-0pt}
  \setlength{\itemsep}{#1}
}{       
  \end{enumerate}
  \vspace*{-0.5em}
}
% Список с маленькими отступами
\newenvironment{MyItemize}[1][4pt]{
  \begin{itemize}
  \setlength{\parskip}{0pt}
  \setlength{\itemsep}{#1}
}{       
  \end{itemize}
}

% Основные математические символы
\def\TODO{{\color{red}\bf TODO}}
\def\C{\mathbb{C}}
\def\Q{\mathbb{Q}}
\def\N{\mathbb{N}}       %
\def\R{\mathbb{R}}       %
\def\F2{\mathbb{F}_2}    %
\def\Z{\mathbb{Z}}       %
\def\INF{\t{+}\infty}    % +inf
\def\EPS{\varepsilon}    %
\def\EMPTY{\varnothing}  %
\def\PHI{\varphi}        %
\def\SO{\Rightarrow}     % =>
\def\EQ{\Leftrightarrow} % <=>
\def\t{\texttt}          % mono font
\def\c#1{{\rm\sc{#1}}}   % font for classes NP, SAT, etc
\def\O{\mathcal{O}}      %
\def\NO{\t{\#}}          % #
\def\XOR{\text{ {\raisebox{-2pt}{\ensuremath{\Hat{}}}} }}
\renewcommand{\le}{\leqslant}
\renewcommand{\ge}{\geqslant}
\newcommand{\q}[1]{\langle #1 \rangle}               % <x>
\newcommand\URL[1]{{\footnotesize{\url{#1}}}}        %
% \newcommand{\sfrac}[2]{{\scriptscriptstyle\frac{#1}{#2}}}  % Очень маленькая дробь
% \newcommand{\mfrac}[2]{{\scriptstyle\frac{#1}{#2}}}    % Небольшая дробь
\newcommand{\sfrac}[2]{{\scriptstyle\frac{#1}{#2}}}  % Очень маленькая дробь
\newcommand{\mfrac}[2]{{\textstyle\frac{#1}{#2}}}    % Небольшая дробь

\newcommand{\fix}[1]{{\color{fixcolor}{#1}}} % \underline
\def\bonus{\t{\red{(*)}}}
\def\ifbonus#1{\ifthenelse{\equal{#1}{}}{}{\bonus}}
\def\smallsquare{$\scalebox{0.5}{$\square$}$}

\newlength{\myItemLength}
\setlength{\myItemLength}{0.3em}
\def\ItemSymbol{\smallsquare}
\def\Item{\vspace*{\myItemLength}\ItemSymbol \ \ }

\newcommand{\LET}{%
  % [line width=0.6pt]
  \begin{tikzpicture}%
  \draw(0.8ex,0) -- (0.8ex,1.6ex);%
  \draw(0,1.6ex) -- (0.8ex,1.6ex);%
  \end{tikzpicture}%
  \hspace*{0.1em}%
}

% Отступы
\def\makeparindent{\hspace*{\parindent}\unskip}
\def\up{\vspace*{-0.5em}}%{\vspace*{-\baselineskip}}
\def\down{\vspace*{0.5em}}
\def\LINE{\vspace*{-1em}\noindent \underline{\hbox to 1\textwidth{{ } \hfil{ } \hfil{ } }}}
\def\BOX#1{\mbox{\fbox{\bf{#1}}}}
\def\Pagebreak{\pagebreak\vspace*{-1.5em}}

% Мелкий заголовок
\newcommand{\THEE}[1]{
  \vspace*{0.5em}
  \noindent{\bf \underline{#1}}%\hspace{0.5em}
  \vspace*{0.2em}
}
% Другой тип мелкого заголовка
\newcommand{\THE}[1]{
  \vspace*{0.5em} $\bullet$
  \noindent{\bf #1}%\hspace{0.5em}
  \vspace*{0.2em}
}

\newenvironment{MyTabbing}{
  \t\bgroup
  \vspace*{-\baselineskip}
  \begin{tabbing}
    aaaa\=aaaa\=aaaa\=aaaa\=aaaa\=aaaa\kill
}{
  \end{tabbing}
  \t\egroup
}

% Код с правильными отступами
\lstnewenvironment{code}{
  \lstset{}
%  \vspace*{-0.2em}
}%
{
%  \vspace*{-0.2em}
}
\lstnewenvironment{codep}{
  \lstset{language=python}
}%
{
}

% Формулы с правильными отступами
\newenvironment{smallformula}{
 
  \vspace*{-0.8em}
}{
  \vspace*{-1.2em}
  
}
\newenvironment{formula}{
 
  \vspace*{-0.4em}
}{
  \vspace*{-0.6em}
  
}

% Большая квадратная скобка
\makeatletter
\newenvironment{sqcases}{%
  \matrix@check\sqcases\env@sqcases
}{%
  \endarray\right.%
}
\def\env@sqcases{%
  \let\@ifnextchar\new@ifnextchar
  \left\lbrack
  \def\arraystretch{1.2}%
  \array{@{}l@{\quad}l@{}}%
}
\makeatother

\DeclareMathOperator{\sort}{sort}

% Определяем основные секции: \begin{Lm}, \begin{Thm}, \begin{Def}, \begin{Rem}
\renewcommand{\qedsymbol}{$\blacksquare$}
\theoremstyle{definition} % жирный заголовок, плоский текст
\newtheorem{Thm}{\underline{Теорема}}[subsection] % нумерация будет "<номер subsection>.<номер теоремы>"
\newtheorem{Lm}[Thm]{\underline{Lm}} % Нумерация такая же, как и у теорем
\newtheorem{Ex}[Thm]{Упражнение} % Нумерация такая же, как и у теорем
\newtheorem{Example}[Thm]{Пример} % Нумерация такая же, как и у теорем
\newtheorem{Solution}[Thm]{Решение}
\newtheorem{Code}[Thm]{Код} % Нумерация такая же, как и у теорем
\theoremstyle{plain} % жирный заголовок, курсивный текст
\newtheorem{Def}[Thm]{Def.} % Нумерация такая же, как и у теорем
\theoremstyle{remark} % курсивный заголовок, плоский текст
\newtheorem{Cons}[Thm]{Следствие} % Нумерация такая же, как и у теорем
\newtheorem{Conj}[Thm]{Гипотеза} % Нумерация такая же, как и у теорем
\newtheorem{Prop}[Thm]{Утверждение} % Нумерация такая же, как и у теорем
\newtheorem{Rem}[Thm]{Замечание} % Нумерация такая же, как и у теорем
\newtheorem{Remark}[Thm]{Замечание} % Нумерация такая же, как и у теорем
\newtheorem{Algo}[Thm]{Алгоритм} % Нумерация такая же, как и у теорем

% Определяем ЗАГОЛОВКИ
\def\SectionName{Алгебра}
\def\AuthorName{Илья Дудников}

\newlength{\sectionvskip}
\setlength{\sectionvskip}{0.5em}
\newcommand{\Section}[4][]{
  % Заголовок
  \pagebreak
%  \ifthenelse{\isempty{#1}}{
    \refstepcounter{section}
%  }{}
  \vspace{0.5em}
%  \ifthenelse{\isempty{#1}}{
%    \addtocontents{toc}{\protect\addvspace{-5pt}}%
    \addcontentsline{toc}{section}{\arabic{section}. #2}
%  }{}
  


  % Запомнили название и автора главы
  \gdef\SectionName{#2}
  \gdef\AuthorName{#4}

  % Заголовок страницы
  \lhead{Конспект, \NAME}
  \chead{}
  \rhead{\SectionName}
  \renewcommand{\headrulewidth}{0.4pt}
    
  \lfoot{}
  \cfoot{\thepage\t{/}\pageref*{LastPage}}
  \rfoot{Автор: \AuthorName}
  \renewcommand{\footrulewidth}{0.4pt}
}

\newcommand{\Subsection}[2][]{
  \refstepcounter{subsection}
  \vspace*{1em}
  \ifthenelse{\equal{#1}{}}
    {\addcontentsline{toc}{subsection}{\arabic{section}.\arabic{subsection}. #2}}
    {\addcontentsline{toc}{subsection}{\arabic{section}.\arabic{subsection}. \bonus\,#2}}
  {\color{blue}\bf\large \arabic{section}.\arabic{subsection}. \ifbonus{#1}\,{#2}} 
  \vspace*{0.5em}
  \makeparindent
}
\newcommand{\Subsubsection}[2][]{
  \refstepcounter{subsubsection}
  \vspace*{1em}
  \ifthenelse{\equal{#1}{}}
    {\addcontentsline{toc}{subsubsection}{\arabic{section}.\arabic{subsection}.\arabic{subsubsection}. #2}}
    {\addcontentsline{toc}{subsubsection}{\arabic{section}.\arabic{subsection}.\arabic{subsubsection}. \bonus\,#2}}
  {\color{blue}\bf\large \arabic{section}.\arabic{subsection}.\arabic{subsubsection}. \ifbonus{#1}\,#2}
  \vspace*{0.5em}
  \makeparindent
}

\newcommand{\Header}{
  \pagestyle{empty}
  \renewcommand{\dateseparator}{--}
  \begin{center}
    {\Large\bf 
    Алгебра 1 семестр ПИ,\\
    \vspace{0.3em}
     \NAME}\\
    \vspace{0.7em}
    {Собрано {\today} в {\currenttime}}
  \end{center}

  \LINE
  \vspace{0em}

  \renewcommand{\baselinestretch}{0.98}\normalsize
  \tableofcontents
  \renewcommand{\baselinestretch}{1.0}\normalsize
  \pagebreak
}

\newcommand{\BeginConspect}{
  \pagestyle{fancy}
  \setcounter{page}{1}
}

\definecolor{mygray}{rgb}{0.7,0.7,0.7}
\definecolor{ltgray}{rgb}{0.9,0.9,0.9}
\definecolor{fixcolor}{rgb}{0.7,0,0}
\definecolor{red2}{rgb}{0.7,0,0}
\definecolor{dkred}{rgb}{0.4,0,0}
\definecolor{dkblue}{rgb}{0,0,0.6}
\definecolor{dkgreen}{rgb}{0,0.6,0}
\definecolor{brown}{rgb}{0.5,0.5,0}

\newcommand{\green}[1]{{\color{green}{#1}}}
\newcommand{\black}[1]{{\color{black}{#1}}}
\newcommand{\red}[1]{{\color{red}{#1}}}
\newcommand{\dkred}[1]{{\color{dkred}{#1}}}
\newcommand{\blue}[1]{{\color{blue}{#1}}}
\newcommand{\dkgreen}[1]{{\color{dkgreen}{#1}}}

\newcommand{\Mod}[1]{\ (\mathrm{mod}\ #1)}

\DeclareMathOperator{\Real}{Re}
\DeclareMathOperator{\Imag}{Im}
%\DeclareMathOperator{\tg}{tg}
%\DeclareMathOperator{\arctg}{arctg}

\begin{document}

\Header

\BeginConspect

\Section{Комплексные числа}{}{Илья Дудников}

\begin{Def}
    Множество $\{(a, b) | a, b \in \R\}$ называется множество комплексных чисел, если:
    \begin{MyList}
        \item $(a, b) = (c, d) \EQ a = c, b = d$
        \item $(a, b) + (c, d) = (a + c, b + d)$
        \item $(a, b) \cdot (c, d) = (ac - bd, ad + bc)$
        \item $a = (a, 0)$    
    \end{MyList} 

    Проверим корректность:
    \begin{MyItemize}
        \item 1 и 4: $a = b \EQ (a, 0) = (b, 0)$
        \item 2 и 4: $a + b = (a, 0) + (b, 0) = (a + b, 0) = a + b$
        \item 3 и 4: $a \cdot b = (a, 0) \cdot (b, 0) = (ab, 0) = ab$   
    \end{MyItemize}
\end{Def}

\begin{Thm}
    $\C$ образует коммутативное кольцо с единицей.
\end{Thm}

\begin{proof}
    $(0, 0)$ -- нейтральный элемент по сложению. $(a, b) : -(a, b) = (-a, -b)$ -- обратный элемент по сложению. Остальные свойства несложно проверяются.
\end{proof}

\begin{Def}
    Множество $K$ называется полем, если $K$ является коммутативным кольцом с единицей и 
    \[\forall x \in K^* = K \setminus \{0\} \ \exists x^{-1} \in K : x \cdot x^{-1} = 1\]
\end{Def}

\begin{Thm}
    $\Z_p (p \text{ -- простое}), \Q, \R, \C$ -- поля.   
\end{Thm}

\begin{proof}
    $\Q, \R$ -- поля.
    
    $\Z_p$ -- коммутативное кольцо с единицей, $\Z_p^*$ -- мультипликативная группа $\SO \Z_p$ -- поле.
    
    $(a, b) \in \C^*, (a, b)^{-1} = \frac{(a, -b)}{a^2 + b^2} = \left(\frac{a}{a^2 + b^2}, \frac{-b}{a^2 + b^2}\right)$ 
    \[(a, b) \cdot \frac{(a, -b)}{a^2 + b^2} = \frac{(a^2 + b^2, 0)}{a^2 + b^2} = (1, 0)\]
\end{proof}

\begin{Def}
    $(a, b)$ и $(a, -b)$ -- комплексно-сопряженные числа. \\
    $|(a, b)| = \sqrt{a^2 + b^2}$ -- модуль комплексного числа. Заметим, что $|(a, 0)| = \sqrt{a^2 + 0} = \sqrt{a^2} = |a|$ \\
    $(a, b) \cdot (a, -b) = a^2 + b^2 = |(a, b)|^2$ 
\end{Def}

\Subsection{Алгебраическая форма записи комплексного числа}

\begin{Def}
    Положим $i = (0, 1)$. Тогда
    \[(a, b) = (a, 0) + (0, b) = (a, 0) \cdot (1, 0) + (b, 0) \cdot (0, 1) = a + bi\]
\end{Def}
\Pagebreak

\Subsection{Геометрическое представление комплексных чисел}

\begin{Def}
    $z = a + bi$. $\Real z = a$ -- вещественная часть числа $z$, $\Imag z = b$ -- мнимая часть.   \\
    $z = a + bi \mapsto $ точка на комплексной плоскости. $(a, b)$ -- радиус-вектор $OM$.  \\
    $\rho = |z| = \sqrt{a^2 + b^2}$ -- длина вектора $OM$. $\PHI = \widehat{(\Real, OM)}$ -- аргумент комплексного числа. \\
    $\arg z = \PHI, \PHI = \PHI_0 + 2\pi k, \PHI_0 \in [0; 2\pi)$ или $\PHI_0 \in (-\pi; pi]$.

    \begin{figure}[h]
        \centering
        \input{images/pic1.pdf_tex}
    \end{figure}
\end{Def}

\Subsection{Тригонометрическая форма записи комплексного числа}

\begin{Def}
    $a = \rho \cos \PHI, b = \rho \sin \PHI \SO z = a + bi = \rho(\cos \PHI + i\sin \PHI)$
    \[\tg \PHI = \frac{b}{a} \SO \PHI = \begin{cases}
        \arctg \frac{b}{a}, z \in \text{ I и II четверти} \\
        \arctg \frac{b}{a} + \pi, z \in \text{III и IV четверти} 
    \end{cases}\] 

\end{Def} 

\begin{Def}[Неравенство треугольника]
    $z_1, z_2 \in \C$
    \begin{MyList}
        \item $|z_1 + z_2| \leqslant |z_1| + |z_2|$
        \item $|z_1 - z_2| \geqslant ||z_1| - |z_2||$  
    \end{MyList} 
\end{Def}

\begin{proof}
    \begin{MyList}
        \item $z_1 = \rho_1(\cos \PHI_1 + i\sin \PHI_1), z_2 = \rho_2(\cos \PHI_2 + i\sin \PHI_2)$ \\
        \begin{gather*}
            |z_1 + z_2|^2 = |\rho_1 \cos \PHI_1 + \rho_2 \cos \PHI_2 + i(\rho_1 \sin \PHI_1 + \rho_2 \sin \PHI_2)|^2 = \\
            = \rho_1^2 \cos^2 \PHI_1 + 2 \rho_1 \rho_2 \cos \PHI_1 \cos \PHI_2 + \rho_2^2 \cos^2 \PHI_2 + \rho_1^2 \sin^2 \PHI_1 + 2 \rho_1 \rho_2 \sin \PHI_1 \sin \PHI_2 + \rho_2^2 \sin^2 \PHI_2 = \\
            = \rho_1^2 + 2\rho_1 \rho_2 \cos(\PHI_1 - \PHI_2) = \rho_2^2 \leqslant \rho_1^2 + 2\rho_1 \rho_2 + \rho_2^2 = (\rho_1 + \rho_2)^2 = (|z_1| + |z_2|)^2
        \end{gather*}

        \item \[
            |z_1| = |z_1 - z_2 + z_2| \leqslant |z_1 - z_2| + |z_2| \SO |z_1| - |z_2| \leqslant |z_1 - z_2| \SO ||z_1| - |z_2|| \leqslant |z_1 - z_2|
        \]
    \end{MyList}
\end{proof}

\begin{Rem}
    $|z_1 + z_2| = |z_1| + z_2| \EQ z_1 \parallel z_2$ 
\end{Rem}

\Pagebreak
\begin{Thm}[Умножение комплексных чисел в тригонометрической форме]
    $z_1 = \rho_1 (\cos \PHI_1 + i\sin \PHI_1), z_2 = \rho_2 (\cos \PHI_2 + i\sin \PHI_2)$. Тогда 
    \[z_1 \cdot z_2 = \rho_1 \cdot \rho_2 (\cos (\PHI_1 + \PHI_2) + i\sin (\PHI_1 + \PHI_2))\] 
\end{Thm}

\begin{proof}
    Достаточно перемножить, заметить формулу косинуса суммы и синуса суммы
\end{proof}

\begin{Cons}[Формула Муавра]
    $z = \rho (\cos \PHI + i\sin \PHI) \SO z^n = \rho^n(\cos n\PHI + i\sin n\PHI)$ 
\end{Cons}

\begin{proof}
    \begin{MyList}
        \item $n \geqslant 0$. По индукции: $n = 1$ очевидно. \\
        $n - 1 \to n:$
        \[z^n = z^{n - 1} \cdot z = \rho^{n - 1}(\cos (n - 1)\PHI + i\sin(n - 1)\PHI) \cdot \rho(\cos \PHI + i\sin\PHI) = \rho^n(\cos n\PHI + i\sin n\PHI)\] 

        \item $n < 0$. Пусть $n = -m, m > 0$. Тогда
        \[z^n = \frac{1}{z^m} = \frac{1}{\rho^m(\cos m\PHI + i\sin m\PHI)} = \rho^{-m} \frac{\cos m\PHI - i\sin m\PHI}{1} = \rho^n (\cos n\PHI + i\sin n\PHI)\]
    \end{MyList}
\end{proof}

\Subsection{Извлечение корней из комплексных чисел}

\begin{Def}
    Корнем $n$-й степени из комплексного числа $z$ называется $w \in \C : w^n = z$ 
\end{Def}

\begin{Thm}
    $\forall z \in \C^* \ \exists n$ корней $n$-й степени $z_k, k = 0, 1, ..., n - 1$
    \[z_k = \sqrt[n]{\rho}(\cos \frac{\PHI + 2\pi k}{n} + i\sin \frac{\PHI + 2\pi k}{n}), z = \rho(\cos \PHI + i\sin \PHI)\]  
\end{Thm}

\begin{proof}
    $w^n = z, w = R(\cos \Theta + i \sin \Theta)$ 
    \begin{gather*}
        \SO (w^n = z): R^n (\cos (n \Theta) + i\sin (n \Theta)) = \rho(\cos \PHI + i\sin \PHI) \SO \\
        \SO R = \sqrt[n]{\rho}, \cos (n \Theta) = \cos \PHI, \sin(n \Theta) = \sin \PHI \SO \\
        \SO n \Theta = \PHI + 2\pi k \SO \Theta = \frac{\PHI + 2\pi k}{n} \SO \text{ любой корень имеет вид } z_k
    \end{gather*}
\end{proof}

\end{document}