\documentclass[12pt]{article}

% Автор: Илья Дудников
% Автор стиля: Сергей Копелиович

\usepackage{cmap}
\usepackage[T2A]{fontenc}
\usepackage[utf8]{inputenc}
\usepackage[russian]{babel}
\usepackage{graphicx}
\usepackage{amsthm,amsmath,amssymb}
\usepackage{listings}
\usepackage{color}
\usepackage{xcolor}
\usepackage{array}
\usepackage{epigraph}
\usepackage{multicol}

\usepackage[russian,colorlinks=true,urlcolor=red,linkcolor=blue]{hyperref}
\usepackage{enumerate}
\usepackage{datetime}
\usepackage{fancyhdr}
\usepackage{lastpage}
\usepackage{verbatim}
\usepackage{tikz}
\usepackage{MnSymbol}
\usetikzlibrary{arrows,decorations.markings,decorations.pathmorphing}
\usepackage{pgfplots}

\usepackage{ifthen}
\usepackage{mathtools}

%\usepackage{tabls}
%\usepackage{tabularx}
%\usepackage{xifthen}
%\listfiles

\def\NAME{Лекции}
\def\SEASON{Конспект лекций по матанализу, ПИ, 1 семестр}

\sloppy
\voffset=-20mm
\textheight=235mm
\hoffset=-22mm
\textwidth=180mm
\headsep=12pt
\footskip=20pt

\parskip=0em
\parindent=0em

\setlength\epigraphwidth{.8\textwidth}

\newlength{\tmplen}
\newlength{\tmpwidth}
\newcounter{listcounter}

% Список с маленькими отступами
\newenvironment{MyList}[1][4pt]{
  \begin{enumerate}[1.]
  \setlength{\parskip}{0pt}
  \setlength{\itemsep}{#1}
}{       
  \end{enumerate}
}
% Вложенный список с маленькими отступами
\newenvironment{InnerMyList}[1][0pt]{
  \vspace*{-0.5em}
  \begin{enumerate}[(a)]
  \setlength{\parskip}{-0pt}
  \setlength{\itemsep}{#1}
}{       
  \end{enumerate}
  \vspace*{-0.5em}
}
% Список с маленькими отступами
\newenvironment{MyItemize}[1][4pt]{
  \begin{itemize}
  \setlength{\parskip}{0pt}
  \setlength{\itemsep}{#1}
}{       
  \end{itemize}
}

% Основные математические символы
\def\TODO{{\color{red}\bf TODO}}
\def\C{\mathbb{C}}       %
\def\Q{\mathbb{Q}}       %
\def\N{\mathbb{N}}       %
\def\R{\mathbb{R}}       %
\def\F2{\mathbb{F}_2}    %
\def\Z{\mathbb{Z}}       %
\def\INF{\t{+}\infty}    % +inf
\def\EPS{\varepsilon}    %
\def\EMPTY{\varnothing}  %
\def\PHI{\varphi}        %
\def\SO{\Rightarrow}     % =>
\def\EQ{\Leftrightarrow} % <=>
\def\t{\texttt}          % mono font
\def\c#1{{\rm\sc{#1}}}   % font for classes NP, SAT, etc
\def\O{\mathcal{O}}      %
\def\NO{\t{\#}}          % #
\def\XOR{\text{ {\raisebox{-2pt}{\ensuremath{\Hat{}}}} }}
\renewcommand{\le}{\leqslant}
\renewcommand{\ge}{\geqslant}
\newcommand{\q}[1]{\langle #1 \rangle}               % <x>
\newcommand\URL[1]{{\footnotesize{\url{#1}}}}        %
% \newcommand{\sfrac}[2]{{\scriptscriptstyle\frac{#1}{#2}}}  % Очень маленькая дробь
% \newcommand{\mfrac}[2]{{\scriptstyle\frac{#1}{#2}}}    % Небольшая дробь
\newcommand{\sfrac}[2]{{\scriptstyle\frac{#1}{#2}}}  % Очень маленькая дробь
\newcommand{\mfrac}[2]{{\textstyle\frac{#1}{#2}}}    % Небольшая дробь

\newcommand{\fix}[1]{{\color{fixcolor}{#1}}} % \underline
\def\bonus{\t{\red{(*)}}}
\def\ifbonus#1{\ifthenelse{\equal{#1}{}}{}{\bonus}}
\def\smallsquare{$\scalebox{0.5}{$\square$}$}

\newlength{\myItemLength}
\setlength{\myItemLength}{0.3em}
\def\ItemSymbol{\smallsquare}
\def\Item{\vspace*{\myItemLength}\ItemSymbol \ \ }

\newcommand{\LET}{%
  % [line width=0.6pt]
  \begin{tikzpicture}%
  \draw(0.8ex,0) -- (0.8ex,1.6ex);%
  \draw(0,1.6ex) -- (0.8ex,1.6ex);%
  \end{tikzpicture}%
  \hspace*{0.1em}%
}

% Отступы
\def\makeparindent{\hspace*{\parindent}\unskip}
\def\up{\vspace*{-0.5em}}%{\vspace*{-\baselineskip}}
\def\down{\vspace*{0.5em}}
\def\LINE{\vspace*{-1em}\noindent \underline{\hbox to 1\textwidth{{ } \hfil{ } \hfil{ } }}}
\def\BOX#1{\mbox{\fbox{\bf{#1}}}}
\def\Pagebreak{\pagebreak\vspace*{-1.5em}}

% Мелкий заголовок
\newcommand{\THEE}[1]{
  \vspace*{0.5em}
  \noindent{\bf \underline{#1}}%\hspace{0.5em}
  \vspace*{0.2em}
}
% Другой тип мелкого заголовка
\newcommand{\THE}[1]{
  \vspace*{0.5em} $\bullet$
  \noindent{\bf #1}%\hspace{0.5em}
  \vspace*{0.2em}
}

\newenvironment{MyTabbing}{
  \t\bgroup
  \vspace*{-\baselineskip}
  \begin{tabbing}
    aaaa\=aaaa\=aaaa\=aaaa\=aaaa\=aaaa\kill
}{
  \end{tabbing}
  \t\egroup
}

% Код с правильными отступами
\lstnewenvironment{code}{
  \lstset{}
%  \vspace*{-0.2em}
}%
{
%  \vspace*{-0.2em}
}
\lstnewenvironment{codep}{
  \lstset{language=python}
}%
{
}

% Формулы с правильными отступами
\newenvironment{smallformula}{
 
  \vspace*{-0.8em}
}{
  \vspace*{-1.2em}
  
}
\newenvironment{formula}{
 
  \vspace*{-0.4em}
}{
  \vspace*{-0.6em}
  
}

% Большая квадратная скобка
\makeatletter
\newenvironment{sqcases}{%
  \matrix@check\sqcases\env@sqcases
}{%
  \endarray\right.%
}
\def\env@sqcases{%
  \let\@ifnextchar\new@ifnextchar
  \left\lbrack
  \def\arraystretch{1.2}%
  \array{@{}l@{\quad}l@{}}%
}
\makeatother

% Определяем основные секции: \begin{Lm}, \begin{Thm}, \begin{Def}, \begin{Rem}
\renewcommand{\qedsymbol}{$\blacksquare$}
\theoremstyle{definition} % жирный заголовок, плоский текст
\newtheorem{Thm}{\underline{Теорема}}[subsection] % нумерация будет "<номер subsection>.<номер теоремы>"
\newtheorem{Lm}[Thm]{\underline{Lm}} % Нумерация такая же, как и у теорем
\newtheorem{Ex}[Thm]{Упражнение} % Нумерация такая же, как и у теорем
\newtheorem{Example}[Thm]{Пример} % Нумерация такая же, как и у теорем
\newtheorem{Code}[Thm]{Код} % Нумерация такая же, как и у теорем
\theoremstyle{plain} % жирный заголовок, курсивный текст
\newtheorem{Def}[Thm]{Def} % Нумерация такая же, как и у теорем
\theoremstyle{remark} % курсивный заголовок, плоский текст
\newtheorem{Cons}[Thm]{Следствие} % Нумерация такая же, как и у теорем
\newtheorem{Conj}[Thm]{Гипотеза} % Нумерация такая же, как и у теорем
\newtheorem{Prop}[Thm]{Утверждение} % Нумерация такая же, как и у теорем
\newtheorem{Rem}[Thm]{Замечание} % Нумерация такая же, как и у теорем
\newtheorem{Remark}[Thm]{Замечание} % Нумерация такая же, как и у теорем
\newtheorem{Algo}[Thm]{Алгоритм} % Нумерация такая же, как и у теорем

% Определяем ЗАГОЛОВКИ
\def\SectionName{unknown}
\def\AuthorName{unknown}

\newlength{\sectionvskip}
\setlength{\sectionvskip}{0.5em}
\newcommand{\Section}[4][]{
  % Заголовок
  \pagebreak
%  \ifthenelse{\isempty{#1}}{
    \refstepcounter{section}
%  }{}
  \vspace{0.5em}
%  \ifthenelse{\isempty{#1}}{
%    \addtocontents{toc}{\protect\addvspace{-5pt}}%
    \addcontentsline{toc}{section}{\arabic{section}. #2}
%  }{}
  \begin{center}
    {\Large \bf Раздел \NO{\arabic{section}}: #2} \\ 
    \vspace{\sectionvskip}
    \ifthenelse{\equal{#3}{}}{}{{\large #3}\\}
  \end{center}

  \LINE

  % Запомнили название и автора главы
  \gdef\SectionName{#2}
  \gdef\AuthorName{#4}

  % Заголовок страницы
  \lhead{\SEASON}
  \chead{}
  \rhead{\SectionName}
  \renewcommand{\headrulewidth}{0.4pt}

  \lfoot{Глава \NO{\arabic{section}}.}
  \cfoot{\thepage\t{/}\pageref*{LastPage}}
  \rfoot{Автор: \AuthorName}
  \renewcommand{\footrulewidth}{0.4pt}
}

\newcommand{\Subsection}[2][]{
  \refstepcounter{subsection}
  \vspace*{1em}
  \ifthenelse{\equal{#1}{}}
    {\addcontentsline{toc}{subsection}{\arabic{section}.\arabic{subsection}. #2}}
    {\addcontentsline{toc}{subsection}{\arabic{section}.\arabic{subsection}. \bonus\,#2}}
  {\color{blue}\bf\large \arabic{section}.\arabic{subsection}. \ifbonus{#1}\,{#2}} 
  \vspace*{0.5em}
  \makeparindent
}
\newcommand{\Subsubsection}[2][]{
  \refstepcounter{subsubsection}
  \vspace*{1em}
  \ifthenelse{\equal{#1}{}}
    {\addcontentsline{toc}{subsubsection}{\arabic{section}.\arabic{subsection}.\arabic{subsubsection}. #2}}
    {\addcontentsline{toc}{subsubsection}{\arabic{section}.\arabic{subsection}.\arabic{subsubsection}. \bonus\,#2}}
  {\color{blue}\bf\large \arabic{section}.\arabic{subsection}.\arabic{subsubsection}. \ifbonus{#1}\,#2}
  \vspace*{0.5em}
  \makeparindent
}

\newcommand{\Header}{
  \pagestyle{empty}
  \renewcommand{\dateseparator}{--}
  \begin{center}
    {\Large\bf 
     Матанализ 2 семестр ПИ,\\
    \vspace{0.3em}
    \NAME}\\
    \vspace{0.7em}
    {Собрано {\today} в {\currenttime}}
  \end{center}

  \LINE
  \vspace{0em}

  \renewcommand{\baselinestretch}{0.98}\normalsize
  \tableofcontents
  \renewcommand{\baselinestretch}{1.0}\normalsize
  \pagebreak
}

\newcommand{\BeginConspect}{
  \pagestyle{fancy}
  \setcounter{page}{1}
}

\definecolor{mygray}{rgb}{0.7,0.7,0.7}
\definecolor{ltgray}{rgb}{0.9,0.9,0.9}
\definecolor{fixcolor}{rgb}{0.7,0,0}
\definecolor{red2}{rgb}{0.7,0,0}
\definecolor{dkred}{rgb}{0.4,0,0}
\definecolor{dkblue}{rgb}{0,0,0.6}
\definecolor{dkgreen}{rgb}{0,0.6,0}
\definecolor{brown}{rgb}{0.5,0.5,0}

\newcommand{\green}[1]{{\color{green}{#1}}}
\newcommand{\black}[1]{{\color{black}{#1}}}
\newcommand{\red}[1]{{\color{red}{#1}}}
\newcommand{\dkred}[1]{{\color{dkred}{#1}}}
\newcommand{\blue}[1]{{\color{blue}{#1}}}
\newcommand{\dkgreen}[1]{{\color{dkgreen}{#1}}}

\newcommand{\Mod}[1]{\ (\mathrm{mod}\ #1)}

\DeclareMathOperator{\Real}{Re}
\DeclareMathOperator{\Imag}{Im}
\DeclareMathOperator{\lcm}{lcm}
\DeclareMathOperator{\sign}{sign}
\DeclareMathOperator{\Si}{Si}
\DeclareMathOperator{\const}{const}

\begin{document}
	\Header

	\BeginConspect

	\Section{Аналитическая геометрия}{}{Илья Дудников}

	\Subsection{Системы координат}
	
	\Subsubsection{Аффинные системы координат}

	\begin{Def}
		Аффинной системой координат на прямой называется взаимно-однозначное соответствие $l \longleftrightarrow \R$. \\
		
		\begin{figure*}[h]
			\centering
			\def\svgwidth{0.3\columnwidth}
			\input{img/as_line.pdf_tex}
		\end{figure*}
		
		Она определяется выбором точки $O$  и выбором вектора $\overline{e}$. АСК = $\{O, \{\overline{e}\}\}$.
	\end{Def}

	\begin{Def}
		АСК на плоскости называется биекция $\pi \longleftrightarrow \R^2$. 

		\begin{figure*}[h]
			\centering
			\def\svgwidth{0.3\columnwidth}
			\input{img/as_plane.pdf_tex}
		\end{figure*}

		Она определяется выбором точки $O$ и векторов $\overline{e}_1, \overline{e}_2 \neq \overline{e}, \overline{e}_1 \not\parallel \overline{e}_2$. 
		АСК = $\{O, \{\overline{e}_1, \overline{e}_2\}\}$.
	\end{Def}

	\begin{Def}
		Если $|\overline{e}_1| = |\overline{e}_2| = 1, \overline{e}_1 \perp \overline{e}_2$, то АСК называется декартовой системой координат. 
	\end{Def}

	\begin{Def}
		АСК в пространстве называется биекция $M \longleftrightarrow \R^3$ .
		Она определяется выбором точки $O$ и векторов $\overline{e}_1, \overline{e}_2, \overline{e}_3 \neq \overline{0}$ -- не компланарны. АСК = $\{O, \{\overline{e}_1, \overline{e}_2, \overline{e}_3\}\}$.
	\end{Def}

	\begin{Def}
		Упорядоченная тройка векторов $(\overline{u}, \overline{v}, \overline{w})$ называется \textbf{правой} , 
		если из конца вектор $\overline{w}$ поворот то $\overline{u}$ к $\overline{v}$ по наименьшему углу выглядит происходящим против часовой стрелки,
		и \textbf{левой} -- в противном случае. 
	\end{Def}

	\Pagebreak
	\Subsubsection{Криволинейные системы координаты}

	\begin{Def}
		Выберем точку $O$ и построим из неё луч $p$, который назовем \textit{полярной осью}. Возьмем теперь произвольную точку $M$ на плоскости и измерим две величины:
		расстояние от $M$ до $O$ и угол между вектором $\overline{OM}$ и полярной осью. Обозначим расстояние за $r$, а угол за $\PHI$.
		Тогда, чтобы избежать неоднозначности, будем считать, что $r > 0, \PHI \in [0, 2\pi)$, и если $r = 0$, то $\PHI = 0$.    
		Такая система координат называется \textbf{полярной}.		

		\begin{figure*}[h]
			\centering
			\def\svgwidth{0.3\columnwidth}
			\input{img/polar_system.pdf_tex}
		\end{figure*}
	\end{Def}

	\begin{Def}
		Полярная система координат, где $r \in \R, \PHI \in \R$, то она называется \textit{обобщенной} полярной системой координат.
	\end{Def}

	\begin{figure*}[h!]
		\centering
		\def\svgwidth{0.3\columnwidth}
		\input{img/coordinate_net.pdf_tex}
		\caption{Координатная сеть полярной системы координат}
	\end{figure*}

	\begin{Def}
		Цилиндрической системой координат называют трёхмерную систему координат, являющуюся расширением полярной системы координат путём добавления третьей координаты (обычно обозначаемой ${\displaystyle z}$), которая задаёт высоту точки над плоскостью.
	\end{Def}

	\begin{Def}
		Сферическая система координат — трёхмерная система координат, в которой каждая точка пространства определяется тремя числами, где r — расстояние до начала координат, а $\theta$ и $\varphi$ — зенитный и азимутальный углы соответственно.
	\end{Def}

	\Subsubsection{Параметризации}

	Построим декартову систему координат. Теперь возьмем какую-то новую систему координат $x', y', z'$.
	Проведем через $x', y'$ плоскость. Если $z'$ не совпадает с $z$, то эта плоскость пересекает плоскость $(x, y)$ по какой-то прямой.
	Отсчитает от вектора $x$ до этой прямой угол $\PHI$. Угол между $z$ и $z'$ обозначим за $\psi$.
	Теперь, мы можем эту прямую поворачивать вокруг оси $z'$ на угол $\delta$, пока она не совпадет с $x'$.
	
	\begin{figure*}
		\centering
		\def\svgwidth{0.3\columnwidth}
		\input{img/parametrisation.pdf_tex}
	\end{figure*}

	Таким образом, мы совместили исходную систему координат с новой СК. То есть
	мы построили соответствие между ( $\psi, \PHI, \delta$ ).
	
	\Subsection{Понятие вектора}

	Пусть $E$ -- евклидово пространство.
	\begin{Def}
		Закрепленный вектор -- упорядоченная пара точек в евклидовом пространстве.
		Обозначение: $\overrightarrow{AB}$, модуль $|\overrightarrow{AB}|$ -- расстояние между точками $A$ и $B$.
	\end{Def}

	\begin{Def}
		Пусть $\{(A, B), A, B \in E\}$ -- множество закрепленных векторов. Введём на нём отношение равенства:
		$(A, B) = (C, D) \EQ$:
		\begin{MyList}
			\item $|\overrightarrow{AB}| = |\overrightarrow{CD}|$ 
			\item $(A, B) || (C, D)$ либо совпадают.
			\item $\overrightarrow{AB} \upuparrows \overrightarrow{CD}$. 
		\end{MyList}
	\end{Def}

	\begin{Rem}
		$\forall A, B \to (A, A) = (B, B)$.
	\end{Rem}

	\begin{Prop}
		Отношение, введённое в прошлом определении -- отношение эквивалентности.
	\end{Prop}

	\begin{proof}
		\begin{MyList}
			\item Рефлексивность: $(A, B) = (A, B)$ -- верно.
			\item Симметричность -- очевидно.
			\item Транзитивность: $(A, B) = (C, D), (C, D) = (F, G) \SO (A, B) = (F, G)$ -- верно.
		\end{MyList}
		Значит множество закрепленных векторов разбивается на классы эквивалентности.
	\end{proof}

	\begin{Def}
		Класс эквивалентности называется \textbf{свободным вектором}.
	\end{Def}

	\Subsection{Сложение и умножение на число}

	Пусть $\overline{a}, \overline{b} \in V$ -- классы.

	\begin{Def}
		Сложение векторов: $V \times V \to V$.
		$[\overrightarrow{OO''}] = \overline{a} + \overline{b}$ 
	\end{Def}

	\begin{Def}
		Пусть $\overline{a} \in V, \lambda \in \R$. Умножение на число на число: $\R \times V \to V$.
	\end{Def}
	
	$(V, +, \cdot)$. Свойства:
	\begin{MyList}
		\item $\forall \overline{a}, \overline{b} \in V \ \overline{a} + \overline{b} = \overline{b} + \overline{a}$.
		\item $\forall \overline{a}, \overline{b}, \overline{c} \in V \ (\overline{a} + \overline{b}) + \overline{c} = \overline{a} + (\overline{b} + \overline{c})$.
		\item $\exists \overline{0} : \forall \overline{a} \ \overline{a} + \overline{0} = \overline{0} + \overline{a} = \overline{a}$.
		\item $\forall \overline{a} \ \exists -\overline{a} : \overline{a} + (-\overline{a}) = \overline{0}$.
		\item $\forall \lambda \in \R, \overline{a}, \overline{b} \in V \ \lambda(\overline{a} + \overline{b}) = \lambda \overline{a} + \lambda \overline{b}$.
		\item $\forall \lambda, \mu \in \R, \overline{a} \in V \ (\lambda + \mu) \overline{a} = \lambda \overline{a} + \mu \overline{a}$.
		\item $\forall \overline{a} \in V \ 1 \cdot \overline{a} = \overline{a}$.
		\item $\forall \lambda, \mu \in \R, \overline{a} \in V \ \lambda(\mu \overline{a}) = (\lambda \mu) \overline{a}$.
	\end{MyList} 

	\begin{Def}
		Множество $(V, +, \cdot)$, удовлетворяющее свойствам 1-8, называется \textbf{векторным пространством}. Элементы -- векторы.
	\end{Def}

	\Subsection{ЛЗ, ЛНЗ, Базис, размерность}

	\begin{Def}
		$\lambda_1 \overline{a}_1 + ... + \lambda_n \overline{a}_n$ -- линейная комбинация. Если $(\lambda_1, ..., \lambda_n) \neq (0, ..., 0)$ -- нетривиальная ЛК.  
	\end{Def}

	\begin{Def}
		$\{\overline{a}_i\}_{i = 1}^n$ -- линейно зависимый, если $\exists$ нетривиальная ЛК $\{\lambda_i\}_{i = 1}^n : \sum_{i = 1}^n \lambda_i \overline{a}_i = 0$   
	\end{Def}

	\begin{Def}
		$\{\overline{a}_i\}_{i = 1}^n$ -- ЛНЗ, если он не ЛЗ. 
	\end{Def}

	Свойства:
	\begin{MyList}
		\item $\{\overline{a} \neq \overline{0}\}$ -- ЛНЗ.
		\item $\{\overline{0}\}$ -- ЛЗ.
		\item $\{\overline{a_1}, ..., \overline{a}_n, \overline{0}\}$ -- ЛЗ.
		\item Пусть $\{\overline{a}_i\}$ -- ЛЗ. Тогда $\{\overline{a}_i, \overline{a}_j\}_{i = 1, j = 1}^{n, m}$ -- ЛЗ.
	\end{MyList}

	\begin{Def}
		$\{\overline{a}_\alpha\}_{\alpha \in \Lambda}$ -- ЛЗ, если в нем $\exists$ ЛЗ конечный поднабор.
	\end{Def}

	\begin{Def}
		ЛНЗ -- набор, который не является ЛЗ.
	\end{Def}

	\begin{Def}
		$\{\overline{a}_\alpha\}_{\alpha \in \Lambda}$ -- полный, если $\forall \overline{v} \in V \ \exists \{\alpha_i\}_{i = 1}^n, \{\lambda_i\}_{i = 1}^n \ \overline{v} = \lambda_1 \overline{a}_{\alpha_1} + ... + \lambda_n \overline{a}_{\alpha_n}$.
	\end{Def}

	\begin{Def}
		$\{\overline{a}_\alpha\}_{\alpha \in \Lambda}$ -- базис $V$, если он полный и ЛНЗ.
	\end{Def}

	\begin{Def}
		Размерность $V$ ( $\dim V$ ) -- мощность базиса.
	\end{Def}

	\begin{Def}
		Векторное пространство $V$ называется конечномерным, если $\exists$ конечный полный набор.
	\end{Def}

\end{document}