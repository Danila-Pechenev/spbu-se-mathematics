\documentclass[12pt]{article}

% Автор стиля: Сергей Копелиович
% Автор конспекта: Илья Дудников

\usepackage{cmap}
\usepackage[T2A]{fontenc}
\usepackage[utf8]{inputenc}
\usepackage[russian]{babel}
\usepackage{graphicx}
\usepackage{amsthm,amsmath,amssymb}
\usepackage{listings}
\usepackage{color}
\usepackage{xcolor}
\usepackage{array}
\usepackage{epigraph}

\usepackage[russian,colorlinks=true,urlcolor=red,linkcolor=blue]{hyperref}
\usepackage{enumerate}
\usepackage{datetime}
\usepackage{fancyhdr}
\usepackage{lastpage}
\usepackage{verbatim}
\usepackage{tikz}
\usetikzlibrary{arrows,decorations.markings,decorations.pathmorphing}
\usepackage{pgfplots}

\usepackage{ifthen}
\usepackage{mathtools}

%\usepackage{tabls}
%\usepackage{tabularx}
%\usepackage{xifthen}
%\listfiles

\def\NAME{Лекция, 09/25/21}

\sloppy
\voffset=-20mm
\textheight=235mm
\hoffset=-22mm
\textwidth=180mm
\headsep=12pt
\footskip=20pt

\parskip=0em
\parindent=0em

\setlength\epigraphwidth{.8\textwidth}

\newlength{\tmplen}
\newlength{\tmpwidth}
\newcounter{listcounter}

% Список с маленькими отступами
\newenvironment{MyList}[1][4pt]{
  \begin{enumerate}[1.]
  \setlength{\parskip}{0pt}
  \setlength{\itemsep}{#1}
}{       
  \end{enumerate}
}
% Вложенный список с маленькими отступами
\newenvironment{InnerMyList}[1][0pt]{
  \vspace*{-0.5em}
  \begin{enumerate}[(a)]
  \setlength{\parskip}{-0pt}
  \setlength{\itemsep}{#1}
}{       
  \end{enumerate}
  \vspace*{-0.5em}
}
% Список с маленькими отступами
\newenvironment{MyItemize}[1][4pt]{
  \begin{itemize}
  \setlength{\parskip}{0pt}
  \setlength{\itemsep}{#1}
}{       
  \end{itemize}
}

% Основные математические символы
\def\TODO{{\color{red}\bf TODO}}
\def\N{\mathbb{N}}       %
\def\R{\mathbb{R}}       %
\def\F2{\mathbb{F}_2}    %
\def\Z{\mathbb{Z}}       %
\def\INF{\t{+}\infty}    % +inf
\def\EPS{\varepsilon}    %
\def\EMPTY{\varnothing}  %
\def\PHI{\varphi}        %
\def\SO{\Rightarrow}     % =>
\def\EQ{\Leftrightarrow} % <=>
\def\t{\texttt}          % mono font
\def\c#1{{\rm\sc{#1}}}   % font for classes NP, SAT, etc
\def\O{\mathcal{O}}      %
\def\NO{\t{\#}}          % #
\def\XOR{\text{ {\raisebox{-2pt}{\ensuremath{\Hat{}}}} }}
\renewcommand{\le}{\leqslant}
\renewcommand{\ge}{\geqslant}
\newcommand{\q}[1]{\langle #1 \rangle}               % <x>
\newcommand\URL[1]{{\footnotesize{\url{#1}}}}        %
% \newcommand{\sfrac}[2]{{\scriptscriptstyle\frac{#1}{#2}}}  % Очень маленькая дробь
% \newcommand{\mfrac}[2]{{\scriptstyle\frac{#1}{#2}}}    % Небольшая дробь
\newcommand{\sfrac}[2]{{\scriptstyle\frac{#1}{#2}}}  % Очень маленькая дробь
\newcommand{\mfrac}[2]{{\textstyle\frac{#1}{#2}}}    % Небольшая дробь

\newcommand{\fix}[1]{{\color{fixcolor}{#1}}} % \underline
\def\bonus{\t{\red{(*)}}}
\def\ifbonus#1{\ifthenelse{\equal{#1}{}}{}{\bonus}}
\def\smallsquare{$\scalebox{0.5}{$\square$}$}

\newlength{\myItemLength}
\setlength{\myItemLength}{0.3em}
\def\ItemSymbol{\smallsquare}
\def\Item{\vspace*{\myItemLength}\ItemSymbol \ \ }

\newcommand{\LET}{%
  % [line width=0.6pt]
  \begin{tikzpicture}%
  \draw(0.8ex,0) -- (0.8ex,1.6ex);%
  \draw(0,1.6ex) -- (0.8ex,1.6ex);%
  \end{tikzpicture}%
  \hspace*{0.1em}%
}

% Отступы
\def\makeparindent{\hspace*{\parindent}\unskip}
\def\up{\vspace*{-0.5em}}%{\vspace*{-\baselineskip}}
\def\down{\vspace*{0.5em}}
\def\LINE{\vspace*{-1em}\noindent \underline{\hbox to 1\textwidth{{ } \hfil{ } \hfil{ } }}}
\def\BOX#1{\mbox{\fbox{\bf{#1}}}}
\def\Pagebreak{\pagebreak\vspace*{-1.5em}}

% Мелкий заголовок
\newcommand{\THEE}[1]{
  \vspace*{0.5em}
  \noindent{\bf \underline{#1}}%\hspace{0.5em}
  \vspace*{0.2em}
}
% Другой тип мелкого заголовка
\newcommand{\THE}[1]{
  \vspace*{0.5em} $\bullet$
  \noindent{\bf #1}%\hspace{0.5em}
  \vspace*{0.2em}
}

\newenvironment{MyTabbing}{
  \t\bgroup
  \vspace*{-\baselineskip}
  \begin{tabbing}
    aaaa\=aaaa\=aaaa\=aaaa\=aaaa\=aaaa\kill
}{
  \end{tabbing}
  \t\egroup
}

% Код с правильными отступами
\lstnewenvironment{code}{
  \lstset{}
%  \vspace*{-0.2em}
}%
{
%  \vspace*{-0.2em}
}
\lstnewenvironment{codep}{
  \lstset{language=python}
}%
{
}

% Формулы с правильными отступами
\newenvironment{smallformula}{
 
  \vspace*{-0.8em}
}{
  \vspace*{-1.2em}
  
}
\newenvironment{formula}{
 
  \vspace*{-0.4em}
}{
  \vspace*{-0.6em}
  
}

% Большая квадратная скобка
\makeatletter
\newenvironment{sqcases}{%
  \matrix@check\sqcases\env@sqcases
}{%
  \endarray\right.%
}
\def\env@sqcases{%
  \let\@ifnextchar\new@ifnextchar
  \left\lbrack
  \def\arraystretch{1.2}%
  \array{@{}l@{\quad}l@{}}%
}
\makeatother

% Определяем основные секции: \begin{Lm}, \begin{Thm}, \begin{Def}, \begin{Rem}
\renewcommand{\qedsymbol}{$\blacksquare$}
\theoremstyle{definition} % жирный заголовок, плоский текст
\newtheorem{Thm}{\underline{Теорема}}[subsection] % нумерация будет "<номер subsection>.<номер теоремы>"
\newtheorem{Lm}[Thm]{\underline{Lm}} % Нумерация такая же, как и у теорем
\newtheorem{Ex}[Thm]{Упражнение} % Нумерация такая же, как и у теорем
\newtheorem{Example}[Thm]{Пример} % Нумерация такая же, как и у теорем
\newtheorem{Code}[Thm]{Код} % Нумерация такая же, как и у теорем
\theoremstyle{plain} % жирный заголовок, курсивный текст
\newtheorem{Def}[Thm]{Def.} % Нумерация такая же, как и у теорем
\theoremstyle{remark} % курсивный заголовок, плоский текст
\newtheorem{Cons}[Thm]{Следствие} % Нумерация такая же, как и у теорем
\newtheorem{Conj}[Thm]{Гипотеза} % Нумерация такая же, как и у теорем
\newtheorem{Prop}[Thm]{Утверждение} % Нумерация такая же, как и у теорем
\newtheorem{Rem}[Thm]{Замечание} % Нумерация такая же, как и у теорем
\newtheorem{Remark}[Thm]{Замечание} % Нумерация такая же, как и у теорем
\newtheorem{Algo}[Thm]{Алгоритм} % Нумерация такая же, как и у теорем

% Определяем ЗАГОЛОВКИ
\def\SectionName{Дискретная математика}
\def\AuthorName{Илья Дудников}

\newlength{\sectionvskip}
\setlength{\sectionvskip}{0.5em}
\newcommand{\Section}[4][]{
  % Заголовок
  \pagebreak
%  \ifthenelse{\isempty{#1}}{
    \refstepcounter{section}
%  }{}
  \vspace{0.5em}
%  \ifthenelse{\isempty{#1}}{
%    \addtocontents{toc}{\protect\addvspace{-5pt}}%
    \addcontentsline{toc}{section}{\arabic{section}. #2}
%  }{}
  


  % Запомнили название и автора главы
  \gdef\SectionName{#2}
  \gdef\AuthorName{#4}

  % Заголовок страницы
  \lhead{Конспект, \NAME}
  \chead{}
  \rhead{\SectionName}
  \renewcommand{\headrulewidth}{0.4pt}
    
  \lfoot{}
  \cfoot{\thepage\t{/}\pageref*{LastPage}}
  \rfoot{Автор: \AuthorName}
  \renewcommand{\footrulewidth}{0.4pt}
}

\newcommand{\Subsection}[2][]{
  \refstepcounter{subsection}
  \vspace*{1em}
  \ifthenelse{\equal{#1}{}}
    {\addcontentsline{toc}{subsection}{\arabic{section}.\arabic{subsection}. #2}}
    {\addcontentsline{toc}{subsection}{\arabic{section}.\arabic{subsection}. \bonus\,#2}}
  {\color{blue}\bf\large \arabic{section}.\arabic{subsection}. \ifbonus{#1}\,{#2}} 
  \vspace*{0.5em}
  \makeparindent
}
\newcommand{\Subsubsection}[2][]{
  \refstepcounter{subsubsection}
  \vspace*{1em}
  \ifthenelse{\equal{#1}{}}
    {\addcontentsline{toc}{subsubsection}{\arabic{section}.\arabic{subsection}.\arabic{subsubsection}. #2}}
    {\addcontentsline{toc}{subsubsection}{\arabic{section}.\arabic{subsection}.\arabic{subsubsection}. \bonus\,#2}}
  {\color{blue}\bf\large \arabic{section}.\arabic{subsection}.\arabic{subsubsection}. \ifbonus{#1}\,#2}
  \vspace*{0.5em}
  \makeparindent
}

\newcommand{\Header}{
  \pagestyle{empty}
  \renewcommand{\dateseparator}{--}
  \begin{center}
    {\Large\bf 
     Дискретная математика 1 семестр ПИ,\\
    \vspace{0.3em}
     \NAME}\\
    \vspace{0.7em}
    {Собрано {\today} в {\currenttime}}
  \end{center}

  \LINE
  \vspace{0em}

  \renewcommand{\baselinestretch}{0.98}\normalsize
  \tableofcontents
  \renewcommand{\baselinestretch}{1.0}\normalsize
  \pagebreak
}

\newcommand{\BeginConspect}{
  \pagestyle{fancy}
  \setcounter{page}{1}
}

\definecolor{mygray}{rgb}{0.7,0.7,0.7}
\definecolor{ltgray}{rgb}{0.9,0.9,0.9}
\definecolor{fixcolor}{rgb}{0.7,0,0}
\definecolor{red2}{rgb}{0.7,0,0}
\definecolor{dkred}{rgb}{0.4,0,0}
\definecolor{dkblue}{rgb}{0,0,0.6}
\definecolor{dkgreen}{rgb}{0,0.6,0}
\definecolor{brown}{rgb}{0.5,0.5,0}

\newcommand{\green}[1]{{\color{green}{#1}}}
\newcommand{\black}[1]{{\color{black}{#1}}}
\newcommand{\red}[1]{{\color{red}{#1}}}
\newcommand{\dkred}[1]{{\color{dkred}{#1}}}
\newcommand{\blue}[1]{{\color{blue}{#1}}}
\newcommand{\dkgreen}[1]{{\color{dkgreen}{#1}}}

\begin{document}

\Header

\BeginConspect

\Section{Основы комбинаторики}{}{Илья Дудников}

\Subsection{Множества}

\begin{Def}
    Множество - совокупность объектов.
\end{Def}

\begin{Def}
    Покрытием множества $A$ называется множество $B = \left\{B_1, B_2, ..., B_k\right\} : \bigcup_i B_i \supset A$ 
\end{Def}

\begin{Def}
    Разбиением множества $A$ называется $\pi (X) = \{X_i\} : $
    \[X_i \neq \varnothing, \bigcup_i X_i = A, \forall i \neq j \to X_i \cap X_j = \varnothing\] 
\end{Def}

\begin{Def}
    Пусть $B, C$ -- разбиения $A$. $B$ называется измельчением $C$, если $B$ -- разбиение $A$ и $\forall i \ \exists j : B_i \subset C_j$   
\end{Def}

\Subsection{Мощность множества}

\begin{MyList}
    \item $|\varnothing| = 0$ 
    \item $X = \left\{x_1, x_2, ..., x_n\right\} \SO |X| = n$ 
    \item $\N$ -- счётное. $\Z$ -- тоже счётное:
    \[f(x) = \begin{cases}
        1, x = 0 \\
        2x, x > 0 \\
        2|x| + 1, x < 0
    \end{cases}\]
    \item $[0, 1]$. Пусть существует $q: \N \to [0, 1]$
    
    \begin{MyList}
        \item $0, a_1 a_2 ... a_k ...$
        \item $0, b_1 b_2 ... b_k ...$ 
        \item $0, c_1 c_2 ... c_k ...$  
    \end{MyList}
    Рассмотрим $\alpha = 0, \alpha_1 \alpha_2 \alpha_3 ... \alpha_k ..., \alpha_1 \neq a_1, \alpha_2 \neq b_2, \alpha_3 \neq c_3$ и т.д. 
    Таким образом, всегда найдётся не пронумерованное число. 

    $|[0, 1]|$ -- континуум 
\end{MyList}

\begin{Def}
    Множество всех подмножеств $A$ обозначается $2^A$ 
\end{Def}

\begin{Prop}
    $|2^A| = 2^{|A|}$  
\end{Prop}

\begin{proof}
    База: $A = \varnothing, |A| = 0, 2^A = \{\varnothing\} \SO |2^A| = 2^{|A|} = 1$
    
    Индукционное предположение: Пусть $\forall A : |A| \leqslant k \to |2^A| = 2^{|A|}$ \\
    Индукционный переход:

    Рассмотрим $A : |A| = k + 1, B_1 \in 2^{A \setminus \{x_{k + 1}\}}, B = \{x_{k + 1}\} \cup B_1$ 

    $2^A = 2^{A \setminus \{x_{k + 1}\}} \cup \{B\}$ 

    \[\begin{cases}
        |2^{A \setminus \{x_{k + 1}\}}| = 2^k \\
        |\{B\}| = 2^k
    \end{cases} \SO 2^A = 2^k + 2^k = 2^{k + 1} = 2^{|A|}\]
\end{proof}

\Pagebreak
\Subsection{Комбинаторика}

\begin{MyList}
    \item $A, B : A \cap B = \varnothing$ $$|A \cup B| = |A| + |B|$$
    \item $A_1, ..., A_n, \forall i, j \to (i \neq j \SO A_i \cap A_j = \varnothing)$
    \[|\bigcup_{i = 1}^n A_i| = \sum_{i=1}^{n} |A_i|\]
    \item $A, B, A \cap B \neq \varnothing$ \[|A \cup B| = |A| + |B| - |A \cap B|\] 
    \item $A_1, ..., A_n$ \[|\bigcup_{i = 1}^n A_i| = \sum_{i=1}^{n} |A_i| - \sum_{i, j=1}^{n} |A_i \cap A_j| + \sum_{i,j,k=1}^{n} |A_i \cap A_j \cap A_k - ... + (-1)^{n + 1}|\bigcap_{i = 1}^n A_i|\]
    \item $A, B$ \[|A \times B| = |A| \cdot |B|\]
    \item $A_1, ..., A_n$ \[|A_1 \times A_2 \times ... \times A_n| = \prod_{i = 1}^n |A_i|\]   
\end{MyList}

\begin{MyList}
    \item Перестановки: $<a_1 ... a_n> = \overline{<a_1 ... a_n>, a_n}$. Тогда
    \[|<1:n>| = |<1:n - 1> \times (1:n)| = |<1:n - 1>| \cdot n = 1 \cdot 2 \cdot ... \cdot n = n!\]

    \item Размещения. $n \cdot (n - 1) \cdot ... \cdot (n - k + 1)$
    \[A_n^k = \frac{n!}{(n - k)!}\]

    \item Сочетания. 
    \[A_n^k = C_n^k \cdot k! \EQ C_n^k = \frac{n!}{k!(n - k)!}\]

    \item Сочетания с повторениями. Выставим все $k$ выбранных объектов в ряд и поставим между ними $n - 1$ перегородку: до первой перегородки будут элементы 1-го типа, от первой до второй перегородки -- 2-го типа и т.д.
    Таким образом, всего $n + k - 1$ место. Нам нужно выбрать $n - 1$ перегородку из этих $n + k - 1$ мест
    \[\overline{C}_n^k = C_{n + k - 1}^{n - 1} = C_{n + k - 1}^k\]
\end{MyList}

\begin{Def}
Пусть дан выпуклый $n$-угольник. Найти количество способов разбить его на треугольники с непересекающимися сторонами

\[C_0 = 1, C_n = \sum_{i=0}^{n - 1} C_i \cdot C_{n - i - 1} - \text{ Числа Каталана}\]

\end{Def}
\end{document}