\documentclass[12pt]{article}

% Автор стиля: Сергей Копелиович
% Автор конспекта: Илья Дудников

\usepackage{cmap}
\usepackage[T2A]{fontenc}
\usepackage[utf8]{inputenc}
\usepackage[russian]{babel}
\usepackage{graphicx}
\usepackage{amsthm,amsmath,amssymb}
\usepackage{listings}
\usepackage{color}
\usepackage{xcolor}
\usepackage{array}
\usepackage{epigraph}

\usepackage[russian,colorlinks=true,urlcolor=red,linkcolor=blue]{hyperref}
\usepackage{enumerate}
\usepackage{datetime}
\usepackage{fancyhdr}
\usepackage{lastpage}
\usepackage{verbatim}
\usepackage{tikz}
\usetikzlibrary{arrows,decorations.markings,decorations.pathmorphing}
\usepackage{pgfplots}

\usepackage{ifthen}
\usepackage{mathtools}

%\usepackage{tabls}
%\usepackage{tabularx}
%\usepackage{xifthen}
%\listfiles

\def\NAME{Лекция, 10/02/21}

\sloppy
\voffset=-20mm
\textheight=235mm
\hoffset=-22mm
\textwidth=180mm
\headsep=12pt
\footskip=20pt

\parskip=0em
\parindent=0em

\setlength\epigraphwidth{.8\textwidth}

\newlength{\tmplen}
\newlength{\tmpwidth}
\newcounter{listcounter}

% Список с маленькими отступами
\newenvironment{MyList}[1][4pt]{
  \begin{enumerate}[1.]
  \setlength{\parskip}{0pt}
  \setlength{\itemsep}{#1}
}{       
  \end{enumerate}
}
% Вложенный список с маленькими отступами
\newenvironment{InnerMyList}[1][0pt]{
  \vspace*{-0.5em}
  \begin{enumerate}[(a)]
  \setlength{\parskip}{-0pt}
  \setlength{\itemsep}{#1}
}{       
  \end{enumerate}
  \vspace*{-0.5em}
}
% Список с маленькими отступами
\newenvironment{MyItemize}[1][4pt]{
  \begin{itemize}
  \setlength{\parskip}{0pt}
  \setlength{\itemsep}{#1}
}{       
  \end{itemize}
}

% Основные математические символы
\def\TODO{{\color{red}\bf TODO}}
\def\N{\mathbb{N}}       %
\def\R{\mathbb{R}}       %
\def\F2{\mathbb{F}_2}    %
\def\Z{\mathbb{Z}}       %
\def\INF{\t{+}\infty}    % +inf
\def\EPS{\varepsilon}    %
\def\EMPTY{\varnothing}  %
\def\PHI{\varphi}        %
\def\SO{\Rightarrow}     % =>
\def\EQ{\Leftrightarrow} % <=>
\def\t{\texttt}          % mono font
\def\c#1{{\rm\sc{#1}}}   % font for classes NP, SAT, etc
\def\O{\mathcal{O}}      %
\def\NO{\t{\#}}          % #
\def\XOR{\text{ {\raisebox{-2pt}{\ensuremath{\Hat{}}}} }}
\renewcommand{\le}{\leqslant}
\renewcommand{\ge}{\geqslant}
\newcommand{\q}[1]{\langle #1 \rangle}               % <x>
\newcommand\URL[1]{{\footnotesize{\url{#1}}}}        %
% \newcommand{\sfrac}[2]{{\scriptscriptstyle\frac{#1}{#2}}}  % Очень маленькая дробь
% \newcommand{\mfrac}[2]{{\scriptstyle\frac{#1}{#2}}}    % Небольшая дробь
\newcommand{\sfrac}[2]{{\scriptstyle\frac{#1}{#2}}}  % Очень маленькая дробь
\newcommand{\mfrac}[2]{{\textstyle\frac{#1}{#2}}}    % Небольшая дробь

\newcommand{\fix}[1]{{\color{fixcolor}{#1}}} % \underline
\def\bonus{\t{\red{(*)}}}
\def\ifbonus#1{\ifthenelse{\equal{#1}{}}{}{\bonus}}
\def\smallsquare{$\scalebox{0.5}{$\square$}$}

\newlength{\myItemLength}
\setlength{\myItemLength}{0.3em}
\def\ItemSymbol{\smallsquare}
\def\Item{\vspace*{\myItemLength}\ItemSymbol \ \ }

\newcommand{\LET}{%
  % [line width=0.6pt]
  \begin{tikzpicture}%
  \draw(0.8ex,0) -- (0.8ex,1.6ex);%
  \draw(0,1.6ex) -- (0.8ex,1.6ex);%
  \end{tikzpicture}%
  \hspace*{0.1em}%
}

% Отступы
\def\makeparindent{\hspace*{\parindent}\unskip}
\def\up{\vspace*{-0.5em}}%{\vspace*{-\baselineskip}}
\def\down{\vspace*{0.5em}}
\def\LINE{\vspace*{-1em}\noindent \underline{\hbox to 1\textwidth{{ } \hfil{ } \hfil{ } }}}
\def\BOX#1{\mbox{\fbox{\bf{#1}}}}
\def\Pagebreak{\pagebreak\vspace*{-1.5em}}

% Мелкий заголовок
\newcommand{\THEE}[1]{
  \vspace*{0.5em}
  \noindent{\bf \underline{#1}}%\hspace{0.5em}
  \vspace*{0.2em}
}
% Другой тип мелкого заголовка
\newcommand{\THE}[1]{
  \vspace*{0.5em} $\bullet$
  \noindent{\bf #1}%\hspace{0.5em}
  \vspace*{0.2em}
}

\newenvironment{MyTabbing}{
  \t\bgroup
  \vspace*{-\baselineskip}
  \begin{tabbing}
    aaaa\=aaaa\=aaaa\=aaaa\=aaaa\=aaaa\kill
}{
  \end{tabbing}
  \t\egroup
}

% Код с правильными отступами
\lstnewenvironment{code}{
  \lstset{}
%  \vspace*{-0.2em}
}%
{
%  \vspace*{-0.2em}
}
\lstnewenvironment{codep}{
  \lstset{language=python}
}%
{
}

% Формулы с правильными отступами
\newenvironment{smallformula}{
 
  \vspace*{-0.8em}
}{
  \vspace*{-1.2em}
  
}
\newenvironment{formula}{
 
  \vspace*{-0.4em}
}{
  \vspace*{-0.6em}
  
}

% Большая квадратная скобка
\makeatletter
\newenvironment{sqcases}{%
  \matrix@check\sqcases\env@sqcases
}{%
  \endarray\right.%
}
\def\env@sqcases{%
  \let\@ifnextchar\new@ifnextchar
  \left\lbrack
  \def\arraystretch{1.2}%
  \array{@{}l@{\quad}l@{}}%
}
\makeatother

\DeclareMathOperator{\sort}{sort}

% Определяем основные секции: \begin{Lm}, \begin{Thm}, \begin{Def}, \begin{Rem}
\renewcommand{\qedsymbol}{$\blacksquare$}
\theoremstyle{definition} % жирный заголовок, плоский текст
\newtheorem{Thm}{\underline{Теорема}}[subsection] % нумерация будет "<номер subsection>.<номер теоремы>"
\newtheorem{Lm}[Thm]{\underline{Lm}} % Нумерация такая же, как и у теорем
\newtheorem{Ex}[Thm]{Упражнение} % Нумерация такая же, как и у теорем
\newtheorem{Example}[Thm]{Пример} % Нумерация такая же, как и у теорем
\newtheorem{Code}[Thm]{Код} % Нумерация такая же, как и у теорем
\theoremstyle{plain} % жирный заголовок, курсивный текст
\newtheorem{Def}[Thm]{Def.} % Нумерация такая же, как и у теорем
\theoremstyle{remark} % курсивный заголовок, плоский текст
\newtheorem{Cons}[Thm]{Следствие} % Нумерация такая же, как и у теорем
\newtheorem{Conj}[Thm]{Гипотеза} % Нумерация такая же, как и у теорем
\newtheorem{Prop}[Thm]{Утверждение} % Нумерация такая же, как и у теорем
\newtheorem{Rem}[Thm]{Замечание} % Нумерация такая же, как и у теорем
\newtheorem{Remark}[Thm]{Замечание} % Нумерация такая же, как и у теорем
\newtheorem{Algo}[Thm]{Алгоритм} % Нумерация такая же, как и у теорем

% Определяем ЗАГОЛОВКИ
\def\SectionName{Дискретная математика}
\def\AuthorName{Илья Дудников}

\newlength{\sectionvskip}
\setlength{\sectionvskip}{0.5em}
\newcommand{\Section}[4][]{
  % Заголовок
  \pagebreak
%  \ifthenelse{\isempty{#1}}{
    \refstepcounter{section}
%  }{}
  \vspace{0.5em}
%  \ifthenelse{\isempty{#1}}{
%    \addtocontents{toc}{\protect\addvspace{-5pt}}%
    \addcontentsline{toc}{section}{\arabic{section}. #2}
%  }{}
  


  % Запомнили название и автора главы
  \gdef\SectionName{#2}
  \gdef\AuthorName{#4}

  % Заголовок страницы
  \lhead{Конспект, \NAME}
  \chead{}
  \rhead{\SectionName}
  \renewcommand{\headrulewidth}{0.4pt}
    
  \lfoot{}
  \cfoot{\thepage\t{/}\pageref*{LastPage}}
  \rfoot{Автор: \AuthorName}
  \renewcommand{\footrulewidth}{0.4pt}
}

\newcommand{\Subsection}[2][]{
  \refstepcounter{subsection}
  \vspace*{1em}
  \ifthenelse{\equal{#1}{}}
    {\addcontentsline{toc}{subsection}{\arabic{section}.\arabic{subsection}. #2}}
    {\addcontentsline{toc}{subsection}{\arabic{section}.\arabic{subsection}. \bonus\,#2}}
  {\color{blue}\bf\large \arabic{section}.\arabic{subsection}. \ifbonus{#1}\,{#2}} 
  \vspace*{0.5em}
  \makeparindent
}
\newcommand{\Subsubsection}[2][]{
  \refstepcounter{subsubsection}
  \vspace*{1em}
  \ifthenelse{\equal{#1}{}}
    {\addcontentsline{toc}{subsubsection}{\arabic{section}.\arabic{subsection}.\arabic{subsubsection}. #2}}
    {\addcontentsline{toc}{subsubsection}{\arabic{section}.\arabic{subsection}.\arabic{subsubsection}. \bonus\,#2}}
  {\color{blue}\bf\large \arabic{section}.\arabic{subsection}.\arabic{subsubsection}. \ifbonus{#1}\,#2}
  \vspace*{0.5em}
  \makeparindent
}

\newcommand{\Header}{
  \pagestyle{empty}
  \renewcommand{\dateseparator}{--}
  \begin{center}
    {\Large\bf 
     Дискретная математика 1 семестр ПИ,\\
    \vspace{0.3em}
     \NAME}\\
    \vspace{0.7em}
    {Собрано {\today} в {\currenttime}}
  \end{center}

  \LINE
  \vspace{0em}

  \renewcommand{\baselinestretch}{0.98}\normalsize
  \tableofcontents
  \renewcommand{\baselinestretch}{1.0}\normalsize
  \pagebreak
}

\newcommand{\BeginConspect}{
  \pagestyle{fancy}
  \setcounter{page}{1}
}

\definecolor{mygray}{rgb}{0.7,0.7,0.7}
\definecolor{ltgray}{rgb}{0.9,0.9,0.9}
\definecolor{fixcolor}{rgb}{0.7,0,0}
\definecolor{red2}{rgb}{0.7,0,0}
\definecolor{dkred}{rgb}{0.4,0,0}
\definecolor{dkblue}{rgb}{0,0,0.6}
\definecolor{dkgreen}{rgb}{0,0.6,0}
\definecolor{brown}{rgb}{0.5,0.5,0}

\newcommand{\green}[1]{{\color{green}{#1}}}
\newcommand{\black}[1]{{\color{black}{#1}}}
\newcommand{\red}[1]{{\color{red}{#1}}}
\newcommand{\dkred}[1]{{\color{dkred}{#1}}}
\newcommand{\blue}[1]{{\color{blue}{#1}}}
\newcommand{\dkgreen}[1]{{\color{dkgreen}{#1}}}

\begin{document}

\Header

\BeginConspect

\Section{Перестановки}{}{Илья Дудников}

\Subsection{Лексикографический порядок перестановок}

\begin{Def}
    Пусть есть две перестановки $X = \{x_1, x_2, ..., x_n\}, Y = \{y_1, y_2, ..., y_n\}$. Тогда 
    \[X < Y \EQ \exists k : x_i = y_i \ \forall i = 1, ..., k \wedge x_{k + 1} < y_{k + 1}\]   
\end{Def}

\begin{Algo}[Поиск следующей перестановки]
    Найдем наибольший убывающий суффикс. 
    Пусть $k : a_{k + 1} > a_{k + 2} > ... > a_n$. Тогда выберем из этого суффикса $a_i : a_i > a_k$ и $a_i$ минимально.
    После этого отсортируем получившийся суффикс. Получим перестановку:
    \[<a_1, a_2, ..., a_i, \sort [a_k, a_k + 1, ..., a_n]>\]
    Она и будет лексикографический следующей.
\end{Algo}

\Section{Числа Стирлинга}{}{Илья Дудников}

\Subsection{Числа Стирлинга}
\begin{Def}
    Пусть $A = \{a_1, ..., a_n\}$. Рассмотрим разбиение этого множества мощности $k$, т.е. $X = \{X_1, ..., X_k\}:$
    $$\forall i, j \to X_i \supset A, X_i \cap X_j = \varnothing, \bigcup_i X_i = A$$
    
    Тогда числами Стирлинга -- количество таких разбиений.
\end{Def}

\begin{MyList}
    \item $k = 2 \SO S(n, 2) = \frac{\sum_{i=1}^{n - 1} C_n^i}{2} = \frac{2^n - 2}{2} = 2^{n - 1} - 1$
    \item Общий случай.
    \begin{MyItemize}
        \item Если $\{a_n\}$ -- элемент разбиения, то таких разбиений $S(n - 1, k - 1)$
        \item $\exists i : a_n \in X_i, |X_i| > 1$. Тогда нужно найти количество разбиений $A \setminus \{a_n\}$ на $k$ множеств, а потом вставить $a_n$ в одно из этих множеств.
        
        Количество способов:
        \[S(n - 1, k) \cdot k\]
    \end{MyItemize}

    Тогда рекуррентная формула:
    \[S(n, k) = S(n - 1, k - 1) + S(n - 1, k) \cdot k\]
    Базовые значения:
    \[
    \begin{array}{cc}
    S(n, 0) = 0 & S(0, 0) = 0 \\ 
    S(k, n) = 0, k > n & S(n, 2) = 2^{n - 1} - 1 \\ 
    S(n, n - 1) = C_n^2 & 
    \end{array} 
    \]
\end{MyList}

\Subsection{Числа Белла}
\begin{Def}
    Числа Белла -- количество разбиений множества.
    \[B(n) = \sum_{i=1}^{n} S(n, i)\]
\end{Def}

\begin{Thm}[Формула чисел Белла]
    Рассмотрим произвольное разбиение множества $A$. $\exists i : a_{n + 1} \in X_i, |X_i| = j$.
    
    $|A \setminus X_i| = n + 1 - j$. Тогда количество способов выбрать $X_i$ равно $C_n^{j - 1} = C_n^{n + 1 - j}$
    
    Количество разбиений $A \setminus X_i$, в свою очередь, равно $B(n + 1 - j)$. Тогда 
    \[B(n + 1) = \sum_{j=1}^{n + 1} C_n^{n + 1 - j} \cdot B(n + 1 - j) = \sum_{k=0}^{n} C_n^k B(k)\]  
\end{Thm}

\Pagebreak
\begin{Thm}[Формула чисел Стирлинга]
    \[S(n, k) = \frac{1}{k!} \sum_{j=0}^{k} (-1)^j \cdot C_k^j (k - j)^n\]    
\end{Thm}

\begin{proof}
    База. $$S(0, k) = 0 = \frac{1}{k!} \sum_{j=0}^{k} (-1)^j \cdot C_k^j (k - j)^0 = \frac{1}{k!} \sum_{j=0}^{k} C_k^j \cdot (-1)^j = \frac{1}{k!} \cdot (1 - 1)^k$$ 
    ИП. \TODO
\end{proof}

\begin{proof}
    Альтернативное доказательство. \\
    Пусть $L = \{\rho \subseteq A \times \{1, ..., k\} | \rho \text{ -- сюръекция}\}$. Заметим, это множество равномощно множеству упорядоченных разбиений мощности $k$. \\
    $\{a_1, ..., a_n\} \to \{1, ..., k\}$. Элементы разбиения имеют следующий вид: $X_i = \{a_k | \rho(a_k) = i\}$. Т.к. отображение сюръективно, то $X_i$ непусты.
    \[S(n, k) = \frac{|L|}{k!}\]
    Чтобы посчитать мощность $L$, из общего количества отображения вычтем количество несюръективных отображений.
    Пусть $P_i = \{\rho \subset A \times \{1, ..., n\} | \forall a \in A \to \rho(a) \neq i\}$. Тогда количество несюръективных отображений равно:
    \[|\bigcup_{i = 1}^k P_i| = \sum_{j=1}^{k} (-1)^{j + 1} \sum_{1 \leqslant i_1 \leqslant i_2 \leqslant i_3 \leqslant ... \leqslant i_j \leqslant k} |P_{i_1} \cap P_{i_2} \cap ... \cap P_{i_j}|\]
    $|P_{i_1} \cap P_{i_2} \cap ... P_{i_j}|$ -- количество отображений из $A$ в $\{1, ..., k\} \setminus \{i_1, ..., i_j\}$ 
    \[|P_{i_1} \cap ... \cap P_{i_j}| = (k - j)^n\]
    \[\sum_{i_1 \leqslant i_2 \leqslant ... \leqslant i_j} |P_{i_1} \cap P_{i_2} \cap ... \cap P_{i_j}| = C_k^j (k - j)^n \]
    Тогда 
    \[|L| = k^n - \sum_{j=1}^{k} (-1)^{j + 1} C_k^j (k - j)^n = k^n + \sum_{j=1}^{k} (-1)^j C_k^j (k - j)^n = \sum_{j=0}^{k} (-1)^j C_k^j (k - j)^n\]
    Тогда искомая формула чисел Стирлинга:
    \[S(n, k) = \frac{1}{k!}\sum_{j=0}^{k} (-1)^j C_k^j (k - j)^n\] 
\end{proof}

\end{document}